\documentclass{article}
\begin{document}
ID: 199504001001
Content:
(a)	Find the remainder when \[2x^2+7x+9\] is divided by x + 4 ;(b)	Solve the equation \[x^3-2x^2-7x+12=0\] giving the solutions to two decimal places where necessary;(Note: Please enter the smaller value of x first in the answer space);(c)	Given that \[2x^2+x+C -= A(x+1)^2+B(x+1)+4 \] for all values of x, find the values of A, B and C .;(d)	Given that 7x-5 and x + 2`are factors of \[14x^3+ax^2+bx+10\] find the value of a and of b .Answers:

ID: 199504001002
Content:
a) A geometric progression has first term a and common ratio r. The second term is -80 and the fifth term is 1250. Calculate ; i) the value of a and r,; ii) the sum of the first seven terms.; b) Find the common ratio of the geometric series whose first term is 18 and whose sum to infinity is 42;c) An arithmetic progression with 19 terms has first term a and common difference d. Show that the sum of the last five terms is 5a + 80d.;The last term of the progression is 52 and the sum of the first five terms added to the sum of the last five terms is 310. Calculate the value of a and of d.Answers:

ID: 199504001003
Content:
a) Express \[63\sin x+16\cos x\]in the form \[R\sin \left ( x+\alpha  \right )\] where R is positive and \[\alpha\] is acute.;Find ;i)	the acute angle x for which $$63\sin x+16\cos x =50$$;ii)	the obtuse angle x for which \[63\sin x+16\cos x=0\];b) By first expressing \[\cos 3x\]as \[\cos \left ( 2x+x \right )\]show that \[\cos 3x -=4\cos ^{3}x-3\cos x\]Hence find all the angles between \[0^{\circ}\]and \[360^{\circ}\]which satisfy the equation \[\cos 3x +\cos ^{2}x=0\];(Note: Please enter the answers in ascending order)Answers:

ID: 199504001004
Content:
img;a) The table shows experimental values of two variables x and y. ; It is known that x and y are related by the equation \[y=\frac{a}{\left ( x+b \right )}\]where a and b are constants.; i) Plot xy against y and obtain a straight line graph;ii) Use your graph to estimate the value of a and of b .;iii)Obtain the value of the gradient of the straight line obtained when \[\frac{1}{y}\]is plotted against x.;img;b) The variables x and y are related in such a way that when y-2x is plotted against \[x^{2}\], a straight line is obtained passing through (1, -2) and (4, 7).;Find ;i)	y in terms of x.;ii)	the values of x when y = 11. (Note: Please enter the smaller value of x first in the answer space)Answers:

ID: 199504001005
Content:
a) Differentiate the following expressions with respect to x.;i)	\[\frac{\left ( 2x-1 \right )}{\left ( x+5 \right )}\]; ii)	\[\sqrt{25-5x^3}\];b) Find the gradient of the curve  \[y^2+2xy=(x-3)^3+16\] at the point (1, 2).;c) Determine the x-coordinate of the stationary point of the curve \[y=x\ln x-2x\]Answers:

ID: 199504001006
Content:
(a) Solve the equation \[2 \lg(x + 2) + \lg 4 = \lg x + 4 \lg 3\].;(b) Solve the equation \[2^{2x}-2^{x+3}+7=0\];(Note: Please enter the smaller value of x first in the answer space);(c)    Sketch the curve \[y=e^{2x}-2\] and calculate the gradient of the curve at the point where the curve meets the x-axis.Answers:

ID: 199504001007
Content:
A particle moves in a straight line so that, at time t seconds after leaving a fixed point O, its velocity v \[ms^{-1}\] is given by `\[v=\frac{27}{(2t+1)^2}-3\];Find ;i) the value of t for which the particle is at instantaneous rest,;ii) the initial acceleration of the particle,;iii) the displacement of the particle from O when \[t=\frac{1}{2}\]Answers:

ID: 199504001008
Content:
Show that  \[(\cos x- \sin x)^{2}=1-\sin 2x\] `;img;The diagram shows part of the curve \[y=\cos x-\sin x\] ; Find the volume generated when the shaded region is rotated through a complete revolution about the x- axis.Answers:

ID: 199504001009
Content:
The parametric equations of a curve are \[x=t-\frac{1}{t}\] , \[y=t+\frac{1}{t}\] , where \[t\neq 0\];i) P is the point on the curve where t = 2. The tangent to the curve at P meets the x-axis at Q. The normal to the curve at P meets the x-axis at R. Calculate the area of the triangle PQR. ; ii)A is the point on the curve where t = k and B is the point on the curve where \[t=\frac{1}{k}\];Show that the line AB is parallel to the y-axis.Answers:

ID: 199504001010
Content:
A car accelerates uniformly from rest, at a rate of \[1.6ms^{-2}\] to a speed of \[20ms^{-1}\] It then continues moving at \[20ms^{-1}\]  It then continues moving at \[20ms^{-1}\] for T seconds before decelerating uniformly to a speed of \[12ms^{-1}\] in a further 5 seconds.;(a)	Sketch the velocity-time diagram for the motion of the car.;Find;(b)	The time taken while accelerating,;(c)	The distance moved while accelerating,;(d)	The distance moved while decelerating.;Given that the total distance moved by the car is 1245 m, find the value of T.;On another occasion the car accelerates uniformly from rest at a point A until it reaches B with a speed of \[15ms^{-1}\]. At B the driver observes an obstruction ahead at a point C and immediately applies the brakes, causing the car to decelerate uniformly to rest 5 m before the obstruction. Given that the car is in motion for 12 seconds, sketch the velocity-time diagram for this motion and find the distance AC.;Given also that the magnitude of the deceleration is four times that of the acceleration, find the distance BC.Answers:

ID: 199504001011
Content:
(a)	    A particle is projected vertically upwards from the ground with an initial speed of \[54ms^{-1}\] Find;(i)	the speed of the particle when it is 117 m above the ground,;(ii)	the length of time for which the particle is above 117 m.;(b)	A stone, P is dropped from rest from the edge of a cliff 80 m high. Find ;(i)	the time P takes to reach the ground,;(ii)	the speed with which P strikes the ground.;One second after P is released, a second stone, Q is also dropped from rest from the edge of the cliff. Find ;(iii)	the height of Q above the ground at the moment when P strikes the ground. At the instant when P strikes the ground, a third stone, R, is projected vertically downwards from the edge of the cliff at a speed of \[u ms^{-1}\]. Given that Q and R strike the ground simultaneously,;(iv)	find the value of u,;(v)	show that the ratio of the final speed of R to the final speed of Q is 17 : 8.Answers:

ID: 199504001012
Content:
a) A river flows between two parallel banks at a speed of \[4ms^{-1}\] A boat, whose speed through still water is \[4ms^{-1}\] leaves a point A, heading in a direction making an angle of \[70^{\circ}\] with bank, as shown in the diagram. The point B is directly opposite A on the other bank and AB = 80 m. At time t seconds after leaving A, the boat reaches the bank at the point C. find ;i) the value of t,;ii) the distance BC,;iii) the angle that the direction of motion of the boat makes with the bank.;b) An aircraft is travelling due north at \[360kmh^{-1}\]  The pilot is steering a course of \[350^{\circ}\] and the aircraft?s speed is \[300kmh^{-1}\]  Find the speed of the wind and the direction from which it is blowing.Answers:

ID: 199504001013
Content:
A projectile is fired from a point A with a speed of \[39ms^{-1}\] at an angle \[\alpha\] to the horizontal, where \[\tan \alpha=\frac{5}{12}\] ;The projectile strikes a horizontal plane, 50 m vertically below A, at the point B .;Find;i) the time taken by the projectile to reach B .;ii) the distance MB,;iii) the magnitude and direction of the velocity of the projectile immediately before impact at B .;At the instant when the projectile is fired from A, an aircraft flying horizontally at \[100ms^{-1}\] releases an object at the point C. Given that the projectile and the object strike B simultaneously, calculate;iv) the vertical height CN,;v) the horizontal distance NB .Answers:

ID: 199504001014
Content:
A particle P, of mass 2.5 kg, is placed on a rough plane, inclined to the horizontal at an angle \[\alpha\], where \[\tan \alpha=\frac{3}{4}\] ;The particle P is attached by a light inextensible string, passing over a light pulley, to a particle Q, of mass M kg, which hangs freely as shown in the diagram. The coefficient of friction between P and the plane is 0.4.;i) Find the value of M for which P is on the point of sliding;a) up the plane,;b) down the plane.;Given M = 0.5, find the acceleration of P down the plane and the tension in the string.Answers:

ID: 199504001015
Content:
a) A particle of mass 0.5 kg is projected with a speed of  \[12ms^{-1}\]down a slope inclined at an angle \[\alpha\] to the horizontal where \[\sin \alpha=0.25\];A constant resisting force P N acting up the slope reduces the speed of the particle to \[2ms^{-1}\]over a distance of 4 m. Find ;i) the loss of kinetic energy of the particle,;ii) the loss of potential energy of the particle,;iii) the value of P.;b) A train of mass 225000 kg accelerates from rest along a straight horizontal track. The engine exerts a constant tractive force of 45000 N. Given that the resistance to motion is 11250 N, find ;i) the acceleration of the train,;ii) the power developed by the engine after 20s.;Later, the engine power is increased to a constant 360kW. Assuming that the resistance to motion is unchanged, find ;iii) the tractive force of the engine when the speed of the train is \[15ms^{-1}\];iv) the maximum possible speed of the train.Answers:

ID: 199504001016
Content:
A bullet is fired horizontally from a gun with a speed of \[320ms^{-1}\] The mass of the bullet is 0.075 kg and the mass of the gun, when empty, is 3 kg. The path of the bullet may be regarded as horizontal. Find ;i) the initial speed of recoil of the gun,;ii) the kinetic energy produced by the explosion.;The bullet strikes a stationary target of mass 3.125 kg, which is free to move, and the bullet becomes embedded in it.;iii) Show that the speed of the combined target and bullet, after the impact, is \[7.5ms^{-1}\];The target containing the bullet is brought to rest from a speed of \[7.5 ms^{-1}\]by a constant force of 180 N in a distance of m. ;iv) Find the value of d.;Bullets, each of mass 0.075 kg, are fired horizontally with a speed of \[320ms^{-1}\]and strike a movable target, without rebounding, at a rate of 20 per second.;v) Find the average force required to keep the target stationary.Answers:

ID: 199504001017
Content:
A particle is in equilibrium under the action of four horizontal forces 4 N, 12 N, 8 N and P N, as shown in the diagram. ;Find the value of P and of \[\theta\].;    b) The diagram shows a boat, A moored by ropes AB and AC to the fixed points B and C. The rope AB runs in a direction \[048^{\circ}\] from A and the rope AC runs due east from A. The ropes are kept taut by the force of the current on A and the tensions in AB and AC are 180 N and 150 N respectively. Find the magnitude of the force of the current on A and the direction of flow of the current.;The diagram shows a particle of mass 2.6 kg, suspended at C by light inelastic strings attached to fixed points A and B which are at the same horizontal level, where AC = 0.5 m,;BC = 1.2 m and AB = 1.3 m. The tensions in AC and BC are \[T_1N\] and \[T_2N\]respectively.;i) Show that angle ACB is a right angle.;ii) Find the value of \[T_1\]and \[T_2\]Answers:

\end{document}
