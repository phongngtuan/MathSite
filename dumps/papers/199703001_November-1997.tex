\documentclass{article}
\begin{document}
ID: 199703001001
Content:
Find the coordinates of the points of intersection of the line x + 2y = 10 and the curve \[2y^2-7y+x=0\];(Note: Please enter the answer with smaller value of x first into the answer space)Answers:

ID: 199703001002
Content:
A line drawn through the point A(4, 6), parallel to the line 2y = x -2, meets the y-axis at the point B .;(a)	Calculate the coordinates of B .;A line drawn through A, perpendicular to AB, meets the line 2y =x-2 at the point C .; (b)	Calculate the coordinates of C .Answers:

ID: 199703001003
Content:
Find the value of the constant c for which the line y = 2x + c is a tangent to the curve \[4x^2-6x+11\];This tangent meets the x-axis at A and the y-axis at B . Calculate the area of the triangle OAB, where O is the origin.Answers:

ID: 199703001004
Content:
Find the range of values of x for which \[3x^2-5x+4>3-x^2\]Answers:

ID: 199703001005
Content:
Find the coefficient of  \[x^{3}\]  in the expansion of \[(9+8x)(1-\frac{x}{3})^9\]Answers:

ID: 199703001006
Content:
Given that \[y=\frac{((3x-2)^{10})}{6}\]  find the value of  \[\frac{\mathrm{d} y}{\mathrm{d} x}\]when \[x=\frac{1}{3}\];The rate of increase of x, when \[x=\frac{1}{3}\]is 2 units per second. Calculate the corresponding rate of change of y.Answers:

ID: 199703001007
Content:
A function f is defined by \[f:x \mapsto  5-\frac{6}{x}\] \[x\neq 0\];(a) Find \[f^{-1}\] and state the value of x for which \[f^{-1}\]  is undefined.;(b)	Find the value of x for which \[f(x)=f^{-1}(x)\];(Note: Please enter the smaller value of x first in the answer space)Answers:

ID: 199703001008
Content:
The diagram shows the parallelogram OABC. Given that \[\vec{OA}=3i+j\] and that ; \[\vec{OC}=4-2j\];i) find \[\vec{OB}\];(Note: Please enter your answer in the form of coordinates);ii) use a scalar product to find the acute angle between the diagonals of the parallelogram.Answers:

ID: 199703001009
Content:
img;The figure shows a circle, centre O, radius 10 cm, and a chord AB such that \[\angle AOB = \frac{2\pi }{5}radians\].;Calculate;(a)	the length of the major arc ACB,;(b)	the area of the shaded region.Answers:

ID: 199703001010
Content:
A particle moves in a straight line so that, t seconds after passing through a fixed point O, its velocity, \[vms^{-1}\] is given by \[v=5t^2+t(1-3p)+p\] where p is a constant.;i) Find an expression for the acceleration of the particle in terms of t and p; ii) Given that the acceleration of the particle is \[3ms^{-2}\]when t = 2, find the value of p; iii) Using your value of p, find the values of t when the particle is at instantaneous rest.(Note: Please enter the smaller value of t first in the answer space)Answers:

ID: 199703001011
Content:
a) The diagram shows part of the curves \[y=x^2-8x+24\] and \[y=8x-x^2\], intersecting at the points A and B . Calculate;i) the coordinates of A and of B,;ii) the area of the shaded region;img; b) The diagram shows part of the curve \[x^2+y^2-4x+5=0\];Calculate the volume generated when the shaded region is rotated through \[360^{\circ}\]  about the x-axis.Answers:

ID: 199703001012
Content:
The gradient at any point on a particular curve is given by the expression \[x^2+\frac{16}{x^{^{2}}}\] where \[x > 0\] Given that the curve passes through the point P(4, 18), find; (a)	the equation of the normal to the curve at P,; (b)	the equation of the curve.;Find the coordinates of the point on the curve where the gradient is a minimum and calculate this minimum value.Answers:

ID: 199703001013
Content:
A circular cylinder, open at one end, is constructed of thin sheet metal whose area is \[432\pi cm^2\] The cylinder has a radius of r cm and a height of h cm.; (a)	Show that the volume, \[V cm^3\] contained by the cylinder is given by \[V=\frac{\pi }{2}(432r-r^3)\];  Given that r can vary,; (b)	Find the value of r for which V is stationary,; (c)	Evaluate the stationary value of V and determine, with working, whether this value is a maximum or a minimum.Answers:

ID: 199703001014
Content:
Find all the angles, between \[0^{\circ}\]  and \[360^{\circ}\]  which satisfy the equation;(a) \[16\sin x-8\sin^2x=5\cos^2x\] ;(b) \[4\sin y\cos y-3\cos^2y=0\]; (c) \[\sec(\frac{(3z)}{2}-18^{\circ})+2=0\]; (Note: Please enter your answers in ascending order)Answers:

ID: 199703001015
Content:
Solutions to this question by accurate drawing will not be accepted.;The points A(-1, 4), B(2, 7), C and D(1, 0) are the four vertices of a parallelogram. The point E lies on BC such that \[BE =\frac{1}{3} BC\]  Lines are drawn, parallel to the y-axis, from A to meet the x-axis at N and from E to meet CD at F.; (a)	Calculate the coordinates of C and of E.;(b)	Find the equation of DC and calculate the coordinates of F. ;(c) Explain why AEFN is a parallelogram and calculate its area.Answers:

ID: 199703001016
Content:
The position vectors of the points A, B and C, relative to an origin O, are a, b and a + 2b respectively. AB and OC meet at D, where \[\frac{(AD)}{(AB)}=p\] and \[\frac{(OD)}{(OC)}=q\] ;Express \[\vec {OD}\]  in terms of;(a)	a, b and p,; (b)	a, b and q.;Hence evaluate p and q.Answers:

ID: 199703001017
Content:
Functions f and g are defined by; \[f:x \mapsto \frac{(3x-1)}{(x-2)} \] \[x\neq 2\];` \[g:x \mapsto \frac{(2x-1)}{(x-3)}\] \[x\neq 3\];(i) Show that \[fg:x \mapsto x\];(ii) Evaluate \[f^{-1}(5)\] \[g^{-1}(4)\]and \[ffg(7)\]Answers:

ID: 199703001018
Content:
Sketch the graphs of 3y = 4x + 2 and 3y= |4x-8| on the same diagram. ;Solve the simultaneous equations;3y = 4x + 2;3y= |4x-8|Answers:

\end{document}
