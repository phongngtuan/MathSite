\documentclass{article}
\begin{document}
ID: 199503001001
Content:
Find the equation of the line which is parallel to 3y = 4x + 5 and which passes through the mid-point of the line joining (-2, 3) and (5, 9). Answers:

ID: 199503001002
Content:
Prove the identity \[\frac{1}{\left ( \tan A+\cot A \right )}-=\sin A\cos A\]Answers:

ID: 199503001003
Content:
Given that the coefficient of \[x^2 \]in the expression of \[(4+kx)(2-x)^6 \] is zero,  find the value of k.Answers:

ID: 199503001004
Content:
Solve the simultaneous equations;\[x+y=xy\] \[2y=x+2\] (Note: Please enter the smaller value of x first in the answer space)Answers:

ID: 199503001005
Content:
The equation of a curve is \[y=(3-x^2)^6\] Find;(a)\[\frac{\mathrm{d} y}{\mathrm{d} x}\] ;(b) the equation of the normal at the point on the curve where x=2Answers:

ID: 199503001006
Content:
The gradient of a curve, at the point (x, y) on the curve, is given by \[\frac{(x^2-4)}{x^2}\]  Given that the curve passes through the point (2, 7), find the equation of the curve.Answers:

ID: 199503001007
Content:
A particle move is a straight line so that , t seconds after leaving a fixed point O, its displacement , s m, is given by \[s=\frac{1}{3}t^3-2t^2+3t\];Given that the particle returns to O when t = T, find the value of T.;Using this value of T, find ;i) the maximum displacement from O of the particle during the interval \[0<=t<=T\] ;ii) the acceleration of the particle at time T seconds.Answers:

ID: 199503001008
Content:
a) Find the range of values of x for which \[4x(4-x)>=15\];b) Find the range of values of c for which the line y = cx + 6 does not meet the curve\[2x^2-xy=3\]Answers:

ID: 199503001009
Content:
The position vectors of points A, B and C, relative to a fixed point O are \[\binom{4}{9}\], \[\binom{6}{-5}\] and \[\binom{7}{12}\] respectively.;Evaluate \[\vec{AB}\cdot \vec{AC}\] and hence determine \[\angle BAC\]Answers:

ID: 199503001010
Content:
Using graph paper, draw, on the same diagram, the graphs of ;     \[y=2-|x-2|\] \[y=\frac{1}{2}x+2\];For \[-1<=x<=5\] How many pairs of values (x, y) satisfy both equations?Answers:

ID: 199503001011
Content:
img; (a) The diagram shows a piece of wire bent to form the perimeter OABCO of a sector of a circle, centre O, radius r cm, where \[\angle OAC\]  is q radians. The wire is of length 100 cm and r and q may vary. Find;(i) the value of r for which the area enclosed by the wire is a maximum,;(ii) the corresponding value of the \[\angle OAC\]  in degrees.;(b) Given that \[y=\frac{2}{(3x^3)}+\frac{5}{2x^2)}-\frac{2}{x}\],  find the two values of x for which y is stationary.(Note:Please enter the smaller value of x first in the answer space); Show that the larger of these values of x corresponds to a minimum value of y.Answers:

ID: 199503001012
Content:
a) Find the volume of the solid formed when the region bounded by the curve \[y^2=8x \] and the line \[y = 2x \] is rotated through \[360^{\circ}\] about the x- axis.;img; b) The diagram shows part of the curve \[y=x^2 \] and of the line \[y=kx\] where k is a positive constant. Calculate the value of k for which the area of the shaded region is 0.288 units.Answers:

ID: 199503001013
Content:
 (a)	A vessel is in the shape of an inverted right circular cone whose base-radius is equal to its height and whose axis is vertical. Liquid is poured into the vessel at a constant rate of \[100cm^3s^{-1}\] The volume of liquid in the vessel is \[\frac{1}{3}\pi x^3cm^3 \] when the depth of liquid is x cm.;img;Calculate, at the instant when the depth of liquid is 10 cm, the rate of increase of;(i)	the depth of the liquid,;(ii)	the area of the horizontal surface of the liquid.;(b)Given that \[y=x^3+3x^2 \]  use calculus to find, in terms of p, the approximate percentage increase in y when x increases from 2 by p%, where p is small.Answers:

ID: 199503001014
Content:
Find all the angles from \[0^{\circ}\] to \[360^{\circ}\]  inclusive which satisfy the equation ;(a) \[\tan(x-30^{\circ})-\tan50^{\circ}=0\] ;(b) \[3 \sin y + \tan y = 0\];(c) \[2\sin^4z+7\cos^2z=4\];(Note: Please enter the answers in ascending order)Answers:

ID: 199503001015
Content:
img;The diagram shows a quadrilateral ABCD where A is (6, 1), B is on the x-axis and C is (1, 3). ;The diagonal BD bisects AC at right angles at M and BD = 3.5 BM. Find;(a)	the equation of BD,;(b)	the x-coordinate of B,;(c)	the coordinate of D,;(d)	the area of the quadrilateral ABCD.;By considering the area of triangle ABD, or otherwise, find the perpendicular distance from D to BA extended.Answers:

ID: 199503001016
Content:
(a) The vector \[\vec{OA}\] has magnitude 100 and has the same direction as \[\binom{7}{24}\] Express \[\vec{OA}\]  as a column vector. The vector \[\vec{OB}\] is \[\binom{24}{99}\] Obtain the unit vector in the direction of \[\vec{AB}\];(Note:Please enter your answers in the form of coordinates);img;(b) In the figure shown the position vectors of A and B with respect to O are a and b respectively. The points P and Q are such that \[\vec{AB}=5\vec{AP}\] and \[\vec{OQ}=2\vec{OP}\] Express \[\vec{OP}\] and \[\vec{BQ}\]  in terms of a and b .;Given that\[\vec{OR}=\lambda a\] and \[\vec{BR}=\mu \vec{BQ}\], express in terms of \[\vec{BR}\];(i)\[\lambda, a, b\] ;(ii) \[\mu, a, b\];and hence evaluate \[\mu\] and \[\lambda \] Answers:

ID: 199503001017
Content:
(a)Functions f and g are defined by;\[f:x \mapsto \frac{6}{(x-2)},x\neq 2\] \[g:x \mapsto kx^2-1\] where k is a constant.;(i)Given that gf(5) = 7, evaluate k.;(ii)Express \[f^2(x) \]  in the form \[\frac{(ax+b)}{(c-x)}\] stating the values of a, b and c.;(b)On graph paper, using the same scale on each axis, draw the graph of \[h:x \mapsto \frac{(2x+2)}{(x+2)}\] for the domain \[-1<=x<=3\];(i) By drawing the appropriate straight line on the graph obtain a solution of the equation \[h(x)=h^{-1}(x)\];(ii) State the domain of\[h^{-1}(x)\];iii)Using the same axes as for h(x), draw, on the same diagram, the graph of \[h^{-1}(x)\]Answers:

\end{document}
