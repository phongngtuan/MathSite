\documentclass{article}
\begin{document}
ID: 200903001001
Content:
The expression $$2x^3 + ax^2 + bx +3$$, where a and b are constants, has a factor of x-1 and leaves a remainder of 15 when divided by x+2. Find the value of a and of b.Answers:

ID: 200903001002
Content:
Given that $$y = \frac{(\ln x)}{x}$$ for $$x>0$$, find the set of values of x for which y is an increasing function of x.Answers:

ID: 200903001003
Content:
Without using a calculator, find the values of the integers a and of b for which the solution of the equation $$x \sqrt{24} = x \sqrt{3} + \sqrt{6}$$ is $$\frac{a+\sqrt{b}}{7} $$.Answers:

ID: 200903001004
Content:
Solve the equation ;(i) $$\lg (x + 14) - \lg (x -2) = 2 \lg 5$$,;(ii) $$\log_2y +\log_4 y =6$$Answers:

ID: 200903001005
Content:
The normal to the curve $$y=1-\frac{3}{\tan x}$$, at the point where the curve crosses the y-axis, passes through the point (k,3). Find the value of k.Answers:

ID: 200903001006
Content:
A curve has the equation $$y = 2x^2 -6x +c$$, where c is a constant.;(i) In the case where c=-20, find the set of x for which $$y\leq0$$.;(ii) Find the value of c for which the line y+2x = 8 is a tangent to the curve.Answers:

ID: 200903001007
Content:
Find the coordinates of the mid-point of the straight line joining the points of intersection of the curve $$x^2 +2y^2 + 5x =68$$ and the line $$2y+3x=9$$.Answers:

ID: 200903001008
Content:
i)Show that $$\cos  3x -\cos  x = -4\sin ^2x \cos  x$$.;ii)Hence, or otherwise, solve, for $$0 \leq x \leq \pi$$ radians, the equation $$\cos 3x + 2\cos x =0$$.;(Note: Please enter your answers in ascending order)Answers:

ID: 200903001009
Content:
The function f is defined, for $$x \geq 0^{\circ}$$, by $$f(x)=3\sin (\frac{x}{3}) -1$$.;(i) State the maximum and minimum values of f(x).;(ii) State the amplitude of f.;(iii) State the period of f.;(iv) Find the smallest value of x such that f(x) =0.;(v) Sketch the graph of $$y = 3\sin (\frac{x}{3}) -1$$ for $$0^{\circ} \leq x\leq 540^{\circ}$$.Answers:

ID: 200903001010
Content:
The mass,  m mg, of a radioactive substance decreases with time, t hours. Measured values of m and t are given in the following table.;img;It is known that m and t are related by the equation $$m = m_0 e^{-kt}$$, where $$m_0$$ and $$k$$ are constants.;(i)	Plot $$\ln m$$ against t for the given data and draw a straight line graph.;(ii)	Use your graph to estimate the value of k and of $$m_0$$.;(iii)	Estimate the number of hours for the mass of the substance to be halved.Answers:

ID: 200903001011
Content:
img;The diagram shows a trapezium ABCD in which AB is parallel to DC and angle $$BAD=90^{\circ}$$. The point A is (0,6) and the point D is (2,-2).;(i) Find the equation of AB.;Given that B lies on the line y=x, find;(ii) the coordinates of B.;Given that the length of DC is twice the length of AB, find ;(iii) the coordinates of C,;(iv) the area of the trapezium ABCD.Answers:

ID: 200903001012
Content:
A curve has the equation $$y=(2x-1)\sqrt{4x-1}$$.;(i) Express $$\frac{\mathrm{d}y}{\mathrm{d} x}$$ in the form $$\frac{(kx)}{\sqrt{4x+1}}$$ where k is a constant.;Hence;(ii) find the rate of change of x when x=2, given that y is changing at a constant rate of 2 units per second,;(iii) evaluate $$\int_0^2\frac{(3x)}{(\sqrt(4x+1))} dx$$.Answers:

\end{document}
