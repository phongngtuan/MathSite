\documentclass{article}
\begin{document}
ID: 200504001001
Content:
Variables V and t are related by the equation $$V=1000e^{kt}$$ ,where k is a constant. Given that V=500 when t=21, find;(i) The value of k,;(ii) The value of V when t=30.Answers:

ID: 200504001002
Content:
The line x + y = 10 meets the curve $$y^2-2x+4$$ at the points A and B . Find the coordinates of the mid-point of AB .Answers:

ID: 200504001003
Content:
(i)	Given that y=1+ln(2x-3), obtain an expression for $$\frac{dy}{dx}$$.;(ii)	Hence find, in terms of p, the approximate value of y when x=2+p, where p is small.Answers:

ID: 200504001004
Content:
The function f is given by $$f: x \mapsto 2+5\sin 3x$$ for $$0 ^{\circ}\leq x\leq180 ^{\circ}$$.;(i) State the amplitude and period of f.;(ii)	Sketch the graph of y=f(x).Answers:

ID: 200504001005
Content:
The binomial expansion of $$(1+px)^n$$, where n > 0, in ascending powers of x is $$1-12x+28p^2x^2+qx^3+...$$. Find the value of n, of p and of q.Answers:

ID: 200504001006
Content:
It is given that $$A= \begin{pmatrix}3 & 1\\  5& p\end{pmatrix}$$ and that $$A+A^{-1}=kl$$, where p and k are constants and I is identity matrix. Evaluate p and k.Answers:

ID: 200504001007
Content:
img;In the diagram $$\vec{OP}=p, \vec{OQ}=q, \vec{PM}= \frac{1}{3}\vec{PQ}$$ and $$\vec{ON}= \frac{2}{5}\vec{OQ}$$;(i)	Given that $$\vec{OX}$$ and $$m \vec{OM}$$, express $$\vec{OX}$$ in terms of m, p and q.;(ii)	Given that $$\vec{PX}=n \vec{PN}$$,express $$\vec{OX}$$ in terms of n, p and q.;(iii)	Hence evaluate m and n.Answers:

ID: 200504001008
Content:
(a)	Find the value of each of the integers p and q for which $$( \frac{25}{16})^{( \frac{-3}{2})}=2^p \times 5^q$$;(b)	(i) Express the equation $$4^x-2^{(x+1)}=3$$ as a quadratic equation in $$2^x$$.;(ii) Hence find the value of x, correct to 2 decimal places.Answers:

ID: 200504001009
Content:
The function $$f(x)=x^3-6x^2+ax+b$$, where a and b are constants, is exactly divisible by $$x-3$$ and leaves a remainder of -55 when divided by $$x + 2$$.;(a)	Find the value of a and of b .;(b)	Solve the equation f(x) = 0;(Note: Please enter your answers in ascending order)Answers:

ID: 200504001010
Content:
A curve is such that $$ \frac{\mathrm{d} ^{2}y}{\mathrm{d} x^{2}}=6x-2$$;The gradient of the curve at the point (2, - 9) is 3.;i) Express y in terms of x;ii) Show that the gradient of the curve is never less than $$\frac{-16}{3}$$Answers:

ID: 200504001011
Content:
(a)	Each day a new agent sells copies of 10 different newpapers,one of which is The Times. A customer buys 3 different newspapers. Calculate the number of ways the customer can select his newspapers;(i)	If there is no restriction,;(ii)	If 1 of 3 newspapers is The Times.;(b)	Calculate the number of different 5-digit numbers which can be formed using the digits 0,1,2,3,4 without repetition and assumming that a number cannot begin with 0.;How many of these 5-digit numbers are even?Answers:

ID: 200504001012
Content:
A curve has the equation $$y = 2 \cos  x -\cos  2x$$, where $$0< x\leq\frac{\pi}{2}$$.;i) Obtain an expression for $$\frac{\mathrm{d} y}{\mathrm{d} x}$$ and $$\frac{\mathrm{d} ^{2}y}{\mathrm{d} x^{2}} $$;ii) Given that $$\sin  2x$$ may be expressed as $$2 \sin  x \cos  x$$ find the x- coordinate of the stationary point of the curve and the nature of this stationary point.;iii) Evaluate $$\int_{\frac{\pi}{3}}^{(\frac{\pi}{2})}y dx$$.Answers:

ID: 200504001013
Content:
img;The diagram, which is not drawn to scale, shows part of the curve $$y=x^2-10x+24$$ cutting the x- axis at Q(4, 0). The tangent to the curve at the point P on the meets the coordinates at S(0, 15) and at T(3.75, 0).;i) Find the coordinates of P;The normal to the curve at P meets the x-axis at R. ;ii) Find the coordinates of R.;iii) Calculate the area of the shaded region bounded by the x- axis, the line PR and the curve PQ.Answers:

\end{document}
