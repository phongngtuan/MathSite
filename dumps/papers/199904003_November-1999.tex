\documentclass{article}
\begin{document}
ID: 199904003001
Content:
Particle MathematicsAnswers:

ID: 199904003002
Content:
Particle MathematicsAnswers:

ID: 199904003003
Content:
Particle MathematicsAnswers:

ID: 199904003004
Content:
Particle MathematicsAnswers:

ID: 199904003005
Content:
Particle MathematicsAnswers:

ID: 199904003006
Content:
A set of 30 cards is made up of cards chosen from a number of packs of ordinary playing cards. The numbers of cards of each type are given in the following table.;img; Thus, for example, there are 2 Kings of Spades and 3 Jacks of Diamonds.  ;;a) One card is taken at random from the set. Events H, K, J are defined as follows. ;H: The card taken is a Heart. ;K: The card taken is a King. ;J: The card taken is a Jack. ;(i) Describe in words what the event K $$\cup$$ H represents and state the probability of this event.;(ii) Describe in words what the event J' $$\cap$$ H' represents and state the probability of this event. ;(iii) Find the conditional probability that the card is a Diamond, given that it is a King.;;b) Two cards are taken from the set, at random and without replacement. Find the probability that both cards are Jacks. Give your answer correct to 3 places of decimals.;;c) Three cards are taken from the set, at random and without replacement. Find the probability that both cards are Jacks. Give your answer correct to 3 places of decimals.Answers:
203: P(K \cup H) = "\frac{7}{15}".
204: P(J^' \cap H^') = "\frac{8}{15}".
205: P(card is a Diamond | K) = "\frac{1}{9}".
206: P(both cards are Jacks) = "0.064".
207: P(3 Kings or 3 Queens or 3 Jacks) = "0.105".

ID: 199904003007
Content:
In a game, 2 red balls and 8 blue balls are placed in a bottle. The bottle is shaken and Mary draws 3 balls at random and without replacement. The number of read balls that she draws is denoted by R. Find the probability distribution of R, and show that $$P(R \geq 1) = \frac{8}{15}$$.  Show that the expectation of R is $$\frac{3}{5}$$ and find the variance of R.  Mary scores 4 points for each red ball that she draws. The balls are now replaced in the bottle and the bottle is shaken again. John draws 3 balls at random and without replacement. He scores 1 point for each blue ball that he draws. Mary's score is denoted by M and John's score is denoted by J. Find the expectation and variance of M-J. Answers:
208: P(R=0) = "\frac{7}{15}".
209: P(R=1) = "\frac{7}{15}".
210: P(R=2) = "\frac{1}{15}".
211: 
212: 
213: Var(R) = "\frac{28}{75}".
214: E(M-J) = "0".
215: Var(M-J) = "\frac{476}{75}".

ID: 199904003008
Content:
The continuous random variable X has cumulative distribution function F given by  $$F(x) = 0$$  if $$x < 0$$   $$a( 6x^{2} - x^{3})$$  if $$0 \leq x \leq 4$$ 1 if $$x > 4$$  Where a is a constant. Find a.  Find the probability density function of X, and use the fact that the graph of this function is symmetrical about $$x=2$$ to write down the expectation of X.  Show that the variance of X is $$\frac{4}{5}$$.  The random variable S is the sum of 100 independent observations of X. Find $$P(S > 185)$$. Answers:
216: a = "\frac{1}{32}".
217: f(x) = "\frac{3}{32}(4x-x^2)", for 0 \leq x \leq 4 and f(x) = "0", otherwise.
218: E(X) = "2".
219: 
220: P( S > 185) = "0.953".

ID: 199904003009
Content:
A telephone enquiry service is so busy that only 80% of calls to it are successfully connected. It may be assumed that all calls are independent. Twelve calls are made at random to the service. Find the probability that at least 9 are successfully connected.   After improving the facilities, the management arranges for a random sample of 120 calls to be made to the service and it is found that 105 of these calls are successfully connected. Test, at the 4% significance level, whether the successful connection rate has improved.   A consumer association carries out its own test using a random sample of 150 calls and finds that the number of unsuccessful calls is 25. Using this sample, find an approximate 92% confidence interval for the proportion of calls that are unsuccessful.  Answers:
221: P(no. of successful calls \geq 9) = "0.795".
222: "reject" (Do not reject/Reject) H_0 and conclude that, at the 4% level of significance, there is " sufficient" (sufficient/insufficient) evidence that the successful connection rate has improved.
223: The 92% confidence interval for q is ["0.113","0.220"].

ID: 199904003010
Content:
a)  A manufacturer of rockets estimates that, on average, 1 in 75 of the rockets fail to burn properly. Using this estimate, and a Poisson distribution, find an approximate value for the probability that, out of 225 randomly chosen rockets, at most 221 burn properly.  State fully what distribution should be used to obtain the exact value of this probability. ;;b)  'Brilliant' fireworks are intended to burn for 40 seconds. A random sample of 50 'Brilliant' fireworks is taken. Each firework in the sample is ignited and the burning time, x seconds, is measured. The results are summarized by $$\Sigma(x-40) = -27$$, $$\Sigma(x-40)2=67$$. Test, at the 5% level of significance, whether the mean burning time of 'Brilliant' fireworks differs from 40 seconds.  State, with a reason, whether, in using the above test, it is necessary to assume that the burning times of 'Brilliant' fireworks have a normal distribution.   Answers:
224: P(at most 221 burn properly) = "0.353".
225: "binomial" distribution should be used to obtain the exact value of this probability. .
226: "reject" (Do not reject/Reject) H_0 and conclude that at 5% level of significance, there is " sufficient" (sufficient/insufficient) evidence that the mean burning time differs from 40 seconds.

ID: 199904003011
Content:
The curve C has polar equation \[r = \frac {\cos \theta}{\sin^2 \theta}, 0 < \theta \leq \frac {1}{2}\pi\]. Obtain the cartesian equation of C, and hence or otherwise sketch C. The area of the finite region enclosed between C and the line \[\theta = \frac {1}{3}\pi\] is denoted by A. Show that \[A = \frac{1}{2}\int_{\alpha}^{\beta} \cot^2 \theta \cos ec^2 \theta d \theta\], stating the values of the constants \[\alpha\] and \[\beta\]. Use the substitution \[u = \cot \theta\] to find the exact value of A. The curve C' is the reflection of C in the line \[\theta = \frac {1}{2} \pi\]. State the polar equation of C', giving the set of values for \[\theta\].Answers:

ID: 199904003012
Content:
a) The complex numbers \[2e^{\frac {1}{12} \pi i}\] and \[2e^{\frac {5}{12} \pi i}\] are represented by the points A and B respectively in an Argand diagram with origin O. Show that triangle OAB is equilateral.;;b) The complex numbers z and w each have modulus R, and have arguments \[\alpha\] and \[\beta\] respectively, where \[0 < \alpha < \beta < \frac {1}{2} \pi\]. In either order,;(i) show that \[z + w = 2R \left \{ \cos \frac {1}{2} (\beta - \alpha) \right \}e^{\frac {1}{2}(\alpha + \beta)i}\],;(ii) express |z + w| and arg (z + w) in terms of R, \[\alpha\] and \[\beta\], as appropriate. Show also that \[|z - w| = 2R\sin \frac {1}{2}(\beta - \alpha)\]. The complex numbers z and w are represented by the points Z and W respectively in an Argand diagram with origin O. Triangle OZW has area \[\Delta\]. Show that \[|z^2 - w^2| = 4\Delta \].Answers:
227: None
228: 
229: |z + w| = "2R\cos{\frac{1}{2}(\beta-\alpha)}".
230: arg(z+w) = "\frac{1}{2}(\alpha+\beta)".
231: 
232: 

ID: 199904003013
Content:
img;It is given that the graph of y = f(x), where \[f(x) = 2 - \frac {1}{(x-1)^2}\], is as shown above. State the equations of the asymptotes. Copy the above diagram and, by sketching another graph on the same diagram, state the number of real roots of the equation 3x = f(x). Taking x = 0.2 as a first approximation, use the Newton-Raphson method once to obtain a second approximation to a root of the equation 3x = f(x), giving your answer correct to 2 significant figures. By sketching appropriate graphs, state the number of real roots of the equation 3|x| = |f(x)|. Use linear interpolation once, on the interval [-1, 0], to obtain an approximation to a root of the equation 3|x| = |f(x)|.Answers:
233: Equations of the asymptotes are x = "1" and y = "2".
234: "1" real roots of the equation 3x = f(x).
235: Second approximation to a root of the equation 3x = f(x) is "0.18".
236: "4" real roots of the equation 3|x| = |f(x)|.
237: An approximation to a root of the equation 3|x| = |f(x)| is "-0.444".

ID: 199904003014
Content:
a) Solve the differential equation \[\frac{\mathrm{d}^2 y}{\mathrm{d} x^2} + 2\frac{\mathrm{d} y}{\mathrm{d} x} + y = x\], given that, when x = 0, y = -2 and \[\frac{\mathrm{d} y}{\mathrm{d} x} = -2\].;;b) Use the substitution y = u - 2x to find the general solution of the differential equation \[\frac{\mathrm{d} y}{\mathrm{d} x} = \frac {8x + 4y + 1}{4x + 2y + 1}\].Answers:
238: General solution of the differential question \frac{dy}{dx} = \frac{8x+4y+1}{4x+2y+1} is x = "(2x+y)^2+(2x+y)".

ID: 199904003015
Content:
img;The diagram shows a circular table, with centre O, and three legs, AA', BB' and CC', attached at points A, B and C to the table. The lengths of the legs are adjusted so that, although the table is standing on a sloping floor, the plane ABC is horizontal. Perpendicular unit vectors i, j and k are defined, with i along \[\overrightarrow{OA}\], and k vertically upwards. The vectors \[\overrightarrow{AA'}\], \[\overrightarrow{BB'}\] and \[\overrightarrow{CC'}\] are parallel to 5i - 12k, -3i + 4j - 12k and -3i - 4j - 12k respectively. Show that all three legs are inclined at the same angle to the vertical. The plane A'B'C' has equation x - 10z = 130. Find the inclination of this plane to the horizontal, and the perpendicular distance from O to this plane. Referred to O as origin, the position vectors of A, B and C are 5i, -3i + 4j and -3i - 4j respectively. Find the lengths of the legs AA', BB' and CC'.Answers:
239: 

\end{document}
