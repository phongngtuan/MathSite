\documentclass{article}
\begin{document}
ID: 201004001001
Content:
Solve the equation $$3\cot^2 \theta + 10 \csc \theta =5$$ for $$0^{\circ} \leq \theta \leq 360^{\circ}$$.;(Note: Please enter your answers in ascending order)Answers:

ID: 201004001002
Content:
img;The diagram shows a triangular piece of land PQR in which $$\angle PQR =90^{\circ}$$, PQ =8m and QR =12m.; A rectangular QUVW is used as the base of a greenhouse, where U, V and W lie on QR, RP and PQ respectively, QU =x cm and QW =y m;(i) Show that $$y =8-\frac{2x}{3}$$;(ii) Express the area, $$A m^2$$, of the base of the greenhouse in terms of x;(iii) Given that x can vary, find the maximum value of A.Answers:

ID: 201004001003
Content:
Without using a calculator, solve, for x and y, the simultaneous equations $$32^x \times 2^y = 1$$,$$3^{x-12} = 27^y = 81^{\frac{1}{x}}$$.Answers:

ID: 201004001004
Content:
(i)Given that the constant term in the binomial expansion of $$(x-\frac{k}{(x-3)})^8$$ is 7, find the value of the positive constant k.;(ii) Using the value of k found in part (i), show that there is no constant term in the expansion of $$(1+x^4)(x-\frac{k}{(x-3)})^8$$.Answers:

ID: 201004001005
Content:
The curve $$y=5-e^{2x}$$ intersects the coordinate axes at the points A and B;(i) Given that the line AB passes through the point with coordinates (ln 5,k), find the value of k.;(ii) In order to solve the equation $$x=\ln \sqrt{9-x}$$, a graph of a suitable straight line is drawn on the same set of axes as the graph of $$y =5-e^{2x}$$. Find the equation of this straight line.Answers:

ID: 201004001006
Content:
img;The diagram shows a point X on a circle and XY is a tangent to the circle. Points A, B and C lie on the circle such that XA bisects angle YXB and YAC is a straight line. The line YC and XB intersects at D.;(i) Prove that AX =AB.;(ii) Prove that CD bisects angle XCB.;(iii) Prove that triangles CDX and CBA are similar.Answers:

ID: 201004001007
Content:
img;The diagram shows a part of the curve $$y =\sqrt{2x+5}$$ passing through the point P and meeting the x-axis at the point Q. The line x=2 passes through P and intersects the x-axis at the point S. Lines from Q meet x=2 at the points R and T such that QR is parallel to the tangent to the curve at P, and RS =ST. Find ;(i) the equation of QR,;(ii) the area of the shaded region.Answers:

ID: 201004001008
Content:
Two particles, P and Q, leaves a point O at the same time and travel in the same direction along the same straight line. Particle P starts with a velocity of $$9 ms^{-1}$$ and moves with a constant acceleration of $$1.5 ms^{-2}$$. Particle Q starts from rest and moves with an acceleration of $$a ms^{-2}$$, where $$a=1+\frac{t}{2}$$ and t seconds is the time since leaving O. Find;(i) the velocity of each particle in terms of t,;(ii) the distance travelled by each particle in terms of t.Hence find;(iii) the distance from O at which Q collides with P,;(iv) the speed of each particle at the point of collisionAnswers:

ID: 201004001009
Content:
img;The diagram shows a triangle ABC with vertices at A(0,5), B(8,14), C(k,15). Given that AB =BC,;(i) find the value of k.;A line is drawn from B to meet the x-axis at D such that AD=CD.;(ii) Find the equation of BD and the coordinates of D.;(iii) Show that the area of the triangle ABC is $$\frac{2}{7}$$ of the area of the quadrilateral ABC.Answers:

ID: 201004001010
Content:
Given that $$\frac{(3x^2+4x-20)}{((2x+1)(x^2+4))} =\frac{A}{(2x+1)} +\frac{(Bx+c)}{(x^2+4)} $$ where A, B and C are constants, find the value of A and of B and show that C=0;(ii) Differentiate $$ln(x^2+4)$$ with respect to x.;(iii) Using the results from parts (i) and (ii), find $$\int\frac{(3x^2+4x-20)}{((2x+1)(x^2+4))} dx$$.Answers:

ID: 201004001011
Content:
The diagram shows the curves $$y=4\cos x$$ and $$y=2+3\sin x$$ for $$0\leq x\leq2 \pi$$ radians. The points A and B are turning points on the curve $$y=2+3\sin x$$ and the point C is a turning point on the curve $$y=4\cos x$$. The curves intersects at the point D and E.;(i) Write down the coordinates of A, B and C.;(ii) Express the equation $$4\cos x =2+3\sin x$$ in the form $$\cos (x+ \alpha) =k$$, where $$\alpha$$ and k are constants to be found.;(iii) Hence find, in radians, the x-coordinate of D and of E.Answers:

\end{document}
