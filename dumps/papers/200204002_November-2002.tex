\documentclass{article}
\begin{document}
ID: 200204002001
Content:
img;Diagram I shows a path, AC, in a park ABCD. It is given that AC = 530 m, BC = 370 m and that AC is perpendicular to BC.;;a) Calculate angle ABC.   [2];;b) Diagram II shows two other paths, AE and CE, in the park. Given that angle CAE = $$25^{\circ}$$ and angle AEC = $$90^{\circ}$$, calculate the length of AE.   [2];;c) Given also that angle ACD = $$70^{\circ}$$ and angle CAD = $$90^{\circ}$$, calculate;;c-i) the length of CD,   [2];;c-ii) the area of the park ABCD.   [3]Answers:

ID: 200204002002
Content:
a) Solve the equation $$(x - 5)^2 = 81$$.   [2];;b) Express as a single fraction in its simplest form $$\frac{a}{2a - 1} - \frac{2}{a + 1}$$ [3];;c) Given that $$A = h(4m + h)$$, Express m in terms of h and A.   [3]Answers:

ID: 200204002003
Content:
a) The points A and B are (-2, 1) and (6, -5) respectively. Calculate;;a-i) the gradient of the line AB,   [1];;a-ii) the equation of the line through A and B.   [2];;b) The points C and D are (4, 5) and (p, q) respectively.;;b-i) Write down, in terms of p and q, the coordinates of the midpoint of CD.   [1];;b-ii) Given that the midpoint of CD is (7, 1), find the coordinates of D.   [2]Answers:

ID: 200204002004
Content:
img;In the diagram, ABFG is a rectangle and BCDE is a square. ACG and GFE are straight lines.;;a) Show that angle ABC = angle FBE.   [2];;b) Prove that triangle ABC is congruent to triangle FBE.   [3];;c) Hence show that ABFG is a square.   [1]Answers:

ID: 200204002005
Content:
img;In the diagram, A is the point (10, 1) and $$\vec{AB} = \begin{bmatrix}-8\\15\end{bmatrix}$$.;;a) Find;;a-i) $$\mid\vec{AB}\mid$$, [2];;a-ii) the coordinates of B.   [1];;b) The point C is (42, 16) and $$\vec{CD} = 3\vec{AB}$$. Find;;b-i) the coordinates of D,   [2];;b-ii) the vector $$\vec{AD}$$.   [1];;c) The point E is (k, 16).;;c-i) Find, in terms of k, the vector $$\vec{AE}$$.   [1];;c-ii) Given that AED is a straight line, find k.   [2];;d) Find $$\frac{Area.of.triangle.ABE}{Area.of.triangle.CDE}$$.     [2]Answers:

ID: 200204002006
Content:
Jane and William saved some money over a number of years.;;a) On 1 May 1998 Jane opened an account by investing $900.;The account paid 6% per annum compound interest.;On 1 May 1999 she invested another $900 into the account.;;a-i) Show that the total sum in the account immediately after this is $1854.   [2];;a-ii) On 1 May 2000 she invested a further $900. Find the sum of money in the account immediately after this.   [2];;b) On 1 May 1998 William invested $900 for 4 years at 6% per annum simple interest.;;b-i) Calculate the interest he received on his investment.   [2];;b-ii) He invested another $900 for 3 years at 6% per annum simple interest on 1 May 1999, ;then $900 for 2 years at 6% per annum simple interest on 1 May 2000,;and a final $900 for 1 year at 6% per annum simple interest on 1 May 2001,;William withdrew all of his money on 1 May 2002.;Calculate how much more money Jane withdrew than William.   [3]Answers:

ID: 200204002007
Content:
A trader bought some paraffin for $500. He paid $x for each litre of paraffin.;;a) Find, in terms of x, an expression for the number of litres he bought.   [1];;b) Due to a leak, he lost 3 litres of paraffin.;He sold the remainder of the paraffin for $1 per litre more than he paid for it.;Write down an expression, in terms of x, for the sum of money he received.   [2];;c) He made a profit of $20.;;c-i) Write down an equation in x to represent this information, and show that it reduces to $$3x^2 + 23x - 500 = 0$$ [3];;c-ii) Solve the equation $$3x^2 + 23x - 500 = 0$$, giving both answers correct to one decimal place.   [4];;d) Find, correct to the nearest whole number, how many litres of paraffin he sold.   [2]Answers:

ID: 200204002008
Content:
img;The diagram shows four points, A, B, C and D, on a piece of horizontal land. ;It is given that AB = 45 metres, AD = 25 metres and BD = 28 metres.;;a) Calculate angle ADB.   [4];;b) Given also that CD = 22 metres and that angle ACD = $$33^{\circ}$$, calculate angle ADC.   [3];;c);img;The line BD is produced beyond D. Calculate the shortest distance from C to this extended line.   [2];;d) D is the foot of a vertical mast, DT. The angle of elevation of the top of the mast, T, from A is $$40^{\circ}$$. Calculate the angle of elevation of T from B.   [3]Answers:

ID: 200204002009
Content:
img;[The area of the curved surface of a cone of radius r and slant height l is $$\pi$$ rl. ;;;The volume of a cone is $$\frac{1}{3} \times base.area \times height$$.] ;Diagram I shows a traditional hut which consist of a circular cylinder with an overhanging roof. The roof is thecurved surface of a cone and is supported by a central vertical pole. ;Diagram II shows a vertical cross-section of the hut. ;BE and CD are horizontal. ;AN = 4.0 m, BM = ME = 3.6 m and BC = DE = 1.3 m.;;a) Show that AB = 4.5 m.   [1];;b) Calculate;;b-i) the volume of the inside of the hut,   [3];;b-ii) the total surface area of the inside of the hut (including the floor).   [4];;c) The sun is directly overhead.;The shadow of the overhanging section of the roof on the ground is a circular ring around the hut.;AP = AQ = 5.5 m.;Calculate;;c-i) PQ,   [2];;c-ii) the area of the circular ring of shadow outside the hut. (Ignore the thickness of the walls.)   [2]Answers:

ID: 200204002010
Content:
Answer the whole of this question on a sheet of a graph paper.;;;A man stood at the top of a tower.;;;He threw a ball vertically upwards.;;;The height, h metres, of the ball above the top of the tower at a time t seconds after it was thrown is given by the formula $$h = 22t - 4.9t^2$$;;;The table below shows some values of t and the corresponding values of h, correct to 1 decimal place.;img;;a) Explain the significance of the value h = -12.5 when t = 5.   [1];;b) Find the value of p.   [1];;c) Using a scale of 2 cm to 1 second, draw a horizontal t-axis for $$0 \leq t \leq 6$$.;Using a scale of 2 cm to 10 metres, draw a vertical h-axis for $$-50 \leq h \leq 30$$.;On your axes, plot the points given in the table and join them with a smooth curve.   [3];;d) Use your graph to find;;d-i) the greatest height of the ball above the top of the tower,   [1];;d-ii) the length of time for which the ball was more than 20 metres above the top of the tower.   [2];;e-i) By drawing a tangent atAnswers:

ID: 200204002011
Content:
a) A class of 27 children took a Mathematics test.;Some of their scores are represented in the stem and leaf diagram below. ;img;The other scores are given below. ;img;;a-i) Construct a single ordered stem and leaf diagram to represent the scores of all 27 children.   [2];img;;a-ii) For the whole class, find;;a-ii-a) The modal score,   [1];;a-ii-b) The median score.   [1];;a-iii) The pass mark for the paper was 24 marks. ;Expressing the answer as a fraction in its simplest form, calculate the probability that two children, chosen at random, from the class both passed the test.   [2];;b) Answer the whole of this part of the question on a sheet of graph paper.;At another school, 300 pupils took an English test.;The table below is the cumulative frequency table for their scores.;img;;b-i) Using a scale of 2 cm to 10 marks, draw a horizontal s-axis for $$0 \leq s \leq 60$$. ;Using a scale of 2 cm to 50 pupils, draw a vertical axis for values from 0 to 300. ;On your axes, draw ;img;Answers:

\end{document}
