\documentclass{article}
\begin{document}
ID: 200101001001
Content:
Solve the simultaneous equations;y = 2x + 1,;6xy = 2x + y + 3.;(Note: Please enter the answer with smaller value of x first in a coordinates form)Answers:

ID: 200101001002
Content:
The lines y - 3x = 1 and  y + 2x = 6 meet at the point A. Find;(a)	the equation of the line through A which passes through the point B(3, 8),;(b)	the equation of the line through A which is perpendicular to AB .Answers:

ID: 200101001003
Content:
Given that \[2\sin A\cos A+(\cos A+\sin A)^2-(2\cos A+\sin A)^2-=p \sin^2A+q\] find the value of the constant p and of the constant q.Answers:

ID: 200101001004
Content:
Show that the tangent to the curve \[y=\frac{3}{x}-\frac{4}{x^{2}}\] at the point (2, 0.5) passes through the origin.Answers:

ID: 200101001005
Content:
The two variables, x and y, are related by the equation \[y=\frac{1}{(x^2+1)^3}\];(a) Obtain an expression for $$\frac{dy}{dx}$$ and the change in y when x changes to x+p; (b) Find the rate of change in x when y changes at a rate of 0.15Answers:

ID: 200101001006
Content:
A curve is such that \[\frac{\mathrm{d} y}{\mathrm{d} x}=5-3x\]The line y = x + 2 meets the curve at the point P, where the gradient of the curve is 12. Find;(a)	the coordinate of P,;(b)	the equation of the curve.Answers:

ID: 200101001007
Content:
A particle moves in a straight line so that, t seconds after leaving a fixed point O, its velocity,; \[v ms^{-1}\], is given by \[v=t^3-3t^2+2t\];Find;i) the acceleration of the particle when t = 1.5,;ii) the values of t when the particle is at instantaneous rest,(Note: Please enter your answers in ascending order);iii) the distance travelled in the interval \[0\leq t\leq 2\]Answers:

ID: 200101001008
Content:
The coefficient of \[x^2\] in the expansion of \[(2+x)(1-ax)^5\] is zero. Find the positive value of a.Answers:

ID: 200101001009
Content:
(a) Find the range of values of x for which \[x^2+7x-9<8x-3\];(b) Find the range of values of c for which \[x^2+7x-9>8x+c\] for all values of x.Answers:

ID: 200101001010
Content:
img;In the triangle OAB shown above, \[\vec{OA}=\alpha,\vec{OB}=b\] and \[\vec{AP}=\frac{2}{5}\vec{AB}\]; (a) Find \[\vec{OP}\]in terms of a and b .;The point Q lies on OB such that QP is parallel to OA. Given that M is the mid-point of QP.;(b)	Find \[\vec{OM}\] in terms of a and b .Answers:

ID: 200101001011
Content:
Functions f and g are defined, for \[x\in \mathbb{R},f:x \mapsto 2x-3,g:x \mapsto\frac{1}{(x+5)},x\neq-5\] ;Solve the equation;i) \[f^{-1}(x)=g^{-1}(2)\] ;ii) \[f^2(x)=-18gf(x),x\neq-1\] ;b) The graph of y = h(x) is a smooth curve passing through the points (1, 1), (2, 3), (8, 6) and (12, 7).;i) Draw, on graph paper, the graph of y = h(x), using a scale of 1 cm to 1 unit on each axis.;ii) Find the values of x in the interval \[1\leq x\leq 12\] for which \[h(x)=h^{-1}(x)\];(Note: Please enter your answers in ascending order)Answers:

ID: 200101001012
Content:
(a)	The two variables, x and y, are related by the equation x + y = 3. Find the maximum value of the variable z, where \[z=5x^3y\];img;(b) The diagram shows a package in the shape of a rectangular block whose sides are of length x cm, 2x cm and y cm. The package is secured by two pieces of string, ABCDA and EFGHE, whose total length is 300 cm. The volume of the package is \[V cm^3\];(i)Show that \[V=300x^2-8x^3\];Given that x varies,;(ii)	find the value of x for which V is stationary and determine the nature of the stationary value.Answers:

ID: 200101001013
Content:
Solutions to this question by accurate drawing will not be accepted.;img;The vertices of the triangle ABC have coordinates (8, 7), (5, -2) and (-2, 5), as shown in the diagram.;AD and CE are perpendicular to BC and AB respectively, and AD and CE meet at the point H. Find;(a) the coordinates of D and of H,;(b)  the ratio AD : HD,;(c) the area of triangle ABC and of triangle HBC.Answers:

ID: 200101001014
Content:
img;(a)	The points A, B, C and D are the vertices of a parallelogram. The position vectors of A, B and D, relative to an origin O, are -2i + 3j, I + 5j, and 6i - j respectively.;(i)	Find the position vector of C.;(ii)	Evaluate \[\vec{AB}\ast \vec{AC}\] and hence find the \[\angle BAC\].;(b) The unit vector p is in the direction of 3i + 4j and the unit vector q is in the direction of 5i + 12j.;(i)	Find p and q.;(ii)	Use a scalar product to show that p - q is perpendicular to p + q.Answers:

ID: 200101001015
Content:
img;The centres, A. B and C, of the arcs of three circles, each of radius 2 cm, are the vertices of an equilateral triangle of side 4 cm. Each side of the equilateral triangle PQR touches two of the arcs as shown, where X and Y are two of the points of contact. Calculate;(a)	the perimeter of the shaded region PXY,;(b)	the area of triangle PQR,;(c)	the total area of the six shaded regions.Answers:

ID: 200101001016
Content:
(a) Find all the angles between \[0^{\circ}\] and \[360^{\circ}\]for which;(i) \[3 \cos x -6 \sec x = 7\],;(ii)  \[(\tan y+1)^2=\sec^2y-3\];(b) Find all the values of z between 0 and 8 for which \[16\sin(\frac{z}{2}-1)=15\];(Note: Please enter your answers in ascending order)Answers:

ID: 200101001017
Content:
img;(a) The diagram shows part of the curve \[y^2=16+6x-x^2\] and the line joining A (0, 1) to B (6, 4). Find the volume generated when the shaded region is rotated through \[360^{\circ}\]about the x-axis.;img;(b) The region P is bounded by the curve \[y=5x-x^2\] the x-axis and the line x = h. The region Q is bounded by the curve, the x-axis and the lines x = h, x = 2h. Given that the area of Q is twice the area of P, find the value of h.Answers:

\end{document}
