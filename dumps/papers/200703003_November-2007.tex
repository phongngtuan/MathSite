\documentclass{article}
\begin{document}
ID: 200703003001
Content:
Show that $$ \frac{2x^{2}-x-19}{x^{2}+3x+2 }-1= \frac{ x^{2}-4x-21}{x^{2}+3x+2}$$. ;Hence, without using a calculator, solve the inequality $$ \frac{2x^{2}-x-19}{x^{2}+3x+2} > 1$$.Answers:
751: 
752: Solution set of the inequality: x < "-3" or " -2" < x < " -1" or x > " 7".

ID: 200703003002
Content:
Functions f and g are defined by ;img img;(i) Only one of the composite functions fg and gf exists. Give a definition (including the domain) of the composite that exists, and explain why the other composite does not exist ;(ii) Find $$f^{-1}(x)$$ and state the domain of $$f^{-1}$$ Answers:
753: "gf" exists because x \mapsto " \frac{1}{(x-3)^2}", for x\in\mathbb{R}, x\neq "0".
754: "fg" does not exist because R_g = [" 0","\infty") \notin D.

ID: 200703003003
Content:
a) Sketch, on an Argand diagram, the locus of points representing the complex number z such that $$|z+2-3i|= \sqrt{13} $$ ;;b) The complex number w is such that $$ww*+2w=3+4i$$ , where w* is the complex conjugate of w. Find w in the form of a + ib, where a and b are real Answers:
755: None
756: w = "-1 + 2i".

ID: 200703003004
Content:
The current I in an electric circuit at time t satisfies the differential equation 4$$\frac{dI}{dt}=2-3I$$  Find I in terms of t, given that I = 2 when t = 0.  State what happens to the current in this circuit for large values of t. Answers:
757: I = "\frac{2}{3}(1+2e^{-\frac{3}{4}t})".
758: For large values of t, the current of this circuit tens to the value "\frac{2}{3}".

ID: 200703003005
Content:
Show that the equation $$y=\frac{2x+7}{x+2} $$ can be written as $$y= A + \frac{B}{x+2} $$, where A and B are constants to be found. Hence state a sequence of transformations which transform the graph of $$y= \frac{1}{x} $$ to the graph of $$y= \frac{2x + 7}{x + 2}$$.  Sketch the graph of $$y= \frac{2x + 7}{x + 2} $$, giving the equations of any asymptotes and the coordinates of any points of intersection with the x- and y-axes.  Answers:
759: A = "2".
760: B = "3".
761: Step 1: A "translation" (scaling/reflection/translation) of the graph of y="\frac{1}{x}" by " 2" units in the " negative" (negative/positive) "x" (x/y)-direction.
762: Step 2: A "scaling" (scaling/reflection/translation) parallel to the " y" (x/y)-axis of the graph of y^{(1)}=" \frac{1}{x+2}" by a factor of "3".
763: Step 3: A "translation" (scaling/reflection/translation) of the graph of y=" \frac{3}{x+2}" by " 2" units in the " positive" (negative/positive) "y" (x/y)-direction.
764: None

ID: 200703003006
Content:
Referred to the origin O, the position vectors of the points A and B are $$i -j +2k$$ and $$2i - 4j +k$$ respectively.;(i) Show that OA is perpendicular to OB;(ii) Find the position vector of the point M on the line segment AB such that AM:MV = 1:2;(iii) The point C has position vector $$-4i + 2j + 2k$$. Use a vector product to find the exact area of the triangle OAC Answers:
765: 
766: The position vector of the point M is "\frac{1}{3}(4i+2j+5k)".
767: Exact area of triangle OAC = "\sqrt{35}" units^2.

ID: 200703003007
Content:
The polynomial P(z) has real coefficients. The equation P(z) = 0 has a root $$re^{i\theta}$$ , where r > 0  and $$0< \theta < \pi $$;(i) Write down a second root in terms of r and $$ \theta $$, and hence show that a quadratic factor of P(z) is $$z^{2}-2 rz \cos \theta +r^{2}$$;(ii) Solve the equation $$z^{2}=-64$$ , expressing the solutions in the form $$re^{i\theta}$$ , where r > 0 and $$-(\pi)< \theta <= \pi $$;(iii) Hence, or otherwise, express as the product of three quadratic factors with real coefficients, giving each factor in non-trigonometrical form. Answers:
768: The second root, in terms of r and  \theta , is "re^{-i\theta}".
769: 
770: Solution of z^6 = -64 are z = "2e^{\pm i\frac{\pi}{2}}" , " 2e^{\pm i\frac{\pi}{6}}" or " 2e^{\pm i\frac{5\pi}{6}}".
771: Expressing z^6-64 as the product of three quadratic factors with real coefficients, "(z+4)(z^2-2\sqrt{3z} + 4)(z^2+2\sqrt{3z}+4)".

ID: 200703003008
Content:
The line l passes through the points A and B with coordinates (1,2,4) and (-2,3,1) respectively. The plane p has equation $$3x-y+2z=17$$ . Find;(i) The coordinates of the point of intersection of l and p,;(ii) The acute angle between l and p,;(iii) The perpendicular distance from A to p.  Answers:
772: The coordinates of the point of intersection of l and p is ("\frac{5}{2}", "\frac{3}{2}", "\frac{11}{1}").
773: The acute angle between l and p is "78.8"^{\circ}.
774: The perpendicular distance from A to p is "\frac{4*\sqrt{14}}{7}" units.

ID: 200703003009
Content:
img;The diagram shows the graph of $$y = e^{x} -3x$$. The two roots of the equation $$e^{x} -3x = 0$$ are denoted by $$\alpha$$ and $$\beta$$, where $$\alpha$$  < $$\beta$$.;(i) Find the values of $$\alpha$$ and $$\beta$$, each correct to 3 decimal places. A sequence of real numbers $$x_{1}, x_{2}, x_{3}, ...$$ satisfies the recurrence relation  $$x_{n+1}= \frac{1}{3} ^{x_{n}}$$ for $$x_{1}= 2$$.;(ii) Prove algebraically that, if the sequence converges, then it converges to either $$\alpha$$ or $$\beta$$.;(iii) Use a calculator to determine the behaviour of the sequence for each of the cases $$x_{1} = 0$$, $$x_{1} = 1$$, $$x_{1} = 2$$.;(iv) By considering $$x_{n+1} - x_{n}$$, prove that $$x_{n+1} < x_{n}$$ if $$\alpha < x_{n} < \beta$$, $$x_{n+1} > x_{n}$$ if $$ x_{n} < \alpha$$ or $$ x_{n} > \beta$$.;(v) State briefly how the results in part (iv) relate to the behaviours determined in part (iii).  Answers:
775: 
776: \alpha = "0.619".
777: \beta = "1.512".
778: 
779:  x_{1} = 0 and 1, Sequence "converges" to " \alpha".
780:  x_{1} = 2, Sequence "diverges".

ID: 200703003010
Content:
A geometric series has common ratio r, and an arithmetic series has first term a and common difference d, where a and d are non-zero. The first three terms of the geometric series are equal to the first, fourth and sixth terms respectively of the arithmetic series.;(i) Show that $$ 3r^{2}-5r+2=0$$.;(ii) Deduce that the geometric series is convergent and find, in terms of a, the sum to infinity.;(iii) The sum of the first n terms of the arithmetic series is denoted by S. Given that a > 0, find the set of possible values of n for which S exceeds 4a. Answers:
781: 
782: 
783:  S_{\infty} = "3a".
784: The set of possible values of n for which S exceeds 4a is "6" \leq n \leq "13", n \in \mathbb{Z}^{+}.

ID: 200703003011
Content:
A curve has parametric equations $$ x = cos^{2} t$$, $$ y = sin^{3} t$$, for $$ 0 \leq t \leq \frac{1}{2} \pi$$.;(i) Sketch the curve.;(ii) The tangent to the curve at the point $$\cos^{2} \theta, \sin^{3} \theta$$, where $$0 < \theta < \frac{1}{2} \pi$$, meets the x- and y-axes at Q and R respectively. The origin is denoted by O. Show that the area of triangle OQR is $$\frac{1}{12} \sin \theta  (3 \cos^{2}  \theta + 2 \sin^{2} \theta)^{2}$$.;(iii) Show that the area under the curve for  $$ 0 \leq t \leq \frac{1}{2} \pi$$ is $$2 \int_{0}^{\frac{1}{2} \pi } \cos t \sin^{4}t dt$$, and use the substitution$$\sin t = u$$ to find this area Answers:
881: None
882: To deduce the area of triangle OQR:;\frac{dx}{dt}="-2\sin{t}\cos{t}";\frac{dy}{dt}="3\sin^{2}{t}\cos{t}";\frac{dy}{dx}=\frac{dy}{dt}/\frac{dx}{dt} = \frac{3\sin^{2}{t}\cos{t}}{-2\sin{t}\cos{t}} = "-\frac{3}{2}\sin{t}";At the point (\cos^{2}{\theta}, \sin^{3}{\theta}), t=\theta \Rightarrow\frac{dy}{dx}="-\frac{3}{2}\sin{\theta}";Equation of the tangent at (\cos^{2}{\theta},\sin^{3}{\theta}): y_1-y_2=m(x_1-x_2);y-\sin^{3}{\theta}=-\frac{3}{2}\sin{\theta}(x-\cos^{2}{\theta});y="-\frac{3}{2}\sin{\theta}(x-\cos^{2}\theta)+\sin^{3}{\theta}";At x=0, y = "\frac{3}{2}\sin{\theta}\cos^{2}{\theta}+\sin^{3}{\theta}";At y=0, "-\frac{3}{2}\sin{\theta}(x-\cos^{2}\theta)+\sin^{3}{\theta}" = 0;-\frac{3}{2}x\sin{\theta}+"\frac{3}{2}\sin{\theta}\cos^{2}{\theta}"+\sin^{3}{\theta} = 0;"\frac{3}{2}x\sin{\theta}"=\frac{3}{2}\sin{\theta}\cos^{2}{\theta}+\sin^{3}{\theta};\Rightarrow x="\cos^{2}{\theta}+\frac{2}{3}\sin^{2}{\theta}";;The area of triangle OQR: \frac{1}{2}xy;=\frac{1}{2}("\cos^{2}{\theta}+\frac{2}{3}\sin^2{\theta}")("\frac{3}{2}\sin{\theta}\cos^{2}{\theta}+\sin^{3}{\theta}") ;=\frac{1}{2}(\frac{3}{2}\sin{\theta}\cos^{4}{\theta}+"\cos^{2}{\theta}\sin^{3}{\theta}"+\cos^{2}{\theta}\sin^{3}{\theta}+"\frac{2}{3}\sin^{5}{\theta}");=\frac{3}{4}\sin{\theta}\cos^{4}{\theta}+\cos^{2}{\theta}\sin^{3}{\theta}+"\frac{1}{3}\sin^{5}{\theta}";= "\frac{1}{12}\sin{\theta}"(9\cos^{4}{\theta} + 12\cos^{2}{\theta}\sin^{2}{\theta}+4\sin^{4}{\theta});= \frac{1}{12}\sin{\theta}("3\cos^{2}{\theta}" + 2\sin^{2}{\theta})^2
883: \frac{dx}{dt} = "-2\sin{t}\cos{t}" \Rightarrow dx = "-2\sin{t}\cos{t}" dt;Area under curve  = \int^{1}_{0} y dx;When x = 1, \cos^2{t} = 1 \Rightarrow t = "0";When x = 0, \cos^2{t} = 0 \Rightarrow t = \pm "\frac{\pi}{2}";\Rightarrow \int^{0}_{-\frac{\pi}{2}} ("\sin^{3}{t}")(-2\sin{t}\cos{t} dt);= 2 \int^{\frac{1}{2}\pi}_{0} \cos{t}\sin^{4}{t} dt ;Using the substitution \sin{t} = u:;\cos{t}\frac{dt}{du} = 1 \Rightarrow dt = "\frac{du}{\cos{t}}";;\therefore Area under curve: 2\int^{\frac{\pi}{2}}_{0} \cos{t}\sin^{4}{t} dt;At t = 0, \sin{0} = u \Rightarrow u = 0;At t=\frac{1}{2}\pi, \sin{\frac{1}{2}\pi} = u \Rightarrow u = 1;\Rightarrow 2 \int^{1}_{0} (\cos{t})("u^4")\left(\frac{du}{\cos{t}}\right) ;= 2 \int^{1}_{0} u^4 du ;= 2["\frac{u^5}{5}"]^{1}_{0};= 2("\frac{1}{5}") - 0;= \frac{2}{5} units^{2}

\end{document}
