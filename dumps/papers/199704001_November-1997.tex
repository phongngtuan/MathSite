\documentclass{article}
\begin{document}
ID: 199704001001
Content:
a) The expressions \[x^3-ax+a^2\] and \[a^3-x^2-17\] have the same remainder when divided by x-2. Find the possible values of ;i) a;ii) the remainder;b) Find the x- coordinate of each of the three points of intersection of the curves \[y=6x^2-5\] and \[y=17x-\frac{6}{x}\];(Note: Please enter your answer in ascending order);c) Find the values of k for which \[x^2-3x+k\] is a factor of \[x^3-5x^2+12\]Answers:

ID: 199704001002
Content:
a) An arithmetic progression is such that the 10th term is 40 and the sum of the first 10 terms is 265. Find the sum of the first 20 terms.;b) The first two terms of a geometric progression are 100 and 120 respectively. The kth term of the progression is the largest term that is less than 5000. Find the value of k and hence evaluate the sum of first k terms.;c) A geometric progression has a first term of 18 and a sum to infinity of 30. Each of the terms in the progression is squared to form a new geometric progression. Find the sum to infinity of the new progression.Answers:

ID: 199704001003
Content:
img;a) The table shows experimental values of two variables x and y. ;Using the vertical axis for xy and the horizontal axis for \[x^3\] plot xy against \[x^3\]and obtain a straight line graph.;Make use of your graph to ;i) express x in terms of y ;ii) estimate the value of x when \[y=\frac{60}{x}\] ;b) The equation \[y=\frac{(x+c)}{(x+d)}\] where c and d are constant, can be represented by a straight line when xy-x is plotted against y as shown in the diagram.;img;Find the value of c and d.Answers:

ID: 199704001004
Content:
The mass, m grams, of a radioactive substance, presented at time t days after first being observed, is given by the formula \[m=24e^{(-0.02t)}\]; a) Find ; i) the value of m when t = 30;ii) the value of t when the mass is half of its value at t = 0,;iii) the rate at which the mass is decreasing when t = 50.;b) Solve the equation \[\lg(20+5x)-\lg(10-x)=1\];c) Given that \[x=\lg a\] is a solution of the equation \[10^{(2x+1)}-7(10^x)=26\] find the value of a.Answers:

ID: 199704001005
Content:
a) Find all angles between \[0^{\circ}\]  and \[360^{\circ}\]   which satisfy the equation;i) \[2\cos2x=4\sin x+3\];ii) \[\sin(y+30^{\circ})=3\cos y\];b) Express \[3\cos x-4\sin x\] in the form \[R\cos(x+\alpha)\] where R is a positive constant and \[\alpha\] is a positive acute angle measured in radians.;A curve has the equation \[y=3\cos x-4\sin x\] For \[0\leq x\leq \pi \];Find the coordinates of ;i) the point on the curve at which y = 1,;ii) the stationary point of the curve.Answers:

ID: 199704001006
Content:
a) Differentiate \[\sin^2x\] with respect to x.;b) Find the value of k for which \[\frac{\mathrm{d} }{\mathrm{d} x}(\frac{(2x-3)}{(x+5)})=\frac{k}{(x+5)^2}\] ;c) Find the gradient of the curve \[y^2=x^2+2xy+8\]  at each of the points where x = 2.;d) A curve has the equation \[y=\frac{c}{(1+2x)^2}\] where c is a constant.;i) Obtain an expression for \[\frac{\mathrm{d} y}{\mathrm{d} x}\]; ii) When x increases from 1 to 1 + p, where p is small, the corresponding change in y is approximately \[\frac{(-8p)}{3}\] ;Find the value of c.Answers:

ID: 199704001007
Content:
a) Find \[\int \sqrt{4x+5}dx\];b) A particle moves ia straight line so that, at time t seconds after passing through a fixed point O, its velocity \[vms^{-1}\] is given by \[v=8\cos(\frac {t}{4})\];  Find ;i) the value of t at which the particle first comes to rest,;ii) the distance travelled by the particle in the first 4 seconds after passing through O.;img;c) The diagram shows part of the curve \[y=\frac{12}{(x+3)}\]; Find ;i) the area of the shaded region A,;ii) the volume obtained when the shaded region B is rotated through \[360^{\circ}\]   about x-axis.Answers:

ID: 199704001008
Content:
The parametric equations of curve are \[x=4t+\frac{9}{t}\]  \[y=2t-5\] ; Find; i) an expression for \[\frac{\mathrm{d} y}{\mathrm{d} x}\] in terms of t,; ii) the coordinates of the point at which the tangent to the curve at (13, - 3) meets the x-axis,;iii) the coordinates of each of the points on the curve  at which the tangent to the curve is parallel to the y-axis. ;Find also;iv) the value of t at each of the points of intersection of the curve with the line x+y=16;v) the cartesian equation of the curve.Answers:

\end{document}
