\documentclass{article}
\begin{document}
ID: 199502002001
Content:
a) A train travels 145 km at an average speed of $$100 kmh^{-1}$$. How many minutes does this journey take? [2];;b) On part of the journey the train takes 25 minutes to travel 40 km. Find the average speed in kilometers per hour. [2];;c) Each of the 400 passengers on the train is given a drink. The drinks are in packs of 12. Find the least possible number of packs required and the number of drinks left over. [2];;d) On each of the 365 days of 1991, the train made two journeys. The average number of passengers on each journey was 400. How many passengers did the train carry in the year? Give your answer in standard form. [2];;e) The number of passengers in 1993 was 5% higher than in 1992. The number in 1994 was 5% higher than in 1993. What was the total percentage increase in the number of passengers from 1992 to 1994? [2]Answers:

ID: 199502002002
Content:
img;The two glasses shown in the diagram are geometrically similar. The height of the smaller glass is 8 cm. The height of the larger glass is 10 cm.;;a) The top of the larger glass has a circumference of 30 cm. Find the circumference of the top of the smaller glass. [2];;b) Both glasses are completely filled with fruit juice. The cost of the fruit juice in the smaller glass is 64 cents. Find the cost of the fruit juice in the larger glass. [3];;c) The volume, V cubic centimeters, and the height, h centimeters, of a third glass are connected by the formula  V= $$\frac{7}{16}h^3$$. The volume of this glass is 300 cm3. Find its height, correct to the nearest millimeter. [3]Answers:

ID: 199502002003
Content:
img;The graph shows the relation between the number (n) of units of electricity used and the total cost (c) of an electricity bill.;;a) Use the graph to find;;a-i) the cost of the bill if 300 units are used, [1];;a-ii) the number of units used when the bill is $32.50. [1];;b) Given that the relation is $$C = pn + q$$,;;b-i) state the value of q and explain its significance, [2];;b-ii) find the value of p and explain its significance, [2];;b-iii) find the total cost of the bill if 1100 units are used. [2]Answers:

ID: 199502002004
Content:
img;The table shows the distribution of the masses of a number of eggs.;;a) Calculate an estimate of the mean mass of the eggs. [2];;b) Draw a histogram to represent this distribution. Label your axes carefully. [3];;c) An egg is chosen at random and is then replaced. Find the probability that its mass lies in the range$$50 < m \leq 60$$. [1];;d) Two eggs are chosen at random. Find the probability that the mass of one is in the range $$50 < m \leq 60$$ and the mass of the other is in the range $$60 < m \leq 80$$. [2]Answers:

ID: 199502002005
Content:
a) ;img; In the diagram, the points R and T lie on a circle with centre O. The tangent at T meets  OR produced at S. The radius of the circle is x cm, TS = 12 cm and RS = 8cm.;;a-i) Express the length OS in terms of x. [1];;a-ii) Explain briefly why $$O \hat TS = 90^{\circ}$$. [1];;a-iii) Write down an equation and solve it to find the value of x. [3];;b) ;img; In the diagram ABC and AED are straight lines. B, C, D and E lie on a circle. $$A \hat BE = 65^{\circ}$$ and $$B \hat DC = 50^{\circ}$$.;;b-i) Find $$C \hat DE$$. [1];;b-ii) Find $$B \hat CE$$. [1];;b-iii) Name the triangle which is similar to triangle ABD, explaining why the two triangles are similar. [2];;b-iv) ;img; By considering a second pair of similar triangles, copy and complete the statement below by filling in the boxes. Name the triangles you have used. [2]Answers:

ID: 199502002006
Content:
img; In a competition, the competitors follow a course OABC, as indicated on the diagram. ;; A is 1000 m due North of O.;;B is 1200 m from A on a bearing of $$120^{\circ}$$.;;;C is 1100 m from B and $$A \hat BC = 90^{\circ}$$.;;a) Using a scale of 1 cm to represent 100 m, make an accurate scale drawing of the course. [You should place the point O half way down the left hand side of a new page.] [2];;b) Use your drawing to find the bearing of O from C. [1];;c) Competitors are told that they have to go from C to a point X. The point X is within the quadrilateral OABC, is 600 m from C and is equidistant from AO and AB.;;c-i) On your scale drawing construct the locus of points which are within the quadrilateral OABC and;;c-i-a) 600 m from C, [1];;c-i-b) Equidistant from AO and AB. [2];;c-ii) Mark clearly the position of X. [1]Answers:

ID: 199502002007
Content:
a) A shop sells two types of radio, Hiblast and Megadef. In a sale all prices are reduced by 5%.;;a-i) A Hiblast radio normally sells for $41. Find its price in the sale. [2];;a-ii) A Megadef radio costs $51 in the sale. Find its normal price. [3];;b-i) Harry assembles Hiblasts and produces one every x minutes. Write down an	expression, in terms of x, for the number he produces in an hour. [1];;b-ii) Marion assembles Megadefs and takes two minutes longer than Harry to produce each one. Write down an expression, in terms of x, for the number of Megadefs she produces in an hour. [1];;b-iii) Harry and Marion together produce a total of 11 radios in an hour. Form an equation and show that it reduces to 11x^2 - 98x - 120 = 0. [2];;b-iv) Solve this equation and hence find how many Megadefs Marion produces in an hour. [3]Answers:

ID: 199502002008
Content:
a) Given that P is the point (1, 1), $$\vec{PQ} = \begin{bmatrix}-3\\2\end{bmatrix}, \vec{PR} = \begin{bmatrix}5\\4\end{bmatrix}$$ and that T is the midpoint of QR, find;;a-i) $$\vec{QR}$$, [1];;a-ii) $$\vec{PT}$$, [1];;a-iii) the coordinates of the point X such that PQXR is a parallelogram. [2];;b) Answer this part of the question on a sheet of graph paper. The triangle ABC has vertices A(0, 1), B(1, 4) and C(2, 2).;;b-i) Using a scale of 1 cm to 1 unit on each axis, draw x and y axes for $$-6 < x \leq 10$$ and $$0 < x \leq 6$$. Draw and label $$\Delta ABC$$. [1];;b-ii) A transformation is represented by the matrix $$\begin{bmatrix}0&-1\\1&0\end{bmatrix}$$. The transformation maps $$\Delta ABC$$ onto $$\Delta DEF$$. Draw and label $$\Delta DEF$$. [2];;b-iii) $$\Delta ABC$$ is transformed onto $$\Delta LMN$$ by a shear, with the x-axis invariant and B(1, 4) mapped onto M(9, 4). Find the coordinates of N and the matrix which represents this shear. [3];;b-iv) Find the matrix representing the transformation which maps $$\Delta DEF$$onto $$\Delta LMN$$. [2]Answers:

ID: 199502002009
Content:
img; A, B and C lie in a straight line on level ground. T is the top a vertical flagpole TC.;;a) John wants to find the height of the flagpole. He measures the angle of elevation of the top of the flagpole from A and finds that it is  . He then walks 7 m to B and finds that the angle of elevation is now  . Calculate;;a-i) $$A \hat TB$$, [1];;a-ii) the length of BT, [3];;a-iii) the height, TC, of the flagpole. [2];;b) On the opposite side of the flagpole from A and B, the ground slopes down to D such that . A rope is stretched from D, which is 15 m from C, to the point R, where CR = 12 m. Calculate the length of the rope DR. [4];;c) A second vertical flagpole DE is to be erected at D. Given that RE is horizontal, calculate the length of DE. [2]Answers:

ID: 199502002010
Content:
Answer the whole of this question on a sheet of graph paper. One thousand candidates took a Mathematics examination which consisted of two papers. Each paper was marked out of 50. ;img;img; Table A gives the distribution of marks and Table B is the corresponding cumulative frequency table.;;a) By comparing the two tables, calculate the values of p, q, r, s and w, x, y, z. [2];;b) Using a horizontal scale of 2 cm to represent 10 marks, and a vertical scale of 2 cm to represent 100 candidates, draw the cumulative frequency curve for Paper 1. [2];;c) Use your curve to estimate, for Paper 1,;;c-i) the median mark, [1];;c-ii) the inter-quartile range, [1];;c-iii) the 30th percentile. [1];;d) The pass mark for Paper 1 was 21. Use your graph to estimate the number of candidates who passed. [1];;e) The top 12% of the candidates taking Paper 1 gained a Distinction. Use your graph to estimate the minimum mark required for a DistinctionAnswers:

ID: 199502002011
Content:
A large area is to be paved with blocks each one metre square. ;img; On Day 1, three blocks are placed in a line, as shown in the diagram. Each following day the paved region is enlarged by adding blocks to surround the previous day;;a) Find;;a-i) the perimeter after Day 4 and after Day 5, [1];;a-ii) an expression, in terms of n, in its simplest form, for the perimeter after Day n. [2];;b) Find;;b-i) the area after Day 4 and after Day 5, [2];;b-ii) an expression, in the form $$an^2 - b$$, for the area after Day n, [2];;b-iii) the total number of blocks which will have been used after Day 15, [1];;b-iv) after which day the area will be $$399 m^2$$. [2];;c) How many blocks will be added during Day 18? [2]Answers:

\end{document}
