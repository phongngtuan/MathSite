\documentclass{article}
\begin{document}
ID: 199602001001
Content:
a) Solve the equation \[2x^3-7x^2-7x+30=0\](Note: Please enter the answer in ascending order);b) When the expression \[x^2+bx+c\] is divided by x-2, the remainder is R. When the expression is divided by x + 1, the remainder is also R.;i)	Find the value of b;When the expression is divided by x-4, the remainder is 2R.;ii)	Find the value of c and of R.;When the expression is divided by x-t, the remainder is 5R.;iii)	Find the two possible values of t.;img;c) The sketch shows part of the graph of \[y=x^3+px^2+qx+r\] where p, q and r are constants.;The points A, B and C have coordinates (-2,0), (2,0) and (4,0) respectively. The curve crosses the y-axis at D. Evaluate p, q and r, and state the coordinates of D.Answers:

ID: 199602001002
Content:
a) The first and last terms of an arithmetic progression are -24 and 72 respectively. The sum of all the terms is 600. Calculate;i) the number of terms in the progression;ii) the common difference;iii)	the sum of the positive terms.;b) All the terms of a geometric progression are positive. the second term is 12 and the fourth term is 6.75. Find;i) the common ratio and the first term,;ii) the sum of the first 10 terms.;c) On January 1st, 1960, a person opened a savings account with £100. Interest, at 5% of the amount standing in the account at the time, was added each year on January 1st, starting in 1961. Given that no withdrawals were made, find the year in which there was more than £520 in the account after the interest had been added.Answers:

ID: 199602001003
Content:
a) Solve the equation \[\lg (x-4)+2\lg 3 = \lg(\frac{x}{2})\] ;b) The first three terms of an arithmetic progression are \[\log_2(32)\] \[\log_2(p)\] \[\log_2(q)\]  The common difference is -3. Evaluate p and q.;c) Show that the curve \[y=e^x-2e^{-x}\]  has no stationary values.;Draw, on graph paper, the curve \[y=e^x-2e^{-x}\]   for values of x at intervals of 0.5 from x = 0 to x = 2. Use your curve to solve the equation \[e^{2x}-3e^x-2=0\]Answers:

ID: 199602001004
Content:
img;a) The table shows experimental values of two variables, x and y.;It is known that x and y are related by the equation \[y=Ae^{-kx}\] where A and k are constants. Using graph paper, plot in y against x for the above data and use your graph to estimate the value of A and of k.;img;b) The diagram shows part of the straight line graph drawn to represent the equation px + qy =xy. Given that the straight line passes through (2,0) and has a gradient of -1.5, find the value of p and of q.;c) Variables x and y are related in such a way that when x - y is plotted against xy a straight line is produced as as shown in the diagram;img; This line passes through the points (1,2) and (5,4). Find y in terms of x.Answers:

ID: 199602001005
Content:
img;The diagram shows two perpendicular lines, AC and BD, where AD =5cm, BC = 12cm and \[\angle DAE = \angle EBC = \theta\] where \[\theta\]  varies.;i)Explain why \[BD= 5\sin \theta + 12\cos \theta\] ;ii) Express BD in the form \[R\sin(\theta + \alpha)\] stating the value of R and of \[\alpha\]  and hence find the value of \[\theta\]   for which BD = 8cm. ;iii) Show that the area of the quadrilateral ABCD is \[\frac{(169\sin 2\theta + 120)}{4}\];iv) Find the maximum area of ABCD as \[\theta\]   varies and state the corresponding value of \[\theta\]  Answers:

ID: 199602001006
Content:
a) Differentiate with respect with x;i) \[x^2\ln(2x+1)\] ;ii) \[\frac{(3x-1)}{\tan x}\] ;b) The point P(2,6) lies on the curve whose equation is \[xy-y^2+12x=0\];Find ;i) the gradient of the curve at P,;ii) the angle at which the tangent to the curve at P makes with the x-axis.;c) The variable y is given in terms of x by \[y=\cos^{2x}-0.5\] where \[0\leq x\leq \frac{\pi }{2}\];Given that x is increasing at 0.5 radians per second,;find the rate of change of y with respect to time when \[x=\frac{\pi }{6}\]Answers:

ID: 199602001007
Content:
(a)	  Find \[\int\frac{(6x^3-1)}{x} dx\];(b)	Show that\[(\cos x-\sin x)^2-=1-\sin 2x\];img;The diagram shows part of the curve \[y= \cos x-\sin x\] The shaded region is rotated through one revolution about the x-axis. Calculate the volume generated.;(c)	Given that \[y=e^x(\cos x-\sin x)\] show that \[\frac{\mathrm{d} y}{\mathrm{d} x}=-2e^x\sin x\] Hence evaluate \[\int_0^\pi  e^x \sin x dx\]Answers:

ID: 199602001008
Content:
The parametric equations of a curve are \[x=2t^2-t\] \[y=t^2+t\].The point A has parameter t=1 and the point B has parameter t=2;The midpoint of AB is M. Find ;i) the coordinates of M,;ii) the gradient of AB,;iii) an expression for \[\frac{\mathrm{d} y}{\mathrm{d} x}\] in terms of t .;The tangent at the point P on the curve is parallel to AB .;iv) Find the value t at the point P.;The point Q gas parameter t=0.1;v) Show that the normal at Q is parallel to PM . Obtain the Cartesian equation of the curve.Answers:

\end{document}
