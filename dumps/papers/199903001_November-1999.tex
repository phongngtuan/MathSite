\documentclass{article}
\begin{document}
ID: 199903001001
Content:
Find the equation of the perpendicular bisector of the line joining the point (-5, 4) to the point (9, -3).Answers:

ID: 199903001002
Content:
Find the range of values of x for which \[3(x+1)^2<16x\]Answers:

ID: 199903001003
Content:
Relative to an origin O the position vectors of the points P and Q are 3i + j and 7i-15j respectively. Given that R is the point such that \[3\vec{PR}=\vec{RQ}\] find a unit vector in the direction \[\vec{OR}\]Answers:

ID: 199903001004
Content:
Find the term independent of x in the expansion of \[(2x-\frac{1}{(2x^2)})^9\]Answers:

ID: 199903001005
Content:
The straight line y = 2p +1 intersects the curve \[y=x+\frac{p^{2}}{x}\] at two distinct points. Find the range of values of p.Answers:

ID: 199903001006
Content:
Show that \[(\tan\theta+\sin\theta)(\tan\theta-\sin\theta)-=\tan^2\theta\sin^2\theta\] Answers:

ID: 199903001007
Content:
The gradient at any point (x, y) on a particular curve is given by \[\frac{\mathrm{d} y}{\mathrm{d} x}=1+\frac{1}{(2x^2)}\];The equation of the tangent at the point P on the curve is y = 3x + 1. Given that the x-coordinate of P is positive, find ;i) the coordinate of P;ii) the equation of the curve.Answers:

ID: 199903001008
Content:
Use calculus to determine, in terms of p, the approximate change in the radius of a circle when the area of the circle increases from \[900\pi\] to\[ (900 + p)\pi\] where p is small.Answers:

ID: 199903001009
Content:
The point P(x, y) lies on the line 7y = x + 23 and is 5 units from the point (2, 0). Calculate the coordinates of the two possible positions of P.(Note: Please enter the answer with smaller value of x first)Answers:

ID: 199903001010
Content:
img;The diagram shows a circle, centre O, of radius 10 cm. ;The line AC is perpendicular to the radius OA, and the line OC intersects the circle at B . Given that angle OCA is 0.5 radians, calculate;(a)	the length of AC,;(b)	the area of the shaded region,;(c)	the perimeter of the shaded region.Answers:

ID: 199903001011
Content:
Two flower beds, one a circle of radius r m, the other a square of side x m, are planned for a large garden. To protect the young plants the two beds are to be surrounded by wire netting; the total length of wire netting to be used is 40 m.;(a)	Express x in terms of r and \[\pi\];The combined area of the two flower beds is \[Am^2\];(b)	Show that \[A=\frac{\pi}{4}(4+\pi)r^2-10\pi r+100\];Given that r may vary,;(c) find the value of r corresponding to the stationary value of A,;(d)show that, when A is stationary, the side of the square is equal in length to the diameter of the circle,;(e)	determine whether the stationary value of A is a maximum or a minimum.Answers:

ID: 199903001012
Content:
a) Calculate, to the nearest \[unit^3\] the volume of the solid of revolution obtained when the region enclosed by the curves \[y=3x^2\]and \[y=x^3\] is rotated through about the x-axis.;img;b) The diagram shows part of the curve \[y=\frac{1}{x^{2}}+2x\] part of the line x = 0.5 and the line OP joining the origin O to P, the minimum point of the curve.;i.Show that the x-coordinate of P is 1.;ii.Find the area of the shaded region.Answers:

ID: 199903001013
Content:
A particle A moves in a straight line so that its displacement, s m, from a point O at time t s, where \[t\geq 0\] is given by \[s=t^3-4t^2-3t+5\];Find ;i) the value of t when the particle is instantaneously at rest and the distance the particle has then travelled,;ii) the value of t, to two decimal places, when the particle has returned to its initial position.;A particle B moves on a parallel straight line so that its acceleration \[a ms^{-2}\] at time t s is given by a= 2t + 1. Given that A and B have the same velocity when t = 5, obtain an expression, in terms of t, for the velocity of B .Answers:

ID: 199903001014
Content:
(a) Find all the angles, between \[0^{\circ}\] and \[360^{\circ}\] which satisfy ;(i) \[4\sin^2x=6-9\cos x\];(ii) \[3 \cos y + \cot y = 0\];(b) Find, to two decimal places, the values of z between 0 and 3 for which \[\tan(2z -1) = 0.6\] (Note: Please enter your answers in ascending order)Answers:

ID: 199903001015
Content:
Solutions to this question by accurate drawing will not be accepted.;img;The diagram shows the trapezium ABCD in which A is the point (1, 2), B is (3, 8), D is (5, 4), \[\angle ABC=90^{\circ}\] and AB is parallel to DC.;(a)	Find the coordinate of C.;The point E lies on BD and is such that the area of triangle CDE is \[\frac{1}{4}\] of the area of triangle CDB;(b)	Find the coordinate of E.;The point F is such that CDFE is a parallelogram.;(c) Find the coordinates of F and the area of the parallelogram CDFE.Answers:

ID: 199903001016
Content:
The position vectors of points P and Q relative to an origin O are \[\binom{4}{2}\] and \[\binom{k}{4}\]respectively. Given that \[\vec{QP}\] is perpendicular to \[\vec{QP}\] use a scalar product to find ;i) the value of k,;ii) \[\angle QOP\];The position vector of the point R relative to the origin O is \[\binom{-8}{6}\] Given that X is a point on RP such that \[\vec{RX}=\lambda \vec{RP}\];iii) express \[\vec{OX}\] as a column vector in terms of \[\lambda\];Given also that \[\vec{OX}=\mu \vec{OQ}\];  iv) find the value of \[\lambda\] and of \[\mu\]Answers:

ID: 199903001017
Content:
Functions f and g are defined by ;$$f(x)=\frac{a}{(3-x)}, x\neq3$$;$$g(x) = 11+bx^2$$;where `x in RR`. Given that $$f^2(5)=4/5$$ and $$fg(2)=1/2$$, evaluate a and b .Answers:

ID: 199903001018
Content:
(a)Find the minimum value of \[(x-2)^2-2\] and the corresponding value of x. Sketch the graph of \[y=|(x-2)^2-2|\] for \[0\leq x\leq 4\];(b) The graph of \[y=h^{-1}(x)\] is a smooth curve passing through the points (1, 0), (1.3, 1), (2, 1.7) and (3, 2).;(i)	Draw, on the graph paper, the graph of \[y=h^{-1}(x)\]using the same scale on each axis.;(ii)On the same diagram draw the graph of \[y=h(x)\]for \[0\leq x\leq 2\]Answers:

\end{document}
