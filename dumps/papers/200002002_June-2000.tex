\documentclass{article}
\begin{document}
ID: 200002002001
Content:
a) The triangle ABC has AB = 36 cm.;;a-i) P is the point on the side AB where AP : PB = 5 : 4. Calculate the length of AP. [2];;a-ii) The length of BC is 10% less than the length of AB. Calculate the length of BC. [2];;a-iii) The length of AB is 20% greater than the length of CA. Calculate the length of CA. [3];;a-iv) Which is the smallest angle of the triangle? Give a reason for your answer. [1];;b) The length and breadth of a rectangle, both measured correct to the nearest centimeter, are 11 cm and 9 cm respectively. Calculate the lower bound of the area of the rectangle. [2]Answers:

ID: 200002002002
Content:
img;The diagram shows A, B, C and D, the four corners of a horizontal rectangular field ABCD. AC = 110 m and  $$D \hat A=43^{\circ}$$.;;a) Calculate the length of DC. [2];;b) TC represents a vertical tree. The angle of elevation of T from A is $$17^{\circ}$$. Calculate;;b-i) the height of the tree, [2];;b-ii) the angle of elevation of T from D. [2]Answers:

ID: 200002002003
Content:
img;The total cost of the electricity supplied to a house is found by adding two charges. These are the first 100 unit at a fixed cost of $22 and cost of additional units at $5 per 100 units ;;a) Find the total cost of 700 units. [1];;b) Find the fixed standing charge. [1];;c) Find the number of units used when the total cost is $80. [2];;d) The total cost when n units are used is C dollars. Write down the formula for C in terms of n. [2]Answers:

ID: 200002002004
Content:
a);img;In the diagram, A, B, C, D and E lie on the circumference of a circle. DE is parallel to CB. $$A \hat CE=48^{\circ}$$, $$C \hat ED=65^{\circ}$$ and $$C \hat BE=73^{\circ}$$. Calculate;;a-i) $$A \hat BE$$, [1];;a-ii) $$A \hat CB$$, [1];;a-iii) $$B \hat EC$$, [1];;a-iv) $$C \hat DE$$. [1];;b);img;[The value of $$\pi$$ is 3.142, correct to 3 decimal places.] In the diagram, O is the centre of a circle of radius 8 cm. PQ is a chord and $$P \hat OQ=150^{\circ}$$. The minor segment of the circle formed by the chord PQ is shaded. Calculate;;b-i) the length of the minor arc PQ, [2];;b-ii) the area of triangle OPQ, [2];;b-iii) the shaded area. [3]Answers:

ID: 200002002005
Content:
a) Simplify the expression 2k - 3(2 - k) + 4. [2];;b) Solve the equation $$\frac{5t}{3}-\frac{t}{2}=1$$. [2];;c) Solve the equation $$y^{2} +3y=6$$, giving both answers correct to 2 decimal places. [4];;d);img;The diagram shows a rectangle ABCD with a square of side x cm removed. AP = 3cm and QC = 4 cm.;;d-i) Find, in terms of x, an expression for the area of the shaded rectangle APSD. [1];;d-ii) The area of the shaded rectangle APSD is double the area of the unshaded rectangle RQCS.;;d-ii-a) Form an equation in x and solve it. [3];;d-ii-b) Hence find the area of the shaded rectangle. [1]Answers:

ID: 200002002006
Content:
The weekly wages of the people who work in a small factory are given in the table below.;img;;a) Calculate the total weekly wages bill for the factory. [1];;b) Find, for the distribution of weekly wages,;;b-i) the mode, [1];;b-ii) the median, [1];;b-iii) the mean. [1];;c) One person is chosen at random from those who work in the factory. Another person is chosen at random from those remaining. Expressing your answer as a fraction in its lowest terms, find the probability that the sum of the wages of these two people is more than $410. [2]Answers:

ID: 200002002007
Content:
img;[The value of $$\pi$$ is 3.142, correct to three decimal places.] [1 litre = $$1000cm^{3} $$.] Some identical bowls are open cylinders each of radius 6 cm and height 4 cm. Each bowl is made from thin metal.;;a) Calculate the area of metal needed to make each bowl, giving your answer correct to the nearest square centimeter. [3];;b);img;A hemispherical pan contained 13 litres of soup. As many bowls as possible are completely filled with soup from the pan.;;b-i) Calculate the number of bowls which are filled. [3];;b-ii) Calculate the volume of soup which is left in the pan, giving your answer in cubic centimeters. [2];;b-iii) [The volume of a sphere of radius r is $$\frac{4}{3}\pi r^{3} $$.] It is given that 13 litres of soup completely filled the pan. Calculate the radius of the hemisphere, giving your answer correct to the nearest millimeter. [2];;c) Michael has two different bowls, which are geometrically similar to each other. The heights of the bowls are in the ration 2 : 3. Write down the ratio of their volumeAnswers:

ID: 200002002008
Content:
Answer the whole of this question on a sheet of graph paper. The table below gives some values of x and the corresponding values of y, given correct to two decimal places, for $$y=\frac{1}{4}(x^{3} -6x^{2} +8x)$$.;img;;a) Using a scale of 2cm to represent 1 unit on each axis, draw a horizontal x-axis for $$-1\leq x\leq5$$ and a vertical y-axis for $$-4\leq y\leq4$$. On your axes, plot the points given in the table and join them with a smooth curve. [3];;b) Describe the symmetry of this curve. [2];;c) Use your graph to solve the equations;;c-i) $$\frac{1}{4}(x^{3} -6x^{2} +8x)=-1$$, [1];;c-ii) $$x^{3} -6x^{2} +8x=8$$. [2];;d) By drawing a tangent, find the gradient of the curve at the origin. [2];;e) The line y = mx intersects the curve $$y=\frac{1}{4}(x^{3} -6x^{2} +8x)$$ at three points. Find the least possible value of m. [2]Answers:

ID: 200002002009
Content:
img;Two ships, Alpha and Beta, left a port, O, at noon. Alpha sailed at 12 km/h on a bearing of $$054^{\circ}$$. Beta sailed at 16 km/h on a bearing of $$130^{\circ}$$.;;a) At 4 p.m., Alpha was at A and Beta was at B. Calculate the distance AB. [4];;b) When Beta had travelled a total distance of 100 km it stopped at C.;img;;b-i) Calculate the time when Beta reached C.[2];;b-ii) Alpha continued on its course until it reached the point D, due North of C. Calculate the distance, OD. [4];;c) An island, R, is 100 km due south of the port, O.;img;Calculate the bearing of R from C. [2]Answers:

ID: 200002002010
Content:
img;In the diagram, O is the origin, A is the point (3, 1), B is (3, 0), C is (5, 1), D is (1, 3) and E is (0, 3). The single transformation P maps $$\Delta  OAB$$ onto $$\Delta  ODE$$. The single transformation Q maps $$\Delta  OAB$$ onto $$\Delta  OCB$$.;;a-i) Describe P completely. [2];;a-ii) Find the matrix which represents P. [2];;b-i) What kind of transformation is Q? [1];;b-ii) The matrix which represents Q is $$\begin{bmatrix}1&n\\0&1\end{bmatrix}$$. Find the value of n. [2];;c) The points O, H and K are the images of O, A and B respectively under the transformation QP. Find;;c-i) the coordinates of H, [2];;c-ii) the matrix which represents QP, [2];;c-iii) the area of $$\Delta  OHK$$. [1]Answers:

ID: 200002002011
Content:
a) $$p=\begin{bmatrix}4\\-3\end{bmatrix}$$, $$q=\begin{bmatrix}-2\\9\end{bmatrix}$$ and $$r=\begin{bmatrix}1\\-2\end{bmatrix}$$.;;a-i) Find |\q|. [1];;a-ii) Express as a column vector;;a-ii-a) 2p + q;;a-ii-b) p - 2r. [2];;a-iii) Write down two facts about the vectors 2p + q and p - 2r. [2];;b);img;In the diagram, $$\vec{OA}=a$$ and $$\vec{OB}=b$$.$$\vec{OM}=3\vec{OA}$$ and $$\vec{ON}=2\vec{OB}$$. C is the point on MN produced where MN = NC.;;b-i) Express, as simply as possible, in terms of a and/or b,;;b-i-a) $$\vec{MO}$$, [1];;b-i-b) $$\vec{MN}$$, [1];;b-i-c) $$\vec{AB}$$, [1];;b-i-d) $$\vec{AC}$$. [2];;b-ii) Write down two facts which your answers to (c) and (d) tell you about A, B and C. [2]Answers:

\end{document}
