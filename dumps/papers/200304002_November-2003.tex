\documentclass{article}
\begin{document}
ID: 200304002001
Content:
a) John and Karen opened bank accounts.;;a-i) John deposited $800 in his account.;This account pays simple interest at the rate of 5% per year.;Calculate the total amount in his account after 3 years.   [2];;a-ii) Karen deposited $800 in her account.;This account pays compound interest at the rate of 5% per year.;Calculate how much more money there is in her account after 3 years than there is in John;;b) Leslie borrowed a sum of money at 10% per year compound interest.;After 3 years he owed a total of $532.40.;Calculate how much he borrowed.   [2]Answers:

ID: 200304002002
Content:
A company manufactures biscuits.;;a) One batch of biscuits contains 300 grams of dried fruit. ;This consists of sultanas and currants, with masses in the ratio 2: 3. ;Find the mass of the sultanas.   [2];;b) The mixture used to make one batch of biscuits has a mass of 18 kg.;The mixture loses 12% of its mass when it is cooked to make the biscuits.;;b-i) Calculate the mass of one batch of biscuits.   [2];;b-ii) Each biscuit has a mass of 12 grams.;One batch of biscuits is put into packets.;Each packet contains 16 biscuits.;Find how many packets can be filled, and the number of biscuits remaining.   [2];;b-iii) The total mass of each packet, including packaging, is 201 grams.;Express the mass of the packaging as a percentage of the total mass of a packet.   [2];;c) A trader sells one packet of biscuits for 80 cents.;He makes a profit of 25% of his cost price.;Calculate the price he paid for a packet of biscuits.   [2]Answers:

ID: 200304002003
Content:
a) Solve the equation $$(2x - 3)(x - 4) = 18$$.   [3];;b) A formula used in connection with a mirror is $$\frac{1}{u} + \frac{1}{v} = \frac{1}{f}$$.;;b-i) Given that v = 9 and f = 5, find u.   [1];;b-ii) Express v in terms of u and f.   [2];;c) A man bought a eggs at r cents per dozen.;He sold them for s cents each.;Find an expression, in terms of a, r and s, for the profit, in cents, that he made.   [2]Answers:

ID: 200304002004
Content:
Answer the whole of this question on a sheet of graph paper.;The speeds of 50 cars being driven along a stretch of road were recorded.;The table below shows the distribution of the speeds of the cars.;img;;a) Using a scale of 1 cm to represent 10 km/h, draw a horizontal axis for speeds up to 110 km/h.;Using a scale of 4 cm to represent 1 unit, draw a vertical axis for frequency densities from 0 to 2 units.;On your axes, draw a histogram to represent the information in the table.   [3];;b) Write down the modal class of the distribution.   [1];;c) In which interval is the upper quartile of the distribution?   [1];;d) Find the probability that one car, selected at random, had a speed of;;d-i) Less than 20 km/h,   [1];;d-ii) More than 60 km/h.   [1];;e) There is a speed limit of 60 km/h on this stretch of road.;Two cars were selected at random.;Calculate the probability that one car was breaking the speed limit and the other was not breaking the limit.   [2]Answers:

ID: 200304002005
Content:
img;The points A, B, C and D lie on a circle as shown on diagram I.;AC cuts BD at P.;AD is parallel to BC.;;a) Show that triangle BPC is an isosceles triangle.   [2];;b) Given that angle ACB = $$32^{\circ}$$ and angle DAB = $$118^{\circ}$$, calculate angle ACD.   [2];;c);img;Diagram II shows the circle in Diagram I and a second circle, centre O.;The two circles intersect at C and D.;AD produced cuts the second circle at F.;BD produced cuts the second circle at E.;Angle DEF = $$110^{\circ}$$.;Calculate;;c-i) angle ACE,   [3];;c-ii) angle COD.   [2]Answers:

ID: 200304002006
Content:
img;ABCD is a rectangle in which AB = 8 cm and BC = 6 cm.;A circular piece of wire, centre O, passes through the vertices of the rectangle as shown in Diagram I.;;a) Show that the radius of the circular wire is 5 cm.   [1];;b) Show that angle AOB = $$106.3^{\circ}$$, correct to 1 decimal place.   [2];;c) Calculate the area of the shaded segment.   [3];;d) The circular wire is cut at A, B, C and D, and the four pieces joined to form the shape shown in Diagram II. ;Calculate the area enclosed by the wires in Diagram II.   [3]Answers:

ID: 200304002007
Content:
img;[The volume of a pyramid = $$\frac{1}{3} \times base.area \times height$$.];The diagram shows a solid traffic bollard.;It consists of a square-based pyramid, VABCD, attached to a cuboid, ABCDPQRS.;The vertical line, VNM, passes through the centres, N and M, of the horizontal squares ABCD and PQRS.;AB = BC = 60 cm and VN = 40 cm.;;a) Calculate;;a-i) VA,   [2];;a-ii) angle VAN,   [2];;a-iii) angle VAP.   [1];;b) Given also that AP = BQ = CR = DS = 80 cm, calculate ;;b-i) The volume of the bollard,   [2];;b-ii) The total surface area of the sides and top of the bollard.   [3];;c) The highway authority needs to paint the sides and tops of 17 of these bollards.;The paint is supplied in tins, each of which contains enough paint to cover $$8 m^2$$.;Find the number of tins of paint needed.   [2]Answers:

ID: 200304002008
Content:
A polar explorer is planning an expedition.;He investigates three possible routes.;;a) If he travels on route A, which is 800 km long, he expects to cover x km per day.;Route B, which is the same distance as route A, has more difficult ice conditions and he would only expect to cover (x - 5) km per day.;Route C, which is 100 km longer than route A, has easier conditions and he would expect to cover (x + 5) km per day.;Write down an expression, in terms of x, for the number of days that he expects to take on;;a-i) Route A,;;aii) Route B,;;a-iii) Route C,;;b) He estimates that route C will take 20 days less than route B.;Form an equation in x, and show that it reduces to $$x^2 + 5x - 450 = 0$$. [4];;c) Solve the equation $$x^2 + 5x - 450 = 0$$, giving both answers correct to 1 decimal place.   [4];;d) Calculate the number of days that he expects to take on route A.   [2]Answers:

ID: 200304002009
Content:
Answer the whole of this question on a sheet of graph paper.;An open rectangular tank has a square base of side x metres.;The volume of the tank is $$36 m^3$$.;;a-i) Find an expression, in terms of x,for the height of the tank.   [1];;a-ii) Hence show that the total external surface area of the tank, A square metres, is given by $$A = x^2 + \frac{144}{x}$$ [1];;b) The table below shows some values of x and the corresponding values of A, ;Correct to 1 decimal place, where $$A = x^2 + \frac{144}{x}$$.;img;;b-i) Find the value of p.   [1];;b-ii) Using a scale of 2 cm to 1 metre, draw a horizontal x-axis for $$2 \leq x \leq 8$$. ;Using a scale of 2 cm to $$10 m^2$$, draw a vertical A-axis for $$40 \leq A \leq 90$$.;On your axes, plot the points given in the table and join them with a smooth curve.   [3];;b-iii) Use your graph to find;;b-iii-a) The values of x for which the surface area is $$65 m^2$$,   [2];;b-iii-b) The gradient of the curve at x = 6,   [2];;b-iii-c) The dimensions of the minimal AAnswers:

ID: 200304002010
Content:
img;An aircraft waiting to land is flying around a triangular circuit ABC.;A, B and C are vertically above three beacons, X, Y and Z.;T is the control tower at the airport, and T, X, Y and Z lie in a horizontal plane.;BC = 18 km, CA = 22 km and AB = 24 km.;;a-i) The plane is flying at 200 km/h.;Calculate the time, in minutes and seconds, that the aircraft takes to complete one circuit.   [2];;a-ii) Calculate the largest angle of triangle ABC.   [4];;b) Z is due West of T. ;The bearing of X from Z is $$042^{\circ}$$ and the bearing of X from T is $$338^{\circ}$$.;;b-i) Find the angles of triangle TXZ.   [2];;b-ii) Calculate TX.   [2];;c) The aircraft is flying at a constant height of 2600 metres.;Calculate the angle of depression of the tower, T, from the aircraft when it is at A.   [2]Answers:

ID: 200304002011
Content:
img;Answer the whole of this question on a sheet of graph paper.;The diagram shows triangle A and the straight line y=x+4.;Triangle A has vertices (3, 2), (3, 4) and (4, 4).;;a) Using a scale of 1 cm to represent 1 unit on each axis, draw, on a sheet of graph paper, axes for values of x and y in the ranges $$-6 \leq x \leq 6$$ and $$-6 \leq y \leq 10$$.;Draw and label triangle A.;Draw the straight line $$y = x + 4$$.   [1];;b) The transformation M is a reflection in the line $$y = x + 4$$.;The transformation M maps triangle A onto triangle B, so that M(A) = B.;Draw and label triangle B.   [2];;c) Triangle C has vertices (-1, 2), (1, 2) and (1, 1).;The rotation R maps triangle A onto triangle C, so that R(A) = C.;Find;;c-i) The angle and direction of this rotation,;;c-ii) The centre of this rotation,;;d) Given that MR(A) = D, draw and label triangle D.   [2];;e) Triangle E has vertices (3, -1), (3, 1) and (4, 0).;The transformation L maps triangleAnswers:

\end{document}
