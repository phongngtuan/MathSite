\documentclass{article}
\begin{document}
ID: 200003003001
Content:
Find \[\frac{\mathrm{d} }{\mathrm{d} x} (\frac {\cos x}{1 - \sin x})\], and simplify your answer.Answers:
240: \frac{d}{dx} \frac{\cos x}{1-\sin x} = "\frac{1}{1-\sin{x}}".

ID: 200003003002
Content:
The diagram shows the graph of y = f(x). The curve passes through the origin O, the point A (a, 0) and the point B (2a, 0). Sketch, on separate clearly labelled diagrams, the graphs of;(i) $$y = f(x-a)$$;(ii) $$y = f(-x)$$;(iii) $$y = |f(x)|$$;imgAnswers:
241: None
242: None
243: None

ID: 200003003003
Content:
Show that the following lines do not intersect: r = -2i + j + 9k + s(2i + 5j + 4k), r = 11i + 8j + 3k + t(3i - j + 5k). ;State whether the lines are parallel or skew, giving a reason for your answer.Answers:
244: 
245: The lines are "skew".

ID: 200003003004
Content:
By considering the derivative as a limit, show that the derivative of \[x^3\] is \[3x^2\].Answers:
246: 

ID: 200003003005
Content:
By means of a suitable substitution, or otherwise, evaluate \[\int_{1}^{3} \frac{x}{(2x-1)^3} dx\].Answers:
247: \int_1^3 \frac{x}{(2x-1)^3} dx = "\frac{8}{25}".

ID: 200003003006
Content:
Let \[y = \ln (1 + \tan x)\].;(i) Show that \[\frac{\mathrm{d}^2 y}{\mathrm{d} x^2} = -1\] when $$x = 0$$.;(ii) Find Maclaurin's series for $$y$$ up to and including the term in \[x^2\].Answers:
248: 
249: Maclaurin's series for y = "x-\frac{1}{2}x^2".

ID: 200003003007
Content:
The angles A and B are such that \[A + B = 120^{\circ}\], \[\cos A + \cos B = \frac{1}{\sqrt{2}}\]. Show that \[\cos (\frac{A - B}{2}) = \frac{1}{\sqrt{2}}\]. ;Hence find the possible values of A and B, given that \[0^{\circ} < A < 120^{\circ}\] and \[0^{\circ} < B < 120^{\circ}\].Answers:
250: 
251: A = (1)"15"^{\circ} or (1)"105"^{\circ}.
252: B = "105"^{\circ} for A(1) or "15"^{\circ} for A(2).

ID: 200003003008
Content:
A ten-digit number is formed by writing down the digits 0, 1, 2, 3, 4, 5, 6, 7, 8, 9 in some order. No number is allowed to start with 0. Find how many such numbers are odd.Answers:
253: Total no. of such 10-digit numbers = "1612800".

ID: 200003003009
Content:
The region R is bounded by the curve \[y = (x^3 + 1)^{\frac{1}{4}}\], the lines x = 1 and x = 3, and the x-axis. Express as an integral the volume V of the solid formed when R is rotated completely about the x-axis. Use the trapezium rule with 4 intervals to estimate V, giving your answer correct to 2 decimal places.Answers:
254: V = "\pi\int_1^3(x^3 +1)^{\frac{1}{2}}" dx.
255: Estimate V = "19.67" units^3.

ID: 200003003010
Content:
Find;(i) \[\int \cos^2 3x dx\] ;(ii) \[\int \tan^2 2x dx\].Answers:
256: \int \cos^2 3x dx = "\frac{1}{2}x + \frac{1}{12} \sin{6x}"+ c.
257: \int \tan^2 2x dx = "\frac{1}{2} \tan{2x} - x"+ c.

ID: 200003003011
Content:
It is given that x and y satisfy the equation \[\tan^{-1} x + \tan^{-1} y + \tan^{-1} (xy) = \frac{7}{12}\pi\].;(i) Find the value of y when x = 1.;(ii) Express \[\frac{\mathrm{d}}{\mathrm{d} x} \tan^{-1} (xy)\] in terms of x, y and \[\frac{\mathrm{d} y}{\mathrm{d} x}\].;(iii) Show that, when x = 1, \[\frac{\mathrm{d} y}{\mathrm{d} x} = -\frac{1}{3} - \frac{1}{2\sqrt{3}}\].Answers:
872: y = "\frac{1}{\sqrt{3}}", when x = 1.
873: \frac{d}{dx}\left[ {{\tan }^{-1}}\left( xy \right) \right] = "\frac{1}{(xy)^{2}+1}"\left( x\frac{dy}{dx}+y \right).
874: \frac{d}{dx}\left[ {{\tan }^{-1}}x+{{\tan }^{-1}}y+{{\tan }^{-1}}\left( xy \right) \right]=\frac{d}{dx}\left[ \frac{7}{12}\pi  \right];"\frac{1}{{{x}^{2}}+1}"+"\frac{1}{{{y}^{2}}+1}"\frac{dy}{dx}+\frac{1}{{{\left( xy \right)}^{2}}+1}\left( x\frac{dy}{dx}+y \right)=0;\frac{1}{{{x}^{2}}+1}+\left[ \frac{1}{{{y}^{2}}+1}+\frac{x}{{{\left( xy \right)}^{2}}+1} \right]\frac{dy}{dx}+\frac{y}{{{\left( xy \right)}^{2}}+1}=0;When:x=1,y=\frac{1}{\sqrt{3}};\Rightarrow \frac{1}{\left( 1 \right)+1}+\left[ \frac{1}{\left( \frac{1}{3} \right)+1}+\frac{\left( 1 \right)}{\left( \frac{1}{3} \right)+1} \right]\frac{dy}{dx}+\frac{\left( \frac{1}{\sqrt{3}} \right)}{\left( \frac{1}{3} \right)+1}=0 ;"\frac{1}{2}"+"\frac{3}{2}"\frac{dy}{dx}+\frac{3}{4\sqrt{3}}=0;\frac{3}{2}\frac{dy}{dx}=-\frac{1}{2}-\frac{3}{4\sqrt{3}};\therefore \frac{dy}{dx}=-\frac{1}{3}-\frac{1}{2\sqrt{3}}

ID: 200003003012
Content:
(i) Prove that \[\cos 3\theta = 4\cos^3 \theta - 3\cos \theta\].;(ii) Use the substitution \[x = 6\cos \theta\] to find the three roots of the equation \[x^3 - 27x + 18 = 0\], giving each root correct to 3 significant figures.;(iii) Find a similar substitution which could be used to solve the equation \[x^3 - 12x + 2 = 0\]. [You are not required to solve this equation.] Answers:
258: 
259: The three roots of the equation are "4.82" and "0.678" and "-5.50".
260: A similar and suitable substitution is x = "4 \cos{\theta}" .

ID: 200003003013
Content:
The expression \[\frac{x^2}{9-x^2}\] can be written in the form \[A + \frac{B}{3 - x} + \frac{C}{3 + x}\].;(i) Find the values of the constants A, B and C.;(ii) Show that \[\int_{0}^{2} \frac{x^2}{9 - x^2} dx = \frac{3}{2}\ln 5 - 2\].;(iii) Hence find the value of \[\int_{0}^{2} \ln (9 - x^2) dx\], giving your answer in terms of ln 5.Answers:
261: A = "-1".
262: B = "\frac{3}{2}".
263: C = "\frac{3}{2}".
264: 
265: \int_0^2 \ln(9-x^2) dx = "5*\ln{5} -4".

ID: 200003003014
Content:
a) The first term of a geometric progression is 3 and the common ratio is r, where |r| < 1. The sum of the first three terms of the progression is \[\frac{8}{9}\] of the sum of the first six terms. Find the sum to infinity.;;b) All the terms of the arithmetic progression \[u_{1}, u_{2}, u_{3},..., u_{n},...\] are positive. Use induction to prove that, for  \[n \geq 2\], \[\frac{1}{u_{1}u_{2}} + \frac{1}{u_{2}u_{3}} + \frac{1}{u_{3}u_{4}} + ... + \frac{1}{u_{n-1}u_{n}} = \frac{n - 1}{u_{1}u_{n}}\].Answers:
266: Sum to infinity = "6".
267: 

ID: 200003003015
Content:
The base ABCD of a cube lies in a horizontal plane. The edges AE, BF, CG and DH are vertical, and EFGH is the top of the cube. Each edge of the cube is of length 1 unit. Calculate;(i) the angle which EC makes with the face CDHG,;(ii) the length FX, where X is the foot of the perpendicular from F to EC,;(iii) the angle between the planes CEF and CEH.Answers:
268: \angle ECH = "35.3" ^{\circ}.
269: FX = "\sqrt{\frac{2}{3}}".
270: \angle FXH = "120" ^{\circ}.

ID: 200003003016
Content:
a) Solve the inequality \[x^2 - 9 \geq (x + 3) (x^2 -3x + 1)\].;;b);(i) Expand \[\frac{1 - x^2}{\sqrt[3]{(1 + 3x)}}\] in ascending powers of x up to and including the term in \[x^3\].;(ii) State the set of values of x for which the series expansion is valid.;(iii) Write down the equation of the tangent at the point (0, 1) on the curve \[y = \frac{1 - x^2}{\sqrt[3]{(1 + 3x)}}\]Answers:
271: x \leq "-3" or x = "2".
272:  \[\frac{1 - x^2}{\sqrt[3]{(1 + 3x)}}\] = "1-x+x^2-\frac{11}{3}x^3", up to and inluding the term in x^3.
273: The expansion is valid for |x| < "\frac{1}{3}".
274: At the point (1), equation of tangent is y = "1-x".

ID: 200003003017
Content:
img;The diagram shows the circle, centre O and radius r, with equation \[x^2 + y^2 = r^2\]. The points A, B, C, D on the circle form a rectangle with sides parallel to the axes. Angle AOD = angle BOC = \[2\alpha\]. The region bounded by the line AB, the line DC and the circular arcs BC and AD is rotated about the x-axis to form a solid of revolution S.;(i) Show that the volume obtained by rotating the shaded part of the region about the x-axis is \[\frac{1}{3} \pi r^3 (\cos^3 \alpha - 3\cos \alpha + 2)\].;(ii) Show that the total volume of S is \[\frac{4}{3} \pi r^3 (1 - \cos^3 \alpha)\].;(iii) Given that the volume of S is half the volume of a sphere of radius r, find the value of \[\alpha\].Answers:
275: 
276: 
277: Value of \alpha = "37.5" { }^{\circ}.

ID: 200003003018
Content:
a) Given that the angle between the vectors \[-i + 3j + \lambda k\] and 2i + 2j + k is \[\cos^{-1} \frac{1}{3}\], find the value of the constant \[\lambda\].;;b) OABC is a tetrahedron (i.e. a solid having four triangular faces). The position vectors of A, B and C with respect to O are a, b and c respectively. The mid-points of BC, CA and AB are X, Y and Z respectively.;(i) The point G on AX is such that \[AG = \frac{2}{3} AX\]. Find the position vector of G. Deduce that AX, BY and CZ are concurrent (i.e. that they all go through the same point).;(ii) Given that L is the mid-point of OA and that H is the mid-point of LX, find the position vector of H. The mid-point of OB is M and the mid-point of OC is N. What statement about the lines LX, MY and NZ can be made?Answers:
278: \lambda = "-\frac{3}{4}".
279: \overrightarrow{OG} =- "\frac{1}{3}a+\frac{1}{3}b+\frac{1}{3}c".
280: G is the "mid-point" of face ABC, therefore AX, BY and CZ are concurrent and go through G .
281: \overrightarrow{OH} = "\frac{1}{4}a+\frac{1}{4}b+\frac{1}{4}c".
282: H is the "centroid" of OABC, therefore LX, MY and NZ are concurrent and go through H.

\end{document}
