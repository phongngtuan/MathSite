\documentclass{article}
\begin{document}
ID: 199501001001
Content:
)The function f is defined as \[f:x \mapsto \frac{\left (3x  \right )}{\left ( x+2 \right )},x\neq 2 \];i) Evaluate \[f^{2}\left ( 4 \right )\];ii) Obtain an expression, in terms of x, for \[f^{-1}\].Answers:
884: Test
885: Test2
887: testing
889: Test4

ID: 199501001002
Content:
Find the equation of the normal to the curve \[y=6-(x-2)^4\] at the point on the curve where \[x=1\].Answers:
886: Test2

ID: 199501001003
Content:
Solve the simultaneous equations ;\[y+4 = x\] and ;\[y^2+17 = 2x^{2}\] (Note: Please enter the smaller value of x first in the answer space.)Answers:

ID: 199501001004
Content:
Calculate the distance of the point A(5, 8) from M, the mid-point of the line joining the points B(-1, 10) and C(3, 2). ;Show that AM is perpendicular to BC.; Calculate the area of triangle ABC.Answers:

ID: 199501001005
Content:
Prove the identity \[\left ( \cot A - \tan A \right )\cos A -=\csc A - 2\sin A\].Answers:

ID: 199501001006
Content:
Given that the coefficient of \[x^{2}\] and \[x^{3}\] in the expansion of \[\left ( 3+x \right )^{20}\] are a and b respectively, evaluate \[\frac{a}{b}\]Answers:

ID: 199501001007
Content:
The two variables x and y are related by the equation \[y = 3x - \frac{4}{x}\];;i) Obtain an expression for \[\frac{\mathrm{d} y}{\mathrm{d} x}\] in terms of x;;ii) Hence find the approximate increase in y as x increases from 2 to 2 + p, where p is small.Answers:

ID: 199501001008
Content:
(a)	Find the range of values of x for which \[x\left ( x+5 \right )\geqslant -6\] (Note: Please enter the smaller value of x first in the answer space.);(b) Given that the straight line y = c-3x  does not intersect the curve xy = 3, find the range of values of c.Answers:

ID: 199501001009
Content:
img;The diagram shows part of a circle, center O, of radius 10 m. The tangents at the points A and B on the circumference of the circle meet at the point P and the angle AOB is 0.8 radians. Calculate;i)	the length of the perimeter of the shaded region.;ii)	The area of the shaded region.Answers:

ID: 199501001010
Content:
img;In the diagram the position vectors of points A and B relative to O are a and b respectively. The lines AB and OC intersect at D and;\[ \vec{BC} = \frac{3}{2} a\] ;Given that \[\vec{OD} = p\vec{OC}\] and that \[\vec{DB} = q\vec{AB}\] express \[\vec{OD}\]  and  \[\vec{OC}\] in terms of;;i) p, a, b,;;ii) q, a, b,;;and hence evaluate p and q.Answers:

ID: 199501001011
Content:
(a) Given that\[y = \frac{16}{x^{2}} + \frac{x^{3}}{3}\] find the stationary value of y and determine whether it is a maximum or a minimum. ;img; (b)The diagram shows a piece of wire bent to form the perimeter of the plane shape PQRSTP which consists of a semi-circle and a rectangle. The radius of the semi-circle is x cm and QR = ST  y cm. The length of the wire is 100 cm and the area of the shape PQRSTP is \[A cm^{2}\].;Show that \[A = 100x-2x^{2}-\frac{\left ( \pi x^{2} \right )}{2}\];;Given that x may vary, show that the maximum value of A is approximately 700.Answers:

ID: 199501001012
Content:
(a)	Calculate the volume of the solid of revolution formed when the region enclosed between the curves \[y = x^{2}\] and \[y^{2}=x\]  is rotated through \[360^{\circ}\]  about the x-axis. ;img; (b) The diagram shows the shaded region bounded by the curve\[y = \left ( x-1 \right )^{2}\] and the lines x+y=3 and x =3.Find;;(i) the coordinates of the point A,;;(ii) the area of the shaded region.;Answers:

ID: 199501001013
Content:
A particle moves in a straight line so that t seconds after leaving a fixed point O, its velocity, \[v ms^{-1}\] is given by \[v=6+pt-t^{2}\] where p is a constant. The particle comes to an instantaneous rest at the point A, 6 seconds after leaving O.;
i) Evaluate p.;
ii) Calculate the distance OA.;
iii) Calculate the acceleration of the particle at A.
Answers:

ID: 199501001014
Content:
A curve is such that \[\frac{\mathrm{d} y}{\mathrm{d} x}=\frac{3}{4}-kx\] where k is a positive constant. Given that the tangents to the curve at the points where x = -1 and 1 are perpendicular, find the value of k.;;Given also that the curve passes through the point (4, 0), find the equation of the curve.Answers:

ID: 199501001015
Content:
Find all the angles between \[0^{\circ}\]and \[360^{\circ}\] which satisfy the equation;;i) \[10 \sin x \cos x = \cos x\] ;;ii) \[5\tan ^{2y}=5\tan y+3\sec ^{2y}\];;iii)	\[\sec \left ( \frac{1}{2z}+107^\circ  \right )=-2\];(Note: Please enter the answers in ascending order)Answers:

ID: 199501001016
Content:
The position vectors of points A and B relative to an origin O are j and -6i ? j respectively.;Using vector methods, find;
i) The value of p such that pi + 2j is perpendicular to \[\vec{AB}\];
ii)\[\angle OBA\];
The point C lies on AB and is such that \[\vec{AC}=k\vec{AB}\] where \[k\neq 0\];Given that \[\vec{OC}\] is a unit vector, evaluate k and obtain the position vector of C relative to O.
Answers:

ID: 199501001017
Content:
Solutions to this question by accurate drawing will not be accepted. ;img; The diagram shows a trapezium ABCE consisting of a parallelogram ABCD and a triangle ADE, where angle \[\angle DAE = 90^{\circ}\]; The equation of the line CDE is 5y = x + 13 and the equation of CB is y = 3x ? 3.  Calculate the coordinates of C.;Given also that the diagonals of the parallelogram intersect at M (5, 5), calculate the coordinates of A, B, D and E.Answers:

ID: 199501001018
Content:
Functions f and g are defined by;
\[f:x \mapsto 3x-1 \];\[g:x \mapsto\frac{2}{x},x\neq 0\];
Where \[x\in \mathbb{R}\];
(i)	Find an expression, in terms of x, for fg and for gf.;
(ii)	Find the values, to two decimal places, of x for which fg(x) = gf(x).
; (Note: Please enter the smaller value of x first in the answer space)Answers:

ID: 199501001019
Content:
Draw on graph paper the graph of \[y=\left | 5-3x \right |+2\]  for \[0\leqslant x\leqslant 4\] Find the range of values of x for which;
(i)	\[y\leqslant 4\];
(ii)\[y\leqslant 3\]Answers:

\end{document}
