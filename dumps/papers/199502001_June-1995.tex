\documentclass{article}
\begin{document}
ID: 199502001001
Content:
a) The remainder when \[x^{3}-3x^{2}-2x+a \] is divided by\[x + 2 \] is twice the remainder when it is divided by \[x ? 3\].  Find the value of a;(b) Factorise completely the expression \[4x^{3}-13x-6\]  and hence solve the equation \[2\left ( 2x^{2}-\frac{3}{x} \right )=13\];(Note: Please enter the smaller value of x first in the answer space);(c) Find the value of p and of q for which \[x^{2}-2x-3\]  is a factor of \[2x^{3}+px^{2}-12x+q\]Answers:

ID: 199502001002
Content:
a) An arithmetic series has first term 7 and second term 6.8. Find the sum of the first 50 terms; b) A funding body gives a grant to a sports organisation each year from 1995. The amount of grant in 1995 is \[\it\unicode{xA3} 10000\] and thereafter is 90% of the grant in each preceding year. Calculate;i) the year in which the value of the grant first falls below \[\it\unicode{xA3} 20000\];
ii) the total amount paid in grants to the sports organisation during the years from 1995 to 2004 inclusive,;
iii)the total amount paid in grants to the sports organisation during the years from 2005 to 2014 inclusive,;
iv) the number of years the grant has been paid when the total amount paid first exceeds \[\it\unicode{xA3} 90000\],Answers:

ID: 199502001003
Content:
(a) When the graph of \[y^{2}\]  against \[\sqrt{x}\]  is drawn, a straight line is obtained which has gradient 1.5 and which passes through the point (10, 16). Determine the relationship connecting x and y and evaluate x when y = 5;(b);img;The table shows experimental values of two variables x and y;It is known that x and y are related by the equation of the form form \[y = px^{2}+ q\sqrt{x}\] where p and q are constants. ;Draw the graph of \[\frac{y}{\sqrt{x}}\]  against \[x\sqrt{x}\]  for the given data and use it;(i) to evaluate p and q;(ii) to estimate the value of x when \[2y = 43\sqrt{x}\]Answers:

ID: 199502001004
Content:
a) Differentiate, with respect to x;i) \[\tan ^{2}x\];ii)\[\frac{x}{\left ( x^{2}-3 \right )}\];b) Find the equation of the tangent to the curve \[x^{2}+5y^{2}+2xy = 4 \]  at the point (1, -1),;;c) Find the x-coordinates, for \[0<x<\frac{\pi }{2}\] of the stationary point on the curve \[y = e ^{3x} cosx\].Answers:

ID: 199502001005
Content:
(a) Given that \[\frac{\left ( \cos \left (A-B  \right ) \right )}{\left ( \cos \left ( A+B \right ) \right )}= \frac{7}{3}\] evaluate \[\tan A\tan B\];img;(b)In the diagram OP = 12 cm, OQ = 5 cm and \[\angle OBP=90^{\circ}\];The angles OPB and AOQ are each equal to \[\theta\] where \[\theta\] is a variable and ;\[0^{\circ}< \theta < 90^{\circ}\];(i)Obtain the values of R and \[\alpha\] for which \[AB = R\sin \left ( \theta -\alpha  \right )\] Hence find;(ii)The value of \[\theta\] for which AB = 3 cm,;(iii)The range of values of \[\theta \]  for which A is between O and BAnswers:

ID: 199502001006
Content:
(a) \[\int\sqrt{\left (3x-1  \right )}\mathrm{d} x\]; (b) Evaluate, to two decimal places,;(i) \[\int_{1}^{5}\frac{\left ( 4 \right )}{\left ( 3x-1 \right )}\mathrm{d} x\];(ii) \[\int_{0}^{1}\sin x\mathrm{d} x\];(c) Find, to two decimal places, the volume obtained when the region bounded by the curve \[y=1+\frac{1}{x}\] the x-axis and the lines x = 2  and x = 3, is rotated through \[360^{\circ}\] about the x-axis.Answers:

ID: 199502001007
Content:
(a) Given that \[2^{x}4^{y}= 128\] and that \[\ln \left (  4x ? y\right )=\ln 2+\ln 5\], calculate the value of x and of y.;(b) Solve the equation;(i) \[\lg \left (  1 ? 2x\right )? 2\lg x= 1 ? \lg \left (2 ? 5x  \right )\];(ii)\[3^{\left ( y+1 \right )}=4^{y}\]; (c) Using graph paper, draw the curve \[y =\ln \left (  x + 1\right )\] for \[0<=x<=4\], taking values of x at unit intervals.By drawing appropriate straight line, obtain and approximate value for the positive root of the equation \[x ? 2\ln \left ( x + 1 \right )= 0\]Answers:

ID: 199502001008
Content:
A curve represented parametrically by  \[x = 1 + 3t, y = t^{2}+ 7t\] ;i) O is the origin and X is a point on the curve such that the gradient of OX is 1.5. Find the two possible values of the parameter t at X.(Note: Please enter the smaller value of t first in the answers space) ;ii) Obtain an expression for \[\frac{\mathrm{d}y }{\mathrm{d} x}\] ;iii) P is a point on the curve with parameter -2 and Q is the point on the curve with parameter -5.The tangents to the curve at P and Q meets at the point R. Find the coordinates of R and the area of the triangle PQR.  ;iv) Obtain the cartesian equation of the curve.Answers:

ID: 199502001009
Content:
(a)	A particle P, moving in a straight line, passes a point A with a constant speed of \[8 ms^{-1}\] Three seconds later a second particle Q, traveling in the same direction as P, passes A with a constant speed of \[10 ms^{-1}\] The particle Q overtakes P at a point B, where AB = s metres, t seconds after P left A. Draw, on the same diagram, the distance-time graphs for the motions of P and Q from A to B, and evaluate s and t.;(b)	A car passes a service depot on a motorway with a speed of \[20 ms^{-1}\]  and a constant retardation of \[0.2 ms^{-2}\] At the same instant a motor-cyclist leaves the service depot, starting from rest, and with a constant acceleration of \[0.6 ms^{-2}\] The motor-cyclist passes the car at time T seconds after leaving the service depot at a point whose distance from the service depot is S metres. Find;(i) the value of T and of S,;(ii) the speed of each vehicle at time T seconds.;(iii) Sketch, on the same diagram, the velocity-time graphs of the motor-cyclist and the car over the time interval of T seconds.;(iv)	Find the time, after leaving the service depot, when the speeds of the vehicles are the same.Answers:

ID: 199502001010
Content:
A particle, P, is dropped from rest at A and strikes a horizontal plane at B, where AB = 20 m. At the same time that P is released, a second particle, Q is projected with a speed of \[15ms^{-1} \]  from a point D, level with A, directly down a smooth plane inclined at an angle of \[30^{\circ}\] to the horizontal. The point E on the plane is at the same level as B .;i) Show that P and Q arrive simultaneously at B and E respectively and find their speeds at these points.;The particle P rebounds from B at \[12ms^{-1} \] and rises to C, the highest point reached after leaving B .;ii) Calculate the distance BC and the time taken by P to go from B to C.;The particle Q continues to move down the plane without interruption and when P is at C, the particle Q is at F.  ;iii) Calculate the speed of Q at F and the distance EF.
Answers:

ID: 199502001011
Content:
An archer shoots an arrow with a speed of \[25ms^{-1} \] in a direction inclined at an angle\[\alpha\] to the horizontal, where \[\sin \alpha  = 0.28\] Find the vertical and horizontal components of the initial velocity.;The arrow strikes the horizontal plane through the point of projection. Find its time of flight and the horizontal distance it travels. ;The diagram shows the path of a second arrow, shot from A with the same speed and direction as the first arrow. This second arrow strikes a target at right angles at B . The target is inclined at an \[\angle \beta \] to the horizontal as shown, where \[\tan \beta =4\] ; Calculate ;i) the vertical component of the velocity of the arrow at B,;ii) the speed of the arrow when it hits the target,;iii) the time of flight from A to B,;iv) the horizontal and vertical distances of B from A. Answers:

ID: 199502001012
Content:
img;(a) The diagram shows a river flowing at \[1.5 ms^{-1}\] between parallel banks.; A rower, whose speed through still water is \[2 ms^{-1}\] rows from A to the point B immediately opposite A. The rower points the boat in a direction inclined at an \[\angle \alpha \] to the bank. Find the value of \[\alpha \] Given that AB = 120 m, find the time taken for the crossing, to the nearest second.;(b) At a particular instant particles P and Q are 36 cm apart and are moving in a horizontal plane with constant speeds and directions as shown in the diagram.;img; Given that they collide 3 seconds later, find;(i) the velocity, in magnitude and direction, of P relative to Q,;(ii) the value of v and of \[\theta\] (Note: Please enter the smaller value of \[\theta\] first in the answer space);(c) A car is traveling at \[80 kmh ^{-1}\] in a direction due north, and the wind is blowing at \[30 kmh ^{-1}\] from a direction \[240^{\circ}\].  Find the angle that a streamer, tied to the car roof, makes with the direction of motion of the car.Answers:

ID: 199502001013
Content:
a) Particles A and B, of mass 0.5 kg and 0.3 kg respectively, are connected by a light extensible string passing over a smooth fixed peg. The particles are held at the same height , 1.8 m, above a horizontal floor and then released. Calculate 
;i) the acceleration of the particles and the tension in the string during the motion,;ii) the time taken for A to reach the floor and its speed on impact.;After A strikes the floor, B rises freely under the action of gravity,;iii) Calculate the greatest height above the floor reached by B ;b) A particle of mass 3 kg slides down a line of greatest slope of a rough plane inclined at an angle of \[27^{\circ}\]  to the horizontal. Given that the coefficient of friction between the particle and the plane is 0.2, calculate the acceleration of the particle down the plane.
Answers:

ID: 199502001014
Content:
A sledge, of mass 20 kg , is held by a cable on a rough plane, inclined at an angle \[\alpha\] to the horizontal, as shown in the diagram. The coefficient of friction between the sledge and the plane is 0.4 and \[\tan \alpha =0.75\];i) Find the tension in the cable when the sledge is about to slip up the plane.;The cable is attached to a winch which pulls the sledge up the plane. Starting from rest at A, the sledge acquires a speed of \[3ms^{-1} \] after being pulled 15 m up the plane to B .;ii) Calculate the total work done by the winch in pulling the sledge from A to B .;The winch is working at 558 W when the speed of the sledge is \[3ms^{-1} \];iii) Calculate the acceleration of the sledge at this instant.Answers:

ID: 199502001015
Content:
a) A bullet, of mass 120 grams, is fired horizontally into a stationary block of mass 2.88 kg. The block is free to move on a smooth horizontal plane. The bullet strikes the block with a speed of \[150ms^{-1} \] and is then embedded in the block. Find ;i) the final speed of the block and the bullet combined,;ii) the impulse exerted by the block on the bullet.;    This impulse occurs over a period of 0.02 seconds.;iii) Find the average resistance if the block to the bullet.;b) Particle A, of mass 0.21 kg, and b, of mass 0.7 kg, are suspended from a fixed point by two light inextensible strings of equal length, as shown in the diagram. Particle A is drawn aside, keeping the string taut, and released from rest at a vertical height of 1.25 m above B, which is at rest.;i) Show that A strikes B with a speed of \[5ms^{-1} \] Immediately after the collision the particles move in opposite directions, each with a speed of \[vms^{-1} \] Calculate  ;ii) the value of v,;iii) the height to which each particle rises after the collision.;The particle collide for a second time and A rebounds with a speed of \[3.6ms^{-1} \].;iv) Find the magnitude and direction of the velocity of B after this second collision.Answers:

ID: 199502001016
Content:
a) Four horizontal forces acting on a on a particle are represented graphically, in magnitude and direction, by the lines joining O to A(2, 1), B(5, -2), C(-3, -3) and D(-1, 2). Given that the resultant of the four forces is to be represented by the line OE, find the coordinates of E.;b) The resultant, R, of the force P, of magnitude 10 N, and the force Q, of magnitude 15 N is perpendicular to P. Find the magnitude of R and the angle between the directions of P and Q.;c) The diagram shows a light extensible string, BAC, attached to two fixed points B and C, passing through a smooth ring A which is suspended by the string. The mass of A is 0.5 kg. The ring is drawn to one side by a horizontal force P N, so that system is in equilibrium with the two parts of the string making angles of \[25^{\circ}\]  and \[40^{\circ}\]  with the vertical.;Find the tension in the string and the value of P.Answers:

\end{document}
