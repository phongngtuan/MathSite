\documentclass{article}
\begin{document}
ID: 200004003001
Content:
Particle MathematicsAnswers:

ID: 200004003002
Content:
Particle MathematicsAnswers:

ID: 200004003003
Content:
Particle MathematicsAnswers:

ID: 200004003004
Content:
Particle MathematicsAnswers:

ID: 200004003005
Content:
Particle MathematicsAnswers:

ID: 200004003006
Content:
A computer generates a random variable X whose probability distribution is given in the following table. img ;Show that Var(X) = 4. Find E($$X^4$$) and Var($$X^2$$). Two independent observations of X are denoted by $$X_1$$ and $$X_2$$. Show that P($$X_1$$+$$X_2$$ =6) = 0.2, and tabulate the probability distribution of $$X_1$$+ $$X_2$$. The sum of 100 independent observations of X is denoted by S. Describe fully the approximate distribution of S.  Answers:
283: 
284: E(X^4) = "598.4".
285: Var(X^2) = "198.4".
286: 
287: Since the no. of observations is large, Central Limit Theorem \Rightarrow S ~ "N"("\mu","\sigma^{2}") approximately.

ID: 200004003007
Content:
img ;The continuous random variable X has probability density function given by ;f(x)= img;A sketch of the grah of the probability density function is given above. Show that $$P(X \leq x) = \frac{1}{2} + \frac{1}{4}x, for 0 < x \leq 2$$, and find a similar expression for $$P(X<=x)$$, for $$-1 \leq x \leq 0$$. Show that $$E(X) = \frac{1}{4}$$, and state the value of $$E(X - \frac{1}{4})$$. Find $$E(\sqrt{X+1})$$. 10 independent observations of X are taken. Find the probability that 8 of these observations are less than 1. Answers:
288: None
289: 
290: 
291: E(X-\frac{1}{4}) = "0".
292: E(\sqrt{X+1}) = "(\frac{1}{2}*\sqrt{3})+\frac{1}{6}".
293: P(8 out of 10 observations are less than 1) = "0.282".

ID: 200004003008
Content:
A large number of tickets are sold in a lottery. Each ticket can win either a small prize or a large prize, but no ticket can win two prizes. 10% of tickets win a small prize and 0.1% of tickets win a large prize.;a) If Charlie has 20 tickets, find the probability that    ;(i) he wins at least 1 prize,   ;(ii)) he wins at most 3 prizes.;;b) If Mary has 400 tickets, use suitable approximations to find the probability that ;(i) she wins at most 35 small prizes,;(ii) she wins at most 2 large prizes. Answers:
294: P(he wins at least 1 prize) = "0.881".
295: P(he wins at most 3 prizes) = "0.863".
296: P(she wins at most 35 small prizes) = "0.227".
297: P(she wins at most 2 large prizes) = "0.992".

ID: 200004003009
Content:
The mass of the fruit in a randomly chosen tin has a normal distribution with mean 278 g and standard deviation 9.2 g. the total mass of the fruit and the juice has an independent normal distribution with mean 432 g and standard deviation 7.8 g. Giving your answers correct to 3 places of decimals, find the probability that;(i) the mass of the juice in a randomly chosen tin is greater than 170 g,;(ii) the total mass of fruit in 4 randomly chosen tins is less than 1112 g. The sugar content of a randomly chosen tin has a normal distribution. A random sample of 15 tins is taken and the sugar content, x g, of each tin is measured. The results are summarized by $$\Sigma x= 748$$, $$\Sigma x^2 = 56243$$. Find a symmetric 80% confidence interval for the mean sugar content of a tin. Answers:
298: P(mass of the juice in a randomly chosen tin is greater than 170 g) = "0.0923".
299: P(total mass of fruit in 4 randomly chosen tins is less than 1112 g) = "0.5".
300: Symmetric 80% confidence interval for the mean sugar content of a tin = ["37.7","62.0"].

ID: 200004003010
Content:
a) A random variable X is known to have a normal distribution with variance 36. The mean of the distribution of X is denoted by $$\mu$$. A random sample of 50 observations of X has mean 20.2, Test at the 1% significance level, the null hypothesis $$\mu$$ = 22 against the alternative hypothesis $$\mu$$ < 22.;;b) A political party elects a new leader. Six months later a poll of 850 electors is asked the name of the leader of the party. Only 124 know her name. Find a symmetric 95% confidence interval for the proportion of the population who know the leader's name. Give the end-points of the interval correct to 3 places of decimals. After a publicity campaign another poll is taken of 953 electors and 263 know the leader's name. Test, at the 5% significance level, whether the proportion of the population who know the leader's name now exceeds 25%.Answers:
301: "do not reject" (Do not reject/Reject) H_0 and conclude that, at the 1% level of significance there is " insufficient" (sufficient/insufficient) evidence that the mean \mu is less than 22.
302: Symmetric 95% confidence interval = ["0.122","0.170"].
303: "reject" (Do not reject/Reject) H_0 and conclude that, at the 5% level of significant, there is " sufficient" (sufficient/insufficient) evidence that p > 0.25..

ID: 200004003011
Content:
In a probability experiment, three containers have the following contents. A jar contains 2 white dice and 3 black dice. A white box contains 5 red balls and 3 green balls. A black box contains 4 red balls and 3 green balls. One die is taken at random from the jar. If the die is white, two balls are taken from the white box, at random and without replacement. If the die is black, two balls are taken from the black box, at random and without   replacement. Events W and M are defined as follows. W: A white die is taken from the jar. M: One red ball and one green ball are obtained. Show that  $$P( M|W ) = \frac{15}{28}$$. Find, giving each of your answers as an exact fraction in its lowest terms,;(i)  $$P( M \cap W )$$;(ii) $$P(W | M)$$;(iii)  $$P( W \cup M )$$;All the dice and balls are now placed in a single container, and four objects are taken at random, each object being replaced before the next one is taken. Find the probability that one object of each colour is obtained. Answers:
304: P(one object of each colour taken) = "\frac{243}{5000}".
305: P(M \cap W) = "\frac{3}{14}".
306: P(W|M) = "\frac{15}{13}".
307: P(M \cup W) = "\frac{26}{35}".

ID: 200004003012
Content:
Sketch the following graphs on a single diagram, stating the x-coordinates of all intersections with the x-axis and the equations of any asymptotes.;(i)  $$y = x( x^2  - 4 )$$.;(ii)  $$y = \frac{x + 1}{(x - 1)^2} $$. ;Use linear interpolation once on the interval [-1, 0] to obtain an approximation to a root of the equation  $$x( x^2  - 4 )$$. The Newton-Raphson method is to be used to find an approximation to another root of the equation. Use this method, with x = 2 as a first approximation, to obtain a second approximation to this root, giving your   answer correct to 2 places of decimals.  Answers:
308: None
309: None
310: Linear interpolation on the root of the equation x( x^2  - 4 ) = "-0.0769".
311: Newton-Raphson method on another root of the equation= "7".
312: Second approximation to the root use for Newton-Raphson method = "2.43".

ID: 200004003013
Content:
Solve the following differential equations, given in each case that y = 1 when x = 0. Give your answers in a form expressing y in terms of x. ;;a)$$\frac{dy}{dx} = 2x + 2xy^2 $$. ;;b)$$( 1 + x^2 )\frac{dy}{dx} + 4xy = \frac{4x}{( 1 + x^2  )^3}$$.Answers:
313: y = "\tan{(x^2 + \frac{\pi}{4})}".
314: y = "\frac{2}{(1+x^2)^3} + \frac{3}{(1+x^2)^2}".

ID: 200004003014
Content:
a) The complex number w has modulus  $$\sqrt 2 $$ and argument  $$ - \frac{3}{4}\pi $$, and the complex number z has modulus 2 and argument  $$ - \frac{1}{3}\pi $$. Find the modulus and argument of wz, giving each answer exactly. By first expressing w and z in the form x + iy, find the exact real and imaginary parts of wz. Hence show that  $$\sin (\frac{1}{12}\pi ) =\frac{\sqrt 3  - 1}{2\sqrt 2}$$.;;b) On a single Argand diagram, sketch the following loci.;(i) $$|z - 2i| = 4$$;(ii)  $$\arg ( z + 2 ) = \frac{1}{4}\pi $$. ;Hence or otherwise find the exact value of z satisfying both equations in parts (i) and (ii).  Answers:
315: |wz| = "2*\sqrt{2}".
316: arg(wz) = "\frac{11}{12}*\pi".
317: w = "-1-i".
318: z = "1-\sqrt{3}i".
319: Re(wz) = "-1-\sqrt{3}".
320: Im(wz) = "\sqrt{3}-1".
321: 
322: None
323: None
324: Exact value of z satisfying both (i) and (ii) = "2\sqrt{2}+i[2+2\sqrt{2}]".

ID: 200004003015
Content:
Relative to the origin O, the points A, B and C have position vectors 5i + 4j + 10k, -4i + 4j - 2k, -5i + 9j + 5k, respectively.;(i) Find the cartesian equation of the line AB.;(ii) Find the length of the projection of  $$vec(AC) $$ onto the line AB.;(iii) Hence or otherwise find the perpendicular distance from C to the line AB, and the position vector of the foot N of the perpendicular from C to the line AB.;(iv) The point D lies on the line CN produced and is such that N is the mid-point of CD. Find the position vector of D. Answers:
325: Cartesian equation of the line AB = "4".
326: Length of projection of \overrightarrow{AC} onto AB = "10" units.
327: Perpendicular distance from C to AB = "5*\sqrt{2}" units.
328: \overrightarrow{ON} = "-i+4j+2k".
329: \overrightarrow{OD} = "3i-j-k".

\end{document}
