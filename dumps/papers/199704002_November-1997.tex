\documentclass{article}
\begin{document}
ID: 199704002001
Content:
a) Evaluate $$\frac{0.2031}{\sqrt[3]{17.95} + 1.292}$$. Giving your answer correct to 3 significant figures. [2];;b) The cost of producing a school magazine is made up from two parts, typing and printing. In 1996 the typing cost $3.00 for every page and the printing cost $18.50 for every 100 copies of the magazine.;;b-i) Show that the total cost of producing 600 copies of a magazine with 32 pages was $207.  [1];;b-ii) The magazines were sold for 40 cents each.;;b-ii-a) Find the number of magazines that needed to be sold so that no loss was made.  [2];;b-ii-b) Calculate the percentage profit that would have been made if all of the 600 magazines were sold. [2];;b-ii-c) In fact 4% of the magazines were given away, and the remainder were sold. Calculate the profit that was actually made. [2];;b-iii) The 1996 cost of typing each page ($3.00) was 25% greater than the 1995 cost. Calculate the cost of typing each page in 1995. [2]Answers:

ID: 199704002002
Content:
a) Express as a single fraction in its simplest form $$\frac{4}{3r - 2} - \frac{3}{r + 5}$$ [3];;b) Given that $$\frac{a}{a + 2} = b$$, ; express a in terms of b. [3];;c) Solve the equation $$3x^2 - 5x - 1 = 0$$ giving the answers correct to two decimal places.  [5]Answers:

ID: 199704002003
Content:
A crane stands on level ground. It may be represented by a tower ABCD, of height 11m, and a jib BR. The jib is of length 20m and can rotate in a vertical plane about B. a vertical cable, RS, carries a load S. the diagrams show two possible positions of the jib, cable and load.;;a);img;Diagram I shows the situation when BS is horizontal and RS = 8m. Calculate;;a-i) The distance BS,;;a-ii) The angle that the jib, BR, makes with the horizontal.;;b) Diagram II shows another situation.;The jib, BR, has been rotated and the length RS increased.; The load, S, is now on the ground at a point 5m from A. Calculate;;b-i) The angle through which the jib has been rotated,;;b-ii) The length by which RS has increased.Answers:

ID: 199704002004
Content:
img;A circle passes through A, B, C and D.; $$B \hat AD = 110^{\circ}$$, $$A \hat BD = 36^{\circ}$$ and $$D \hat BC = 54^{\circ}$$.;;a) Find;;a-i) $$C \hat AD$$,  [1];;a-ii) $$B \hat PA$$. [2];;b) CDX, BAX, DAY and CBY are straight lines.;;b-i) Calculate $$A \hat DX$$, showing your reasoning. [2];;b-ii) Explain why X, D, B and Y lie on a circle. [1];;b-iii) State where the centre of the circle XDBY lies. [1]Answers:

ID: 199704002005
Content:
a) John runs at a constant speed, taking 0.17 seconds to run each metre. ;Show that the time he takes to run 70 metres is approximately 12 seconds. [1];;b) Alan runs at a constant speed, taking a seconds to run each metre. ; His sister, Betty, also runs at a constant speed, taking b seconds to run each metre.;;b-i) They ran a race over a distance of 50 metres, which Alan won.;;b-i-a) Write down an expression, in terms of a and b, for the difference between their times. [1];;b-i-b) Given that Alan won the race by 0.5 seconds, form an equation in a and b and show that it simplifies to ;;100b - 100a ;;b-ii) Next day they ran another race at the same speeds, but Betty was given a start of 3 metres, so that she ran 47 metres. ;She won this race by 0.1 seconds. ;Write down another equation in a and b and simplify it. [2];;b-iii) Solve these two equations to find the value of b. (You are not asked to find the value of a.) [2]Answers:

ID: 199704002006
Content:
img;A series of diagrams of shaded and unshaded small triangles is shown below.;img;;The shaded triangles are those which have at least one side on the edge of the big triangle.;;All of the other small triangles are unshaded.; Copy the table below, which shows numbers of small triangles.;;a) Complete the column for Diagram 4 in your table.;;b) By considering the number patterns in your table,;;b-i) Complete the column for Diagram 5,;;b-ii) Find, in terms of n, expressions for x, y, and z.;;c) Find the number of unshaded small triangles in Diagram 100.Answers:

ID: 199704002007
Content:
img;[The value of $$\pi$$ is 3.142 correct to three decimal places.] ; Diagram I shows a barn and Diagram II shows the cross-section of its end. ;;A farmer needs to order a new roof for his barn. The roof is represented by ABC, the arc of a circle of radius r, centre O. ACDE is a rectangle.;The farmer measures AC, CD, BF and the length of the barn.;;a) Given that AC = 8m and BF = 2m,;;a-i) Write down an expression, in terms of r, for the length of OF, [1];;a-ii) Show that the radius, r, of the circle is 5 metres. [2];;b) Show that angle AOC is approximately $$106^{\circ}$$. [2];;c) Given that the length of the barn is 12 metres, calculate the curved area of the roof (shaded in Diagram I). [3];;d) Given also that CD = 7 metres, calculate the volume of the barn. [4]Answers:

ID: 199704002008
Content:
Answer the whole of this question on a sheet of graph paper.;;a) Table I below gives some values of x and the corresponding values of y, correct to two decimal places, where $$y = x(1 + x)(3 - x)$$;;a-i) Find the value of p. [1];;a-ii) Using a scale of 4 cm to represent 1 unit draw a horizontal x-axis for $$0 \leq x \leq 3$$. ;Using a scale of 2 cm to represent 1 unit draw a vertical y-axis for $$0 \leq x \leq 8$$.;On your axes plot the points given in the table and join them with a smooth curve. [3];;a-iii) Use your graph to find the greatest value of $$x(1 + x)(3 - x)$$ in the interval $$0 \leq x \leq 3$$.;;b) Table II shows some corresponding values of x and y where $$y = 2^x$$.;;b-i) Find the value of q, correct to 2 decimal places. [1];;b-ii) On the axes used in part (i)(2) draw the graph of $$y = 2^x$$. [2];;c) From your graphs, find the values of x in the interval $$0 \leq x \leq 3$$, for which [4];;c-i) $$x(1 + x)(3 - x) = 3$$, ;;c-ii) $$x(1 + x)(3 - x) > 2^x$$.Answers:

ID: 199704002009
Content:
img;In the diagram, ABC represents a horizontal triangular field and AD represents a vertical tree in the corner of the field. A path runs along the edge BC of the field. ;;AB = 83 m, AC = 46 m and angle BAC = $$67^{\circ}$$.;;a) The angle of elevation of the top of the tree when viewed from B is $$14^{\circ}$$. ; Calculate the height of the tree. [2];;b) Calculate the length of the path BC. [4];;c) Calculate the area of the field ABC. [2];;d) Calculate the shortest distance from A to the path BC. [2];;e) Calculate the greatest angle of elevation of the top of the tree when viewed from any point on the path. [2]Answers:

ID: 199704002010
Content:
a) A box containing 250 apples was opened and each apple was weighed. The distribution of the masses of the apples is given in the following table.;img;;a-i) When a histogram is drawn to illustrate this information, the height of the column representing apples with mass m in the interval $$60 < m \leq 100$$ is 10 cm.;;Calculate the height of the column that represents values of m in $$160 < m \leq 220$$.;;a-ii) Calculate an estimate of the mean mass of the apples in the box. [3];;b) In this part of the question all probabilities should be given as exact decimals. ;The ticket machine in a car park takes 50 cent coins and $1 coins. ;A ticket costs $1.50.;The probability that the machine will accept a particular 50 cent coin is 0.9 and that it will accept a particular $1 coin is 0.8.;;b-i) What is the probability that the machine will not accept a particular 50 cent coin? [1];;b-ii) Leslie put one 50 cent coin and one $1 coin into the machine.;;Calculate the probability that the machine will not accept both coins;; b-iii) What is the probability that 3 50 cent coins will be accepted?Answers:

ID: 199704002011
Content:
img;Triangle ABC has vertices A(-6, 12), B(8, 4) and C(2, 2).;;a) Triangle FGH has vertices F(-12, 6), G(-4, 8) and H(-2, 2). [4];;a-i) Describe fully the single transformation that maps $$\Delta ABC$$ onto $$\Delta FGH$$.;;a-ii) Write down the matrix that represents this transformation.;;b) Write down the column vector which represents the translation that maps C onto B. [1];;c) ABCP is a trapezium in which $$\vec{PA}$$ is parallel to $$\vec{CB}$$ and PA = $$\frac{1}{2}$$CB.;Find the coordinates of P. [2];;d) An enlargement maps $$\Delta ABC$$  onto $$\Delta ADE$$.;;d-i) Which point is the centre of the enlargement? [1];;d-ii) Given that D is the point (k, 0), find [4];;d-ii-a) the value of k,;;d-ii-b) the scale factor of the enlargement,;;d-ii-c) the coordinates of E,;;d-ii-d) the ratio of the area of $$\Delta ABC$$  to the area of $$\Delta ADE$$.Answers:

\end{document}
