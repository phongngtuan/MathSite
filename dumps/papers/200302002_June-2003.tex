\documentclass{article}
\begin{document}
ID: 200302002001
Content:
a-i) Evaluate $$\frac{4.8^{2} -1.7^{2}}{4.8 \times 1.7}$$. [1];;a-ii) Find a value of x for which $$\sin x^{\circ}= \tan 12^{\circ}+\cos 46^{\circ}$$. [1];;b);img;The diagram shows a framework ABCD. AD = 2.2 m, BD = 1.9 m and $$BCD=42^{\circ}$$, $$A \hat BD=B \hat DC=90^{\circ}$$. Calculate;;b-i) $$A \hat DB$$, [2];;b-ii) BC. [3];;c) A vertical flagpole, 18 m high, stands on horizontal ground. Calculate the angle of elevation of the top of the flagpole from a point, on the ground, 25 m from its base. [2]Answers:

ID: 200302002002
Content:
a) Factorize completely $$20t^{2} -5$$. [2];;b) Express as a single fraction in its simplest form $$\frac{7}{2x}-\frac{5}{3x}$$. [2];;c) Tickets for a concert were priced at $5, $8 and $12. The number of $5 tickets sold was twice the number of $8 tickets. The number of $12 tickets sold was 80 more than the number of $8 tickets. The number of $8 tickets sold was x.;;c-i) Find an expression, in terms of x, for the total sum of money received from the sale of the tickets. [1];;c-ii) Given that $9360 was received from the sale of the tickets, form an equation in x. Solve this equation and hence find the total number of tickets that were sold. [3]Answers:

ID: 200302002003
Content:
In 2001 the price of one litre of petrol was 72 cents.;;a) 65% of this price is tax ;;a-i) Find, in its simplest form, the ratio of tax to the other cost. Give your answer in the form m : n, where m and n are integers. [1];;a-ii) Calculate how much tax is paid on one litre of petrol. [1];;b) Maureen bought as many complete litres of petrol as she could with a $20 note ($1 = 100 cents).;;b-i) Calculate how many litres she bought. [1];;b-ii) Calculate how much change she received. [1];;c) In 2002 the price of one litre of petrol was 81 cents. Calculate the percentage increase in the price of petrol from 2001 to 2002. [2];;d) The price of petrol in 2001 was 10% less than the price in 2000. Calculate the price of one litre of petrol in 2000. [3];;e) AndrewAnswers:

ID: 200302002004
Content:
img;BD is a diameter of the circle, centre O. C and A are two points on the circle. AB and DC, when produced, meet at E. $$A \hat OB=110^{\circ}$$ and $$B \hat DC=23^{\circ}$$.;;a) Find;;a-i) $$A \hat DO$$, [1];;a-ii) $$B \hat AC$$, [1];;a-iii) $$C \hat BD$$, [1];;a-iv) $$C \hat EB$$. [1];;b) M is the midpoint of CD.;;b-i) Explain why triangle OMD is similar to triangle BCD. [2];;b-ii) Write down the value of $$\frac{Area.of.\Delta OMD}{Area.of.\Delta BCD}$$. [1]Answers:

ID: 200302002005
Content:
img;a) One hundred and sixty students took an examination. The table shows the marks needed for each grade. The cumulative frequency curve shows the distribution of their marks.;;a-i) Use the graph to estimate;;a-i-a) the median, [1];;a-i-b) the inter-quartile range, [2];;a-i-c) the number of students who were awarded a Grade C. [2];;a-ii) A pie chart was drawn to illustrate the grades to the students. Calculate the angle of the sector which represented the number of students who were awarded a Grade C. [2];;b) An ordinary unbiased die has faces numbered 1, 2, 3, 4, 5 and 6. Sarah and Terry each threw this die once. Expressing each answer as a fraction in its lowest terms, find the probability that;;b-i) Sarah threw a 7, [1];;b-ii) they both threw a 6, [1];;b-iii) neither threw an even number, [1];;b-iv) Sarah threw exactly four more than Terry. [1]Answers:

ID: 200302002006
Content:
img;The natural numbers 1, 2, 3, ... are written, in a clockwise direction, on a circular grid as shown in the diagram. There are four numbers in each ring. The numbers 1, 2, 3 and 4 are in the first ring. The numbers 5, 6, 7 and 8 are in the second ring. The following numbers fill up the other rings in the same way.;;a) Write down the numbers in the fourth ring. [1];;b) Write down the largest number in the tenth ring. [1];;c) The sum, Sn, of the four numbers in the nth ring, where n = 1, 2 and 3, is given in the table below.;img;;c-i) Write down the value of S4. [1];;c-ii) Find, in its simplest form, an expression, in terms of r, for Sr. [2];;c-iii) In which ring is the sum of the four numbers equal to 1018? [1]Answers:

ID: 200302002007
Content:
img;[The value of $$\pi$$ is 3.142, correct to three decimal places.] ; [The surface area of a sphere is $$4 \pi r^{2} $$.]; [The volume of a sphere is $$\frac{4}{3} \pi r^{3} $$.] ; A closed container is made by joining together a cylinder of radius 9 cm and a hemisphere of radius 9 cm as shown in Diagram I. The length of the cylinder is 18 cm. The container rests on a horizontal surface and is exactly half full of water.;;a) Calculate the surface area of the inside of the container that is in contact with the water. Give your answer correct to the nearest square centimeter. [4];;b) Show that the volume of the water is $$972\pi cm^{3} $$. [2];;c);img;The container is held with its axis vertical, the hemisphere being at the bottom, as shown in Diagram II. Calculate the depth of the water. [4];;d);img;The container is now placed with its circular end on a horizontal surface as shown in Diagram III. Find the depth of the water. [2]Answers:

ID: 200302002008
Content:
Answer the whole of this question on a sheet of graph paper.;;Temperatures were recorded over a nine hour period. The table below shows the temperature, $$y ^{\circ}C$$, at various times.;img;;a) Using a scale of 1 cm to represent 1 hour, draw a horizontal x-axis for $$0\leq x\leq 9$$. Using a scale of 2 cm to represent $$1^{\circ}C$$, draw a vertical y-axis for $$-2\leq y\leq 4$$. On your axes, plot the points given in the table and join them with a smooth curve. [3];;b) Use your graph to find an estimate for;;b-i) the temperature when x = 5.5, [1];;b-ii) the difference between the highest and lowest temperatures, [1];;b-iii) how long, in hours and minutes, the temperature was above $$2^{\circ}C$$. [2];;c-i) By drawing a tangent, find the gradient of the curve at that point where x = 8. [2];;c-ii) State briefly what this gradient represents. [1];;d) The curve from x = 0 to x = 2 has the equation $$y=x^{2} +Bx+C$$. Find the value of C and the value of B. [2]Answers:

ID: 200302002009
Content:
img;The diagram shows the position of a harbor, H, and three islands A, B and C. C is due North of H. The bearing of A from H is $$062^{\circ}$$ and $$H \hat AB=128^{\circ}$$. HA = 54 km and AB = 31 km.;;a) Calculate the distance HB. [4];;b) Find the bearing of B from A. [1];;c) The bearing of A from C is 133$$^{\circ}$$. Calculate the distance AC. [4];;d) A lightship, L, is positioned due North of H and equidistant from A and H. Calculate the distance HL. [3]Answers:

ID: 200302002010
Content:
img;Diagram I shows a quadrilateral, ABCD, in which DA = AB = x centimeters and BC = CD = y centimeters. $$A \hat BC=C \hat DA=90^{\circ}$$.;;a) Show that the area of this quadrilateral is xy square centimeters. [1];;b) Five of these quadrilaterals are joined together to make the shape shown in Diagram II. The total area of this shape is $$80 cm^{2} $$.;;b-i) Show that the outside perimeter, P centimeters, of this shape is given by $$P=10x+\frac{32}{x}$$. [2];;b-ii-a) In the case when P = 38, show that $$5x^{2} -19x+16=0$$. [2];;b-ii-b) Solve the equation $$5x^{2} -19x+16=0$$, giving both answers correct to two decimal places. [4];;b-ii-c) Find the two possible values of y when P = 38. [1];;b-iii-a) Calculate the value of P when x = y. [1];;b-iii-b) What is the special name given to the quadrilateral ABCD when x = y? [1]Answers:

ID: 200302002011
Content:
img;The diagram shows triangles A, B, C and D. ;;a) Describe fully the single transformation which maps A onto B. [2];;b) Find the matrix that represents the single transformation which maps A onto C. [2];;c) A is mapped onto D by a clockwise rotation. Find;;c-i) the angle of this rotation. [1];;c-ii) the coordinates of the centre of this rotation. [1];;d) The matrix $$\begin{bmatrix}2&0\\0&1\end{bmatrix}$$ represents the transformation which maps triangle A onto triangle E.;;d-i) Find the coordinates of the vertices of triangle E. [2];;d-ii) Describe fully the transformation that is represented by the matrix $$\begin{bmatrix}2&0\\0&1\end{bmatrix}$$. [2];;d-iii) Find the matrix that represents the single transformation which maps triangle E onto triangle A. [2]Answers:

\end{document}
