\documentclass{article}
\begin{document}
ID: 199902002001
Content:
a) In a chocolate, the ratio of the masses of cocoa : milk : other ingredients is 3 : 2 : 5. ;;a-i) What fraction of the chocolate is cocoa? [1];;a-ii) The mass of a chocolate is 28 g. Calculate the mass of milk in a chocolate. [2];;b) The chocolates were sold in boxes. In 1996 each box was $4.80. Fiona had $29 and bought as many boxes as possible.;;b-i) How many boxes did she buy? [1];;b-ii) How much money did she have left? [1];;c) The price in 1998 was 10% more than the 1996 price of $4.80. Calculate the price in 1998. [1];;d) The price of $4.80 was an increase of 20% of the price in 1990. Calculate the price in 1990. [3]Answers:

ID: 199902002002
Content:
a) The first four terms of a sequence are 8, 13, 18 and 23.;;a-i) Write down the 10th term. [1];;a-ii) The nth term is 5n + k. Find the value of k. [1];;b) The nth term of another sequence is 7n - 1. Find the value of the term of this sequence which is closest to 1999. [2];;c-i) Express as a single fraction $$\frac{2x}{5}-\frac{x}{4}$$. [2];;c-ii) Simplify $$\frac{3a}{5c^{2}} \times \frac{10c^{3}}{a^{2}}$$. [2];;d) A clock gains s seconds in one hour. Write down, in its simplest form, an expression in terms of s and d for the number of minutes it gains in d days. [2]Answers:

ID: 199902002003
Content:
a);img;In the diagram, AB = 8 cm, BC = 3 cm and BE is parallel to CD.;;;;a-i) Find the value of;;a-i-a) $$\frac{BE}{CD}$$, [1];;a-i-b) $$\frac{Area.of.triangle.ABE}{Area.of.triangle.ACD}$$, [1];;a-i-c) $$\frac{Area.of.triangle.ABE}{Area.of.quadrilateral.BCDE}$$. [2];;a-ii) What is the special name given to the quadrilateral BCDE? [1];;b);img;Twenty points, labeled 1, 2, ..., 20, are marked at equal intervals around the circumference of a circle, centre O. Some of these points are shown and labeled in the diagram. H and K are two such points which are next to each other. X, Y and Z are the points labeled 1, 11 and 15 respectively. Calculate;;b-i) $$H \hat OK$$, [1];;b-ii) $$H \hat XK$$, [1];;b-iii) the three angles of the triangle XYZ. [2]Answers:

ID: 199902002004
Content:
img;In the diagram, BCD is a straight line, BA is parallel to CE, ED = CD, $$B \hat AC=40^{\circ}$$, $$A \hat BC=72^{\circ}$$, $$C \hat EA=82^{\circ}$$.;;a) Calculate;;a-i) $$A \hat CE$$, [1];;a-ii) $$C \hat AE$$, [1];;a-iii) $$C \hat DE$$. [2];;b) Given that BC = 8 cm, calculate the length of AC. [3]Answers:

ID: 199902002005
Content:
a) The number of children in 40 families is given in the table below.;img;Calculate the mean number of children in a family. [2];;b) Some students were asked which colour they liked best. The results are shown in the pie chart.;img;;b-i) Three times as many students said Blue as said Green. Calculate the angle of the sector which represents the number of students who said Green. [2];;;;b-ii) Eight more students said Red than said Yellow. Calculate the number of students who said Red. [2];;c) There were 12 girls and 3 boys in a group of children. One child was chosen at random from the group. Another child was chosen at random from the remaining children. Expressing each answer as a fraction in its simplest form, calculate the probability that;;c-i) the first child chosen was a girl, [1];;c-ii) the first child chosen was a girl and the second was a boy, [1];;c-iii) a child of each sex was chosen. [1]Answers:

ID: 199902002006
Content:
a);img;A, B and C are three points on horizontal ground. BT is a vertical mast of height 20 m. The top of the mast is joined to A and C by straight wires. Angle BCT = $$31^{\circ}$$. ;;a-i) Calculate the length of the wire CT. [2];;a-ii) Given that AB is 30 m, calculate the angle of elevation of T from A. [2];;b);img;In the triangle shown, XY = 5 cm, YZ = 7 cm and ZX = 6 cm. Calculate $$Y \hat XZ$$. [4]Answers:

ID: 199902002007
Content:
img;Two rectangles, A and B, each have an area of $$11cm^{2} $$. The length of rectangle A is x cm. The length of rectangle B is (x + 3) cm.;;a) Find, in terms of x, an expression for the width of;;a-i) rectangle A, [1];;a-ii) rectangle B. [1];;b) Given that the width of rectangle A is 2 cm greater than the width of rectangle B, form an equation in x and show that it simplifies to $$2x^{2} +6x-33=0$$. [3];;c) Solve the equation $$2x^{2} +6x-33=0$$, giving both answers correct to 2 decimal places. [5];;d) Hence find the width of rectangle B. [2]Answers:

ID: 199902002008
Content:
a);img;[The value of $$\pi$$ is 3.142 correct to 3 decimal places.] A well is a vertical open cylinder of radius 1.2 m and height 5 m. The well contains water to a depth of 3 m.;;a-i) Calculate the total internal area of the curved surface of the well and the bottom of the well. [3];;a-ii) Calculate the volume, in litres, of water in the well. [$$1m^{3} $$ = 1000 litres] [2];;a-iii) 270 litres of water is removed from the well. Calculate, correct to the nearest centimeter, the resulting fall in the water level. [3];;b) At 9 a.m. a storage tank contained 400 litres of water. Water was removed from the tank at a constant rate of 5 litres per minute. At 10 a.m. 270 litres of water was added. A further 270 litres was added at the end of each hour after that.;;b-i) Calculate the number of litres of water in the tank at 10.05 a.m. [2];;b-ii) Calculate the time when the tank was empty for the first time. [2]Answers:

ID: 199902002009
Content:
Answer the whole of this question on a sheet of graph paper. A particle was projected directly up a slope. Its distance, d metres, from the bottom of the slope, t seconds after it was projected, it is given in the table below.;img;;;a) Using a horizontal scale of 2 cm to represent 1 second, and a vertical scale of 1 cm to represent 1 metre, draw a graph of d against t. [3];;b) Use your graph to find the distance of the particle from the bottom of the slope when t = 2.5. [1];;c) What happened to the particle after approximately 6 seconds? [1];;d-i) By drawing a tangent, find the gradient of your curve when t = 4. [2];;d-ii) State briefly what this gradient represents. [1];;e) Calculate the average speed of the particle during the first 6 seconds. [1];;f) At the instant the particle was projected, another particle was projected down the slope from point 10 metres from the bottom. This particle moved directly down the slope at a constant speed of 2 m/s.;;f-i) On the same axesAnswers:

ID: 199902002010
Content:
Answer the whole of this question on a sheet of graph paper. The lengths of 200 leaves were measured. The cumulative frequencies are given in the table below.;img;;a) Using a horizontal scale of 2 cm to represent 10 mm, and a vertical scale of 2 cm to represent 50 leaves, draw a cumulative frequency curve to illustrate this information. [3];;b) Showing your method clearly, use your graph to estimate;;b-i) the median, [1];;b-ii) the inter-quartile range, [2];;b-iii) the number of leaves which are longer than 55 mm. [2];;c) The frequency distribution for these results is given in the table below.;img;;c-i) Write down the value of p and the value of q. [1];;c-ii) Draw a histogram to illustrate this information. Use a horizontal scale of 2 cm to represent 10 mm. Use a vertical scale for frequency density such that the height of the first column is 3 cm. [3]Answers:

ID: 199902002011
Content:
img;The diagram shows triangles A, B and C.;;a) Describe fully the single transformation which maps A onto B. [2];;b) The single transformation which maps A onto C is a stretch. Find;;b-i) the scale factor,;;b-ii) the invariant line. [2];;c) M is the matrix $$\begin{bmatrix}0&-1\\-1&0\end{bmatrix}$$. P is the matrix which represents a rotation of $$90^{\circ}$$ anticlockwise about the origin.;;c-i) The image of A under the transformation represented by M is K. Find the coordinates of the vertices of K. [2];;c-ii) Describe fully the single transformation represented by $$P^{-1}$$. [1];;c-iii) Find the matrix P. [2];;c-vi) By considering the effects of transformations of A, or otherwise, find the matrix Q such that M = QP. [3]Answers:

\end{document}
