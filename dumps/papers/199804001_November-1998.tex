\documentclass{article}
\begin{document}
ID: 199804001001
Content:
(a)Find the value of k for which \[x^2+(k-1)x+k^2-16\] is exactly divisible by x ? 3 but not divisible by x + 4 ;(b)Solve the equation \[8x^3-2x^2-5x-1=0\]  Hence find the values of \[\theta\] between \[0^{\circ}\]  and \[180^{\circ}\]  which satisfy the equation \[8\tan^2\theta-2\tan\theta-5=\cot\theta\](Note: Please enter your answers in ascending order)Answers:

ID: 199804001002
Content:
a) An arithmetic progression is such that the 5th term is 3 times the 2nd term.;i) Show that the sum of the first 8 terms is 4 times the sum of the 1st 4 terms.;ii) Given further that the sum of the 5th, 6th, 7th and 8th term is 240, calculate the value of the first term.;b) The first term of a geometric progression is a and the common ratio is r. Given that a+96r=0 and that the sum to infinity is 32, find the 8th term.;c) The first and second terms of a geometric progression are 10 and 11 respectively. Find the least number of terms such that their sum exceeds 8000.Answers:

ID: 199804001003
Content:
img;a) The table shows experimental values of two variables, x and y.  ;It is known that x and y are related by the equation \[y+10=Ak^x\] where A and k are constants. Using graph paper plot lg(y + 10) against x for the above data and use your graph to estimate ;i) the value of A and of k,;ii) the value of x when y = 0.;b) The variables x and y are related by the equation \[px^3+qy^2=1\];img;The diagram shows the straight-line graph of y2 against x3 which passes through the point \[(\frac{1}{2},\frac{1}{4})\] ;i) Given that the gradient of this line is \[\frac{3}{4}\] calculate the value of p and q,;ii) Given also that this line passes through (k, 4), find the value of k.Answers:

ID: 199804001004
Content:
a) Solve, for \[0^{\circ}\leq \theta\leq 360^{\circ}\] \[\cos2\theta=\cos\theta\] ;b)Express \[4\sin x - 2\cos x\] in the form of \[R\sin(x - \alpha)\] where R is a positive constant and \[\alpha\] is a positive acute angle measured in radians. ;Hence find the x- coordinates, between 0 and \[2\pi\] of the points of intersection of the curves whose equations are \[y = 4\sin x\] and \[y = 2\cos x + 1\];c) Given that \[\frac{(\cos(A-B))}{(\cos(A+B))}=\frac{-9}{7}\] find the value of \[\tan A\tan B\] (Note: Please enter your answers in ascending order)Answers:

ID: 199804001005
Content:
(a) Solve the equation \[2^{x+1}+2^x=9\] ;(b) Solve the simultaneous equations;\[3^x=27(3^y)\];and \[\lg(x+2y)=\lg5+\lg3\]Answers:

ID: 199804001006
Content:
A particle moves in a straight line so that at time, t, seconds after leaving a fixed point O, its velocity \[vms^{-1}\] is given by \[v=20e^{\frac{-t}{4}}\]; i) Sketch the velocity - time curve.;ii) Find the value of t when v = 10.;iii) Find the acceleration of the particle when v = 10.;iv) Obtain an expression, in terms of t, for the displacement from O of the particle at time t seconds.Answers:

ID: 199804001007
Content:
a) Differentiate with respect to x,;i) \[(3-2x)^{10}\];ii) \[x^2lnx\];b) Find the equation of the tangent of the curve  \[y^2=x^2y+6x\]  at the point (2, 6).;c) A curve has the equation \[y=\frac{\cos x}{\sin x-4}\]; Find,;i) an expression for \[\frac{\mathrm{d} y}{\mathrm{d} x}\];ii) the values of x between 0 and \[2\pi\] for which y is stationary.Answers:

ID: 199804001008
Content:
a) A curve is such that \[\frac{\mathrm{d} y}{\mathrm{d} x}=3x^2+\frac{2}{x}\];Given that the curve passes through the point (1, 3), find the equation of the curve.;b) Show that \[\frac{\mathrm{d} }{\mathrm{d} x}=(2x+2\sin x)=4\cos^2x\];Hence or otherwise, evaluate \[\int_\frac{\pi}{2}^\pi cos^2xdx\];img;c) The figure shows part of the curve \[y=\frac{1}{\sqrt{(4x+1)}}\];Find;i) the area of the shaded region,;ii)the volume generated when the shaded region is rotated through \[360^{\circ}\]   about the x-axis.Answers:

ID: 199804001009
Content:
in terms of the parameter t, the equations of a curve are \[x=t^2-t\], \[y=2t+1\];i) Find the value of t at the point P on the curve where the gradient is \[\frac{2}{5}\];ii) Show that the equation of the normal of P is 2y + 5x = 44.;iii) Find the value of t at the point where the normal at P again intersects the curve.Answers:

ID: 199804001010
Content:
a) The parametric equations of a curve are \[x=3\sin\alpha+\cos\alpha\] \[y=\sin\alpha-2\cos\alpha\];Express each of \[\sin\alpha and \cos\alpha\] in terms of x and y. Hence obtain the equation of the curve.;b) The Cartesian equation of the curve is \[(y-3)^2=1+x^2\];Given that x may be defined parametrically by \[x=\cot \theta \]and that y = 1 when \[\theta=\frac{\pi }{6}\] express y in terms of \[\csc \theta\]Answers:

\end{document}
