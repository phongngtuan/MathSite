\documentclass{article}
\begin{document}
ID: 200004001001
Content:
The curve whose equation is \[y=(2x^2+3x-9)(x-k)\] where k is a constant, has a turning point where x = -1.;(i)	Calculate the value of k.;(ii)	Calculate the value of x at the other turning point on the curve.;(iii)	Draw a rough sketch of the curve and find the set of values of x for which y > 0.;(Note: Please enter your answers in ascending order)Answers:

ID: 200004001002
Content:
f(x) and g(x) are given by; \[f(x)=x^4+3x^3-12x^2+2x+4\]; \[g(x)=x^4+2x^3-8x^2+x-2\];(i)	Solve completely the equation \[f(x) - g(x) = 0\];(Note: Please enter your answers in ascending order);(ii) f(x) and g(x) hav e a common factor \[x-\alpha\] Find the value of \[\alpha\]Answers:

ID: 200004001003
Content:
The parametric equations of a curve are \[x=4t+1,y=1+t^3\];The line y = x intersects the curve at points A, B and C . The coordinates of A are negative and B lies between A and C .;i) Find the coordinates of A, B and C .;ii) Show that AB = BC;iii) Obtain an expression for \[\frac{\mathrm{d} y}{\mathrm{d} x}\] in terms of t.;iv) Show that the tangent to the curve at A is parallel to the tangent at C and find the equation of each of these tangents;v) Given that the tangent at A meets the curve again at the point D, find the value of t at D.;vi) Obtain the cartesian equation of the curve.Answers:

ID: 200004001004
Content:
(a)	Solve the equation \[2^{x+2}=2^x+10\];(b)	The variables \[\theta\] and t are related by the equation \[\theta=\theta_0e^{-kt}\] where \[\theta_0\] and k are constants. When t = 30, \[\theta=\frac{1}{2}\theta_0\];(i) Show that the value of k, correct to 4 decimal places, is 0.0231. ;When t = 40, \[\theta=28\];(ii) Calculate the value of \[\theta_0\]. ;When t = 50, calculate;(iii) \[\theta_0\];(iv) \[\frac{\mathrm{d} \theta}{\mathrm{d} t}\];Find the average rate of change of \[\theta\] with respect to t over the interval \[0\leq t\leq 50\]Answers:

ID: 200004001005
Content:
img;(a) The table shows experimental values of two variables x and y. ;Using the vertical axis for xy and the horizontal axis for \[x^2\] plot xy against \[x^2\] and obtain a straight line graph. Use your graph to ;(i)	express y in terms of x,;(ii)	estimate the value of x when \[y=\frac{30}{x}\];(b)	Variables x and y are related in such a way that, when y-x is plotted against \[x^2\] a straight line is produced which passes through the points A(4, 6), B(3, 4) and P(p, 4.48), as shown in the diagram.;img ;Find;(i)	y in terms of x,;(ii)	the value of p,;(iii)	the value of x and of y at the point P.Answers:

ID: 200004001006
Content:
(a) Given that \[(144p^4)^{\frac{3}{2}}\div (216p^{-3})^{\frac{-2}{3}}=2^x3^yp^z\] evaluate x, y and z.;(b) Without using tables or a calculator, find the value of k such that \[(\frac{1}{\sqrt6}-\frac{\sqrt{24}}{3}+\frac{49}{\sqrt{294}}) \times  \frac{3}{\sqrt2} = k\sqrt3\] Answers:

ID: 200004001007
Content:
(a) Given that \[4\lg(x\sqrt y)=1.5+\lg x-\lg y\]where x and y are both positive, express, in its simplest form,y in terms of x.;(b) Given that \[p=log_4q\] express, in terms of p,;i) \[log_4(\frac{1}{q})\];ii) \[log_2 8q\]Answers:

ID: 200004001008
Content:
img;The diagram shows an isosceles triangle AOB, with OA = OB = 2 cm and an isosceles triangle COD with OC = OD = 4 cm. \[\angle AOD=90^{\circ}\] and \[\angle COD=x^{\circ}\];The sum of the areas of triangles AOB and COD is \[Scm^2\];i) Show that \[S=2\cos x+8\sin x\];ii) Express S in the form \[R\sin(x+\alpha)\] stating the value of R and of \[\alpha\];Given that x can vary, find ;iii) the value of x for which S = 6,;iv) the maximum value of S and the corresponding value of x.;v) the value of x for which the area of triangle AOB is half the area of triangle COD.Answers:

ID: 200004001009
Content:
i) \[\int_\frac{\pi}{4}^\pi 3\cos2xdx\];ii) \[\int_0^2\sqrt{4x+1}dx\];b) Find \[\frac{\mathrm{d} }{\mathrm{d} x}(\frac{1}{(9-4x^2)})\]  and hence evaluate \[\int_0^1 \frac{x}{(9-4x^2)^2}dx\];img;c) The diagram shows parts of the curves \[y = 1 + \sin2x\] and \[y=1-x^2\]  and the line x = 0.5. Find the area of the shaded region.Answers:

ID: 200004001010
Content:
(a)	Find the equation of the tangent to the curve \[xy+x^2=2y\] at the point on the curve where x = 1.;(b)	Given that \[y=x^2\sin3x\] find the value of \[\frac{\mathrm{d} y}{\mathrm{d} x}\] where x = 2;(c)	Given that \[y= x\ln x - x\] find an expression for \[\frac{\mathrm{d} y}{\mathrm{d} x}\]  Hence find, in terms of p, the approximate change in y when x changes from \[e^2\] to \[e^2+p\] where p is small.Answers:

\end{document}
