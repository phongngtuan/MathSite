\documentclass{article}
\begin{document}
ID: 200002001001
Content:
(a)	Solve the equation \[5x^3-3x^2-32x-12=0\];(b) When the function \[f(x)-=2x^3-9x^2+Ax+B\] where A and B are constants, is divided by x + 1 or by x - 2 there is a remainder of 3 in each case.;(i)	Find the value of A and of B .;(ii) Solve the equation \[f(x) - 3 = 0\] ;(iii) Solve the equation \[f(2x) - 3 = 0\];(Note: Please enter your answers in ascending order)Answers:

ID: 200002001002
Content:
a) In an arithmetic progression the 20th term is 92 and the sum of the first 20 terms is 890. Find;i) the first term and the common difference,;ii) the sum of the first 10 terms.;b) In a geometric progression the second term is a and the common ratio is r, where |r| < 1. The sum of the first n terms is \[S_n\] and the sum to infinity is \[S_{oo}\] `;Show that \[S_{oo}-S_n=\frac{(ar^n)}{(1-r)}\]; A geometric progression has a first term of 5 and a common ratio of 0.8. The sum of the first n terms differs from the sum to infinity by less than 0.6. Find the least possible value of n.Answers:

ID: 200002001003
Content:
img;a) The table shows experimental values of the variables x and y;Using the vertical axis for \[\frac{1}{y}\] and the horizontal axis for \[\frac{1}{x}\]against \[\frac{1}{x}\] and draw a vertical straight line graph. Use this graph to express y in terms of x.;img;b) The diagram shows part of the straight line graph obtained by \[\lg y\] against \[\lg(x + 1)\] Find;i) y in terms of x,;ii) the value of y when x = 3.Answers:

ID: 200002001004
Content:
(a)	The curve  \[y=px^q-15\] passes through the points (2, 25) and (3, 120). Calculate the value of p and of q.;(b)	Solve the equation \[3^{2x}-3^{x+1}-10=0\];(c)	Using graph paper, draw the graph of \[y=e^{2x}\] for values of x, at intervals of 0.25, from x = 0 to x = 1 inclusively. By drawing a suitable straight line on your graph, obtain an approximate solution to the equation \[x=\ln \sqrt{(9-4x)}\]Answers:

ID: 200002001005
Content:
img;The diagram shows two perpendicular lines, OA and AB, of length 5 cm and 2 cm respectively. The line BN is perpendicular to ON. The line OA is inclined to an angle \[\theta\] to ON.;(a)Show that \[ON=5\cos\theta + 2\sin\theta\];(b)	Find the value of R and of \[\alpha\] for which \[ON=R\cos(\theta-\alpha)\];(c)	Find the particular value of \[\theta\] for which ON = 3 cm.;(d) State which line in the diagram has a length of R cm and which angle in the diagram has a value of \[\alpha\];(e)	Express BN in terms of R and \[(\theta-\alpha)\];(f) Show that the area of triangle OBN is \[\frac{(29\sin 2(\theta-\alpha))}{4}cm^2\];(g) Given that \[\theta\] can vary, find the maximum value of the area of triangle OBN and the corresponding value of \[\theta\]Answers:

ID: 200002001006
Content:
(a)	Given that \[y=\frac{(3e^{2x})}{(2x+1)}\]  find the value of k for which \[\frac{\mathrm{d} y}{\mathrm{d} x}=\frac{(kxy)}{(2x+1)}\];(b) A curve has the equation  \[y^2+3y \ln x=10\] Find the value of \[\frac{\mathrm{d} y}{\mathrm{d} x}\] at the point (e, 2) on the curve.;(c) A curve has the equation \[y=2\tan^2x-7\tan x\] Find;(i)	an expression for the gradient of the curve,;(ii)	the x-coordinate of each of the stationary points of the curve for which \[0\leq x\leq 2\pi\] radians.Answers:

ID: 200002001007
Content:
A particle moves in a straight line so that , at time t seconds after passing through a fixed point, its velocity \[v m s{-1}\] is given by \[v = 6 \cos 2t\] Find ;i) the two smallest positive values of t for which  the particle is at instantaneous rest,;ii) the distance between the positions of instantaneous rest corresponding to these two values of t,;iii) the greatest magnitude of the acceleration.Answers:

ID: 200002001008
Content:
img;The diagram shows part of the curve \[y=x+\frac{1}{x}\] and of the lines y = x, x = 2 and x = 3. Calculate;i. the area of the shaded region, ;ii. the volume obtained when the shaded region is rotated through \[360^{\circ}\] about the x-axis.Answers:

ID: 200002001009
Content:
The parametric equations of a curve are \[x=t^2-t, y=t^2+t\];At the point A on the curve t = - 1 and at the point B on the curve \[t=\frac{1}{2}\] ;The normals to the curve at A and B intersect at the point C.;i) Find the coordinates of C.;At the point D on the curve the tangent to the curve is parallel to the line AB .;ii) Find the value of t at D.;Find the cartesian equation of the curve.Answers:

\end{document}
