\documentclass{article}
\begin{document}
ID: 200904001001
Content:
A and B are acute angles that $$\sin  (A-B) = \frac{3}{8}$$ and $$\sin A\cos B = \frac{5}{8}$$. Without using a calculator, find the value of ;(i) $$\cos A\sin B$$;(ii) $$\sin (A+B)$$,;(iii) $$\frac{\tan A}{\tan B}$$.Answers:

ID: 200904001002
Content:
(i) Express $$\frac{7}{(2x^2 -x-6)}$$ in partial fractions.;(ii) Hence evaluate $$\int_3^9 \frac{7}{(2x^2 -x-6)}dx$$.Answers:

ID: 200904001003
Content:
(i) Use the substitution $$u=2^x$$ to express the equation $$8^x -2^{(x+2)} = 15$$ as a cubic equation in u.;(ii) Show that u=3 is the only real solution of this equation.;(iii) Hence solve the equation $$8^x -2^{(x+2)} =15$$.Answers:

ID: 200904001004
Content:
img;The diagram shows an isosceles triangle ABC in which AC=BC. Lines are drawn from $$A$$ and B to meet BC and AC at P and Q respectively. The lines AP and BQ intersect at X. Given that PC=QC, show that;(i) AXB is an isosceles triangle,;(ii) PX = QX.Answers:

ID: 200904001005
Content:
Write down the first three terms in the expansion, in ascending powers of x, of $$(2-\frac{x}{4})$$, where n is a positive integer greater than 2.;The first two terms in the expansion, in ascending powers of x, of $$(1+x)(2-\frac{x}{4})^n$$ are $$a+bx^2$$, where a and b are constants.;(ii) Find the value of n.;(iii) Hence find the value of a and of b.Answers:

ID: 200904001006
Content:
img;The diagram shows part of the curve $$y=1+2\cos x$$, meeting the  x-axis at the point A and B .;(i) Show that the x-coordinate of A is $$\frac{2\pi }{3} $$ and find the x-coordinate of B .;(ii) Find the total area of the shaded regions.Answers:

ID: 200904001007
Content:
Find the coordinates of all the points at which the graph of $$y=|3x-5|-2$$ meets the coordinate axes.;(ii) Sketch the graph of $$y=|3x-5|-2$$.;(iii) Solve the equation $$x=|3x-5|-2$$.;(Note: Please enter your answers in ascending order)Answers:

ID: 200904001008
Content:
A motorcycle is driven along a straight horizontal road. As it passes a point A the brakes are applied and the motorcycle slows down, coming to a rest at a point B. For the journey from A to B, the distance, s metres, of the motorcycle from A, t seconds after passing A, is given by $$s =400(1-e^{-\frac{t}{10}})-16t$$.;(i) Find an expression, in terms of t, for the velocity of the motorcycle during the journey from A to B. ;(ii) Find an expression, in terms of t, for the acceleration of the motorcycle during the journey from Ato B.;(iii) Find the velocity of the motorcycle at A.;(iv) Show that the time taken for the journey from A to B is approximately 9.163 seconds.;(v) Find the average speed of the motorcycle for the journey from A to BAnswers:

ID: 200904001009
Content:
img;The diagram shows a circle with centre C(2,-1) and radius 5.;(i) Given that the equation of the circle is $$x^2 +y^2+2gx+2fy+c=0$$, find the value of each of the constants g, f and c.;The points A and B lie on the circle such that the line AC is parallel to the x-axis and the line AB passes through the origin O;(ii) Write down the coordinates of A.;(iii) Find the equation of AB.;(iv) Find the coordinates of B.Answers:

ID: 200904001010
Content:
A curve is such that $$\frac{\mathrm{d} ^{2}y}{\mathrm{d} x^{2}}= 6x-6$$. The curve passes through the point (3,10) and at this point the gradient of the curve is 12. Find the coordinates of the stationary point of the curve and determine the nature of this stationary point.Answers:

ID: 200904001011
Content:
img;The diagram shows three fixed points O, A and D such that OA = 17cm, OD = 31cm and $$\angle AOD =90^{\circ}$$. The lines AB and DC are perpendicular to the line OC which makes an angle $$\theta$$ with the line OD. The angle $$\theta$$ can vary in such a way that the point B lies between the points O and C.;(i) Show that $$AB + BC + CD = (48\cos  \theta + 14 \sin  \theta)cm$$.;(ii) Find the values of $$\theta$$ for which AB+BC+CD = 49cm.;(iii) State the maximum value of AB+BC+CD and the corresponding value of $$\theta$$.Answers:

\end{document}
