\documentclass{article}
\begin{document}
ID: 199703003001
Content:
By means of the substitution \[y = 8^{x}\], or otherwise, find the exact values of x which satisfy the equation \[64^{x} - 5(8^{x}) + 4 = 0\].Answers:
1: x="\frac{2}{3}" or "0".

ID: 199703003002
Content:
img;A smooth curve passes through the points (1, 2),(1.5, 2.8),(2, 3.4),(2.5, 3.8) and (3, 3.6) as shown in the diagram. Use the trapezium rule with four intervals to find an approximate value for the area of the region between the curve, the x-axis and the lines x = 1 and x = 3. State with a reason, whether this approximate value is greater than or less than the true area of the region.Answers:
2: Area = "6.4".
3: Approximate value is "lesser" (lesser/greater) than the true area of the region because the curve is concave "downwards" (upwards/downwards).

ID: 199703003003
Content:
img;The diagram shows a cube with each side of length 1 unit. The base ABCD lies in a horizontal plane, the edges AE, BF, CG and DH are vertical, and EFGH is the top of the cube. Calculate the size of the angle which the diagonal BH makes with the base ABCD, giving your answer to the nearest degree.Answers:
4: \angle DBH = "35"^{\circ}.

ID: 199703003004
Content:
The points A, B and C have position vectors given respectively by \[a = 2i + 3j + 4k\], \[b = 5i - 2j + k\], \[c = 3i + 5j + 2k\]. Calculate angle ABC, giving your answer to the nearest degree.Answers:
5: \angle ABC = "24"^{\circ}.

ID: 199703003005
Content:
By completing the square, find the greatest and least values, as \[\Theta\] varies, of \[cos^{2} - cos\Theta + 6\].Answers:

ID: 199703003006
Content:
Solve the inequality \[|1 + x| < 3\]. Hence write down the solution of \[-4 < |1 + x| < 3\].Answers:
6: "-4" < "x" < "3".

ID: 199703003007
Content:
Given that \[6x^{3} + cx^{2} - 29x + d\] is divisible by both \[2x - 1\] and \[x + 3\], find the value of c and the value of d.Answers:

ID: 199703003008
Content:
Given that x is sufficiently small for \[x^{3}\] and higher powers of x to be neglected, and that \[\cos x - 4\sin x = 6x\], show that a quadratic equation for x is \[x^{2} + 20x - 2 = 0\]. Hence find an approximate value for x, giving your answer correct to 3 significant figures.Answers:
7: This answer hasn't been provided
8: Approximate value for x = "0.0995", since x is small..

ID: 199703003009
Content:
Find the general solution, in radians, of the equation \[7\sin^{2}x - \cos^{2}x = 1\].Answers:
9: x="n\pi \pm \frac{1}{6}\pi".

ID: 199703003010
Content:
The number 105840 can be expressed in prime factors as \[2^{4} \times 3^{3} \times 5^{1} \times 7^{2}\]. Excluding 1 and 105840, how many positive integers are factors of 105840? ;[Hint: one factor of 105840 is 90, which can be expressed as \[2^{1} \times 3^{2} \times 5^{1} \times 7^{0}\].Answers:
10: No. of positive factors of 105840 (excluding 1 and 105840) = "118".

ID: 199703003011
Content:
Express \[5\sin\theta + 12\cos\theta\] in the form \[R\sin(\theta + \alpha)\], where \[R\] is positive and \[\alpha\] is acute, giving the value of \[\alpha\] to the nearest \[0.1^\circ\]. Hence solve the equation \[6\sec\theta - 5\tan\theta = 12\], for values of \[\theta\] lying between \[0^\circ\] and \[360^\circ\], giving your answers to the nearest  \[0.1^\circ\].Answers:
11: R = "13".
12: \alpha = "67.38".
13: \theta = "85.1"^{\circ} or "320.1"^{\circ}.

ID: 199703003012
Content:
In this question, you may use the result \[\frac{\mathrm{d} }{\mathrm{d} x} \sec x = \sec x \tan x\]. It is given that \[y = \ln \sec x.\];(i) Prove that \[\frac{\mathrm{d}^{2} y}{\mathrm{d} x^{2}} = \sec^{2}x\].;(ii) Find Maclaurin's series for y in ascending powers of x, up to and including the term in \[x^{4}\].Answers:
14: 
15: Maclaurin's series for y(up to x^4) is "\frac{1}{2}x^2 + \frac{1}{12}x^4".

ID: 199703003013
Content:
a);img;img; A sector containing an angle of \[\frac{1}{4}\pi\] radians is cut from a circular piece of paper of radius r, as shown in Fig. 1. The straight edges of the paper which remains are then brought together, without overlapping, to form a cone, as shown in Fig. 2. The semi-vertical angle of this cone is denoted by \[\theta\]. Calculate the size of this semi-vertical angle, giving your answer in radians, correct to 3 significant figures.;;b);img; The point D on the side BC of triangle ABC is such that \[BD = \frac{2}{3} BC\]. Angle \[ABD = 60^\circ\], AB = c, BC = a, AC = b and AD = x (see Fig. 3).;(i) By considering triangle ABD, prove that \[9x^{2} = 9c^{2} - 4a^{2} + 6ax\].;(ii) By considering also triangle ACD, deduce that \[x = \frac{a^{2} + 3b^{2} - 3c^{2}}{3a}\].Answers:
16: 
17: 

ID: 199703003014
Content:
a) A piece of wire of length 8 cm is cut into two pieces, one of length x cm, the other of length (8 - x) cm. The piece of length x cm is bent to form a circle with circumference x cm. The other piece is bent to form a square with perimeter (8 - x) cm. Show that, as x varies, the sum of the areas enclosed by these two pieces of wire is a minimum when the radius of the circle is \[\frac{4}{4 + \pi}\] cm.;;b) A spherical balloon is being inflated in such a way that its volume is increasing at a constant rate of \[150cm^{3}s^{-1}\]. At time t seconds, the radius of the balloon is r cm.;(i) Find \[\frac{\mathrm{d} r}{\mathrm{d} t}\] when r = 50.;(ii) Find the rate of increase of the surface area of the balloon when its radius is 50cm. ;[The formulae for the volume and surface area of a sphere are \[V = \frac{4}{3}\pi r^{3}\], \[A = 4\pi r^{2}\].]Answers:
18: 
19: \frac{dr}{dx} = "\frac{0.015}{\pi}".
20: "6" cm^2/s

ID: 199703003015
Content:
A bank has an account for investors. Interest is added to the account at the end of each year at a fixed rate of 5% of the amount in the account at the beginning of that year. A man and a woman both invest money.;;a) The man decides to invest $$x$$ at the beginning of one year and then a further $$x$$ at the beginning of the second and each subsequent year. He also decides that he will not draw any money out of the account, but just leave it, and any interest, to build up.;(i) How much will there be in the account at the end of 1 year, including the interest?;(ii) Show that, at the end of n years, when the interest for the last year has been added, he will have a total of ;\[21(1.05^{n} - 1)x\] in his account.;(iii) After how many complete years will he have, for the first time, at least $$12x$$ in his account?;;b) The woman decides that, to assist her in her everyday expenses, she will withdraw the interest as soon as it has been added. She invests $$y$$ at the beginning of each year. Show that, at the end of n years, she will have received a total of \[\frac {1}{40} n(n + 1)y\] in interest.Answers:
21: At the end of 1 year, the account will have | "1.05x" |.
22: 
23: Required no. of years = "10".
24: 

ID: 199703003016
Content:
a) O is the origin and A is the point on the curve \[y = \tan x\] where \[x = \frac {1}{4} \pi\]. Show that the area of the region enclosed by the chord OA and the arc OA of the curve is \[\frac {1}{8} \pi - \ln \sqrt{2}\].;;b) A portion of the curve \[ay = x^{2}\], where a is a positive constant, is rotated about the vertical axis Oy to form the curved surface of an open bowl. The bowl has a horizontal circular base of radius r and a horizontal circular rim of radius 3r. Prove that the depth of the bowl is \[\frac {8r^{2}}{a}\].  Find the volume of the bowl in terms of r and a. Given that the volume of the bowl is \[\frac {1}{10} \pi a^{3}\], find the depth of the bowl in terms of a only.Answers:
25: 
26: 
27: Volume of the bowl in terms of r and a = "\frac{40 \pi r^4}{a}".
28: Depth of bowl = "\frac{2a}{5}".

ID: 199703003017
Content:
(i) Express \[f(x) = \frac {3x + 5}{(x + 1)(x + 2)(x + 3)}\] in partial fractions.;(ii) Given that x is sufficiently small for \[x^{3}\] and higher powers of x to be neglected, show that \[f(x) \approx \frac {5}{6} - \frac {37}{36}x + \frac {227}{216}x^{2}\].;(iii) Use the quadratic approximation in (ii) to estimate the value of \[\int_{0}^{0.1} f(x)dx\].;(iv) Prove, by induction or otherwise, that \[\sum_{r=1}^{n} \frac {3r + 5}{(r+1)(r+2)(r+3)} = \frac {7}{6} - \frac {3n + 7}{(n+2)(n+3)}\].Answers:
29: $f(x)$= "\frac{1}{x+1}+\frac{1}{x+2}-\frac{2}{x+3}"
31: $\int \limits_{0}^{0.1}{f\left( x \right) dx} \approx$ "0.0785".

ID: 199703003018
Content:
Find;(i) \[\int \frac {1}{1 + \cos 2x} dx\],;(ii) \[\int x^{2} e^{3x} dx\],;(iii)\[\int_{0}^{2} \frac {x^{2}}{1 + x^{6}} dx\], by using the substitution \[y = x^{3}\], giving your answer correct to 3 decimal places.Answers:
33: $\int \frac{1}{1+\cos 2x} dx$ = "\frac{1}{2} \tan{x}"+ c.
34: $\int x^2 e^{3x} dx$ = "\frac{x^2}{3} e^{3x} - \frac{2x}{9} e^{3x} + \frac{2}{27} e^{3x}"+ c.
35: $\int_0^2 \frac{x^2}{1+x^6} dx$ = "0.482".

ID: 199703003019
Content:
a) Given that \[x^{2} - 2xy + 2y^{2} = 4\], find an expression for \[\frac{\mathrm{d} y}{\mathrm{d} x}\] in terms of x and y. Find the coordinates of each point on the curve \[x^{2} - 2xy + 2y^{2} = 4\] at which the tangent is parallel to the x-axis.;;b) A curve is defined by the parametric equations \[x = t^{2}, y = t^{3}\]. Show that the equation of the tangent to the curve at the point \[P(p^{2}, p^{3})\] is \[2y - 3px + p^{3} = 0\]. Show that there is just one point on the curve at which the tangent passes through the point (-3, -5), and determine the coordinates of this point.Answers:
36: \frac{dy}{dx} = "\frac{x-y}{x-2y}".
37: Coordinates are ("2","2") and ("-2","-2").
38: 
39: 
40: Coordinate of the point at which the tangent passes through the point (-3, -5) = ("1","1").

\end{document}
