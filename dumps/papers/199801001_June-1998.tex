\documentclass{article}
\begin{document}
ID: 199801001001
Content:
Solve the simultaneous equations;4xy + 5 = 0,;3y + x = 1.Answers:

ID: 199801001002
Content:
Variables p and q are connected by the equation \[pq^2=144\]; Find an expression, in terms of q for \[\frac{\mathrm{d} p}{\mathrm{d} q}\] and hence find the approximate change in p as q increases from 6 to 6 + k, where k is small.Answers:

ID: 199801001003
Content:
img;The diagram shows part of the curve \[y=4\sqrt{x}\] and part of the line y = 2x, intersecting at O and at P. Find ; (a)	the coordinates of P,; (b)	the volume obtained when the shaded region is rotated through \[360^{\circ}\]  about the x-axis.Answers:

ID: 199801001004
Content:
A curve has the equation \[y=\frac{(3x-4)^5}{16}\]  Find;(a) an expression for  \[\frac{\mathrm{d} y}{\mathrm{d} x}\] and the equation of the tangent to the curve at the point where x = 2.;(b) A particle P moves along the curve. When P is at the point where x = 2, the x-coordinate of P is increasing at the rate of 0.02 units per second. Find the corresponding rate of change of the y-coordinate of P.Answers:

ID: 199801001005
Content:
(a)	Find the values of c for which the equation  \[4x^2-(c+2)x+c=1\] has equal roots.(Note: Please enter the answer with smaller value first) ;(b)	Find the range of values of k for which the curve \[y=\frac{16}{x}\] intersects the straight line y = k-x at two distinct points.Answers:

ID: 199801001006
Content:
A particle moves in a straight line so that, t seconds after passing through a fixed point O, its velocity, \[vms^{-1}\] is given by \[v=5t-3t^2+2\] The particle comes to instantaneous rest at the point Q. Find; i) the acceleration of the particle at Q,; ii) the distance OQ;iii) the total distance travelled in the time interval t = 0 to t = 3.Answers:

ID: 199801001007
Content:
Find;(a)	the coefficient of x in the expansion of \[(x-\frac{2}{x})^7\];(b)	the coefficient of \[x^3\]in the expansion of \[(2+5x)(1-\frac{x}{2})^8\]Answers:

ID: 199801001008
Content:
Show that \[\frac{1}{(1+\sin\theta)}+\frac{1}{(1-\sin\theta)}-=2\sec^2\theta\]Answers:

ID: 199801001009
Content:
Find the range of the function \[f:x \mapsto 2x^2-6x+7\] for the domain \[0\leq x\leq 4\]Answers:

ID: 199801001010
Content:
(a) Find the x-coordinate of each of the stationary points on the curve \[y=(2x+1)(4-x^2)\] and determine the nature of each of the stationary points.;(b) A sports club wishes to use 720 m of fencing to make six equal-sized rectangular courts placed adjacent to each other as shown in the diagram. Given that each court measures x m by y m, show that the total area of all six courts, \[A m^2\] is given by \[A=\frac{3}{4}(720x-9x^2)\] Given that x and y may vary, find the dimension of each court for which A is a maximum.Answers:

ID: 199801001011
Content:
img;The figure shows a trapezium ABCD in which AD is parallel to BC and AB is parallel to the y-axis. The coordinates of A, D and C are (2, 12), (6, 4) and (13, 0) respectively. The point X lies on BC such that \[\angle BXA=90^{\circ}\];(a)	Find the equation of BC and of AX.;(b)	Deduce the coordinates of B and of X.;(c)	Determine the ratio BC : AD in the form n : 1.;(d)	Find the length of AX and deduce the area of the trapezium ABCD.Answers:

ID: 199801001012
Content:
(a)	Solve, giving all the solutions from \[0^{\circ}\]  and \[360^{\circ}\]   ; i)	\[2 \sin x + 3 \cos x = 0\] ;(ii) \[2\tan^2y=5\sec y+1\];(b)	Find all the values of z, between 0 and 10, for which \[2\sin(\frac{(\pi z)}{4})=1\] where \[\frac{(\pi z)}{4}\]  is in radians.;(Note: Please enter your answers in ascending order)Answers:

ID: 199801001013
Content:
img;In the diagram OAXB is a sector of a circle, centre O, of radius 6 cm and CAYB is a sector of another circle, centre C, of radius 10 cm. The \[\angle OAB = 2.4 radians\] Calculate;(a)	the length of the arc AXB,;(b)	the area of the sector OAXB,;(c)	the length of the chord AB,;(d)	the \[\angle ACB\] in radians,;(e)	the perimeter of the shaded region,;(f)	the area of the shaded region.Answers:

ID: 199801001014
Content:
(a)	The position vectors of points A and B relative to an origin O are \[\binom{2}{3}\] and \[\binom{5}{1}\] respectively. The point C is such that OACB is a parallelogram. Find the value of \[\vec{AB}\ast \vec{AC}\]and hence find the  \[\angle ACB\]; (b)	The position vectors of three points P, Q and R relative to an origin O are p, q and 4p respectively. The point M is the midpoint of PQ and the point S lies on OM produced such that \[OS = \frac{8}{5}OM\] ;Express \[\vec{OS}\]and \[\vec{QR}\]in terms of p and q. Deduce that S lies on QR and find the ratio QS : SR.Answers:

ID: 199801001015
Content:
img;The diagram shows part of the curve \[y=\frac{x^{2}}{9}+\frac{x}{6}+k\] where k is a constant. P is the point on the curve where x = 6. Given that the area of the region A is \[23 units^2\] show that k = 2. The diagram also shows the normal to the curve at the point P. This normal meets the y-axis at Q. Find;i)	the coordinates of Q,;ii)	the area of the region B .Answers:

ID: 199801001016
Content:
Functions f and g are defined, for \[x\epsilon \mathbb{R}\] by ; \[f:x \mapsto3x-a\] ; \[g:x \mapsto \frac{b}{x}\]  \[x\neq 0\] ;where a and b are constants. Given that \[f^2(2)=10\] and \[fg(2) = 16\] find the value of a and of b .Answers:

ID: 199801001017
Content:
Using graph paper,draw accurately on the same diagram, for \[-3\leq x\leq 3\] the graphs of 2y = |x - 2| and y =x+|2x|. On each of the axis use 1 cm to represent one unit.;Hence, or otherwise, solve the equation\[\frac{(|x-2|)}{2}=x+|2x|\]Answers:

\end{document}
