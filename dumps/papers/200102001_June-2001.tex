\documentclass{article}
\begin{document}
ID: 200102001001
Content:
(a)	The sum of the remainders obtained when \[x^2+(k+8)x+k\] is divided by x-2 and when it is divided by x + 1 is 0. Find the value of k.;(b) Find the value of a and of b for which \[ax^3-11x^2+ax+b\] is exactly divisible by \[x^2-4x-5\] ;(c)	Find the x-coordinate of each of the points of intersection of the curve \[4x^2(x-1)\] with the line \[y = 19 x-10\];(Note: Please enter your answers in ascending order)Answers:

ID: 200102001002
Content:
img;a) The table shows experimental values of two variables, x and y, which are known to be related by the equation \[y=\frac{a}{x}+\frac{b}{x^{2}}\];Using the vertical axis for xy, plot xy against \[\frac{1}{x}\] and draw a straight line graph. Use your graph to estimate ;i) the value of a and b,;ii) the value of x for which \[y=\frac{13}{x}\];b) Variables x and y are related by the equation \[y^2=\frac{a}{x^{n}}\]; When lg y is plotted against lg x, a straight line graph is obtained passing through \[(\frac{1}{2},\frac{1}{4})\] and (1, -1) as shown in the diagram.;img; Find the value of a and of n.Answers:

ID: 200102001003
Content:
(a)	The value of a precious stone at the beginning of 1960 was $2500. This value increased continuously so that, after a period of t years, the value of the stone was given by the expression \[\$ 2500(1.05)^t\] Find;(i)	the value of the stone, to the nearest dollars, at the beginning of 1980.;(ii)	the year in which the value of the stone first reached \[\$10 000\];(b)	Solve the equations;(i) \[2\lg x=1+\lg \frac{(4x-15)}{2}\];(ii) \[4^y-7(2^y)=8\]Answers:

ID: 200102001004
Content:
A function f is defined by \[f:x \mapsto \ln(2x-1)\]for the domain x > 1. Express \[f^{-1}\] in similar form and state the domain and range of \[f^{-1}\]Answers:

ID: 200102001005
Content:
An arithmetic progression is such that the eighth term is twice the second term and the eleventh term is 18. Find;i) the first term and the common difference;ii) the sum of the first 26 terms;iii)the smallest number of terms of the progression whose sum exceeds 300.;b) A geometric progression, with a positive common ratio, has first term of 75 and a third term of 48.;Find ;i) the common ratio;ii) the sum to infinity;iii) the value of the largest term in the progression which is less than 0.2Answers:

ID: 200102001006
Content:
(a)	Find the gradient of each of the following curves at the point where x = 1.;(i) \[y=x^2e^{-x}\] ;(ii)	\[y=\frac{(2x+6)}{(3x-1)}\]`;(b)	Find the equation of the tangent to the curve \[y^2=x^2+2y+6\]  at the point (3, 5).;(c)	Given that \[y = A \sin 2x + B \cos 2x\] where A and B are constants, is a solution of the equation \[2 \frac{\mathrm{d} y}{\mathrm{d} x}+3y=11\cos 2x+2\sin 2x\] for all values of x, find the value of A and of B .Answers:

ID: 200102001007
Content:
img;The figure shows part of the curve \[y = 3 \sin x + 2 \cos x \] for \[0\leq x\leq \pi\] The points P and Q on the curve are such that y is a maximum at P and Q has coordinates (k, 1).;i) Express \[3 \sin x + 2 \cos x  \] in the form \[R\sin(x + \alpha)\]where \[R > 0 \]and \[0\leq \alpha\leq \frac{\pi }{2}\];ii) Find the coordinates of P.;iii) Find the value of k.;b) A curve has equation \[y=8\sin x\cos^3x\];Find an expression for \[\frac{\mathrm{d} y}{\mathrm{d} x}\] and hence find the coordinates of the stationary point in the interval \[0< x<\frac{\pi }{2}\] radians.Answers:

ID: 200102001008
Content:
Find;(i) \[\int(2+3x)^5dx\] ;(ii) \[\int 6e^{\frac{x}{2}}dx\];img;(b) The figure shows part of the curve \[y=2+\frac{8}{(x-1)}\] and the lines x = 2 and x = 5. Find, correct to 2 decimal places, the area of the shaded region.;img;(c) The figure shows the region bounded by the curve y = tan x, the x-axis and the line \[x=\frac{\pi }{4}\] Use the identity \[\sec^2x=1+\tan^2x\] to find, correct to 2 decimal places, the volume obtained when the region is rotated through \[2\pi \] radians about the x-axis.Answers:

ID: 200102001009
Content:
The parametric equations of a curve are \[x=t^2+2t-6, y=t^2-3t-4\];Find;i) an expression for \[\frac{\mathrm{d} y}{\mathrm{d} x}\]in terms of t,;ii) the coordinates of the stationary point on the curve,;iii) the coordinates of the point on the curve at which the tangent is parallel to the y-axis,;iv) the equation of the normal to the curve at the point where t = 1,;v) the value of t at the point where this normal meets the curve again.;Draw, on graph paper, using a scale of 1 cm to 1 unit on each axis, the graph of the curve for \[- 2<  t< 3\]Answers:

\end{document}
