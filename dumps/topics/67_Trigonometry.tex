\documentclass{article}
\begin{document}
ID: 199502002006
Content:
img; In a competition, the competitors follow a course OABC, as indicated on the diagram. ;; A is 1000 m due North of O.;;B is 1200 m from A on a bearing of $$120^{\circ}$$.;;;C is 1100 m from B and $$A \hat BC = 90^{\circ}$$.;;a) Using a scale of 1 cm to represent 100 m, make an accurate scale drawing of the course. [You should place the point O half way down the left hand side of a new page.] [2];;b) Use your drawing to find the bearing of O from C. [1];;c) Competitors are told that they have to go from C to a point X. The point X is within the quadrilateral OABC, is 600 m from C and is equidistant from AO and AB.;;c-i) On your scale drawing construct the locus of points which are within the quadrilateral OABC and;;c-i-a) 600 m from C, [1];;c-i-b) Equidistant from AO and AB. [2];;c-ii) Mark clearly the position of X. [1]Answers:

ID: 199502002009
Content:
img; A, B and C lie in a straight line on level ground. T is the top a vertical flagpole TC.;;a) John wants to find the height of the flagpole. He measures the angle of elevation of the top of the flagpole from A and finds that it is  . He then walks 7 m to B and finds that the angle of elevation is now  . Calculate;;a-i) $$A \hat TB$$, [1];;a-ii) the length of BT, [3];;a-iii) the height, TC, of the flagpole. [2];;b) On the opposite side of the flagpole from A and B, the ground slopes down to D such that . A rope is stretched from D, which is 15 m from C, to the point R, where CR = 12 m. Calculate the length of the rope DR. [4];;c) A second vertical flagpole DE is to be erected at D. Given that RE is horizontal, calculate the length of DE. [2]Answers:

ID: 199601002016
Content:
img;ABC is a triangle in which AC = 13 cm, BC = 11 cm and AB = 20 cm. N is a point on BC produced, where CN = 5 cm and AN = 12 cm.;;a) Explain why $$A \hat NB$$ is a right angle. [1];;b) Expressing your answers as fractions, find ;;b-i) cos $$A \hat BN$$, [1];;b-ii) sin $$A \hat CB$$. [1]Answers:

ID: 199602002008
Content:
img;In the diagram, A, B and C represent three towns. They are joined by straight roads. The distance AC = 20 km, BC = 15 km and $$A \hat CB = 110^{\circ}$$.;;a) Calculate;;a-i) the area of triangle ABC, [2];;a-ii) the distance AB, [5];;a-iii) the shortest distance from C to the road AB. [2];;b) ;img;A picnic place is situated at P where $$P \hat CB = 18^{\circ}$$ and $$B \hat PC = 140^{\circ}$$. Calculate the distance BP. [3]Answers:

ID: 199703002013
Content:
img;The bearing of B from A is $$072^{\circ}$$.;;a) Find the bearing of A from B. [1];;b) C is due South of B and BA = BC. Find the bearing of A from C. [2]Answers:

ID: 199703002016
Content:
img;In the diagram ACX is a straight line. ; AB = 12 cm, sin $$B \hat AC$$ = $$\frac{1}{2}$$ and sin $$B \hat CA$$= $$\frac{3}{4}$$.;;a) Write down the value of sin $$B \hat CX$$. [1];;b) Calculate BC. [3]Answers:

ID: 199704002003
Content:
A crane stands on level ground. It may be represented by a tower ABCD, of height 11m, and a jib BR. The jib is of length 20m and can rotate in a vertical plane about B. a vertical cable, RS, carries a load S. the diagrams show two possible positions of the jib, cable and load.;;a);img;Diagram I shows the situation when BS is horizontal and RS = 8m. Calculate;;a-i) The distance BS,;;a-ii) The angle that the jib, BR, makes with the horizontal.;;b) Diagram II shows another situation.;The jib, BR, has been rotated and the length RS increased.; The load, S, is now on the ground at a point 5m from A. Calculate;;b-i) The angle through which the jib has been rotated,;;b-ii) The length by which RS has increased.Answers:

ID: 199704002009
Content:
img;In the diagram, ABC represents a horizontal triangular field and AD represents a vertical tree in the corner of the field. A path runs along the edge BC of the field. ;;AB = 83 m, AC = 46 m and angle BAC = $$67^{\circ}$$.;;a) The angle of elevation of the top of the tree when viewed from B is $$14^{\circ}$$. ; Calculate the height of the tree. [2];;b) Calculate the length of the path BC. [4];;c) Calculate the area of the field ABC. [2];;d) Calculate the shortest distance from A to the path BC. [2];;e) Calculate the greatest angle of elevation of the top of the tree when viewed from any point on the path. [2]Answers:

ID: 199801002020
Content:
img;img;ABC is a triangle with AB = 5 cm, BC = 4 cm and angle ABC = $$120^{\circ}$$. AB is produced to D and angle BCD = $$90^{\circ}$$. Using as much information given in the table below as is necessary, calculate;;a) the area of triangle ABC, [2];;b) the length of BD. [3]Answers:

ID: 199802002004
Content:
img;Two coastguard stations, A and B, are 150 km apart with A due North of B. The coastguards are attempting to find the position of a ship. Radio signals indicate that this ship is ;I	on a bearing of $$146^{\circ}$$ from A, ;II within 100 km of B and ;III nearer to B than A. ; Mark the position of A as shown on the diagram.;;a) Using a scale of 1 cm to 50 km,;;a-i) mark the position of B,;;a-ii) draw the 3 loci, corresponding to I, II and III. [5];;b) On your drawing label the two extreme positions of the ship, $$S_1$$ and $$S_2$$. [1];;c) The bearing of the ship from B is $$x^{\circ}$$. By considering the two extreme positions, $$S_1$$ and $$S_2$$, of the ship, copy and complete the possible statement ... < x < .... [2]Answers:

ID: 199802002007
Content:
img;Three buoys, A, B and C, are positioned in a lake to provide a course for a yacht race. AB = 800 m, $$A \hat BC=32^{\circ}$$, $$B \hat AC=22^{\circ}$$ and N is the point on AB which is 200 m from A.;;a) Show that the distance AC is 524 m, correct to the nearest metre. [2];;b) Calculate the distance NC. [4];;c) A helicopter, H, is hovering at a point vertically above N.;;c-i) The angle of elevation of the helicopter from A is $$12^{\circ}$$. Calculate the height of the helicopter. [2];;c-ii) P is the point on AC which is nearest to the helicopter. Calculate the angle of elevator of the helicopter from P. [4]Answers:

ID: 199804002005
Content:
img;A triangular sheet of cardboard, ABC, has sides whose lengths are AB = 31 cm, AC = 53 cm and BC = 47 cm.;;a) Calculate $$A \hat CB$$, giving your answer correct to one decimal place. [4];;b) A triangle ABD is to be cut from triangle ABC. The area of triangle ABD is 300$$cm^2$$ and the length of BD is 23 cm. Calculate $$A \hat CB$$, giving your answer correct to one decimal place. [3]Answers:

ID: 199804002009
Content:
img;The diagram shows A, B, C and D, the four corners of a horizontal rectangular field ABCD. The corner B is 82 metres from A on a bearing of $$021^{\circ}$$ and C is 173 metres from A.;;a) Calculate ;;a-i) The bearing of C from B, [1];;a-ii) $$B \hat AC$$, [2];;a-iii) The bearing of C from A. [1];;b) A hot air balloon was hovering at E, which is vertically above C. The angle of elevation of the bottom of the balloon from D was $$35^{\circ}$$. Calculate ;;b-i) the height of the bottom of the balloon above C, [2];;b-ii) the angle of elevation of the bottom of the balloon from B. [4];;c) A bird hovering at a height of 40 metres above the field. It spots it prey on the ground at an angle of depression of $$63^{\circ}$$. Calculate the distance that the bird must fly to catch its prey. [2]Answers:

ID: 199901002014
Content:
img;In the diagram, ABD is a straight line, AB = 10 cm, BC = 6 cm and $$B \hat CD$$ is a right angle. $$C \hat BD$$ = x$$^{\circ}$$, where sin x$$^{\circ}$$ = 0.6, cos x$$^{\circ}$$ = 0.8 and tan x$$^{\circ}$$ = 0.75. Calculate;;a) CD, [1];;b) $$\cos A \hat BC$$. [1];;c) the area of triangle ABC. [2]Answers:

ID: 199901002020
Content:
img;In Figure 1, the points A, B, C and D are the centres of four spheres, each of radius 4 cm, which rest on a horizontal table. Each sphere touches two of the other spheres, so that ABCD is a square of side 8 cm.;;a) Given that N is the midpoint of AC, explain why $$AN^{2} =32$$. [1];;b) A fifth sphere, with centre E and radius 5 cm, is now placed on top of the other four spheres so that it touches each one of them, as shown in Figure 2. The centres of the five spheres form a pyramid, as shown in Figure 3.;;b-i) Write down the length AE. [1];;b-ii) Calculate the length EN. [2];;b-iii) Calculate the height of E above the table. [1]Answers:

ID: 199902002006
Content:
a);img;A, B and C are three points on horizontal ground. BT is a vertical mast of height 20 m. The top of the mast is joined to A and C by straight wires. Angle BCT = $$31^{\circ}$$. ;;a-i) Calculate the length of the wire CT. [2];;a-ii) Given that AB is 30 m, calculate the angle of elevation of T from A. [2];;b);img;In the triangle shown, XY = 5 cm, YZ = 7 cm and ZX = 6 cm. Calculate $$Y \hat XZ$$. [4]Answers:

ID: 199904002001
Content:
img;A ladder FT stands on horizontal ground at F and leans against a vertical wall at T.;The point W, on the ground, is vertically below T.;The ladder can be extended to various lengths.;The diagrams above show three positions of this ladder.;;a) In Diagram I, FT = 5.5 m and angle TFW = $$65^{\circ}$$. Calculate FW. [2];;b) In Diagram II, TW = 7.3 m and FW = 2.6 m. Calculate the angle which the ladder makes with the ground. [2];;c) In Diagram III, TW = 9.3 m and angle TFW = $$71^{\circ}$$. Calculate by how much the ladder has been extended from its original length of 5.5 m. [3]Answers:

ID: 199904002009
Content:
A radio mast AB, of height 20 m, stands at the top of a slope which is inclined at $$18^{\circ}$$ to the horizontal.;;a);img;The mast is supported by a wire AC attached to a point C on the slope, where BC = 30 m. Calculate;;a-i) Angle ABC, [1];;a-ii) The length of the wire AC. [4];;b);img;When the sun is in a certain position, the shadow cast by the mast lies down the slope, shown in the diagram by the line BD.;Given that angle ADB = $$42^{\circ}$$.;Calculate;;b-i) The angle of elevation of the sun, [1];;b-ii) Angle DAB, [1];;b-iii) The length of the shadow BD. [3];;c);img;The mast is supported by another wire AF.;The points B, E and F lie on horizontal ground.;Given that angle BEF = $$90^{\circ}$$, BE = 12 m and EF = 15 m, calculate the length of the wire AF. [2]Answers:

ID: 200001002005
Content:
img;The flag A is mapped onto the flag B by a clockwise rotation of $$z^{\circ}$$ about the centre P. Write down;;a) the value of z, [1];;b) the coordinates of P. [1]Answers:

ID: 200001002023
Content:
a);img;ABC is an equilateral triangle of side 2 units. AN is perpendicular to BC. Show that $$cos 60^{\circ}=\frac{1}{2}$$. [1];;b);img;In the diagram, which is not accurate: RRAnswers:

ID: 200002002002
Content:
img;The diagram shows A, B, C and D, the four corners of a horizontal rectangular field ABCD. AC = 110 m and  $$D \hat A=43^{\circ}$$.;;a) Calculate the length of DC. [2];;b) TC represents a vertical tree. The angle of elevation of T from A is $$17^{\circ}$$. Calculate;;b-i) the height of the tree, [2];;b-ii) the angle of elevation of T from D. [2]Answers:

ID: 200002002009
Content:
img;Two ships, Alpha and Beta, left a port, O, at noon. Alpha sailed at 12 km/h on a bearing of $$054^{\circ}$$. Beta sailed at 16 km/h on a bearing of $$130^{\circ}$$.;;a) At 4 p.m., Alpha was at A and Beta was at B. Calculate the distance AB. [4];;b) When Beta had travelled a total distance of 100 km it stopped at C.;img;;b-i) Calculate the time when Beta reached C.[2];;b-ii) Alpha continued on its course until it reached the point D, due North of C. Calculate the distance, OD. [4];;c) An island, R, is 100 km due south of the port, O.;img;Calculate the bearing of R from C. [2]Answers:

ID: 200003002015
Content:
img;In the diagram, ABC is a straight line, $$C \hat BD = 68.2^{\circ}, B \hat CD = 90^{\circ},$$ CD = 5 cm and AD = 8cm. Using as much of the information given in the table below as is necessary, find;;a) sin x, giving your answer as a fraction, [1];;b) cos y, [1];;c) BC. [1]Answers:

ID: 200004002005
Content:
img;The diagram shows four towns, A, B, C and D. C is 110 km due East of A. $$A \hat BC = 90^{\circ}$$ and AB = 70 km.;;a) Calculate;;a-i) the distance BC, [2];;a-ii) the bearing of B from A. [3];;b) Given that D is due South of C and that $$C \hat AD = 28^{\circ}$$, calculate the distance CD. [2];;c) An aircraft flies from A to B, then from B to C, then from C to D and finally from D to A. Calculate the total distance that it flies. [3]Answers:

ID: 200004002008
Content:
img;The diagram shows two horizontal triangular fields, ABC and ACD, which are surrounded by hedges.;It is given that DAB is a straight line, AC = 65 m, $$C \hat AB = 60^{\circ}$$ and $$A \hat BC = 72^{\circ}$$.;;a) Calculate the length of the hedge BC. [2];;b) The hedge AD has length 84 m. Calculate;;b-i) the area of $$\Delta ACD$$. Calculate;;b-ii) the length of the hedge CD. [4];;c) A vertical tree is growing at C. The angle of elevation of the top of the tree from A is $$14^{\circ}$$.;;c-i) Calculate the height of the tree. [2];;c-ii) A boy has climbed exactly half way up the tree. Calculate the angle of depression of D when viewed by the boy. [2]Answers:

ID: 200101002023
Content:
a);img;The diagram shows a hollow cone of height 10 cm. AB is a diameter of the circular top. C is the vertex of the cone and $$A \hat CB=60^{\circ}$$. Using as much of the information given in the table as is necessary, calculate the radius of the circular top. [2];;b);img;Diagram I shows another hollow cone whose sloping edge is of length t cm. The radius of the circular top is r cm. The cone is cut along its sloping edge and laid flat to form the sector OPQ of a circle of radius t cm as shown in Diagram II. ;;b-i) Find an expression, in terms of r, for the length of the arc PQ. [1];;b-ii) It is given that  t = 5r.;;b-ii-a) Calculate $$P \hat OQ$$. [2];;b-ii-b) Given also that $$t=\sqrt{40}$$, calculate the area of the sector OPQ, expressing your answer as a multiple of $$\pi$$. [2]Answers:

ID: 200103002017
Content:
img;A, B and C are three towns.;C is equidistant from A and B.;The bearing of C from A is $$132^{\circ}$$ and $$B \hat AC = 75^{\circ}$$.;Find;;a-i) the acute angle ACB, [1];;a-ii) the reflex angle ACB, [1];;b) the bearing of A from C, [1];;c) the bearing of A from B.    [1]Answers:

ID: 200103002021
Content:
img;The diagram shows three points A(-2, 7), B(-2, 2) and C(6, -4). Find;;a) the length BC,    [2];;b) the area of triangle ABC,    [2];;c) the value of $$\sin A \hat BC$$.    [1]Answers:

ID: 200104002001
Content:
img;In the diagram, AB is a vertical wall.;A beam, CD, of length 11 metres, rests with one end, D, on horizontal ground.;It is held in place by two cables, BC and BD.;Given that AD = 8 metres, BD = 15 metres and angle BDC = $$55^{\circ}$$, calculate;;a) the length AB,;;b) the length of the cable BC,;;c) the angle between the beam CD and the ground.Answers:

ID: 200201002007
Content:
img;The diagram shows a lighthouse, L, and two ports P and Q. Q is due east of L and $$P \hat LQ=80^{\circ}$$. P and Q are each 10 km from L. Find;;a) $$L \hat QP$$, [1];;b) the bearing of Q from P, [1];;c) the bearing of L from P. [1]Answers:

ID: 200201002017
Content:
a) Factorize completely 18rc - 3rd - 6tc + td. [2];;b) Solve the equation $$\frac{4}{x+3}=\frac{3}{2x}$$. [2]Answers:

ID: 200204002001
Content:
img;Diagram I shows a path, AC, in a park ABCD. It is given that AC = 530 m, BC = 370 m and that AC is perpendicular to BC.;;a) Calculate angle ABC.   [2];;b) Diagram II shows two other paths, AE and CE, in the park. Given that angle CAE = $$25^{\circ}$$ and angle AEC = $$90^{\circ}$$, calculate the length of AE.   [2];;c) Given also that angle ACD = $$70^{\circ}$$ and angle CAD = $$90^{\circ}$$, calculate;;c-i) the length of CD,   [2];;c-ii) the area of the park ABCD.   [3]Answers:

ID: 200204002008
Content:
img;The diagram shows four points, A, B, C and D, on a piece of horizontal land. ;It is given that AB = 45 metres, AD = 25 metres and BD = 28 metres.;;a) Calculate angle ADB.   [4];;b) Given also that CD = 22 metres and that angle ACD = $$33^{\circ}$$, calculate angle ADC.   [3];;c);img;The line BD is produced beyond D. Calculate the shortest distance from C to this extended line.   [2];;d) D is the foot of a vertical mast, DT. The angle of elevation of the top of the mast, T, from A is $$40^{\circ}$$. Calculate the angle of elevation of T from B.   [3]Answers:

ID: 200301002022
Content:
It is given that $$\sin 30^{\circ}=0.5$$ and $$\cos30^{\circ} = 0.866$$.;;a) Write down the value of;;a-i) $$\cos 150^{\circ}$$, [1];;a-ii) $$\cos 60^{\circ}$$. [1];;b) A triangle has sides of length 6 cm and 5 cm. The angle between these two sides is $$150^{\circ}$$. Calculate the area of the triangle. [2]Answers:

ID: 200302002001
Content:
a-i) Evaluate $$\frac{4.8^{2} -1.7^{2}}{4.8 \times 1.7}$$. [1];;a-ii) Find a value of x for which $$\sin x^{\circ}= \tan 12^{\circ}+\cos 46^{\circ}$$. [1];;b);img;The diagram shows a framework ABCD. AD = 2.2 m, BD = 1.9 m and $$BCD=42^{\circ}$$, $$A \hat BD=B \hat DC=90^{\circ}$$. Calculate;;b-i) $$A \hat DB$$, [2];;b-ii) BC. [3];;c) A vertical flagpole, 18 m high, stands on horizontal ground. Calculate the angle of elevation of the top of the flagpole from a point, on the ground, 25 m from its base. [2]Answers:

ID: 200302002009
Content:
img;The diagram shows the position of a harbor, H, and three islands A, B and C. C is due North of H. The bearing of A from H is $$062^{\circ}$$ and $$H \hat AB=128^{\circ}$$. HA = 54 km and AB = 31 km.;;a) Calculate the distance HB. [4];;b) Find the bearing of B from A. [1];;c) The bearing of A from C is 133$$^{\circ}$$. Calculate the distance AC. [4];;d) A lightship, L, is positioned due North of H and equidistant from A and H. Calculate the distance HL. [3]Answers:

ID: 200304002010
Content:
img;An aircraft waiting to land is flying around a triangular circuit ABC.;A, B and C are vertically above three beacons, X, Y and Z.;T is the control tower at the airport, and T, X, Y and Z lie in a horizontal plane.;BC = 18 km, CA = 22 km and AB = 24 km.;;a-i) The plane is flying at 200 km/h.;Calculate the time, in minutes and seconds, that the aircraft takes to complete one circuit.   [2];;a-ii) Calculate the largest angle of triangle ABC.   [4];;b) Z is due West of T. ;The bearing of X from Z is $$042^{\circ}$$ and the bearing of X from T is $$338^{\circ}$$.;;b-i) Find the angles of triangle TXZ.   [2];;b-ii) Calculate TX.   [2];;c) The aircraft is flying at a constant height of 2600 metres.;Calculate the angle of depression of the tower, T, from the aircraft when it is at A.   [2]Answers:

ID: 200403002008
Content:
img;In the diagram, BCD is a straight line. BC = 5 cm, AB = 12 cm, AC = 13 cm and $$A \hat BC=90^{\circ}$$. Find;;a) $$\tan B \hat AC$$, [1];;b) $$\cos A \hat CD$$. [1]  Give both answers as fractions.Answers:

ID: 200403002009
Content:
img;The diagram shows the positions of A and B. Find the bearing of;;a) A from B, [1];;b) B from A. [1]Answers:

ID: 200404002001
Content:
img;The diagram represents some beams which support part of a roof. AD and BC are horizontal and CDE is vertical. AC = 8 m, $$\vec{BAC}=78^{\circ}$$, $$\vec{ACD}=35^{\circ}$$ and  $$\vec{CAE}=90^{\circ}$$. Calculate the length of the beam;;a) AD, [2];;b) CE, [2];;c) AB. [3]Answers:

ID: 200404002008
Content:
img;The diagram shows the design of a company symbol. It consists of three circles. ; The smallest circle has centre O and radius 2x centimeters. The largest circle has centre O and radius 2y centimeters. The third circle touches both the other two circles as shown. The three regions formed are colored red, yellow and green as shown.;;a) Explain fully why the radius of the third circle is (x + y) centimeters. [2];;b) Write down, in terms of $$\pi$$, x and y expressions for the area of the region that is colored;;b-i) yellow, [1];;b-ii) green. [1];;c) The area of the green region is twice the area of the yellow region. Use this information to write down an equation involving x and y, and show that it simplifies to $$y^{2} -6xy=5x^{2} =0$$. [3];;d-i) Factorize $$y^{2} -6xy+5x^{2} $$. [1];;d-ii) Solve the equation $$y^{2} -6xy+5x^{2}=0$$, expressing y in terms of x. [2];;e) Calculate the fraction of the design that is colored yellow. [2]Answers:

ID: 200504002008
Content:
img;Three points A, B and C, lie on a horizontal field. Angle BAC = $$75^{\circ}$$ and the bearing of C from A is 217$$^{\circ}$$. AB = 72 m and AC = 60 m.;;a) Calculate;;a-i) the bearing of B from A, [1];;a-ii) BC, [4];;a-iii) angle ABC, [3];;a-iv) the bearing of C from B. [1];;b) A girl is standing at B is flying kite. The kite, K, is vertically above A. The string, BK, attached to the kite is at $$24^{\circ}$$ to the horizontal. Calculate the angle of elevation of the kite when viewed from C. [3]Answers:

ID: 200604002009
Content:
img;A vertical flagpole, BF stands at the top of a hill. AB is the steepest path up the hill. N lies vertically below B and $$A \hat NB=90^{\circ}$$. AN = 100 m and AB = 104 m.;;a) Show that BN = 28.6 m. [1];;b) It is given that $$F \hat AN=25^{\circ}$$.;;b-i) Write down the size of the angle of depression of A from F. [1];;b-ii) Calculate the height, BF, of the flagpole. [3];;c);img;The diagram shows three other straight paths (CB, DB and ACD) on the hill. The path ACD is horizontal and $$B \hat AC=N \hat AC=90^{\circ}$$. CN and DN are horizontal lines.;;c-i) Given that AC = 60 m, calculate $$B \hat CN$$. [4];;c-ii) Given that $$B \hat DN=10^{\circ}$$, calculate $$D \hat BA$$. [3]Answers:

ID: 200703002012
Content:
img;In the triangle ABC, angle ABC = $$90^{\circ}$$ and BC is produced to D.;;a) Write down $$\cos A \hat CD$$. [1];;b) Calculate the perpendicular distance from B to AC. [1]Answers:

ID: 200704002005
Content:
In the pentagon ABCDE, angles BCD, BDA and AED are each $$90^{\circ}$$ and the angle ADE is $$18^{\circ}$$. CD = 5 cm, BD = 6 cm and AD = 14 cm. Calculate;;a) angle CDB, [2];;b) AE, [2];;c) the radius of the circle through A, B and D. [2]Answers:

ID: 200704002007
Content:
img;In the diagram, the rectangle ABCD represents a vertical cliff face. The bottom of the cliff, AB, runs from West to East, and is at sea level. A yacht is in the sea at Y. Angle BAY = $$75^{\circ}$$, angle AYB = $$63^{\circ}$$ and AB = 35 m.;;a) Find the bearing of Y from B. [2];;b) Show that BY = 37.9 m, correct to three significant figures. [2];;c) Calculate the area of triangle ABY. [2];;d) Calculate the shortest distance from the yacht to the cliff. [2];;e) The angle of depression of the yacht when viewed from C is $$18^{\circ}$$.;;e-i) Find the height of the cliff. [2];;e-ii) Calculate the greatest possible value of the angle of elevation of the top of the cliff when viewed from the yacht. [2] Answers:

ID: 200803002009
Content:
img;In triangle ABC, AB = 15 cm, BC = 8 cm and AC = 17 cm.;;a) Explain why angle ABC is a right angle.;;b) BA is produced to D and AD = 5cm. ;;b-i) Find the area of triangle DAC.;;b-ii) Write down cos $$\angle DAC$$.Answers:

ID: 200803002023
Content:
The scale drawing in the answer space below shows the positions of the towns A, B and C.;img; A is due North of B.;;a) Find the bearing of C from A. ;;b) The town D is on a bearing of $$052^{\circ}$$ from A and on a bearing of 335^{\circ} from C. Find and label the position of town D.;;c) A television mast is to be erected equidistant from A, B and C. By constructing perpendicular bisectors, find and label the position of mast M.;;d) Give that AB = 60 km, calculate the distance from D to the mast.Answers:

ID: 200804002002
Content:
img;A rectangular advertising hoarding, ABCD, is strengthened by three struts, AQ, AP and PQ. ; AQ =9.4 m, AP = 12.1 m, angle PAQ = $$32^{\circ}$$ and angle QAD = $$49^{\circ}$$ Calculate;;a) AD;;b) PB;;c) The area of trangle APQ;;d) PQAnswers:

ID: 200903002009
Content:
img;The diagram shows a prism whose cross-section is an isosceles triangle. AB = AC = 7.43 cm, $$\angle BAC = 38^{\circ}$$ and the length of the prism is 20 cm. Calculate;;a) The area of the triangle ABC,;;b) The volume of the prism.Answers:

ID: 200903002019
Content:
img;a-i) Calculate $$\frac{494.6}{56.33 \times 98.12}$$, showing all the figures on your calculator display.;;a-ii) Give your answer correct to 1 decimal place.;;b) Sue is trying to estimate the distance across a river from A to B. She paces out the distance from B to C as 280 m and measures angle ACB as $$30^{\circ}$$. Calculate AB, giving your answer to a reasonable degree of accuracy.Answers:

ID: 200903002022
Content:
img;a) A and B lie on the circle centre O, of radius 10 cm. $$\angle AOB$$ = 2.3 radians. Find the length of major arc AB;;b) Find two values of x, in radians, such that $$sin x = \frac{1}{2}$$Answers:

ID: 200904002003
Content:
img;A crane stands on level ground. ; It may be represented by a vertical tower AE of height 6 m and a jib AB of length 12 m. ; A vertical cable hangs from B and is attached to a load on the ground at the point D. The jib inclined at an angle of $$20^{\circ}$$ to the horizontal line AC.;;a) Calculate BD.;;b) The load is lifted from D as the jib is rotated in a vertical plane about A. ; When the jib is in the position AG, the load is lowered to the point F on the ground, vertically below G. ; The line AC cuts GF at N. ; EF = 5m. ; Calculate;;b-i) GN,;;b-ii) The angle through which the jib has rotated.Answers:

ID: 200904002008
Content:
img;In the diagram, A is a point at sea level at the foot of a vertical cliff. ; Two buoys, B and C are on the surface of the sea. ; AB = 550 m and AC = 645 m. ; The bearings of B from A is $$062^{\circ}$$ and the bearing of C from A is $$100^{\circ}$$.;;a) Calculate;;a-i) BC,;;a-ii) angle ACB,;;a-iii)Bearing C from B;;b) D is a point at the top of the cliff vertically above A. The angle of elevation of D from C is $$7^{\circ}$$. Calculate;;b-i) AD,;;b-ii) The greatest possible angle of elevation of D from a point on BC.Answers:

ID: 201004002001
Content:
img;The diagram shows a field, ABCD, which is crossed by two paths, DM and DB. DM is perpendicular to AB and DB is perpendicular to BC. ; DM = 43 m, DB = 72 m, angle BDA = $$62^{\circ}$$ and angle BCD = $$23^{\circ}$$.;;a) Calculate;;a-i) MB, ;;a-ii) AB,;;a-iii) CD;;b) B is due east of A. ; Calculate the bearing of D from B.Answers:

\end{document}
