\documentclass{article}
\begin{document}
ID: 199501002008
Content:
Given that $$\vec{OP} = u$$, $$\vec{OQ} = v$$ and $$\vec{PR} = w$$, express $$vec{QR}$$ in terms of u, v and w. [2]Answers:

ID: 199702002008
Content:
a);img;In the diagram, OABCDE is a regular hexagon. $$\vec{OA}=a$$ and $$\vec{OC}=c$$. Express the following vectors, as simply as possible, in terms of a and/or c.;;a-i) $$\vec{CD}$$. [1];;a-ii) $$\vec{AC}$$. [1];;a-iii) $$\vec{OB}$$. [2];;b);img;In the diagram, A, B, C, and D are four markers on a horizontal field. BD = 15 m, DC = 12m, $$A \hat BD=35^{\circ}$$, $$A \hat DB=90^{\circ}$$ and $$B \hat DC=160^{\circ}$$.;;b-i) Calculate the distance AB. [2];;b-ii) Calculate the distance BC. [4];;b-iii) A vertical pole of height 10 m is positioned at D. Calculate the angle of elevation of the top of the pole from C. [2]Answers:

ID: 199703002017
Content:
img;On the grid in the answer space $$\vec{OA}$$ = a and $$\vec{OB}$$ = b. The point Y is also marked.;;a) On the grid mark clearly the point X, such that $$\vec{OX}$$ = 3b - 2a. ;;b) Write down $$\vec{OY}$$ in terms of a and b. [1];;c) It is given that $$\vec{OP}$$ = a + nb, where n takes all values from -3 to 3. On the grid draw the locus of P. [3]Answers:

ID: 199803002024
Content:
img;In the diagram, OABC is a parallelogram. The position vectors of the points A and B are given by $$\vec{OA} = \begin{bmatrix}5\\1\end{bmatrix}$$, $$\vec{OB} = \begin{bmatrix}8\\6\end{bmatrix}$$. ;;a) Find |$$\vec{OB}$$|. [1];;b) Express as column vectors;;b-i) $$\vec{AB}$$, [2];;b-ii) $$\vec{CA}$$. [2];;c) The point P lies on CA produced and $$\vec{AP} = h \vec{CA}$$.;;c-i) Show that $$\vec{OP} = \begin{bmatrix}5+2h\\1-4h\end{bmatrix}$$. [1];;c-ii) Given that P lies on the x-axis, find;;c-ii-a) the value of h, [1];;c-ii-b) the coordinates of P. [1]Answers:

ID: 199901002013
Content:
Given that $$u=\begin{bmatrix}6\\-8\end{bmatrix}$$, $$v=\begin{bmatrix}-9\\10\end{bmatrix}$$ and $$w=\begin{bmatrix}15\\p\end{bmatrix}$$, find;;a-i) $$|\u|$$, [1];;a-ii) 2u + v. [1];;b) Given that the vector w is parallel to the vector  u, calculate the value of p. [1]Answers:

ID: 199903002024
Content:
img;In the diagram, $$\vec{OA} = \begin{bmatrix}8\\6\end{bmatrix}$$ and $$\vec{OB} = \begin{bmatrix}3\\-2\end{bmatrix}$$.;;a) Find;;a-i) $$\mid\vec{OA}\mid$$, [1];;a-ii) $$\vec{BA}$$. [1];;b) Given that $$\vec{BC} = \begin{bmatrix}x\\9\end{bmatrix}$$ and $$\vec{OB} = k \vec{OA}$$, Find the value of;;b-i) k, [1];;b-ii) x, [1];;b-iii) the ratio $$\frac{area.of.triangle OAB}{area.of.triangle ACB}$$. [1]Answers:

ID: 200001002003
Content:
A is the point (7, 3) and B is the point (5, 11). Find;;a) the coordinates of the midpoint of AB, [1];;b) the vector $$\vec{AB}$$. [1]Answers:

ID: 200002002011
Content:
a) $$p=\begin{bmatrix}4\\-3\end{bmatrix}$$, $$q=\begin{bmatrix}-2\\9\end{bmatrix}$$ and $$r=\begin{bmatrix}1\\-2\end{bmatrix}$$.;;a-i) Find |\q|. [1];;a-ii) Express as a column vector;;a-ii-a) 2p + q;;a-ii-b) p - 2r. [2];;a-iii) Write down two facts about the vectors 2p + q and p - 2r. [2];;b);img;In the diagram, $$\vec{OA}=a$$ and $$\vec{OB}=b$$.$$\vec{OM}=3\vec{OA}$$ and $$\vec{ON}=2\vec{OB}$$. C is the point on MN produced where MN = NC.;;b-i) Express, as simply as possible, in terms of a and/or b,;;b-i-a) $$\vec{MO}$$, [1];;b-i-b) $$\vec{MN}$$, [1];;b-i-c) $$\vec{AB}$$, [1];;b-i-d) $$\vec{AC}$$. [2];;b-ii) Write down two facts which your answers to (c) and (d) tell you about A, B and C. [2]Answers:

ID: 200003002010
Content:
p = $$\begin{bmatrix}1\\-4\end{bmatrix}$$, q = $$\begin{bmatrix}-3\\4\end{bmatrix}$$ and r = $$\begin{bmatrix}m\\2\end{bmatrix}$$;;a) Find |q|. [1];;b) Express 2p - q as a column vector. [1];;c) Given that p is parallel to r, find m.Answers:

ID: 200102002010
Content:
img;In the diagram, $$\vec{AD}=p$$, $$\vec{DB}=q$$ and $$AC=2AD$$.;;a) Express, as simply as possible, in terms of p and q,;;a-i) $$\vec{AB}$$, [1];;a-ii) $$\vec{CB}$$. [1];;b) E is a point such that $$\vec{BE}=p$$.;;b-i) Express $$\vec{AE}$$ in terms of p and q. [1];;b-ii) Explain why ABEC is a trapezium. [1];;c) Given that |\p|=8, |\q|=7 and $$A \hat DB=57^{\circ}$$, calculate;;c-i) the perpendicular distance from B to AB, [2];;c-ii) the area of the trapezium ABEC, [2];;c-iii) the length of AB. [4]Answers:

ID: 200104002011
Content:
img;In the diagram, OAB is a triangle.;C is the point on AB such that AC:CB = 2:1.;The side OB is produced to the point D such that OB: BD=3:2.;It is given that $$\vec{OA} = a$$ and $$\vec{OB} = b$$;;a) Express, as simply as possible, in terms of a and/or b,;;a-i) $$\vec{AB}$$;;a-ii) $$\vec{AC}$$;;a-iii) $$\vec{OC}$$,;;a-iv) $$\vec{OD}$$.;;b) Show that $$\vec{CD} = b -\frac{1}{3} a$$.;;c) It is given that E is the point on OA such that $$\vec{OE} = \frac{5}{9} a$$.;Express, as simply as possible, in terms of a and b, the vector $$\vec{ED}$$.;;d-i) Show that $$\vec{ED} = k \vec{CD}$$, where k is a constant.;;d-ii) Write down two facts about ED and CD.;;e) Calculate $$\frac{the.area.of.triangle. AEC}{the.area.of.triangle. OEC}$$Answers:

ID: 200201002006
Content:
P is the point (1, 1) and Q is the point (5, -2).;;a) A translation maps P onto Q. Write down the column vector which represents this translation. [1];;b) Find the coordinates of the midpoint of PQ. [1]Answers:

ID: 200202002011
Content:
img;a) $$\vec{OP}= \begin{bmatrix}-9\\40\end{bmatrix}$$ and $$\vec(OQ)=\begin{bmatrix}3\\-16\end{bmatrix}$$. Find;;a-i) |$$\vec{OP}$$|, [2];;a-ii) $$\vec{PQ}$$. [1];;b) In the diagram, ABC and EBD are two straight lines. Angle EAB = angle CDB. AB = 2 cm, BC = 6 cm and BD = 4 cm.;;b-i) Explain why triangle ABE is similar to triangle DBC. [1];;b-ii) Explaining your working fully, show that BE = 3cm. [2];;b-iii) Write down, as a fraction in its lowest terms, the value of $$\frac{Area.of.triangle.ABE}{Area.of.triangle.DBC}$$ . [1];;b-iv) It is given that $$\vec{AB}=p$$ and $$\vec{DB}=q$$. Express each of the following in terms of p and/or q;;b-iv-a) $$\vec{BC}$$, [1];;b-iv-b) $$\vec{BE}$$, [1];;b-iv-c) $$\vec{AE}$$, [1];;b-iv-d) $$\vec{DC}$$. [1];;v) Use your answers to parts (iv)(c) and (d) to explain why AE is not parallel to DC. [1]Answers:

ID: 200204002005
Content:
img;In the diagram, A is the point (10, 1) and $$\vec{AB} = \begin{bmatrix}-8\\15\end{bmatrix}$$.;;a) Find;;a-i) $$\mid\vec{AB}\mid$$, [2];;a-ii) the coordinates of B.   [1];;b) The point C is (42, 16) and $$\vec{CD} = 3\vec{AB}$$. Find;;b-i) the coordinates of D,   [2];;b-ii) the vector $$\vec{AD}$$.   [1];;c) The point E is (k, 16).;;c-i) Find, in terms of k, the vector $$\vec{AE}$$.   [1];;c-ii) Given that AED is a straight line, find k.   [2];;d) Find $$\frac{Area.of.triangle.ABE}{Area.of.triangle.CDE}$$.     [2]Answers:

ID: 200301002010
Content:
$$a=\begin{bmatrix}-2\\4\end{bmatrix}$$, $$b=\begin{bmatrix}-3\\2\end{bmatrix}$$, $$c=\begin{bmatrix}u\\10\end{bmatrix}$$.;;a) Express 2a + b as a column vector. [1];;b) Given that the vector c is parallel to the vector a, calculate the value of u. [1]Answers:

ID: 200301002023
Content:
img;In the diagram, $$\vec{OP}=o$$, $$\vec{OQ}=q$$ and $$\vec{OR}=r$$. The midpoints of PQ and QR are E and F respectively.;;a) Express, as simply as possible, in terms of p and/or q,;;a-i) $$\vec{PE}$$, [1];;a-ii) $$\vec{OE}$$. [1];;b) Hence write down $$\vec{OF}$$. [1];;c) Find $$\vec{EF}$$. [1];;d) Write down two facts about EF and PR. [1]Answers:

ID: 200303002020
Content:
img;In the diagram, OABC is a parallelogram, $$\vec{OA} = 4p - q$$ and $$\vec{OC} = p+5q$$.;;a) Express, as simply as possible, in terms of p and q,;;a-i) $$\vec{BC}$$,   [1];;a-ii) $$\vec{AC}$$.   [1];;b) D is the point such that $$\vec{OD} = -p + 2q$$.;;b-i) Explain why $$\vec{AC}$$ is parallel to $$\vec{OD}$$.   [1];;b-ii) Given that the area of triangle OAC is 18 square units, find the area of triangle OCD.   [1]Answers:

ID: 200403002013
Content:
$$\vec{AB}= \begin{bmatrix}8\\-4\end{bmatrix}$$, $$\vec{BC}=\begin{bmatrix}6\\4\end{bmatrix}$$.;;a) Express $$\vec{AC}$$ as a column vector. [1];;b) It is given that $$\vec{CD}=\begin{bmatrix}-11\\h\end{bmatrix}$$. Find the two possible values of h which will make ABCD trapezium. You may use the grid below to help you with your investigation. [2];img;Answers:

ID: 200404002004
Content:
img;In the diagram, ABCD is a square. Points P, Q, R and S lie on AB, BC, CD and DA so that AP = BQ = CR = DS.;;a) Giving all your reasons, prove that;;a-i) PB = QC, [2];;a-ii) triangle BPQ is congruent to triangle CQR, [3];;a-iii) PQR is a right angle. [2];;b) Write down two reasons to prove that PQRS is a square. [2]Answers:

ID: 200503002012
Content:
img;In the diagram, AB is parallel to DC and $$A \hat CB = C \hat DA$$.;;a) Explain why triangles ABC and CAD are similar. [1];;b) Given that AB = 4 cm, BC = 7cm, AC = 6 cm and CD = 9 cm, calculate AD. [2]Answers:

ID: 200504002011
Content:
img;A regular hexagon, ABCDEF, has centre O. $$\vec{OA}=a$$ and $$\vec{OB}=b$$.;;a) Express, as simply as possible, in terms of a and/or b,;;a-i) $$\vec{DO}$$, [1];;a-ii) $$\vec{AB}$$, [1];;a-iii) $$\vec{DB}$$. [1];;b) Explain why |a|=|b|=|b-a|. [1];;c) The points X, Y and Z are such that $$\vec{OX}=a+b$$, $$\vec{OY}=a-2b$$ and $$\vec{OZ}=b-2a$$.;;c-i) Express, as simply as possible, in terms of a and/or b,;;c-i-a) $$\vec{AX}$$, [1];;c-i-b) $$\vec{YX}$$. [1];;c-ii) What can be deduced about Y, A and X? [1];;d) Express, as simply as possible, in terms of a and/or b, the vector $$\vec{XZ}$$. [1];;e) Show that triangle XYZ is equilateral. [2];;f) Calculate $$\frac{Area.of.triangle.OAB}{Area.of.triangle.XYZ}$$ . [2]Answers:

ID: 200603002012
Content:
img;In the diagram, $$\vec{OA}=4a$$, $$\vec{OC}=2c$$ and $$\vec{CB}=a$$.;;a) Express $$\vec{BA}$$ in terms of a and c. [1];;b) $$\vec{OP}=2a-\frac{4}{3}c$$. Explain why $$\vec{OP}$$ is parallel to $$\vec{BA}$$. [1];;c) Find $$\frac{Area.of.triangle.OBA}{Area.of.triangle.OPA}$$ . [1]Answers:

ID: 200704002009
Content:
img;a) The points L is (3, -2) and the point M is (9, 6).;;a-i) Find the gradient of the line LM. [1];;a-ii) Find the equation of the line LM. [2];;a-iii) Express $$\vec{LM}$$ as a column vector. [1];;a-iv) The point N is such that $$\vec{MN}=4 \vec{LM}$$ and O is the origin. Find the column vector $$\vec{ON}$$. [2];;b) In the diagram, $$\vec{OA}=a$$, $$\vec{OB}=b$$, $$\vec{AP}=\frac{1}{4}\vec{AB}$$ and $$\vec{BC}=3 \vec{OA}$$.;;b-i) Express each of the following in terms of a and b,;;b-i-a) $$\vec{OC}$$, [1];;b-i-b) $$\vec{AP}$$, [1];;b-i-c) $$\vec{OP}$$. [1];;b-ii) What can you deduce about O, P and C? [1];;b-iii) Find $$\frac{Area.of.\Delta OAP}{Area.of.\Delta OBP}$$. [2]Answers:

ID: 200803002021
Content:
img;In the diagram, $$\vec{OA}$$ = 3a and $$\vec{OB}$$ = 4b. ; X is the point on AB such that AX = $$\frac{3}{4}$$AB. ; Y is the point such that $$\vec{BY}= \frac{1}{3} \vec{OA}$$;;a) Find in the form pa + qb,;;a-i) $$\vec{AB}$$;;a-ii) $$\vec{AX}$$;;a-iii) $$\vec{OX}$$;;a-iv) $$\vec{XY}$$;;b) Use your answers to parts (a)(iii) and (a)(iv) to explain why O, X and Y lie in a straight line.Answers:

ID: 200903002026
Content:
img;The position vectors of A and B, relative to O, are 6a and 6b respectively. ; $$\vec{OL} = \vec{LA}$$ and $$\vec{BM} = \frac{1}{3} \vec{BA}$$;;a) Express each of the following in terms of a and b;;a-i) $$\vec{BM}$$;;a-ii) $$\vec{OM}$$;;a-iii) $$\vec{ML}$$;;b) Find the position vector of P, such that $$ \vec{LP} = 3 \vec{ML}$$;;c) Make two statements about the points, O, B and P.;;d) Find the position vector N such that LMBN is a parallelogram.Answers:

\end{document}
