\documentclass{article}
\begin{document}
ID: 199704003007
Content:
In a large population the proportion having blood group A is 35%. Specimens of blood from the first five people attending a clinic are to be tested. It can be assumed that these five people are a random sample from the population. The random variable X denotes the number of people in the sample who are found to have blood group A.;;a) Show that $$P(X \leq 2) = 0.765$$, correct to 3 decimal places.;b)Three such samples of five people are taken. Find;(i) The probability that each of these three samples has more than two people with blood group A;(ii) The probability that one of these three samples has exactly one person with blood group A, another has exactly two people with blood group A, and the remaining sample has more than two people with blood group A. One hundred such samples of five people are taken. Using a suitable approximation, find the probability that the number of these samples that contain two or fewer people with blood group A will be at least 70. Answers:
46: 
47: P((no. of samples out of 3 with X > 2 )= 3) = "0.013".
48: P(X=1, X=2 and X > 2) = "0.148".
49: P(Y \ge 70) = "0.951".

ID: 199904003008
Content:
The continuous random variable X has cumulative distribution function F given by  $$F(x) = 0$$  if $$x < 0$$   $$a( 6x^{2} - x^{3})$$  if $$0 \leq x \leq 4$$ 1 if $$x > 4$$  Where a is a constant. Find a.  Find the probability density function of X, and use the fact that the graph of this function is symmetrical about $$x=2$$ to write down the expectation of X.  Show that the variance of X is $$\frac{4}{5}$$.  The random variable S is the sum of 100 independent observations of X. Find $$P(S > 185)$$. Answers:
216: a = "\frac{1}{32}".
217: f(x) = "\frac{3}{32}(4x-x^2)", for 0 \leq x \leq 4 and f(x) = "0", otherwise.
218: E(X) = "2".
219: 
220: P( S > 185) = "0.953".

ID: 200004003007
Content:
img ;The continuous random variable X has probability density function given by ;f(x)= img;A sketch of the grah of the probability density function is given above. Show that $$P(X \leq x) = \frac{1}{2} + \frac{1}{4}x, for 0 < x \leq 2$$, and find a similar expression for $$P(X<=x)$$, for $$-1 \leq x \leq 0$$. Show that $$E(X) = \frac{1}{4}$$, and state the value of $$E(X - \frac{1}{4})$$. Find $$E(\sqrt{X+1})$$. 10 independent observations of X are taken. Find the probability that 8 of these observations are less than 1. Answers:
288: None
289: 
290: 
291: E(X-\frac{1}{4}) = "0".
292: E(\sqrt{X+1}) = "(\frac{1}{2}*\sqrt{3})+\frac{1}{6}".
293: P(8 out of 10 observations are less than 1) = "0.282".

ID: 200104003009
Content:
A company sends a leaflet to 8000 customers. The leaflet describes two offers and a special prize. The company estimates that the probability that a randomly chosen customer will claim the 'free offer' is 0.4, and that, independently, the probability that the customer will claim the 'cheap offer' is 0.2. Each free offer costs the company $5 and each cheap offer costs the company $3. Write down the expected numbers of customers claiming each offer and hence show that the expected total cost to the company is $20800. Using a suitable approximation, find the probability that the total cost to the company of the offers exceeds $20700. The company also estimates that the probability that a randomly chosen customer will claim the special prize is 0.0005. Each special prize costs the company $1000. Assuming independence, find the probability that the total cost to the company of the special prize exceeds $4000. Answers:
383: 
384: P(total cost of the offers exceeds $20700) = "0.658".
385: P(total cost of special prizes exceeds $4000) = "0.371".

ID: 200204003027
Content:
The random variable X has a binomial distribution with mean 12 and variance 8. Find P(X=12). Using a suitable approximation, find an approximate value for P(X>13).Answers:
471: P(X = 12) = "0.140".
472: P(X > 13) = "0.298".

ID: 200304003023
Content:
The random variable X has the binomial distribution B(5,p), where 0 < p < 1. It is given that Var(X) = $$\frac{1}{4}E(X)$$. Find $$E(X^2).$$Answers:
531: E(X^2) = "15".

ID: 200504003022
Content:
In the UK, the failure rate for the treatment by IVF is 80%. Find the probability that there are exactly 6 failures in 10 randomly chosen patients receiving the treatment. It is given that there are fewer than 8 failures in the 10 treatments. Find the conditional probability that there are exactly 6 failures. Give your answer correct to 3 decimal places.Answers:
652: P(X = 6) = "0.088".
653: P(X = 6 | X < 8) = "0.273".

\end{document}
