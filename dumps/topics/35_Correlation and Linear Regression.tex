\documentclass{article}
\begin{document}
ID: 199903003010
Content:
Expand \[(1 + y)^{14}\] as a series of ascending powers of y up to and including the term in \[y^3\]. Simplify the coefficients. In the expansion of \[(1 + x + kx^2)^{14}\], where k is a constant, the coefficient of \[x^3\] is zero. By writing \[x + kx^2\] as y, or otherwise, find the value of k.Answers:
177: \[(1 + y)^{14}\] = "1+14y+91y^2+364y^3".
178: k = "-2".

ID: 200004003012
Content:
Sketch the following graphs on a single diagram, stating the x-coordinates of all intersections with the x-axis and the equations of any asymptotes.;(i)  $$y = x( x^2  - 4 )$$.;(ii)  $$y = \frac{x + 1}{(x - 1)^2} $$. ;Use linear interpolation once on the interval [-1, 0] to obtain an approximation to a root of the equation  $$x( x^2  - 4 )$$. The Newton-Raphson method is to be used to find an approximation to another root of the equation. Use this method, with x = 2 as a first approximation, to obtain a second approximation to this root, giving your   answer correct to 2 places of decimals.  Answers:
308: None
309: None
310: Linear interpolation on the root of the equation x( x^2  - 4 ) = "-0.0769".
311: Newton-Raphson method on another root of the equation= "7".
312: Second approximation to the root use for Newton-Raphson method = "2.43".

ID: 200104003012
Content:
Sketch the graph of  $$y = \frac{6 - x}{4 - x}$$, stating the equations of the asymptotes. By first sketching the graph of a suitable function on your diagram, find an interval of the form [n, n + 1], where n is an integer, containing the larger root,  $$\alpha $$, of the equation  $$\frac{6 - x}{4 - x} = 28e^{- x} $$. Taking your value of n as the first approximation, use the Newton-Raphson process to obtain a second approximation to  $$\alpha $$. Give your answer correct to 2 places of decimals. On another diagram sketch the graph of  $$y = \left| \frac{6 - x}{4 - x} \right|$$. Use linear interpolation once, on the interval [5.5, 6], to find an approximation to a root of the equation  $$\left| \frac{6 - x}{4 - x}\right| = 28e^{- x}$$. Give your answer correct to 2 places of decimals.  Answers:
391: None
392: x = "4" is an asymptote.
393: y = "1" is an asymptote.
394: The interval \alpha \in ["6","7"].
395: Second approximation to \alpha = "6.13".
396: None
397: Linear interpolation on [5.5, 6]: \beta \approx "5.88".

ID: 200604003005
Content:
Using a graphical argument, or otherwise, show that the equation  $$x^3  + x = 100$$ has exactly one real root,  $$\alpha $$. Find a pair of consecutive integers between which  $$\alpha $$ lies, and use linear interpolation once to find an initial approximation,  $$x_1 $$, to  $$\alpha $$. Give your answer correct to 1 decimal place. The iteration  $$x_{n + 1}  = 3 ( 100 - x_n ) $$, with initial approximation  $$x_1 $$, converges. Explain why the iteration converges to  $$\alpha $$, and use the iteration to find  $$\alpha $$ correct to 3 decimal places. Answers:
714: 
715: Pair of consecutive integers between which  \alpha  lies = ["4","5"].
716: Linear interpolation = "4.5".
717: \alpha = "4.570".

ID: 200704003011
Content:
Research is being carried out into how the concentration of a drug in the bloodstream varies with time, measured from when the drug is given. Observations at successive times give the data shown in the following table.;img;;It is given that the value of the product moment correlation coefficient for this data is -0.912, correct to 3 decimal places. The scatter diagram for the data is shown below.;img;;Calculate the equation of the regression line of x on t.;Calculate the corresponding estimated value of x when t = 300, and comment on the suitability of linear model.;The variable y is defined by $$y = \ln x$$. For the variables y and t,;(i) calculate the product moment correlation coefficient and comment on its values,;(ii) calculate the equation of the appropriate regression line.;Use a regression line to give the ebst estimate that you can of the time when the drug concentration is 15 micro grams per litre.Answers:
812: Equation of the regression line of x on t: x = "-0.260t + 66.2".
813: Corresponding estimated value of x when t = 300 is "-11.7".
814: "not a good" (Not a good/Good) model because the concentration of the drug in the bloddsteam cannot be " negative" (negative/positive).
815: The product moment correlation coefficient is "-0.994".
816: This indicates a "stronger" (weaker/stronger) correlation is between y and t as it is "closer" (further/closer) to 1 than the value obtained from linear model.
817: Equation of regression line of y on t is y = "-0.0123t+4.62".
818: The best estimate of time when drug concentration is 15 mg \ell^{-1} is "155" min.

\end{document}
