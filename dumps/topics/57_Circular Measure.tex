\documentclass{article}
\begin{document}
ID: 199504001002
Content:
a) A geometric progression has first term a and common ratio r. The second term is -80 and the fifth term is 1250. Calculate ; i) the value of a and r,; ii) the sum of the first seven terms.; b) Find the common ratio of the geometric series whose first term is 18 and whose sum to infinity is 42;c) An arithmetic progression with 19 terms has first term a and common difference d. Show that the sum of the last five terms is 5a + 80d.;The last term of the progression is 52 and the sum of the first five terms added to the sum of the last five terms is 310. Calculate the value of a and of d.Answers:

ID: 199601001009
Content:
a) A circle of radius 1.5 cm has a sector with area \[6.25 cm^2\] Calculate the perimeter of this sector.;img;b) The diagram shows two points A and B on the circumference of a circle, centre O. Given that \[\angle AOB\] is 1.8 radians, find the ratio of the area of the segment, shown shaded in the diagram, to the area of the circle.Answers:

ID: 199701001015
Content:
img;The diagram shows the shape XAYBX formed by two intersecting circles. The radii of the circles, centres P and Q, are 5 cm and 3 cm respectively. At each of the points of intersection, A and B, the radius of one circle is perpendicular to the radius of the other.;(a)	Show that angle APB is approximately 1.08 radians.;(b)	Find the perimeter of the shape XAYBX.;(c)	Find the area of the shape XAYBX.Answers:

ID: 199703001009
Content:
img;The figure shows a circle, centre O, radius 10 cm, and a chord AB such that \[\angle AOB = \frac{2\pi }{5}radians\].;Calculate;(a)	the length of the major arc ACB,;(b)	the area of the shaded region.Answers:

ID: 199801001013
Content:
img;In the diagram OAXB is a sector of a circle, centre O, of radius 6 cm and CAYB is a sector of another circle, centre C, of radius 10 cm. The \[\angle OAB = 2.4 radians\] Calculate;(a)	the length of the arc AXB,;(b)	the area of the sector OAXB,;(c)	the length of the chord AB,;(d)	the \[\angle ACB\] in radians,;(e)	the perimeter of the shaded region,;(f)	the area of the shaded region.Answers:

ID: 199803001010
Content:
img;In the diagram, OAB is a sector of a circle, centre O, of radius 8 cm and angle AOB = 0.92 radians. ;The line AD is the perpendicular from A to OB . The line AC is perpendicular to OA and meets OB produced at C. Find;(a)	the perimeter of the region ADB, marked P,;(b)	the area of the region ABC, marked Q.Answers:

ID: 199901001005
Content:
img;The diagram shows a sector OAB of a circle, centre O and radius 20 cm. The \[\angle AOB = 0.6 radians\]and AC is perpendicular to OB . Find the area of the shaded region.Answers:

ID: 200103001006
Content:
img;The diagram shows a sector, AOB, of a circle, centre O, radius r cm, where the acute angle AOB is \[\theta radians\] ;Given that the perimeter of the sector is 14 cm and that the area of the sector is \[10 cm^2\] evaluate \[r\] and \[\theta\]Answers:

\end{document}
