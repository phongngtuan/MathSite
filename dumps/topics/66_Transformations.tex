\documentclass{article}
\begin{document}
ID: 199501002003
Content:
img;The diagram shows the points A(4, 4), B(6, 2) and C(0, 2). The line AB is mapped onto the line AB by an enlargement centre C, scale factor $$\frac{-1}{2}$$. On the diagram draw and label AB. [2]Answers:

ID: 199502002008
Content:
a) Given that P is the point (1, 1), $$\vec{PQ} = \begin{bmatrix}-3\\2\end{bmatrix}, \vec{PR} = \begin{bmatrix}5\\4\end{bmatrix}$$ and that T is the midpoint of QR, find;;a-i) $$\vec{QR}$$, [1];;a-ii) $$\vec{PT}$$, [1];;a-iii) the coordinates of the point X such that PQXR is a parallelogram. [2];;b) Answer this part of the question on a sheet of graph paper. The triangle ABC has vertices A(0, 1), B(1, 4) and C(2, 2).;;b-i) Using a scale of 1 cm to 1 unit on each axis, draw x and y axes for $$-6 < x \leq 10$$ and $$0 < x \leq 6$$. Draw and label $$\Delta ABC$$. [1];;b-ii) A transformation is represented by the matrix $$\begin{bmatrix}0&-1\\1&0\end{bmatrix}$$. The transformation maps $$\Delta ABC$$ onto $$\Delta DEF$$. Draw and label $$\Delta DEF$$. [2];;b-iii) $$\Delta ABC$$ is transformed onto $$\Delta LMN$$ by a shear, with the x-axis invariant and B(1, 4) mapped onto M(9, 4). Find the coordinates of N and the matrix which represents this shear. [3];;b-iv) Find the matrix representing the transformation which maps $$\Delta DEF$$onto $$\Delta LMN$$. [2]Answers:

ID: 199502002011
Content:
A large area is to be paved with blocks each one metre square. ;img; On Day 1, three blocks are placed in a line, as shown in the diagram. Each following day the paved region is enlarged by adding blocks to surround the previous day;;a) Find;;a-i) the perimeter after Day 4 and after Day 5, [1];;a-ii) an expression, in terms of n, in its simplest form, for the perimeter after Day n. [2];;b) Find;;b-i) the area after Day 4 and after Day 5, [2];;b-ii) an expression, in the form $$an^2 - b$$, for the area after Day n, [2];;b-iii) the total number of blocks which will have been used after Day 15, [1];;b-iv) after which day the area will be $$399 m^2$$. [2];;c) How many blocks will be added during Day 18? [2]Answers:

ID: 199601002023
Content:
img;a) A translation maps $$\Delta  A$$ onto $$\Delta  B$$. Write down the column vector representing this translation. [1];;b) A rotation maps $$\Delta  A$$ onto $$\Delta  C$$. Write down the coordinates of the centre of this rotation. [2];;c) An enlargement maps $$\Delta  A$$ onto $$\Delta  D$$. Write down the coordinates of the centre of this enlargement and its scale factor. [2];;d) A shear maps $$\Delta  A$$ onto $$\Delta  E$$. Write down the shear factor. [1]Answers:

ID: 199701002019
Content:
The diagram in the answer space shows the triangles A and B.;img;;a) Triangle A is mapped onto triangle C by the translation $$\begin{bmatrix}5\\-1\end{bmatrix}$$. On the diagram, draw an label triangle C. [1];;b) Describe fully the single transformation which maps triangle A onto triangle B. [3]Answers:

ID: 199702002009
Content:
The line l and the triangle ABC are shown on the graph.;img;;a-i) Write down the gradient of the line AB. [1];;a-ii) Find the equation of the line AB. [1];;a-iii) The line x = 2 is the axis of symmetry of the quadrilateral ABCD. Write down the coordinates of the point D. [1];;b) The triangle ABC is reflected in the line l.;;b-i) Find the coordinates of the image of the point C. [1];;b-ii) Write down the matrix which represents this reflection. [2];;c) An enlargement scale factor 2, maps triangle ABC onto triangle LMN. The point A maps onto the point L(2, 2).;;c-i) Find the coordinates of the centre of the enlargement. [1];;c-ii) Find the coordinates of N, the image of C. [2];;d) A shear, with the x-axis invariant, maps triangle ABC onto triangle PQR. The point A maps onto the point P (6, 2).;;d-i) Find the coordinates of Q, the image of B. [2];;d-ii) State the length of QR. [1]Answers:

ID: 199704002011
Content:
img;Triangle ABC has vertices A(-6, 12), B(8, 4) and C(2, 2).;;a) Triangle FGH has vertices F(-12, 6), G(-4, 8) and H(-2, 2). [4];;a-i) Describe fully the single transformation that maps $$\Delta ABC$$ onto $$\Delta FGH$$.;;a-ii) Write down the matrix that represents this transformation.;;b) Write down the column vector which represents the translation that maps C onto B. [1];;c) ABCP is a trapezium in which $$\vec{PA}$$ is parallel to $$\vec{CB}$$ and PA = $$\frac{1}{2}$$CB.;Find the coordinates of P. [2];;d) An enlargement maps $$\Delta ABC$$  onto $$\Delta ADE$$.;;d-i) Which point is the centre of the enlargement? [1];;d-ii) Given that D is the point (k, 0), find [4];;d-ii-a) the value of k,;;d-ii-b) the scale factor of the enlargement,;;d-ii-c) the coordinates of E,;;d-ii-d) the ratio of the area of $$\Delta ABC$$  to the area of $$\Delta ADE$$.Answers:

ID: 199802002009
Content:
img;The diagram shows triangles $$T_1$$, $$T_2$$, $$T_3$$ and $$T_4$$.;;a) Describe fully the single transformation which maps $$T_1$$ onto $$T_2$$. [2];;b) Find the matrix representing the transformation which maps $$T_1$$ onto $$T_3$$. [2];;c) $$T_1$$ is mapped onto $$T_4$$ by an enlargement. Find;;c-i) the scale factor of this enlargement.;;c-ii) the coordinates of the centre of this enlargement. [2];;d) The matrix $$\begin{bmatrix}2&1\\0&3\end{bmatrix}$$ represents the transformation which maps $$T_1$$ onto triangle $$T_5$$. Find;;d-i) the coordinates of the vertices of $$T_5$$, [2];;d-ii) the area of $$T_5$$ [2];;d-iii) the matrix representing the transformation which maps $$T_5$$ onto $$T_1$$. [2]Answers:

ID: 199804002011
Content:
Answer the whole of this question on a sheet of graph paper.;Triangle A has vertices (4, 1), (4, 5), and (6, 1).;Triangle B has vertices (-1, 0), (3, 0) and (-1, -2).;Triangle C has vertices (2, 1), (2, 5) and (3, 1).;;a) Using a scale of 1 cm to represent 1 unit on each axis, draw axes for values of x and y in the ranges $$-4\leq x \leq 12$$ and $$-4\leq y \leq 12$$. Draw and label the triangles A, B and C. [1];;b) Describe fully the single transformation which maps triangle A onto triangle B. [3];;c) Triangle C is mapped onto triangle A by means of a stretch of scale factor2 in which the y-axis is invariant. Find the matrix representing the transformation which maps triangle C onto triangle A. [2];;d) The transformation represented by the matrix $$\begin{bmatrix}1&0\\ \frac{1}{2}&1\end{bmatrix}$$maps triangle A onto triangle D.;;d-i) Draw and label triangle D. [2];;d-ii) Describe fully the single transformation which maps triangle A onto triangle D. [2];;e) Find the matrix that represents the single transformationAnswers:

ID: 199901002017
Content:
img;The diagram in the answer space shows $$\Delta  ABC$$, $$\Delta  PQR$$ and the point D.;;a) A translation maps the point A (3, 6) onto D (12, 6). Write down the column vector that represents this translation. [1];;b) An enlargement maps $$\Delta  ABC$$ onto $$\Delta  PQR$$. Write down the scale factor of this enlargement. [1];;c) A shear in which the x-axis is invariant maps $$\Delta  ABC$$ onto $$\Delta  DEF$$. Draw $$\Delta  DEF$$ on the diagram. [2]Answers:

ID: 199902002011
Content:
img;The diagram shows triangles A, B and C.;;a) Describe fully the single transformation which maps A onto B. [2];;b) The single transformation which maps A onto C is a stretch. Find;;b-i) the scale factor,;;b-ii) the invariant line. [2];;c) M is the matrix $$\begin{bmatrix}0&-1\\-1&0\end{bmatrix}$$. P is the matrix which represents a rotation of $$90^{\circ}$$ anticlockwise about the origin.;;c-i) The image of A under the transformation represented by M is K. Find the coordinates of the vertices of K. [2];;c-ii) Describe fully the single transformation represented by $$P^{-1}$$. [1];;c-iii) Find the matrix P. [2];;c-vi) By considering the effects of transformations of A, or otherwise, find the matrix Q such that M = QP. [3]Answers:

ID: 199903002022
Content:
img;The diagram in the answer space shows triangle A, triangle B and the line y = x.;Triangle A can be mapped onto triangle B by a combination of two transformations.;The first is a reflection in the line y = x.;;a) On the diagram, draw the triangle after this reflection. [1];;b) Describe fully the second transformation. [2];;c) State the matrix of the combined transformation which maps A onto B. [2]Answers:

ID: 199903002025
Content:
Two corners, A and B, of a horizontal triangular field are 240 m apart.;The diagram below is part of a scale drawing of the filed.;img;;a) Find the scale of the drawing in the from 1: n. [1];;b) Find the bearing of A from B. [1];;c) The third corner, C, of the field is south of AB.;It is 220 m from A and 170 m from B.;Using ruler and compasses only, find and label the position of C on the scale drawing.   [3];;d) A tree, T, in the field is equidistant from the three corners A, B and C.;;d-i) Showing your construction clearly, find and label the position of the tree.;;d-ii) Find the distance of the tree from the corners of the field.   [1]Answers:

ID: 200002002010
Content:
img;In the diagram, O is the origin, A is the point (3, 1), B is (3, 0), C is (5, 1), D is (1, 3) and E is (0, 3). The single transformation P maps $$\Delta  OAB$$ onto $$\Delta  ODE$$. The single transformation Q maps $$\Delta  OAB$$ onto $$\Delta  OCB$$.;;a-i) Describe P completely. [2];;a-ii) Find the matrix which represents P. [2];;b-i) What kind of transformation is Q? [1];;b-ii) The matrix which represents Q is $$\begin{bmatrix}1&n\\0&1\end{bmatrix}$$. Find the value of n. [2];;c) The points O, H and K are the images of O, A and B respectively under the transformation QP. Find;;c-i) the coordinates of H, [2];;c-ii) the matrix which represents QP, [2];;c-iii) the area of $$\Delta  OHK$$. [1]Answers:

ID: 200003002024
Content:
img;The diagram below shows three triangles, A, B and C.;;a) Describe fully the single transformation which maps triangle a onto;;a-i) Triangle B, [2];;a-ii) Triangle C. [2];;b) Triangle A is mapped onto triangle D by an enlargement with centre (1, 0) and scale factor -2. Draw triangle D on the diagram.Answers:

ID: 200004002003
Content:
img;In the diagram, AC = 8cm, CE = 4cm and the area of triangle EBC = $$4.2cm^2$$.;;a) Calculate the area of triangle ABC. [1];;b) An enlargement, with centre A, maps triangle ABC onto triangle ADE. Calculate the scale factor of the enlargement. [1];;c) Another enlargement, with centre E, maps triangle EBC onto triangle EFA. BC = 3.6cm. Calculate;;c-i) the length of AF, [1];;c-ii) the area of triangle EFA, [2];;c-iii) the area of triangle BAF. [2]Answers:

ID: 200102002011
Content:
a) A shop sells two packs of fireworks. Pack A contains 15 rockets and 25 fountains. Pack B contains 5 rockets, 20 mines and 10 fountains. The prices of single fireworks are: rockets $2, mines $3, fountains $1. This information can be represented by the matrices P and Q below. ; ; $$P=\begin{bmatrix}15&0&25\\5&20&10\end{bmatrix}, Q=\begin{bmatrix}2\\3\\1\end{bmatrix}$$;;a-i) Find PQ. [2];;a-ii) Explain what the numbers in your answer represent. [1];;b) The inverse of $$\begin{bmatrix}x&-1\\2&0\end{bmatrix}$$ is $$\begin{bmatrix}0&y\\-1&-2\end{bmatrix}$$. Find the values of x and the value of y. [3];;c);img;The diagram shows the triangles X, Y and Z. The single transformation D maps X onto Y. The enlargement E maps X onto Z.;;c-i) Describe D completely. [2];;c-ii) Find the coordinates of the centre of the enlargement E. [2];;c-iii) O is the origin. Find the coordinates of DE(O). [2]Answers:

ID: 200201002011
Content:
img;The diagram shows triangles A, B and C.;;a) An enlargement maps triangle A onto triangle B. Write down the scale factor of this enlargement. [1];;b) Describe completely the single transformation which will map triangle A onto triangle C. [2]Answers:

ID: 200201002015
Content:
img;In the diagram, TC, TD and AB are straight lines.;;a) Construct the locus of the points which are equidistant from TC and TD.;;b) Construct the locus of the points which are equidistant from A and B.;;c) The two loci meet at P. AB is a chord of a circle, centre P. Draw the circle. [4]Answers:

ID: 200304002011
Content:
img;Answer the whole of this question on a sheet of graph paper.;The diagram shows triangle A and the straight line y=x+4.;Triangle A has vertices (3, 2), (3, 4) and (4, 4).;;a) Using a scale of 1 cm to represent 1 unit on each axis, draw, on a sheet of graph paper, axes for values of x and y in the ranges $$-6 \leq x \leq 6$$ and $$-6 \leq y \leq 10$$.;Draw and label triangle A.;Draw the straight line $$y = x + 4$$.   [1];;b) The transformation M is a reflection in the line $$y = x + 4$$.;The transformation M maps triangle A onto triangle B, so that M(A) = B.;Draw and label triangle B.   [2];;c) Triangle C has vertices (-1, 2), (1, 2) and (1, 1).;The rotation R maps triangle A onto triangle C, so that R(A) = C.;Find;;c-i) The angle and direction of this rotation,;;c-ii) The centre of this rotation,;;d) Given that MR(A) = D, draw and label triangle D.   [2];;e) Triangle E has vertices (3, -1), (3, 1) and (4, 0).;The transformation L maps triangleAnswers:

ID: 200503002022
Content:
The diagram below shows the point P and triangles A, B and C.;img;;a) The reflection, M, maps $$\Delta  A$$ onto $$\Delta  B$$. Given that M(P) = Q, write down the coordinates of Q. [1];;b) The rotation R, maps $$\Delta  A$$  onto $$\Delta  C$$. Find;;b-i) the coordinates of the centre of this rotation, [1];;b-ii) the angle and direction of this rotation. [1];;c) The enlargement, E, with centre (-1, 0) and scale factor -2, maps $$\Delta  A$$  onto $$\Delta  D$$. Draw and label $$\Delta  D$$  on the diagram. [2];;d) Given that FE(A) = A, describe fully the single transformation F. [1]Answers:

ID: 200603002024
Content:
img;The diagram shows triangles T, B and C.;;a) The enlargement, with centre (0, 0) and scale factor 2, maps $$\Delta  T$$ onto $$\Delta  A$$. Draw $$\Delta  A$$ on the diagram above. [1];;b) Describe fully the single transformation which maps $$\Delta  T $$ onto $$\Delta  B$$. [2];;c) A reflection maps $$\Delta  T$$ onto $$\Delta  C$$. Write down the equation of the line of reflection. [2]Answers:

ID: 200703002025
Content:
img;The diagram shows the point P(1, 2) and triangles A, B and C.;;a) The translation, T, maps $$\Delta  A$$ onto $$\Delta  B$$. Write down the column vector representing T. [1];;b) An anticlockwise rotation maps $$\Delta  A$$ onto $$\Delta  C$$.;;b-i) Write down the coordinates of the centre of rotation. [1];;b-ii) On the diagram, draw the path that P would follow under this anticlockwise rotation. [1];;c) The enlargement, E, has centre (0, 1) and scale factor -3. Write down the coordinates of E(P). [1];;d) The shear, S, with the x-axis invariant and scale factor 2, maps $$\Delta  B$$ onto $$\Delta  D$$. Draw $$\Delta  D$$ on the diagram. [2]Answers:

\end{document}
