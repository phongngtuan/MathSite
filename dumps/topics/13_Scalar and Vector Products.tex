\documentclass{article}
\begin{document}
ID: 199803003013
Content:
Relative to a fixed origin O, the points A, B and C have position vectors given respectively by $$a = 2i + 3j - k$$, ;$$b = 5i - 2j + 3k$$, $$c = 4i + j - 2k$$. Find;(i) the length of AB, correct to 3 significant figures,;(ii) angle BAC, correct to the nearest degree,;(iii) the area of triangle ABC, correct to 3 significant figures. Show that, for all values of the parameter t, the point P with position vector $$p = (2 + 3t)i + (3 - 5t)j + (-1 + 4t)k$$ lies on the line through A and B. Find p such that OP is perpendicular to AB. Answers:
99: Length of AB = "7.07".
100: \angle BAC = "56"^{\circ}.
101: Area of \DeltaABC = "8.75" units^{2}.
102: 
103: p = "\frac{1}{50}(139i+85j+2k)".

ID: 199804003015
Content:
img; Three vertical flagpoles, OF, AG, BH, stand with their bases on horizontal ground. The flagpoles have heights 10m, 14m, 18m, and their bases are O, A, B respectively, where OA = 4m and OB = 8m, and angle AOB is a right angle. The point O is taken as the origin, with unit vectors i along OA, j along OB and k vertically upwards.;(i) Find, in the form ax + by + cz = 10, the equation of the plane FGH.;(ii) Find the angle between the plane FGH and the horizontal, giving your answer correct to the nearest \[0.1^{\circ}\].;(iii) Find the perpendicular distance from the mid-point of AF to the line GH, giving 3 significant figures in your answer.Answers:
165: Perpendicular distance from AF to GH = "8.77" units.

ID: 199903003006
Content:
The angle between the vector \[\lambda i + 3j - 6k\] and the vector I is \[120^{\circ}\]. Find the exact value of the constant \[\lambda\].Answers:
170: \lambda = "-\sqrt{15}".

ID: 199904003015
Content:
img;The diagram shows a circular table, with centre O, and three legs, AA', BB' and CC', attached at points A, B and C to the table. The lengths of the legs are adjusted so that, although the table is standing on a sloping floor, the plane ABC is horizontal. Perpendicular unit vectors i, j and k are defined, with i along \[\overrightarrow{OA}\], and k vertically upwards. The vectors \[\overrightarrow{AA'}\], \[\overrightarrow{BB'}\] and \[\overrightarrow{CC'}\] are parallel to 5i - 12k, -3i + 4j - 12k and -3i - 4j - 12k respectively. Show that all three legs are inclined at the same angle to the vertical. The plane A'B'C' has equation x - 10z = 130. Find the inclination of this plane to the horizontal, and the perpendicular distance from O to this plane. Referred to O as origin, the position vectors of A, B and C are 5i, -3i + 4j and -3i - 4j respectively. Find the lengths of the legs AA', BB' and CC'.Answers:
239: 

ID: 200004003015
Content:
Relative to the origin O, the points A, B and C have position vectors 5i + 4j + 10k, -4i + 4j - 2k, -5i + 9j + 5k, respectively.;(i) Find the cartesian equation of the line AB.;(ii) Find the length of the projection of  $$vec(AC) $$ onto the line AB.;(iii) Hence or otherwise find the perpendicular distance from C to the line AB, and the position vector of the foot N of the perpendicular from C to the line AB.;(iv) The point D lies on the line CN produced and is such that N is the mid-point of CD. Find the position vector of D. Answers:
325: Cartesian equation of the line AB = "4".
326: Length of projection of \overrightarrow{AC} onto AB = "10" units.
327: Perpendicular distance from C to AB = "5*\sqrt{2}" units.
328: \overrightarrow{ON} = "-i+4j+2k".
329: \overrightarrow{OD} = "3i-j-k".

ID: 200104003015
Content:
a)  The points P and Q have position vectors 3i - j + k and 9i - 7j - 2k respectively. Show that PQ = 9. Find the unit vector in the direction of  $$\vec{PQ}$$, and find also a cartesian equation for the line PQ. The line l, which passes through P, has equation  $$\frac{x - 3}{- 2} = \frac{y + 1}{1} = \frac{z - 1}{2}$$. Find;(i) the length of the projection of PQ onto l,;(ii) the length of the perpendicular from Q to l.;;b) By expanding (b-c) . (b-c), simplify  $$| b |^2  + | c |^2  -  (b-c)( b-c)$$. By taking  $$b = \vec{AC}$$ and  $$c = \vec{AB}$$, deduce the cosine formula for triangle ABC.Answers:
410: 
411: Unit vector in the direction of \overrightarrow{PQ} = "\frac{2}{3}i-\frac{2}{3}j-\frac{1}{3}k".
412: Length of the projection of PQ onto l = "8" units.
413:  Length of perpendicular from Q to l = "\sqrt{17}" units.
414: (b-c).(b-c) = "|b|^2+|c|^2 - 2bc".
415: Cosine formula for triangle ABC is "AC^2+AB^2-2(AC)(AB)\cos{\angle BAC}".

ID: 200403003003
Content:
Referred to an origin O, the position vectors of four non-collinear points A, B, C and D are a, b, c and d respectively. Given that a - b = d - c, show that ABCD is a parallelogram. Given also that |a - c| = |b - d|, identify the shape of the parallelogram ABCD, justifying your answer.  Answers:
551: None

ID: 200603003014
Content:
img;A curve has parametric equations x = ct, $$y=\frac{c}{t} $$, where c is a positive constant. Three points P $$\frac{(cp,c)}{p}$$, Q $$\frac{(cq,c)}{q}$$ and R $$\frac{(cr,c)}{r}$$ on the curve are shown in the diagram. ;(i) Prove that the gradient of QR is $$- \frac{1}{qr} $$.  ;(ii) Given that the line through P perpendicular to QR meets the curve at V $$\frac{(cv,c)}{v}$$ , find v in terms of p, q and r. ;(iii) Find the gradient of the normal at P.;(iv) The normal at P meets the curve again at S $$\frac{(cs,c)}{s}$$. Show that $$s= - \frac{1}{p^{3}} $$.;(v) Given that angle QPR is $$90^{\circ}$$, prove that QR is parallel to the normal at P.  Answers:
700: 
701: v = "-\frac{1}{prq}".
702: Gradient of normal at  P(t=p), m_P' = "p^2".
703: 
704: 

ID: 200603003015
Content:
The points A, B, C and D have position vectors i - 2j + 5k, i + 3j, 10i + j + 2k and -2i + 4j + 5k respectively, with respect to an origin O. The point P on AB is such that $$AP:PB= \lambda :1- \lambda $$ and the point Q on CD is such that $$CQ:QD= \mu :1- \mu $$. Find $$ \vec {OP}$$ and $$ \vec {OQ}$$ in terms of $$\lambda$$ and $$\mu$$ respectively. Given that PQ is perpendicular to both AB and CD,;(i) show that $$ \vec {PQ} =i+2j+2k$$, ;(ii) find the area of triangle ABQ. Answers:
705:  \overrightarrow{OP} = "i+(5\lambda -2)j+5(1-\lambda)k " .
706: \overrightarrow{OQ} = "(10-12\mu)i+(1+3\mu)j+(2+3\mu)k" .
707: 
708: Area of \Delta ABQ = "\frac{15}{\sqrt{2}}" units^2.

ID: 200703003006
Content:
Referred to the origin O, the position vectors of the points A and B are $$i -j +2k$$ and $$2i - 4j +k$$ respectively.;(i) Show that OA is perpendicular to OB;(ii) Find the position vector of the point M on the line segment AB such that AM:MV = 1:2;(iii) The point C has position vector $$-4i + 2j + 2k$$. Use a vector product to find the exact area of the triangle OAC Answers:
765: 
766: The position vector of the point M is "\frac{1}{3}(4i+2j+5k)".
767: Exact area of triangle OAC = "\sqrt{35}" units^2.

\end{document}
