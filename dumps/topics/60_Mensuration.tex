\documentclass{article}
\begin{document}
ID: 199501002025
Content:
In a factory, waste liquid is poured into cylindrical drums. The area of the base of each drum is $$2000 cm^2$$ and its height is 90 cm.;;a) Calculate the volume of each drum, giving your answer in cubic metres. [2];;b) When full, the drums are emptied into a tank. ;img; Both the base of the tank, DCGH, and its top, ABFE, are horizontal rectangles. Each of the vertical sides ABCD and EFGH is a trapezium. AB = EF = 4.5m, DC = HG = 3.5 m, AE = BF = CG = DH = 2 m and the perpendicular height of the tank is 1.5 m. Calculate;;b-i) the area of the trapezium ABCD, [2];;b-ii) the volume of the tank. [1];;c) How many full drums of waste can be emptied into the tank? [1]Answers:

ID: 199502002002
Content:
img;The two glasses shown in the diagram are geometrically similar. The height of the smaller glass is 8 cm. The height of the larger glass is 10 cm.;;a) The top of the larger glass has a circumference of 30 cm. Find the circumference of the top of the smaller glass. [2];;b) Both glasses are completely filled with fruit juice. The cost of the fruit juice in the smaller glass is 64 cents. Find the cost of the fruit juice in the larger glass. [3];;c) The volume, V cubic centimeters, and the height, h centimeters, of a third glass are connected by the formula  V= $$\frac{7}{16}h^3$$. The volume of this glass is 300 cm3. Find its height, correct to the nearest millimeter. [3]Answers:

ID: 199602002007
Content:
[The value of $$\pi$$ is 3.142, correct to 3 decimal places.] A large Traffic Marker consists of a solid cone, of height 40 cm and radius 9 cm, with a solid cylindrical base of diameter 30 cm and thickness 2 cm.;img;;a-i) Calculate the volume of the cone. [Volume of a cone = $$\frac{1}{3} \pi r^2h$$.] [2];;a-ii) Calculate the total volume of the Marker. [2];;b) Every part of the surface of the Marker is painted orange.;;b-i) Calculate the slant height of the cone and hence the area of the painted part of the cone. [Curved surface area of a cone = $$\pi r l$$, where l is the slant height.] [3];;b-ii) Calculate the area of the painted part of the base. [3];;c) A small Traffic Marker is geometrically similar to a large one, and the diameter of its base is 15 cm.;;c-i) Write down the ratio of the volume of a Small Marker to that of a Large Marker. ;;c-ii) Hence calculate the volume of a Small Marker. [2]Answers:

ID: 199602002009
Content:
Answer the whole of this question on a sheet of graph paper. The volume of an open rectangular box, made of thin metal, is $$7500cm^3$$. The lengths of the edges of the base of the box are 30 cm and x cm.;img;;a) Find, in terms of x, an expression for;;a-i) the area of the base of the box, [1];;a-ii) the height of the box. [1];;b) The total external area, of the base and the four sides, is $$Acm^2$$. Show that $$A = 500 + 30x + \frac{15000}{x}$$. [3];;c) The table below shows some values of x and the corresponding values of A. ;img; The values of A are given correct to the nearest integer, where appropriate. Using a scale of 2 cm to 5 units draw a horizontal x axis for $$10 \leq x \leq 45$$. Using a scale of 2 cm to 100 units draw a vertical A axis for $$1800 \leq A \leq 2300$$. Plot the points represented by the values in the table and join them with a smooth curve. [3];;d) Use your graph to find;;d-i) the range of values of x for which $$A\leq 2000$$, [2];;d-ii) the minimum value of A, [1];;d-iii) the height for minimum value of AAnswers:

ID: 199702002007
Content:
a);img;[The curved surface area of a cone of radius r and slant height l is $$\pi rl$$.] The diameter of the base of a cone is 16 cm. Given that the slant height of the cone is 10 cm, calculate;;a-i) the curved surface area of the cone, leaving your answer as a multiple of \pi, [1];;a-ii) the height of the cone. [2];;b) [The value of $$\pi$$ is 3.142 correct to three decimal places.];img;In Diagram 1, O is the centre of the circle of radius 6 cm and P is the mid-point of the chord AB. The length of OP is 3 cm.;;b-i) Show that $$A \hat OB=120^{\circ}$$. [1];;b-ii) Calculate the area of triangle AOB. [2];;b-iii) Calculate the area of the shaded sector. [3];;b-iv) Diagram 2 shows a cylindrical vessel resting on a horizontal surface.;img;The vessel, which has radius 6 cm and length 20 cm, contains liquid to a depth of 9 cm. Calculate the volume, in cubic centimeters, of liquid in the vessel. [3]Answers:

ID: 199704002007
Content:
img;[The value of $$\pi$$ is 3.142 correct to three decimal places.] ; Diagram I shows a barn and Diagram II shows the cross-section of its end. ;;A farmer needs to order a new roof for his barn. The roof is represented by ABC, the arc of a circle of radius r, centre O. ACDE is a rectangle.;The farmer measures AC, CD, BF and the length of the barn.;;a) Given that AC = 8m and BF = 2m,;;a-i) Write down an expression, in terms of r, for the length of OF, [1];;a-ii) Show that the radius, r, of the circle is 5 metres. [2];;b) Show that angle AOC is approximately $$106^{\circ}$$. [2];;c) Given that the length of the barn is 12 metres, calculate the curved area of the roof (shaded in Diagram I). [3];;d) Given also that CD = 7 metres, calculate the volume of the barn. [4]Answers:

ID: 199801002017
Content:
img;The diagram, which shows the sector AOB of a circle, represents a piece of card. The radius of the sector is 24 cm and the angle AOB is $$120^{\circ}$$.;;a) Calculate, as a multiple of $$\pi$$, the length of the arc AB. [2];;b) The card is used to make a hollow cone by joining the edges OA and OB. Calculate the radius of the base of the cone. [2]Answers:

ID: 199802002010
Content:
img;The volume of a rectangular block of candle wax is $$375cm^{3} $$. It has a length of 20 cm and a width of 7.5 cm.;;a) Calculate;;a-i) the height of the block, [1];;a-ii) the number of these blocks that can be packed into a box whose internal dimensions are 100 cm by 15 cm by 10 cm. [1];;b) The block is melted down and $$125cm^{3} $$ of the wax is poured into each of three separate moulds. The first mould is a cube, the second a cylinder of height 12 cm, and the third a pyramid with a square base of side x cm and height 2x cm.  ; [The volume of a pyramid = $$\frac{1}{3} \times Area Of The Base \times height$$. The value of $$\pi$$ is 3.142 correct to 3 decimal places.] ; Calculate;;b-i) the length of the edge of the cube, [1];;b-ii) the radius of the cylinder, [2];;b-iii) the value of x. [3];;c) The pyramid is melted down. The wax is used to make a number of pyramid candles, geometrically similar to the original pyramid, but with half its height. How many of these smaller pyramid can be made?Answers:

ID: 199803002020
Content:
[The value of $$\pi$$ is 3.14 correct to 2 decimal places.];;a) The inside radius of a cylindrical pot is 20 cm.;It contains water to a depth of 10 cm.;Calculate, in cubic centimeters, the volume of water in the pot. [2];;b) The inside radius of another cylindrical pot is 10 cm.;It contains $$2100 cm^3$$ of water.;Estimate, correct to the nearest centimeter, the depth of water in this pot. [2]Answers:

ID: 199804002007
Content:
img;[The value of $$\pi$$ is 3.142 correct to three decimal places.];;Diagrams I and II represent the cross-section of a video cassette. ;A tape runs from one circular spool, centre A, past B, C and D, to a second circular spool, centre E.;Each end of the tape is fixed to one of the spools, both of which have a radius of 1.2 cm.;Initially as much tape as possible is wound on to the spool with centre A.;It is represented on Diagram I by the shaded area, whose outer radius is 3.9 cm.;;a) Show that this shaded area is approximately 43.3 $$cm^2$$. [3];;b) Diagram II represents the situation when some of the tape has been wound onto the second spool. The total shaded area remains unaltered, so that it is always approximately 43.3 $$cm^2$$. At a certain time there are equal lengths of tape on each spool. Calculate the outer radius of the tape on one of the spools at that time. [4];;c) The tape runs at a speed of 23.4 millimetres per second past C. It takes 3 hours for the tape to run. What is the length of the tape?Answers:

ID: 199902002008
Content:
a);img;[The value of $$\pi$$ is 3.142 correct to 3 decimal places.] A well is a vertical open cylinder of radius 1.2 m and height 5 m. The well contains water to a depth of 3 m.;;a-i) Calculate the total internal area of the curved surface of the well and the bottom of the well. [3];;a-ii) Calculate the volume, in litres, of water in the well. [$$1m^{3} $$ = 1000 litres] [2];;a-iii) 270 litres of water is removed from the well. Calculate, correct to the nearest centimeter, the resulting fall in the water level. [3];;b) At 9 a.m. a storage tank contained 400 litres of water. Water was removed from the tank at a constant rate of 5 litres per minute. At 10 a.m. 270 litres of water was added. A further 270 litres was added at the end of each hour after that.;;b-i) Calculate the number of litres of water in the tank at 10.05 a.m. [2];;b-ii) Calculate the time when the tank was empty for the first time. [2]Answers:

ID: 199903002008
Content:
a) The volume of a cube is 200 cubic centimeters. Find the length of an edge of the cube, correct to the nearest centimeter. [1];;b) The volume of a sphere of radius r is $$\frac{4}{3} \pi r^3$$. Find, correct to one significant figure, the volume of a sphere with radius 10 cm. [1]Answers:

ID: 199904002007
Content:
img;[The value of $$\pi$$ is 3.142 correct to three decimal places];Earth is excavated to make a railway tunnel.;The tunnel is a cylinder of radius 5 m and length 450 m.;;a) Calculate the volume of earth removed. [2];;b) A level surface is laid inside the tunnel to carry the railway lines.;The diagram shows the circular cross-section of the tunnel.;The level surface is represented by AB, the centre of the circle is O and angle AOB = $$90^{\circ}$$.;The space below AB is filled with rubble. Calculate;;b-i) The area of triangle AOB, [1];;b-ii) The volume of rubble used in the 450 length of tunnel. [4];;c) Steel girders are erected above the tracks to strengthen the tunnel.;Some of these are shown in the diagram above.;The girders are erected at 6 m interval along the length of the tunnel, with one at each end.;;c-i) How many girders are erected? [2];;c-ii) Calculate the length of each girder. [2];;c-iii) Calculate total length of steel required in the 450 length tunnelAnswers:

ID: 200002002007
Content:
img;[The value of $$\pi$$ is 3.142, correct to three decimal places.] [1 litre = $$1000cm^{3} $$.] Some identical bowls are open cylinders each of radius 6 cm and height 4 cm. Each bowl is made from thin metal.;;a) Calculate the area of metal needed to make each bowl, giving your answer correct to the nearest square centimeter. [3];;b);img;A hemispherical pan contained 13 litres of soup. As many bowls as possible are completely filled with soup from the pan.;;b-i) Calculate the number of bowls which are filled. [3];;b-ii) Calculate the volume of soup which is left in the pan, giving your answer in cubic centimeters. [2];;b-iii) [The volume of a sphere of radius r is $$\frac{4}{3}\pi r^{3} $$.] It is given that 13 litres of soup completely filled the pan. Calculate the radius of the hemisphere, giving your answer correct to the nearest millimeter. [2];;c) Michael has two different bowls, which are geometrically similar to each other. The heights of the bowls are in the ration 2 : 3. Write down the ratio of their volumeAnswers:

ID: 200003002003
Content:
img;A rectangular block of wood has dimensions 24 cm by 9 cm by 7 cm. It is cut up into children of dimensions 3cm by 3cm by 3cm;;a) Find the largest number of bricks that can be cut from the block. [1];;b) Find the volume of wood that is left. [1]Answers:

ID: 200003002013
Content:
img;[The value of $$\pi$$ is 3.14 correct to three significant figures.] In the diagram, the circle, centre O, passes through A and B. The radius of the circle is 4 cm and $$A \hat OB = 45^{\circ}$$.;;a) Find the area of the minor sector AOB. [2];;b) The tangent at A meets OB produced at T. Find the shaded area. [1]Answers:

ID: 200004002007
Content:
img;[The value of $$\pi$$ is 3.142 correct to three decimal places.];Diagram I shows an open rectangular box of height 15 cm.;The box contains 10 cylindrical tins.;The tins touch one another and the sides of the box.;Each tin has radius 6 cm and height 15 cm.;;a-i) Each tin has a label wrapped round it, which exactly covers its vertical surface. Calculate the area of paper needed for one label. [2];;a-ii) Calculate the volume of one tin. [2];;b) Diagram II shows the view of the box and the tins from above.;PQRS is the rectangular cross-section of the box.;The points A, B and C are the centres of the circular tops of three adjacent tins which touch one another. The midpoint of AB is N.;;b-i) Write down the length of AC. [1];;b-ii) Calculate;;b-ii-a) the length of CN, [2];;b-ii-b) the length of PS. [2];;c) The height of the box is also 15 cm. Calculate the volume of the space in the box which is not occupied by the tins. [3]Answers:

ID: 200102002007
Content:
img;[The value of $$\pi$$ is 3.142, correct to three decimal places.] ; [The volume of a sphere is $$\frac{4}{3}\pi r^{3} $$.] ; Sarah makes biscuits. The amount of mixture required to make one biscuit is $$18cm^{3} $$. Before it is cooked, the mixture is rolled into a sphere.;;a) Calculate the radius of the sphere. [2];;b-i) After it is cooked, the biscuit becomes a cylinder of radius 3 cm and height 0.7 cm. The increase in volume is due to air being trapped in the biscuit. ; Calculate the volume of air contained in a biscuit. [2];;b-ii) Express this volume of air as a percentage of the total volume of the mixture. [2];;c-i);img;The biscuits are packed in a box. The cross-section of the box is a regular hexagon, containing 7 biscuits, arranged as shown in Diagram I.;;Three of the biscuits are shown in Diagram II. O is the centre of the hexagon and of the middle biscuit. B is the point where two biscuits touch. A and C are the centres of these biscuits. E is the midpoint of the side DF of the. Estimate the value of DF Answers:

ID: 200104002005
Content:
img;[The value of $$\pi$$ is 3.142 correct to three decimal places.];A water tank, shown in Diagram I, is a circular cylinder of radius 24cm and height 125cm.;It is open at one end and full of water.;;a) Calculate;;a-i) the volume in litres, of water in the tank,;;a-ii) the total area, in square metres, of the outside of the open tank.;;b) Diagram II shows a rectangular trough of length 150cm and width 20cm.;The trough was completely filled with $$48 000cm^3$$ of water from the tank.;Calculate the depth of the trough.;;c) After the trough had been filled, water started to leak from the tank.;In 2 hours 30 minutes it was found that 20 000 cm3 ran out of the tank.;Calculate the rate at which the level of water in the tank was falling.;Express your answer in centimetres per hour.Answers:

ID: 200201002024
Content:
a);img;The diagram shows a 10 cm cube.;;a-i) A triangular pyramid is cut from the corner of the cube at A. The cut is made halfway along each of the edges meeting at A as shown. Calculate the volume, in cubic centimeters, of the pyramid. [The volume of a pyramid = $$\frac{1}{3} \times area.of.base \times height$$] [2];;a-ii);img;From another 10 cm cube, a second similar pyramid is cut from the corner at P. The volume is 8 times the volume of the first pyramid. On the diagram, draw the lines where the cut is made.[1];;b) Another 10 cm cube is cut as shown. A prism containing the corners B and C is removed. Calculate the volume which remains. [2]Answers:

ID: 200202002004
Content:
a) Show that the interior angle of a regular hexagon is $$120^{\circ}$$. [2];;b);img;In the diagram, ABCDEF is a regular hexagon. ABPQ and FARS are two squares.;;b-i) Calculate;;b-i-a) reflex $$P \hat BC$$, [1];;b-i-b) obtuse $$P \hat AS$$ , [2];;b-i-c) acute $$R \hat BA$$. [2];;b-ii) What is the special name given to triangle AQR? [1]Answers:

ID: 200202002006
Content:
Answer the whole of this question on a sheet of graph paper. ; The masses of 80 parcels sent out by a garden centre are given in the table below.;img;;a) Using a scale of 1 cm to represent 1 kg, draw a horizontal axis for $$0\leq m\leq 15$$. Choose a suitable scale for the vertical axis and draw a histogram to represent this data. [3];;b) Estimate the number of parcels which had a mass greater than 9 kg. [1];;c) Calculate an estimate of the mean mass. [3];;d) One parcel was chosen at random and not replaced. A second parcel was chosen at random from the remainder. Giving each answer as a fraction in its lowest terms, find the probability that;;d-i) both parcels were chosen from the $$6<m\leq 10$$ group, [1];;d-ii) one parcel was chosen from the $$6<m\leq10$$ group and the other parcel was not chosen from the $$6<m\leq10$$ group. [2]Answers:

ID: 200203002013
Content:
img;In the diagram, OAB is a quadrant of a circle, radius 14 cm.;A semicircle is drawn on OB as diameter.;Taking $$\pi$$ to be $$22/7$$, calculate;;a) the arc length AB,   [1];;b) the perimeter of the shaded part of the diagram.   [2]Answers:

ID: 200204002009
Content:
img;[The area of the curved surface of a cone of radius r and slant height l is $$\pi$$ rl. ;;;The volume of a cone is $$\frac{1}{3} \times base.area \times height$$.] ;Diagram I shows a traditional hut which consist of a circular cylinder with an overhanging roof. The roof is thecurved surface of a cone and is supported by a central vertical pole. ;Diagram II shows a vertical cross-section of the hut. ;BE and CD are horizontal. ;AN = 4.0 m, BM = ME = 3.6 m and BC = DE = 1.3 m.;;a) Show that AB = 4.5 m.   [1];;b) Calculate;;b-i) the volume of the inside of the hut,   [3];;b-ii) the total surface area of the inside of the hut (including the floor).   [4];;c) The sun is directly overhead.;The shadow of the overhanging section of the roof on the ground is a circular ring around the hut.;AP = AQ = 5.5 m.;Calculate;;c-i) PQ,   [2];;c-ii) the area of the circular ring of shadow outside the hut. (Ignore the thickness of the walls.)   [2]Answers:

ID: 200302002007
Content:
img;[The value of $$\pi$$ is 3.142, correct to three decimal places.] ; [The surface area of a sphere is $$4 \pi r^{2} $$.]; [The volume of a sphere is $$\frac{4}{3} \pi r^{3} $$.] ; A closed container is made by joining together a cylinder of radius 9 cm and a hemisphere of radius 9 cm as shown in Diagram I. The length of the cylinder is 18 cm. The container rests on a horizontal surface and is exactly half full of water.;;a) Calculate the surface area of the inside of the container that is in contact with the water. Give your answer correct to the nearest square centimeter. [4];;b) Show that the volume of the water is $$972\pi cm^{3} $$. [2];;c);img;The container is held with its axis vertical, the hemisphere being at the bottom, as shown in Diagram II. Calculate the depth of the water. [4];;d);img;The container is now placed with its circular end on a horizontal surface as shown in Diagram III. Find the depth of the water. [2]Answers:

ID: 200303002015
Content:
img;The two circles shown have radii x and 3x.;;a) A point is chosen, at random, inside the larger circle. Find, in its simplest fractional form, the probability that this point is in the shaded area.   [2];;b) Find, in its simplest form, the ratio circumference of large circle: sum of circumference of the two circles.   [2]Answers:

ID: 200304002007
Content:
img;[The volume of a pyramid = $$\frac{1}{3} \times base.area \times height$$.];The diagram shows a solid traffic bollard.;It consists of a square-based pyramid, VABCD, attached to a cuboid, ABCDPQRS.;The vertical line, VNM, passes through the centres, N and M, of the horizontal squares ABCD and PQRS.;AB = BC = 60 cm and VN = 40 cm.;;a) Calculate;;a-i) VA,   [2];;a-ii) angle VAN,   [2];;a-iii) angle VAP.   [1];;b) Given also that AP = BQ = CR = DS = 80 cm, calculate ;;b-i) The volume of the bollard,   [2];;b-ii) The total surface area of the sides and top of the bollard.   [3];;c) The highway authority needs to paint the sides and tops of 17 of these bollards.;The paint is supplied in tins, each of which contains enough paint to cover $$8 m^2$$.;Find the number of tins of paint needed.   [2]Answers:

ID: 200403002007
Content:
img;A block of wood is cuboid, 10 cm by 6 cm by 2 cm. Find;;a) its volume, [1];;b) its surface area. [1]Answers:

ID: 200404002010
Content:
img;Answer the whole of this question on a sheet of a graph paper. ;; A solid cylinder of radius r centimeters and height h centimeters has a volume of $$100\pi cm^{3} $$.;;a-i) Show that $$h=\frac{100}{r^{2}}$$. [1];;a-ii) The cylinder has a total surface area of $$\pi$$y square centimeters. Show that $$y=2r^{2} +\frac{200}{r}$$. [1];;b) The table below shows some values of r and the corresponding values of y, correct to the nearest whole number.;img;;b-i) Find the value of p. [1];;b-ii) Using a scale of 2 cm to represent 1 cm, draw a horizontal r-axis for $$1\leq r\leq 6$$. Using a scale of 2 cm to represent $$20 cm^{2} $$, draw a vertical y-axis for $$70\leq y\leq 200$$. On your axes, plot the points given in the table and join them with a smooth curve. [3];;c) Use your graph to find the values of r for which y = 100. [2];;d) By drawing a tangent, find the gradient of the graph at the point where r = 2. [2];;e) Use your graph to find;;e-i) the value of r for which y is least, [1];;e-ii) the smallest possible value of the total surface area of the cylinderAnswers:

ID: 200503002016
Content:
img;In the diagram, the circle, centre O, passes through A and B. The tangent at A meets OB produced at C. The radius of the circle is 3 cm and $$A \hat OB=45^{\circ}$$.;;a) The area of the shaded region can be written as $$(p-q \pi) cm^{2} $$. Find the values of p and q. [2];;b) The perimeter of the shaded region can be written as $$(v \pi + \sqrt{t})cm$$. Find the values of v and t. [2]Answers:

ID: 200504002009
Content:
img;[The surface area of a sphere = $$4 \pi r^{2} $$.] ;[The volume of a sphere = $$\frac{4}{3} \pi r^{3} $$.]; [The area of the curve surface of a cone of radius r and slant height l is $$\pi rl$$.];[The volume of a cone = $$\frac{1}{3} \times  base.area \times height$$.];;A solid cone has a base radius of 5 cm and height 12 cm. A solid hemisphere has a radius of 5 cm. A metal toy is formed by joining the plane faces of the cone and the hemisphere.;;a) Show that the length of the slant edge of the cone is 13 cm. [1];;b) Calculate;;b-i) the surface area of the toy, [4];;b-ii) the volume of the toy. [3];;c) A solid metal cylinder has a radius of 1.5 m and height 2 m. The cylinder was melted down and all of the metal was used to make a large number of these toys. Calculate the number of toys that were made. [4]Answers:

ID: 200603002015
Content:
img;The points A and B lie on the circumference of a circle, centre O. The area of the circle is $$81\pi $$;;a) the radius of the circle, [1];;b) the length of the minor arc, giving your answer in the form $$k \pi$$, [2];;c) the acute angle AOB. [1]Answers:

ID: 200604002007
Content:
img;[Surface area of a sphere = $$4\pi r^{2} $$] ;[Volume of a sphere = $$\frac{4}{3} \pi r^{3} $$] ;;;A hot water tank is made by joining a hemisphere of radius 30 cm to an open cylinder of radius 30 cm and height 70 cm.;;a) Calculate the total surface area, including the base, of the outside of the tank. [4];;b) The tank is full of water.;;b-i) Calculate the number of litres of water in the tank. [3];;b-ii) The water drains from the tank at a rate of 3 litres per second. Calculate the time, in minutes and seconds, to empty the tank. [2];;b-iii);img;All of the water from the tank runs into a bath, which it just completely fills. The bath is a prism whose cross-section is a trapezium. The lengths of the parallel sides of the trapezium are 0.4 m and 0.6 m. The depth of the bath is 0.3 m. Calculate the length of the bath. [3]Answers:

ID: 200703002023
Content:
img;The diagram shows the arcs AB and CD of two circles, centre O, with radii 4 cm and 8 cm respectively. OAC and OBD are straight lines and angle COD = $$60^{\circ}$$. Calculate;;a) the perimeter of the shaded region, giving your answer in the form $$a+b\pi$$, [2];;b) the area of the shaded region, giving your answer as a multiple of $$\pi$$, [2];;c) the ratio of the shaded area to the area of the sector AOB. Give your answer in the form m : n where m and n are integers. [1]Answers:

ID: 200704002006
Content:
img;The diagram shows a concrete slab, ABCDEF, in the form of a right angled triangular prism. DE = 50 cm, AE = 150 cm and AB = 180 cm.;;a) Calculate the volume, in cubic metres, of the block. [3];;b) The concrete is a mixture of cement, sand and stones in the ratio 3 : 5 : 6 by volume. Show that, correct to three significant figures, the volume of cement needed to make the slab is $$0.145 m^{3} $$. [2];;c) The mass of 1 cubic metre of cement is 1420 kg. Calculate the mass of cement needed to make the slab. [2];;d) Cement is sold in bags, each of which contains a 25 kg of cement. How many bags of cement should be bought to make the block? [2];;e) The price of each bag of cement is $8.50, which includes 5% Goods and Services Tax. Calculate, correct to the nearest cent, the total tax paid for the cement bought to make the slab. [3]Answers:

ID: 200803002006
Content:
The diagram shows a quadrant of a circle, centre o and radius of 6 cm. ; C is the midpoint of OB and a semi-circle is drawn with OC as diameter. ; Find the perimeter of the shaded region. ; Give your answer in the form $$a + b\pi$$.Answers:

ID: 200804002008
Content:
img;Diagram I shows a pencil. ; It is made up of a cylinder and a cone. ; The cylinder has radius 0.4 cm and height 16 cm. ; The cone has base radius 0.4 cm and height 2 cm.;;a) Calculate;;a-i) The slant height of the cone,;;a-ii) The total surface area of the pencil.;;b) Calculate the volume of the pencil.;;c);img;Diagram II shows twelve of these pencils which just fit into a box;;c-i) Show that the volume of the inside of the box is $$138.24 cm^3$$;;c-ii) Calculate the percentage of the volume of the box that is not occupied by the pencils.Answers:

ID: 200903002014
Content:
img;The diagram shows a toy made from a cone and a hemisphere. ; The cone and the hemisphere both have radius 2.8 cm. ; The cone has a slant height of 7.2 cm. ; Calculate the surface area of the toyAnswers:

ID: 200903002015
Content:
The volume of two gold spheres are $$640 cm^3$$ and $$1250 cm^3$$.;;a) Find, in its simplest form, the ratio of;;a-i) The smaller radius to the larger radius,;;a-ii) The smaller surface area to the larger surface area;;b) The larger sphere has a mass of 25 kg. ; Find the mass of the smaller sphere.Answers:

ID: 200904002007
Content:
img;The diagram shows an ornament which is made of wire. ; It consists of a regular pentagon, ABCDE, of side 4 cm, five equal semicircles with diameter 4 cm and an outer circle, centre O. ; Each semicircle touches the outer circle as shown.;;a) Find the length of the perpendicular from O to AB.;;b) Show that the radius of the outer circle is 4.75 cm, correct to three significant figures.;;c) Find the total length of wire needed to make the ornament.;;d) Show that the area enclosed by the wire pentagon is $$27.5 cm^2$$, correct to three significant figures.;;e) Find the area of the region shaded on the diagram.Answers:

ID: 201004002008
Content:
a);img;The cross-section of a tunnel, shaded in the figure above, is a major segment of a circle centre O and radius 8 m. ; The total perimeter of the major sector POQR is 44m. ; Calculate;;a-i) The magnitude, in radians, of reflex angle POQ,;;a-ii) The area of triangle POQ,;;a-iii) The total area of cross-section of the tunnel.;;b);img;A traffic bollard consists of a pyramid, VABCD attached to a cuboid ABCDEFGH, ; ABCD and EFGH are square of side 10 cm. ; The vertical height, VN, of the pyramid is 12 cm, and VA = VB = VC = VD. ; AE = BF = CG = DH = 30 cm. Calculate;;b-i) The volume of the bollard, ;;b-ii) The surface area of the bollard. ; [Do not include the area of the bottom of the bollard.]Answers:

\end{document}
