\documentclass{article}
\begin{document}
ID: 199501002004
Content:
On the farm the numbers of workers picking up fruit on eleven days last March were 3, 8, 9, 12, 12, 15, 12, 13, 10, 8, 4. Find;;a) the median, [1];;b) the mode of this distribution. [1]Answers:

ID: 199502002010
Content:
Answer the whole of this question on a sheet of graph paper. One thousand candidates took a Mathematics examination which consisted of two papers. Each paper was marked out of 50. ;img;img; Table A gives the distribution of marks and Table B is the corresponding cumulative frequency table.;;a) By comparing the two tables, calculate the values of p, q, r, s and w, x, y, z. [2];;b) Using a horizontal scale of 2 cm to represent 10 marks, and a vertical scale of 2 cm to represent 100 candidates, draw the cumulative frequency curve for Paper 1. [2];;c) Use your curve to estimate, for Paper 1,;;c-i) the median mark, [1];;c-ii) the inter-quartile range, [1];;c-iii) the 30th percentile. [1];;d) The pass mark for Paper 1 was 21. Use your graph to estimate the number of candidates who passed. [1];;e) The top 12% of the candidates taking Paper 1 gained a Distinction. Use your graph to estimate the minimum mark required for a DistinctionAnswers:

ID: 199601002018
Content:
The times taken by an athlete to run 800 metres in three successive races were 2 minutes 0.8 seconds, 1 minute 59.1 seconds and 2 minutes 1.6 seconds.;;a) Calculate the average of these times. [2];;b) In order to qualify for an award of 8 minutes in 2400 metres, find the time taken of the final raceAnswers:

ID: 199602002003
Content:
a) A car uses 1 litre of petrol when travelling 8 km on dirt tracks and 1 litre when travelling 14 km on ordinary roads. On a particular journey of 180 km, 40 km was on dirt tracks and the remainder was on ordinary roads. Calculate;;a-i) the number of litres of petrol used on this journey. [2];;a-ii) the average number of kilometers travelled per litre of petrol. [2];;a-iii) the average speed in kilometers per hour for this journey, given that it took 3 hours 20 minutes. [2];;b) ;img; The diagram represents a gauge showing how much petrol is in a tank. The tank holds 42 litres when it is full.;;b-i) Estimate the number of litres in the tank when the needle is in the position shown. [1];;b-ii) It is given that OP = OQ = 5 cm and  $$P \hat OQ = 90^{\circ}$$. Arc AB has radius 3 cm and centre O. Calculate the shaded area ABQP. [The value of $$\pi$$ is 3.142, correct to 3 decimal places.] [3]Answers:

ID: 199602002006
Content:
A long ruler was fastened to a wall and used to measure the heights of 120 children. ;img; The diagram shows the cumulative frequency graph of these heights.;;a) Use the graph to estimate;;a-i) the median, [1];;a-ii)  the inter-quartile range, [2];;a-iii) the number of children whose height is greater than 170 cm. [1];;b) Several days later it was noticed that the ruler had been wrongly positioned, and that all heights should be 3 cm less. State what adjustment, if any, should be made to your results for part (a) (i) and (a) (ii) in order to give the correct value of;;b-i) the median, [1];;b-ii) the inter-quartile range. [1]Answers:

ID: 199701002017
Content:
Miss Jones asked the children in her class, ;;a-i) How many children are in her class? [1];;a-ii) What is the mode of the distribution? [1];;b) The information is also to be represented in a pie chart.;img; Calculate the angle of the sector representing Yellow. [2]Answers:

ID: 199701002020
Content:
img;The cumulative frequency curve shows the distribution of the population of the United States of America in 1950. Use the curve to estimate;;a) the median age of the distribution, [1];;b) the upper quartile of the distribution, [1];;c) the probability that an American chosen at random would be more than 60 years old. [2]Answers:

ID: 199703002008
Content:
img;The bar chart shows the number of goals scored by a team in 19 matches. Find;;a) the mode of the distribution, [1];;b) the median of the distribution, [1];;c) the total number of goals scored. [1]Answers:

ID: 199704002010
Content:
a) A box containing 250 apples was opened and each apple was weighed. The distribution of the masses of the apples is given in the following table.;img;;a-i) When a histogram is drawn to illustrate this information, the height of the column representing apples with mass m in the interval $$60 < m \leq 100$$ is 10 cm.;;Calculate the height of the column that represents values of m in $$160 < m \leq 220$$.;;a-ii) Calculate an estimate of the mean mass of the apples in the box. [3];;b) In this part of the question all probabilities should be given as exact decimals. ;The ticket machine in a car park takes 50 cent coins and $1 coins. ;A ticket costs $1.50.;The probability that the machine will accept a particular 50 cent coin is 0.9 and that it will accept a particular $1 coin is 0.8.;;b-i) What is the probability that the machine will not accept a particular 50 cent coin? [1];;b-ii) Leslie put one 50 cent coin and one $1 coin into the machine.;;Calculate the probability that the machine will not accept both coins;; b-iii) What is the probability that 3 50 cent coins will be accepted?Answers:

ID: 199801002023
Content:
On a certain stretch of road, the speeds of 100 cars were recorded. The results are summarized in the table below.;img;;a);img;On the grid, draw a frequency polygon to show this information. [2];;b) Calculate an estimate of the mean speed of these cars. [3]Answers:

ID: 199802002008
Content:
img;The population of a small island, A, is 3000. The histogram shows the distribution of ages of the population.;;a) Find the values of r and s in the frequency table below. [2];img;;b) Copy and complete the cumulative frequency table below. [1];img;;c) Using a horizontal scale of 2 cm to represent 10 years and vertical scale of 2 cm to represent 500 people, draw a smooth cumulative frequency curve to illustrate this information. [3];;d) Use your graph to estimate;;d-i) the median age of the population, [1];;d-ii) the number of people who are more than 57 years old. [1];;e) A second island, B, also has a population of 3000. For this population, the median age is 23, the lower quartile is 15, the inter-quartile range is 20 and the age of the oldest habitant is 70. On the axes you used for part (c), draw the cumulative frequency curve to illustrate this information. [3];;f) Which island is likely to have the greater population in 10 yearsAnswers:

ID: 199802002011
Content:
A toy firm makes a profit of $y when it produces x toys, where y is given by the formula $$y=\frac{1}{10}(3x-20)(100-x)$$ for $$0\leq x\leq 80$$. The table below shows the profit which the firm makes when producing different numbers of toys.;img;;a) Find the value of p. [1];;b) Using a scale of 2 cm to represent 10 toys, draw a horizontal axis for $$0\leq x\leq 80$$. Using a scale of 2 cm to represent $100, draw a vertical axis for $$-200\leq y\leq 700$$. On your axes plot the points given in the table and join them with a smooth curve. [3];;c) Use your graph to estimate;;c-i) the profit made when 25 toys are produced. [1];;c-ii-a) the number of toys produced to give the maximum profit, [1];;c-ii-b) the profit per toy when this number is produced, [2];;c-iii) the value of y when x = 5, explaining the significance of your answer. [2];;d-i) On the same axes draw the line whose equation is y = 9x. [1];;d-ii) Hence find the two values of x for which the profit is $9 per toy. [1]Answers:

ID: 199803002007
Content:
A group of 30 teachers was asked the colour of their car. The replies are given below.;img;;a) By making tally marks, or otherwise, obtain the frequency distribution of the colours. [2];img;;b) State the mode of this distribution. [1]Answers:

ID: 199803002017
Content:
img;The time taken for some students to perform a calculation is shown in the table. Part of the corresponding histogram is shown alongside.;;a) Find the value of p and the value of q. [2];;b) Complete the histogram. [2]Answers:

ID: 199804002010
Content:
Answer the whole of this question on a sheet of graph paper.;The length of time taken by all of the 600 pupils of a school to complete a task is given in the table below.;img;;a) Calculate an estimate of the mean time taken to complete the task. [3];;b) Copy and complete the cumulative frequency table for the time taken to complete the task. [1];img;;c) Using a horizontal scale of 2cm to represent 10 minutes for times between 30 minutes and 100 minutes, and a vertical scale of 2cm to represent 100 pupils, draw a cumulative frequency curve to illustrate this information. [2];;d) Use your graph to find;;d-i) the median time taken, [1];;d-ii) the interquartile range, [2];;d-iii) the 70th percentile, [1];;d-iv) the probability that a pupil chosen at random from the school took more than 60 minutes. [2]Answers:

ID: 199901002005
Content:
The temperature on the surface of the moon in the middle of the day was $$126^{\circ}C$$. The temperature on the surface of the moon in the middle of the night was $$-154^{\circ}C$$.;;a) By how much did the temperature decreases during this period? [1];;b) Find the average of the temperatures in the middle of the day and the middle of the night. [1]Answers:

ID: 199902002010
Content:
Answer the whole of this question on a sheet of graph paper. The lengths of 200 leaves were measured. The cumulative frequencies are given in the table below.;img;;a) Using a horizontal scale of 2 cm to represent 10 mm, and a vertical scale of 2 cm to represent 50 leaves, draw a cumulative frequency curve to illustrate this information. [3];;b) Showing your method clearly, use your graph to estimate;;b-i) the median, [1];;b-ii) the inter-quartile range, [2];;b-iii) the number of leaves which are longer than 55 mm. [2];;c) The frequency distribution for these results is given in the table below.;img;;c-i) Write down the value of p and the value of q. [1];;c-ii) Draw a histogram to illustrate this information. Use a horizontal scale of 2 cm to represent 10 mm. Use a vertical scale for frequency density such that the height of the first column is 3 cm. [3]Answers:

ID: 199904002006
Content:
img;Six hundred candidates took a mathematics examination which consisted of two papers. Each paper was marked out of 100. The diagram shows, on the same axes, the cumulative frequency curves for Paper 1 and Paper 2.;;a) Use the graph for Paper 1 to estimate;;a-i) The median, [1];;a-ii) The interquartile range, [2];;a-iii) The number of candidates who scored more than 45 marks. [1];;b) A candidate scored 60 on Paper 1. Use the two graphs to estimate this candidate;;c) State, with a reason, which you think was the more difficult paper. [1]Answers:

ID: 199904002008
Content:
a);img;The table shows the ages in years of 120 members of a sports club.;;a-i) State the modal class of this distribution. [1];;a-ii) In which class does the median lie? [1];;a-iii) Calculate an estimate of the mean age of the 120 members of the sports club. [3];;b) A bag contains 15 identical discs.;There are 8 red, 4 blue and 3 white discs.;A disc is picked out at random and not replaced.;A second disc is then picked out at random and not replaced.;The tree diagram below shows the possible outcomes and some of their probabilities.;img;;b-i) Calculate the values of p, q, r and s shown on the tree diagram. [2];;b-ii) Expressing each of your answers as a fraction in its lowest terms, calculate the probability that;;b-ii-a) Both discs will be red, [1];;b-ii-b) One disc will be red and the other blue. [2];;b-iii) A third disc is now picked out at random. Calculate the probability that none of the three discs is white. [2]Answers:

ID: 200001002019
Content:
160 candidates took an examination. The mark distribution produced the following information: the median mark was 39, the inter-quartile range was 23 marks, the lower quartile was 27 marks, the highest mark scored was 68, the lowest mark scored was 6.;;a) Calculate the upper quartile of this distribution. [1];;b) On the axes shown draw the cumulative frequency curve to represent this information. [3];img;Answers:

ID: 200001002024
Content:
An examination was taken by 160 pupils. The frequency distribution of the marks out of 100 is shown in the table below:;img;;a) Complete the cumulative frequency table for this distribution. [1];img;;b) A pie chart is drawn to represent this information.;img;Calculate the angle of the sector representing the number of pupils scoring more than 70 marks. [2];;c) A histogram is drawn to represent this information. Complete the histogram. [2]Answers:

ID: 200002002006
Content:
The weekly wages of the people who work in a small factory are given in the table below.;img;;a) Calculate the total weekly wages bill for the factory. [1];;b) Find, for the distribution of weekly wages,;;b-i) the mode, [1];;b-ii) the median, [1];;b-iii) the mean. [1];;c) One person is chosen at random from those who work in the factory. Another person is chosen at random from those remaining. Expressing your answer as a fraction in its lowest terms, find the probability that the sum of the wages of these two people is more than $410. [2]Answers:

ID: 200003002020
Content:
Some children were asked how many television programme they had watched on the previous day. The table shown the result.;img;;a) Write down the largest possible value of x given that the mode is 1. [1];;b) Write down the largest possible value of x given that the median is 1. [1];;c) Calculate the value of x given that the mean is 1. [2]Answers:

ID: 200004002010
Content:
Answer the whole of this question on a sheet of graph paper.;One day a farmer collected 340 eggs from his chickens.;The table below shows the distribution of the masses of the eggs.;img;;a) When a histogram is drawn to illustrate this information, the rectangle representing the eggs with masses in the interval $$42 < m \leq 46$$ has width 2 cm and height 3 cm.;Find the width and the height of the rectangle representing the eggs with masses in the interval $$46 < m \leq 48$$ [2];;b) Copy and complete the cumulative frequency table below. [1];img;;c) Using a scale of 2 cm to represent 5 grams, draw a horizontal m-axis for $$30 \leq m \leq 70$$. Using a scale of 2 cm to represent 50 eggs, draw a vertical axis for values from 0 to 340. On your axes, draw a smooth cumulative frequency curve to illustrate this information. [3];;d) Use your graph to find;;d-i) the median mass of the eggs, [1];;d-ii) the interquartile range. [2];;e) The farmer classifies 60 of the eggs to be ;;e-i) Use yAnswers:

ID: 200101002016
Content:
In a school of 120 children, a teacher found out the distance, d metres, that each child could swim. The results were grouped as shown in the table.;;a) A pie chart is drawn to represent this information.;img;Calculate the angle which represents the number of children in Group B. [2];;b) A histogram is drawn to represent the information. Calculate the frequency densities for each group. [2]Answers:

ID: 200101002021
Content:
Eleven people work for a company. Five office workers each have a car allowance of $10000. Six managers have car allowances of $12000, $18000, $20000, $24000, $28000 and $35000.;;a) State;;a-i) the median car allowance, [1];;a-ii) the mode of this distribution. [1];;b) Calculate the mean car allowance. [2];;c) Two people are chosen at random from the eleven members of the company. Calculate the probability that one is a manager and one an office worker. Express your answer as a fraction. [2]Answers:

ID: 200102002009
Content:
a) Over a 40 day period, the number of students absent from school was recorded. The results are given in the table above.;img;For example, on 8 of the days there were 5, 6, 7, 8 or 9 students absent from school.;;a-i) Which is the modal class? [1];;a-ii) Calculate an estimate of the mean number of students absent per day. [3];;b) The diagram shows the cumulative frequency graph of the marks scored by 160 students in an examination.;img;;b-i) Use the graph to estimate;;b-i-a) the median mark, [1];;b-i-b) the inter-quartile range, [2];;b-i-c) the number of students who scored 70 marks or more. [1];;b-ii) Before combining these marks with those of another exam, the teacher scaled them. To obtain the scaled mark, the teacher multiplied each student;;b-ii-a) the median scaled mark, [1];;b-ii-b) the inter-quartile range of the scaled marks, [1];;b-ii-c) the number of students who had a scaled mark of 80 or more. [2]Answers:

ID: 200103002014
Content:
img;The graph is a cumulative frequency curve showing the marks gained by 300 candidates in an examination.;;a) Use the curve to estimate;;a-i) the median mark,    [1];;a-ii) the number of candidates who gained 33 marks or less.    [1];;b) It is given that 70 candidates achieved a grade A. Use the curve to estimate the smallest mark required for a grade A.    [1]Answers:

ID: 200104002008
Content:
Answer the whole of this question on a sheet of graph paper.;;;The length of time taken by 80 drivers to complete a journey is given in the table below.;img;;a) Using a scale of 2 cm to represent 10 minutes, draw a horizontal axis for times between 60 minutes and 130 minutes.;Choose a suitable scale for the vertical axis and draw a histogram to represent the information in the table.;;b) In which interval does the median of the distribution lie?;;c) Calculate an estimate of the mean time taken to complete the journey.;;d) One driver is chosen at random.;Expressing your answer as a fraction in its lowest terms, calculate the probability that she took 90 minutes or less for the journey.;;e) Two drivers are chosen at random.;Expressing each answer as a fraction in its lowest terms, calculate the probability that;;e-i) both took more than 110 minutes for the journey,;;e-ii) one took 80 minutes or less for the journey and the other took more than 110 minutes.Answers:

ID: 200201002005
Content:
Mr. Smith asked the children in his class ;img;;a);img;By making tally marks, or otherwise, obtain the frequency distribution of the colours. [1];;b) State the mode of this distribution. [1]Answers:

ID: 200201002019
Content:
img;The cumulative frequency curve shows the distribution of the times of 300 competitors in a women;;a) The race was won by Tegla. Find her time, giving your answer in hours and minutes. [1];;b) Find the median time in hours and minutes. [1];;c) The qualifying time for the Olympic Games was achieved by ten percent of the runners. The race began at 11.30. At what time did the last qualifying athlete finish the race? Express your answer using the 24 hour clock. [2]Answers:

ID: 200202002005
Content:
img;[The value of $$\pi$$ is 3.142, correct to three decimal places.] ; [The volume of a sphere is $$\frac{4}{3}\pi r^{3} $$.] ; The diagrams show two ways of packaging 4 identical balls. The radius of each ball is 3 cm. Diagram I shows a closed rectangular box with a square base. Each ball touches the top, the bottom and two sides of the box. Each ball also touches two other balls. Diagram II shows a closed cylinder. The balls touch the ends and the side of the cylinder.;;a-i) Write down the dimensions of the rectangular box. [1];;a-ii) Calculate the total surface area of the outside of this box. [2];;b) Calculate the total surface area of the outside of the cylinder. [2];;c) Calculate the total volume of the 4 balls. [2];;d) Calculate, correct to three decimal places, the value of $$\frac{Volume.of.the.cylinder}{Volume.of.the.box}$$ . [2];;e) Hence state which of the two containers has more space not occupied by the balls. [1]Answers:

ID: 200203002008
Content:
a) An article in a newspaper reported that the number of crimes had been reduced by half from 1991 to 2001. The article contained the bar chart shown here.;img;Explain why this bar chart might be considered misleading.[1];;b) The histogram alongside shows the distribution of times taken by a group of students to travel to school.;11 students took at least 5 but less than 10 minutes.;Complete the table in the answer space.   [2];img;Answers:

ID: 200203002015
Content:
img;The dot diagram shows the number of children living in the houses on a certain road. Find;;a) the percentage of houses that have at least 3 children living in them,   [1];;b) the probability that two houses, chosen at random, would each have more than 3 children living in them,   [1];;c) the mean number of children.   [1]Answers:

ID: 200204002011
Content:
a) A class of 27 children took a Mathematics test.;Some of their scores are represented in the stem and leaf diagram below. ;img;The other scores are given below. ;img;;a-i) Construct a single ordered stem and leaf diagram to represent the scores of all 27 children.   [2];img;;a-ii) For the whole class, find;;a-ii-a) The modal score,   [1];;a-ii-b) The median score.   [1];;a-iii) The pass mark for the paper was 24 marks. ;Expressing the answer as a fraction in its simplest form, calculate the probability that two children, chosen at random, from the class both passed the test.   [2];;b) Answer the whole of this part of the question on a sheet of graph paper.;At another school, 300 pupils took an English test.;The table below is the cumulative frequency table for their scores.;img;;b-i) Using a scale of 2 cm to 10 marks, draw a horizontal s-axis for $$0 \leq s \leq 60$$. ;Using a scale of 2 cm to 50 pupils, draw a vertical axis for values from 0 to 300. ;On your axes, draw ;img;Answers:

ID: 200301002025
Content:
The numbers of goals scored in 20 football matches were 5, 0, 5, 4, 1, 0, 5, 5, 1, 3, 4, 5, 0, 0, 5, 5, 3, 2, 5, 4.;;a-i);img;Complete the table. [1];;a-ii);img;Using the axes, represent the information as a bar chart. [2];;b) State the median. [1];;c) Calculate the mean number of goals. [2]Answers:

ID: 200302002005
Content:
img;a) One hundred and sixty students took an examination. The table shows the marks needed for each grade. The cumulative frequency curve shows the distribution of their marks.;;a-i) Use the graph to estimate;;a-i-a) the median, [1];;a-i-b) the inter-quartile range, [2];;a-i-c) the number of students who were awarded a Grade C. [2];;a-ii) A pie chart was drawn to illustrate the grades to the students. Calculate the angle of the sector which represented the number of students who were awarded a Grade C. [2];;b) An ordinary unbiased die has faces numbered 1, 2, 3, 4, 5 and 6. Sarah and Terry each threw this die once. Expressing each answer as a fraction in its lowest terms, find the probability that;;b-i) Sarah threw a 7, [1];;b-ii) they both threw a 6, [1];;b-iii) neither threw an even number, [1];;b-iv) Sarah threw exactly four more than Terry. [1]Answers:

ID: 200303002011
Content:
img;All the students from two schools, A and B, take the same examination paper. ;The cumulative frequency curves show the results for the two schools.;;a) Estimate the median mark of the students from school A.   [1];;b) Estimate the percentage of the students from school B who gained more than 80 marks.   [1];;c) State, with a reason, which school achieved the better results.   [1]Answers:

ID: 200304002004
Content:
Answer the whole of this question on a sheet of graph paper.;The speeds of 50 cars being driven along a stretch of road were recorded.;The table below shows the distribution of the speeds of the cars.;img;;a) Using a scale of 1 cm to represent 10 km/h, draw a horizontal axis for speeds up to 110 km/h.;Using a scale of 4 cm to represent 1 unit, draw a vertical axis for frequency densities from 0 to 2 units.;On your axes, draw a histogram to represent the information in the table.   [3];;b) Write down the modal class of the distribution.   [1];;c) In which interval is the upper quartile of the distribution?   [1];;d) Find the probability that one car, selected at random, had a speed of;;d-i) Less than 20 km/h,   [1];;d-ii) More than 60 km/h.   [1];;e) There is a speed limit of 60 km/h on this stretch of road.;Two cars were selected at random.;Calculate the probability that one car was breaking the speed limit and the other was not breaking the limit.   [2]Answers:

ID: 200403002019
Content:
img;The lengths of 40 nails were measured. Their lengths, in centimeters, are summarized in the table below.;;a) On the axes in the answer space, draw the histogram which represents this information. [2];img;;b) Calculate an estimate of the mean length of the nails. [2]Answers:

ID: 200404002011
Content:
Answer the whole of this question on a sheet of graph paper. ;; The table below shows the marks obtained in tests of English and Mathematics by 140 students.;img;;a) Copy and complete the cumulative frequency table below. [2];img;;b) Using a scale of 2 cm to represent 20 marks, draw a horizontal x-axis for $$0\leq x\leq 100$$. Using a scale of 2 cm to represent 20 pupils, draw a vertical axis for values from 0 to 140. On your axes, draw and label both smooth cumulative frequency curves to illustrate this information. [3];;c) Use your curves to find;;c-i) the upper quartile mark for English, [1];;c-ii) the inter-quartile range for English, [1];;c-iii) the median mark for English and the median mark for Mathematics. [1];;d) State, with a reason, which you think is the easier test. [1];;e) One student is chosen at random. It may be assumed that the marks gained in the two subjects are independent. Expressing each answer as a fraction in its lowest terms, calculate the probability ;;e-i) student gains more than 60 marks on both papers;;eii)student gains more than 80 marks on one paper, but not on the otherAnswers:

ID: 200503002019
Content:
img;The diagram above is the cumulative frequency curve for the heights of 400 plants which were grown in Field A. Use the graph to find;;a) the number of plants that grew to a height of more than 30 cm, [1];;b) the inter-quartile range. [1];;c) Another 400 plants were grown in Field B. The cumulative frequency distribution of the heights of these plants is shown in the table.;img;; On the same axes as for Field A, draw the cumulative frequency curve for the plants grown in Field B. [2];;d) By comparing the two curves, state, with a reason, which Field produced the taller plants. [1]Answers:

ID: 200504002004
Content:
img;The table shows the number of cars owned by each of 25 families.;;a) Draw a dot diagram to represent the information in the table. [2];;b) Find;;b-i) the median number of cars, [1];;b-ii) the modal number of cars, [1];;b-iii) the mean number of cars. [1];;c) A family is chosen at random. Find the probability that it owns 3 cars. [1];;d) Two families are chosen at random. Find the probability that one family owns 2 cars and the other owns 4 cars. [2];;e) A car is chosen at random. Find the probability that it belongs to a family which owns 2 cars. [2]Answers:

ID: 200603002006
Content:
The temperature at the bottom of a mountain was $$8^{\circ}C$$. The temperature at the top was $$-26^{\circ}C$$. Find;;a) the difference between the two temperature, [1];;b) the mean of the two temperatures. [1]Answers:

ID: 200604002011
Content:
img;Answer the whole of this question on a sheet of graph paper.;;The diagram shows the histogram which represents the heights of the pupils in a small school.;;a-i) On your graph paper, copy and complete this frequency table that represents the	distribution. [2];img;;a-ii) Hence copy and complete this cumulative frequency table that represents the distribution. [1];img;;b) Using a scale of 2 cm to represent 10 cm, draw a horizontal h-axis for $$130\leq h\leq 190$$. Using a scale of 1 cm to represent 10 pupils, draw a vertical axis. On your axes, draw a smooth cumulative frequency curve to illustrate the information. [3];;c) Use your graph to find;;c-i) the median height of the pupils, [1];;c-ii) the lower quartile height, [1];;c-iii) the inter-quartile range. [1];;d) One student is chosen at random. Use your frequency table to find the probability that the student;;e) Two students are chosen at random. Calculate the probability that one has a height greater than 170 cm and thAnswers:

ID: 200703002015
Content:
A class of 11 boys and 9 girls took a test. The girls;;a) Find the mean mark for the girls. [1];;b) The mean mark for the boys was 6.0. Find the mean mark for the whole class. [2]Answers:

ID: 200703002022
Content:
img;Use as much information from the table as necessary to answer the following.;;a) Find the ratio of the population of Singapore to the population of Australia. Give your answer in the form 1 : n. [1];;b) How many more people live in Malaysia than in Singapore? Give your answer in standard form. [2];;c) Calculate the average number of people per square kilometer in the UK. [2]Answers:

ID: 200704002011
Content:
a) The masses of 150 men were measured. The cumulative frequency curve below shows the distribution of their masses.;img;;a-i) Use the curve to estimate;;a-i-a) the median mass, [1];;a-i-b) the upper quartile mass, [1];;a-i-c) the inter-quartile range. [1];;a-ii) It was stated that 18% of these men were considered overweight. Find the least mass for a man to be considered overweight. [2];;a-iii) Use the curve to draw up a table to show the number of men whose masses, w kilograms, fall into each of the intervals of width 10 kg, starting with the interval $$40<w\leq 50$$. On a sheet of graph paper, draw a histogram to represent the distribution of masses. [3];;b) A pack of 16 cards contains 4 cards of each of the colors red, blue, green and yellow.	The four cards of each color are numbered 1, 2, 3 and 4.;;b-i) The pack is shuffled and one card is drawn at random. Find, as a fraction, the probability that it is a red card numbered 4 or a blue card. [1];;b-ii) The card is replaAnswers:

ID: 200803002002
Content:
img;In a survey, 48 children were asked how they travelled to school. ; The result of the survey are shown in the bar chart.;;a) Express the total number of children who walked or cycled as a fraction of the total number of children. ; Give your answer in its lowest terms.;;b) The same information is to be shown in a pie chart. ; Find the angle which represents the children who travelled by car.Answers:

ID: 200803002014
Content:
a) A group of students took a test. ; Their scores are shown in the stem and leaf diagram;img;;a-i) Write down the modal score.;;a-ii) Find the median score.;;b) The box and whisker diagram below shows the masses of a number of bananas.;img; Find the interquartile range.Answers:

ID: 200804002010
Content:
a) The total mass of the tomatoes produced by each of 40 tomato plants was measured.; The cumulative frequency curve below shows the distribution of the masses;img;;a-i) Copy and complete the grouped frequency table of the mass of tomatoes on each plant.;;a-ii) Using your grouped frequency table, calculate an estimate of ;;a-ii-a) The mean mass of tomatoes produced by each plant,;;a-ii-b) The standard deviation.;;a-iii) The tomatoes produced by another group of 40 plants have the same median but a larger standard deviation ; Describe how the cumulative frequency curve will differ from the given curve.;;b) A bag contains six identical balls number 1, 2, 3, 4, 5 and 6. Two balls are drawn at random, one after the other, from the bag without replacement.;;b-i) Draw the possibility diagram to show the outcome of the draw;;b-ii) Find, as a fraction in its simplest form, the probability that ;;b-ii-a) Both balls have an even number;;b-ii-b) The sum of numbers drawn is 8;;b-ii-b)Another ball is drawn and at least one of the balls drawn is a multiple of 3Answers:

ID: 200903002003
Content:
The distribution of colour of cars in a car park is shown in the table. ;img;;a) Write down the modal colour.;;b) This distribution is to be shown in a pie chart. Calculate the angle representing the colour green.Answers:

ID: 200903002020
Content:
img;The cumulative frequency graph shows the distribution of marks of 800 students in a Mathematics examination;;a) Find the median mark.;;b) Find the interquartile range.;;c) To be awarded a Grade A, a student has to get a mark of 78 or more. ; Estimate the number of students who are awarded a Grade A.Answers:

ID: 201004002010
Content:
The sixty member of a group of pupils were asked how many books they had read this term. ; The results of this survey are summarized in the table below. ;img;;a) Find;;a-i) The values of a, b, c, d and e,;;a-ii) The mean and standard deviation of the number of books read.;;b) Find the probability that one pupil chosen at random from the group had read exactly 6 books.;;c) Two pupils are chosen at random from the group. ; Find the probability that both had read more than 4 books.Answers:

\end{document}
