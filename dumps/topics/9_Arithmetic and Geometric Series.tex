\documentclass{article}
\begin{document}
ID: 199703003015
Content:
A bank has an account for investors. Interest is added to the account at the end of each year at a fixed rate of 5% of the amount in the account at the beginning of that year. A man and a woman both invest money.;;a) The man decides to invest $$x$$ at the beginning of one year and then a further $$x$$ at the beginning of the second and each subsequent year. He also decides that he will not draw any money out of the account, but just leave it, and any interest, to build up.;(i) How much will there be in the account at the end of 1 year, including the interest?;(ii) Show that, at the end of n years, when the interest for the last year has been added, he will have a total of ;\[21(1.05^{n} - 1)x\] in his account.;(iii) After how many complete years will he have, for the first time, at least $$12x$$ in his account?;;b) The woman decides that, to assist her in her everyday expenses, she will withdraw the interest as soon as it has been added. She invests $$y$$ at the beginning of each year. Show that, at the end of n years, she will have received a total of \[\frac {1}{40} n(n + 1)y\] in interest.Answers:
21: At the end of 1 year, the account will have | "1.05x" |.
22: 
23: Required no. of years = "10".
24: 

ID: 199803003003
Content:
The first term of a geometric progression is 10 and its sum to infinity is 15. ;Find the third term of the progression. Answers:
83: The 3rd term = "\frac{10}{9}".

ID: 200003003014
Content:
a) The first term of a geometric progression is 3 and the common ratio is r, where |r| < 1. The sum of the first three terms of the progression is \[\frac{8}{9}\] of the sum of the first six terms. Find the sum to infinity.;;b) All the terms of the arithmetic progression \[u_{1}, u_{2}, u_{3},..., u_{n},...\] are positive. Use induction to prove that, for  \[n \geq 2\], \[\frac{1}{u_{1}u_{2}} + \frac{1}{u_{2}u_{3}} + \frac{1}{u_{3}u_{4}} + ... + \frac{1}{u_{n-1}u_{n}} = \frac{n - 1}{u_{1}u_{n}}\].Answers:
266: Sum to infinity = "6".
267: 

ID: 200203003003
Content:
The nth term of a series is  $$2^{n - 2}  + 3n$$. Find the sum of the first N terms. Answers:
420: Sum of the first N terms = "\frac{1}{2}(2^N - 1)+\frac{3}{2}N(N+1)".

ID: 200304003004
Content:
The first, second and fourth term of a convergent geometric progression are consecutive terms of an arithmetic progression. Prove that the common ratio of the geometric progression is  $$\frac{-1 + \sqrt 5}{2}$$. The first term of the geometric progression is positive. Show that the sum of the first 5 terms of this progression is greater than nine tenths of the sum to infinity.   Answers:
527: 

ID: 200404003002
Content:
At the end of 1995 the population of Urbis was 46650 and by the end of 2000 it had risen to 54200. On the assumption that the populations at the end of each year form a geometric progression, find;(i) the population at the end of 2006. Give you answer to 4 significant figure.;(ii) the year in which the population reaches 100000. Answers:
579: Population at the end of 2006 = "64570".
580: Year in which the population reaches 100000 = "27".

ID: 200603003001
Content:
The sum, $$S_{n}$$ , of the first n terms of a geometric progression is given by $$S_{n}=6- \frac{2}{3^{n-1}} $$. ;Find the first term and the common ratio. Answers:
676: 1^{st}  term of geometric progression = "4".
677: Common ratio of geometric progression = "\frac{1}{3}".

ID: 200703003010
Content:
A geometric series has common ratio r, and an arithmetic series has first term a and common difference d, where a and d are non-zero. The first three terms of the geometric series are equal to the first, fourth and sixth terms respectively of the arithmetic series.;(i) Show that $$ 3r^{2}-5r+2=0$$.;(ii) Deduce that the geometric series is convergent and find, in terms of a, the sum to infinity.;(iii) The sum of the first n terms of the arithmetic series is denoted by S. Given that a > 0, find the set of possible values of n for which S exceeds 4a. Answers:
781: 
782: 
783:  S_{\infty} = "3a".
784: The set of possible values of n for which S exceeds 4a is "6" \leq n \leq "13", n \in \mathbb{Z}^{+}.

\end{document}
