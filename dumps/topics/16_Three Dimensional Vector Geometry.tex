\documentclass{article}
\begin{document}
ID: 199704003015
Content:
The plane \[\prod \] passes through the origin O and is perpendicular to 2i + 2j + k. The line L has cartesian equation \[\frac {x - 4}{1} = \frac {y - 8}{3} = \frac {z - 4}{2}\]. The points A and B have position vectors i + 2j - 3k and 3i + 5j + 2k respectively.;(i) State a cartesian equation for \[\prod \].;(ii) The point C on L has position vector c, where c.k = 0. Find c, and hence state an alternative equation for L of the form \[\frac {x - p}{1} = \frac {y - q}{m} = \frac {z}{n}\], giving numerical values for p, q, l, m and n.;(iii) Find the exact value of the length of the perpendicular from A to L.;(iv) Find, in either order, the exact value of the length of the projection of \[\overrightarrow{AB}\] onto;a) the normal to  \[\prod \],;b) the plane  \[\prod \].Answers:
862: A cartesian equation for \prod is "2x+2y+z" = "0".
863: c = "2i+2j".
864: p="2".
865: q="2".
866: l="1".
867: m="3".
868: n="2".
869: Length of perpendicular from A to L = "\sqrt{\frac{13}{2}}".
870: Length of the projection of \overrightarrow{AB} onto the normal to  \prod = "5" units.
871: Length of the projection of \overrightarrow{AB} onto the plane  \prod = "\sqrt{13}" units.

ID: 200003003003
Content:
Show that the following lines do not intersect: r = -2i + j + 9k + s(2i + 5j + 4k), r = 11i + 8j + 3k + t(3i - j + 5k). ;State whether the lines are parallel or skew, giving a reason for your answer.Answers:
244: 
245: The lines are "skew".

ID: 200003003018
Content:
a) Given that the angle between the vectors \[-i + 3j + \lambda k\] and 2i + 2j + k is \[\cos^{-1} \frac{1}{3}\], find the value of the constant \[\lambda\].;;b) OABC is a tetrahedron (i.e. a solid having four triangular faces). The position vectors of A, B and C with respect to O are a, b and c respectively. The mid-points of BC, CA and AB are X, Y and Z respectively.;(i) The point G on AX is such that \[AG = \frac{2}{3} AX\]. Find the position vector of G. Deduce that AX, BY and CZ are concurrent (i.e. that they all go through the same point).;(ii) Given that L is the mid-point of OA and that H is the mid-point of LX, find the position vector of H. The mid-point of OB is M and the mid-point of OC is N. What statement about the lines LX, MY and NZ can be made?Answers:
278: \lambda = "-\frac{3}{4}".
279: \overrightarrow{OG} =- "\frac{1}{3}a+\frac{1}{3}b+\frac{1}{3}c".
280: G is the "mid-point" of face ABC, therefore AX, BY and CZ are concurrent and go through G .
281: \overrightarrow{OH} = "\frac{1}{4}a+\frac{1}{4}b+\frac{1}{4}c".
282: H is the "centroid" of OABC, therefore LX, MY and NZ are concurrent and go through H.

ID: 200103003004
Content:
The points A and B have position vectors 3i + 2j and i - j respectively, with respect to the origin O. The line l has vector equation r = 5i + 5j + t(2i - j). Find the acute angle between l and the line passing through A and B.Answers:
337: Acute angle between l and the line passing through A and B= "82.9" ^{\circ}.

ID: 200204003004
Content:
(i) Show that the lines given by  $$r = ( 5i + 2j + 4k ) + \lambda ( i + 3j + k )$$ and  $$r = ( 3i + j + k ) + \mu ( 4i + 7j + 5k )$$ intersect, and find their point of intersection.;(ii) Calculate the acute angle between the lines. Answers:
457: 
458: The point of intersection = "7i+8j+6k".
459: Acute angle between the lines = "17.6"^{\circ}.

ID: 200303003005
Content:
Referred to the origin O, the position vectors of points A and B are 4i - 11j + 4k and 7i + j + 7k respectively.;(i) Find a vector equation for the line l passing through A and B.;(ii) Find the position vector of the point P on l such that OP is perpendicular to l. Answers:
819: Vector equation of the line l = "(4i-11j+4k)+t(i+4j+k)", t \in R.

ID: 200403003016
Content:
The equation of the line L is  $$r =  \begin{pmatrix} 1\\ 3\\ 7 \end{pmatrix} + t \begin{pmatrix} 2\\ -1\\ 5 \end{pmatrix}$$. The points A and B have position vectors  $$ \begin{pmatrix} 9\\ 3\\ 26 \end{pmatrix}$$ and  $$ \begin{pmatrix} 13\\ 9\\ \alpha \end{pmatrix}$$ respectively. The line L intersects the line through A and B at P. Find  $$\alpha $$ and the acute angle between line L and AB. Give your answer correct to 1 decimal place. The point C has position vector  $$ \begin{pmatrix} 2\\ 5\\ 1 \end{pmatrix}$$ and the foot of the perpendicular from C to L is Q. Find the length of CQ. ;Answers:
820: \alpha = "34".
821: \theta = "44.6" ^{\circ}.
822: |\overrightarrow{CQ}| = "\sqrt{11}" units.

ID: 200503003015
Content:
The lines $$l_1$$ and $$l_2$$ have equations  $$r = \begin{pmatrix}0\\ 1\\ 2\end{pmatrix} + s\begin{pmatrix}1\\ 0\\ 3\end{pmatrix} $$ and $$r = \begin{pmatrix}-2\\ 3\\ 1\end{pmatrix} + t\begin{pmatrix}2\\ b\\ 0\end{pmatrix}$$ respectively, where b is a constant.;(i) For the case where the lines intersect, calculate the acute angle between them. Give your answer correct to 1 decimal places.;(ii) For the case where b = 1, find the position vectors of the points P on $$l_1$$ and Q on $$l_2$$ such that O, P and Q are collinear, where O is the origin. Answers:
826: Acute angle = "78.3" ^{\circ}.
827: \overrightarrow{OP} = "\frac{1}{25}"("-14","25","8").
828: \overrightarrow{OQ} = "\frac{1}{8}"("-14","25","8").

ID: 200703003008
Content:
The line l passes through the points A and B with coordinates (1,2,4) and (-2,3,1) respectively. The plane p has equation $$3x-y+2z=17$$ . Find;(i) The coordinates of the point of intersection of l and p,;(ii) The acute angle between l and p,;(iii) The perpendicular distance from A to p.  Answers:
772: The coordinates of the point of intersection of l and p is ("\frac{5}{2}", "\frac{3}{2}", "\frac{11}{1}").
773: The acute angle between l and p is "78.8"^{\circ}.
774: The perpendicular distance from A to p is "\frac{4*\sqrt{14}}{7}" units.

\end{document}
