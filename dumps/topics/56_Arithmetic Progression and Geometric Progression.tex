\documentclass{article}
\begin{document}
ID: 199502001002
Content:
a) An arithmetic series has first term 7 and second term 6.8. Find the sum of the first 50 terms; b) A funding body gives a grant to a sports organisation each year from 1995. The amount of grant in 1995 is \[\it\unicode{xA3} 10000\] and thereafter is 90% of the grant in each preceding year. Calculate;i) the year in which the value of the grant first falls below \[\it\unicode{xA3} 20000\];
ii) the total amount paid in grants to the sports organisation during the years from 1995 to 2004 inclusive,;
iii)the total amount paid in grants to the sports organisation during the years from 2005 to 2014 inclusive,;
iv) the number of years the grant has been paid when the total amount paid first exceeds \[\it\unicode{xA3} 90000\],Answers:

ID: 199602001002
Content:
a) The first and last terms of an arithmetic progression are -24 and 72 respectively. The sum of all the terms is 600. Calculate;i) the number of terms in the progression;ii) the common difference;iii)	the sum of the positive terms.;b) All the terms of a geometric progression are positive. the second term is 12 and the fourth term is 6.75. Find;i) the common ratio and the first term,;ii) the sum of the first 10 terms.;c) On January 1st, 1960, a person opened a savings account with £100. Interest, at 5% of the amount standing in the account at the time, was added each year on January 1st, starting in 1961. Given that no withdrawals were made, find the year in which there was more than £520 in the account after the interest had been added.Answers:

ID: 199604001003
Content:
a) The length of the perimeter of a hexagon is 36 cm, The lengths of the sides of the hexagon are in arithmetic progression and the length of the longest side is five times the length of the shortest side. Find the length of each side.;b) A gardener has the task of digging an area of \[800m^2\] On the first day he digs an area of \[10m^2\] On each successive day he digs an area 1.2 times the area he dug the previous day, until the day when the task is completed. Find the number of days needed to complete the task.;c) The sum of n terms of an arithmetic progression is given by the formula \[S_n=2n^2+n\];Find;i) the first term;ii) the common difference;iii) the tenth term.Answers:

ID: 199702001002
Content:
a) An infinite geometric progression has a first term of 50 and a fourth term of 10.8. Find ;i) the fifth term,;ii) the sum to infinity;iii) the value of n given that the nth term is the first term in the progression that is less than 0.25.;b) During 1336 a company increased its sales of television sets at a constant rate of 200 sets per month. Thus the number of television sets sold in February was 200 more in January, the number of television sets sold in march was 200 more than in February, and this pattern continued month by month throughout the year. Given that the company sold 38400 television sets in 1996, calculate the number of television sets sold in ;i) January;ii) December;c) Find the sum of all numbers between 200 and 1000 which are exactly divisible by 15.Answers:

ID: 199704001002
Content:
a) An arithmetic progression is such that the 10th term is 40 and the sum of the first 10 terms is 265. Find the sum of the first 20 terms.;b) The first two terms of a geometric progression are 100 and 120 respectively. The kth term of the progression is the largest term that is less than 5000. Find the value of k and hence evaluate the sum of first k terms.;c) A geometric progression has a first term of 18 and a sum to infinity of 30. Each of the terms in the progression is squared to form a new geometric progression. Find the sum to infinity of the new progression.Answers:

ID: 199802001002
Content:
a) A polygon has sides whose lengths are in arithmetic progression. The lengths of the shortest and longest sides are 1.3 m and 5.7 m respectively. Given that the length of the perimeter is 42 m, calculate the number of sides.;b) The first term of a geometric progression is 25 and the sum of the first three terms is 61. Find the possible values of the common ratio.;Given also that this progression has a sum to infinity, evaluate this sum.;c) A man undertakes a long-distance walk of 480 km. On the first day he walks 35 km. On each subsequent day he walks 95% of the distance he had walked the previous say, until his walk is completed. He commences his walk on 1st June. Find ;i) the distance, to the nearest km, that he walks on 10th June,;ii) the date on which he completes his 480 km walk.Answers:

ID: 199804001002
Content:
a) An arithmetic progression is such that the 5th term is 3 times the 2nd term.;i) Show that the sum of the first 8 terms is 4 times the sum of the 1st 4 terms.;ii) Given further that the sum of the 5th, 6th, 7th and 8th term is 240, calculate the value of the first term.;b) The first term of a geometric progression is a and the common ratio is r. Given that a+96r=0 and that the sum to infinity is 32, find the 8th term.;c) The first and second terms of a geometric progression are 10 and 11 respectively. Find the least number of terms such that their sum exceeds 8000.Answers:

ID: 199902001003
Content:
a) A geometric progression has first term a and common ratio r. The 3rd term is 32 and the eighth term is -243. Calculate ;i) the value of r and of a,;ii) the sum of the first ten terms.;b) Find the common ratio of a geometric series whose sum to infinity is 4 times its first term.;c) An arithmetic progression with first term 8 and common difference d, consists of 101 terms. Given that the sun of the last 3 terms is 3 times the sum of the first 3 terms, find;i) the value of d,;ii) the sum of the last 48 terms.Answers:

ID: 199904001002
Content:
a) The arithmetic progression has first term -91 and common difference 6. Given that the arithmetic progression has three times as many positive terms as it has negative terms, find;i) the total number of terms in the progression,;ii) the sum of the positive terms.;b) The ninth term of an arithmetic progression is 6. Find the sum of the first 17 terms.;c) A geometric progression is such that the sum of the first 3 terms is 0.973 times the sum to infinity. Find the common ratio.;Given that the first term of the progression is 823, find the sixth term, giving the answer to the nearest whole number.Answers:

ID: 200002001002
Content:
a) In an arithmetic progression the 20th term is 92 and the sum of the first 20 terms is 890. Find;i) the first term and the common difference,;ii) the sum of the first 10 terms.;b) In a geometric progression the second term is a and the common ratio is r, where |r| < 1. The sum of the first n terms is \[S_n\] and the sum to infinity is \[S_{oo}\] `;Show that \[S_{oo}-S_n=\frac{(ar^n)}{(1-r)}\]; A geometric progression has a first term of 5 and a common ratio of 0.8. The sum of the first n terms differs from the sum to infinity by less than 0.6. Find the least possible value of n.Answers:

ID: 200102001005
Content:
An arithmetic progression is such that the eighth term is twice the second term and the eleventh term is 18. Find;i) the first term and the common difference;ii) the sum of the first 26 terms;iii)the smallest number of terms of the progression whose sum exceeds 300.;b) A geometric progression, with a positive common ratio, has first term of 75 and a third term of 48.;Find ;i) the common ratio;ii) the sum to infinity;iii) the value of the largest term in the progression which is less than 0.2Answers:

\end{document}
