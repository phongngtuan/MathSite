\documentclass{article}
\begin{document}
ID: 200004003006
Content:
A computer generates a random variable X whose probability distribution is given in the following table. img ;Show that Var(X) = 4. Find E($$X^4$$) and Var($$X^2$$). Two independent observations of X are denoted by $$X_1$$ and $$X_2$$. Show that P($$X_1$$+$$X_2$$ =6) = 0.2, and tabulate the probability distribution of $$X_1$$+ $$X_2$$. The sum of 100 independent observations of X is denoted by S. Describe fully the approximate distribution of S.  Answers:
283: 
284: E(X^4) = "598.4".
285: Var(X^2) = "198.4".
286: 
287: Since the no. of observations is large, Central Limit Theorem \Rightarrow S ~ "N"("\mu","\sigma^{2}") approximately.

ID: 200104003006
Content:
A random variable X has the probability distribution given in the following table. ;img;;(i) Given that E(X) = 4, find p and q.;(ii) Show that Var(X) = 1.;(iii) Find E(|X-4|).;(iv) Then independent observations of X are taken. Find the probability that the value 3 is obtained at the most three times.  Answers:
365: p = "\frac{1}{10}".
366: q = "\frac{4}{10}".
367: 
368: E(|X-4|) = "\frac{8}{10}".
369: P(the value 3 is obtained at the most three times) = "0.879".

ID: 200104003010
Content:
A random sample of 90 batteries, used in a particular model of mobile phone, is tested and the 'standby time', x hours, is measured. The results are summarized by $$\Sigma x=3040.8$$ and $$\Sigma x^2 = 115773.66$$. Test, at the 1% significance level, whether the mean standby-time is less than 36.0 h. This type of battery is advertised as having a 'talk-time' of 5 hours. In a random sample of 50 of these batteries, 6 are found to have a talk-time of less than 5 hours. Find a 90% confidence interval for the population proportion of this type of battery that have a talk-time of less than 5 hours. In a test at the 55 significance level it is found that there is significant evidence that the population mean talk-time is less than 5 hours. ;a) There is significant evidence at the 10% significance level that the population mean talk-time is less than 5 hours.;b) There is significant evidence at the 5% significance level that the population mean talk-time is not 5 hours. ;Using only this information, and giving a reason in each case, state whether each of the following statements is;(i) necessarily true,;(ii) necessarily false, or;(iii) neither necessarily true nor necessarily false. Answers:

ID: 200204003024
Content:
The speeds of 120 randomly selected cars are measured as they pass a camera on a motorway. Denoting the speed by x km per hour, the results are summarized by $$\Sigma (x-100) = -221$$, $$\Sigma (x-100)2=4708$$. Giving your answer correct to 2 places of decimals, find;(i) unbiased estimates of the population mean and variance,;(ii) a 96% confidence interval for the population mean. Answers:
464: \hat{\mu} = "98.16" kmh^{-1}.
465: \hat{\sigma} = "36.14" kmh^{-1}.
466: Symmetric 96% confidence interval for \mu = ["97.03","99.29"].

ID: 200404003027
Content:
The mass of each parcel in a batch of 60 randomly chosen parcels is accurately measured. the masses in kg are denoted by $$x_i + \frac{y_i}{1000}$$, where $$x_i$$ is an integer and $$0 \leq y_i < 1000$$, for $$i=1,2,...,60$$. Regarding $$y_1, y_2,...,y_{60}$$ as observations of a continuous random variable Y, state a suitable model for the distribution of Y. The formula $$\Sigma x_i +30$$ is proposed as an approximation for the total mass in kg of the batch. Justify this formula. Given that $$Var(Y) = \frac{106}{12}$$, find an approximate value for the probability that the formula $$\Sigma x_i+30$$ gives the total mass of the batch to within $$\pm$$ 5 kg. Give your answer correct to 3 significant figures. Answers:
597: A suitable model for the distribution of Y is "uniform" distribution.
598: 
599: Approximate value = "0.975".

ID: 200404003029
Content:
a) In a small school there are 3 classes, each of 30 children, and 2 classes, each of 20 children. A student takes a sample of 10 children from the school, by taking 2 children at random from each class. State, with a reason, whether this gives a random sample of 10 children from the school. ;;b) A random sample of 50 children is taken from a large school. The number of children in this sample with access to a computer at home is 34. The overall proportion of children at the school with access to a computer at home is denoted by p. Test, at the 5% significance level, the null hypothesis p = 0.8 against the alternative hypothesis p <0.8, and find a symmetric 95% confidence interval for p. Give your answer correct to 3 significant figures.Answers:
603: The procedure "does not gives" (does not gives/gives) a random sample because the children " do not have" (do not have/have) equal chanes of being selected.
604: 
605: Symmetric 95% confidence interval for p is ["0.551","0.809"].

ID: 200504003027
Content:
In a sawmill, wood is cut into planks. The length of a randomly chosen plank is denoted by X m. The random variable X has a continuous uniform (rectangular) distribution on the interval $$l - 0.03 \leq x \leq l + 0.03$$, where l is a constant. Write down the probability density function of X, and hence show that the variance of X is .0003. Each day a random sample of 50 planks is taken and the total combined length of the planks is denoted by T m. By considering the approximate distribution of the sample mean, find the approximate distribution of T. Planks are intended to have a length of 2 m. Values of T are found for 5 randomly chosen days, and it is found that $$\Sigma t = 499.53$$. Test, at the 1% significance level, whether l < 2. Answers:
829: Probability density of X = "\frac{1}{0.06}", where "l - 0.03" \leq x \leq "l+0.03" or " 0", otherwise.
830: 
831: T \sim N("50l","0.015") approximately.
832: At 1% level of significance, there is "insufficient" (sufficient/insufficient) evidence that l<2.

ID: 200604003022
Content:
In a poll of 800 electors, the number supporting George Berry as Presidential candidate is 417. Find a 99% confidence interval for the percentage of the electorate that supports George Berry as presidential candidate. State any assumption, or assumptions, that you need to make.Answers:
718: The 99% confidence interval for p is ["0.476","0.567"].
719: Assumption made is that the electors were "randomly selected".

ID: 200604003027
Content:
A calculator generates random numbers between 0 and 1. The corresponding continuous random variable is X. State the distribution of X, and calculates its variance.  A second calculator generates random numbers between -1 and 2. The corresponding continuous random variable is Y, whether y has the same distribution as 3X -1. State the mean and variance of Y.  The random variable T is given by T = $$\lambda$$ X + $$(1-\lambda)$$Y, where $$\lambda$$ is a constant and where X and Y correspond to independent random number s generated by the two calculators. Show that E(T) does not depend on $$\lambda$$, and find the value of $$\lambda$$ for which Var(T) is as small as possible. Answers:
729: Distribution of X is X \sim U ("0","1").
730: Var(X) = "\frac{1}{12}".
731: E(Y) = "\frac{1}{2}".
732: Var(Y) = "\frac{3}{4}".
733: 
734: Var(T) is smallest when \lambda = "\frac{9}{10}".

ID: 200604003029
Content:
A researcher is investigating the distribution of the amount of time per week that teenagers spend playing computer games. Using the data from a large sample, the researcher obtains the result that (3.52, 4.14) is a 95% confidence interval for $$\mu$$, the population mean number of hours in a week that teenagers spend playing computer games. Explain what is meant by '(3.52, 4.14) is a 95% confidence interval for $$\mu$$', and explain why , in obtaining the confidence interval, it is not necessary to make any assumptions about the distribution. Calculate a 99% confidence interval for $$\mu$$, correcting your answer to 3 significance figures. The researcher plans to continue the investigation one year later by calculating confidence intervals for $$\mu$$ based on two large independent random samples $$S_1$$  and $$S_2$$ . Using $$I_1$$, a 95% confidence interval $$I_1$$  and a 99% confidence interval $$I_1$$  will be calculated. Using $$S_2$$, a 95% confidence interval $$I_2$$  will be calculated. Find the probability that, in the form of percentage,;(i) Both $${I_1}^'$$ and $$I_1$$ will contain $$\mu$$ ;(ii) Exactly one of $${I_1}^'$$  and $$I_1$$ will contain $$\mu$$;(iii) Both $$I_1$$  and $$I_2$$ will contain $$\mu$$;(iv) Exactly one of $$I_1$$  and $$I_2$$  will contain $$\mu$$Answers:
739: It meant that there is a 95% condicence level that the interval (3.52,4.14) "encloses" the true value of \mu.
740: A symmetric 99% confidence interval for \mu = ["3.42","4.24"].
741: P(both I_1 and {I_1}^' to contain \mu) = "95"%.
742: P(exactly one of I_1 and {I_1}^' to contain \mu) = "4"%.
743:  P(both I_1 and I_2 to contain \mu) = "90.25"%.
744:  P(exactly one of I_1 and I_2 to contain \mu) = "9.5"%.

ID: 200704003005
Content:
(i) Give a real-life example of a situation in which quota sampling could be used. Explain why quota sampling would be appropriate in this situation, and describe briefly any disadvantage that quota sampling has.;(ii) Explain briefly whether it would be possible to use stratified sampling in the situation you have described in part(i)Answers:

ID: 200704003007
Content:
A large number of students in a college have completed a geography project. The time, x hours, taken by a student to complete the project i noted for a random sample of 150 students. The result are summarized by $$ \sum x = 4626, \sum x^2 = 147691$$. ;;Find unbiased estimates of the population mean and variance.;Test, at the 5% significance level, whether the population mean time for a student to complete the project exceeds 30hours.;State, giving a reason, whether any assumptions about the population are needed in order for the test to be valid.Answers:
796: Unbiased estimate of mean, \mu = "30.84".
797: Unbiased estimate of population variance, s^2 = "33.7".
798: "reject" (Do not reject/Reject) H_0 and conclude that, at the 5% level of significance, there is "sufficient" (sufficient/insufficient) evidence that the mean time for a student to complete the project exceeds 30hours.
799: "No assumption" (No assumption/Assumption) is needed in order for the test to be valid because the sample size is " large" (small/large), by Central Limit Theorem, the sample mean is approximately " normally" distributed.

\end{document}
