\documentclass{article}
\begin{document}
ID: 199501001010
Content:
img;In the diagram the position vectors of points A and B relative to O are a and b respectively. The lines AB and OC intersect at D and;\[ \vec{BC} = \frac{3}{2} a\] ;Given that \[\vec{OD} = p\vec{OC}\] and that \[\vec{DB} = q\vec{AB}\] express \[\vec{OD}\]  and  \[\vec{OC}\] in terms of;;i) p, a, b,;;ii) q, a, b,;;and hence evaluate p and q.Answers:

ID: 199501001016
Content:
The position vectors of points A and B relative to an origin O are j and -6i ? j respectively.;Using vector methods, find;
i) The value of p such that pi + 2j is perpendicular to \[\vec{AB}\];
ii)\[\angle OBA\];
The point C lies on AB and is such that \[\vec{AC}=k\vec{AB}\] where \[k\neq 0\];Given that \[\vec{OC}\] is a unit vector, evaluate k and obtain the position vector of C relative to O.
Answers:

ID: 199502001016
Content:
a) Four horizontal forces acting on a on a particle are represented graphically, in magnitude and direction, by the lines joining O to A(2, 1), B(5, -2), C(-3, -3) and D(-1, 2). Given that the resultant of the four forces is to be represented by the line OE, find the coordinates of E.;b) The resultant, R, of the force P, of magnitude 10 N, and the force Q, of magnitude 15 N is perpendicular to P. Find the magnitude of R and the angle between the directions of P and Q.;c) The diagram shows a light extensible string, BAC, attached to two fixed points B and C, passing through a smooth ring A which is suspended by the string. The mass of A is 0.5 kg. The ring is drawn to one side by a horizontal force P N, so that system is in equilibrium with the two parts of the string making angles of \[25^{\circ}\]  and \[40^{\circ}\]  with the vertical.;Find the tension in the string and the value of P.Answers:

ID: 199503001009
Content:
The position vectors of points A, B and C, relative to a fixed point O are \[\binom{4}{9}\], \[\binom{6}{-5}\] and \[\binom{7}{12}\] respectively.;Evaluate \[\vec{AB}\cdot \vec{AC}\] and hence determine \[\angle BAC\]Answers:

ID: 199601001008
Content:
The position vectors of the points A and B, with respect to an origin O, are a and b respectively. Given that a = 5i + 2j and b = i + 11j, find ;a) a.b,;b) \[\angle AOB\];c) the magnitude of a + 2bAnswers:

ID: 199601001017
Content:
a) The point P lies on the line AB such that \[\vec{AP}=\frac{1}{4}\vec{AB}\] ;The position vectors, relative to an origin O, of the points A, B and P are a, b and p respectively. Express p in terms of a and b .;Given that 2p and 3a are the position vectors, relative to O, of the points Q and R respectively, show that B, Q and R are collinear. ;img;b) In the diagram, Z is the point of intersection of the diagonals of the quadrilateral OJKL. The position vectors, relative to O, of the points J and L are 5v and 4u + v, respectively. Find \[\vec{JL}\] in terms of u and v.;Given that \[\vec{JZ} = s\vec{JL}\] find \[\vec{OZ}\] in terms of u, v and s.;The position vector, relative to O, of the point K is 3u + 12v. Given that \[\vec{OZ} = t\vec{OK}\]  evaluate s and t.Answers:

ID: 199603001008
Content:
Given that a = 10j and b = -8i + j, find the unit vector which is parallel to a-3bAnswers:

ID: 199603001014
Content:
The three points O, A and B are such that \[\vec{OA}=\binom{10}{5}\] and \[\vec{OB}=\binom{1}{3}\];i) Using vector methods calculate the \[\angle AOB\];ii) Find the length of AB .;Given that \[\vec{OC}=\binom{c}{3c}\] where \[c > 1\] and that \[|\vec{AB}|=|\vec{AC}|\] calculate the value of c .;Given that D is a point on OB produced and that \[\angle BAD = 90^{\circ}\] find the vector \[\vec{OD}\]Answers:

ID: 199701001010
Content:
The position vectors of points A, B and C, relative to an origin O, are a, b and c respectively. The mid-point of AB is M.;(a)	Find the position vector of M, in terms of a and b. ;The point D lies on the line CM such that \[\vec{CD}=2\vec{DM}\] ;(b)	Find the position vector of D, in terms of a, b and c.Answers:

ID: 199701001017
Content:
The position vectors of points A and B, relative to an origin O, are \[\vec{OA}=10i+9j\] and \[\vec{OB}=-14i+2j\];(a)	Use a vector method to find  \[\angle AOB\]; A third point, P, has position vector \[\vec{OP}=pi+13j\];(b)	Find the vectors \[\vec{AP}\] and \[\vec{BP}\]in terms of p, I and j.;(c)	Use a vector method to find the values of p for which AP and BP are perpendicular.;(d)	Find the value of p for which AP and BP are equal in length.Answers:

ID: 199703001008
Content:
The diagram shows the parallelogram OABC. Given that \[\vec{OA}=3i+j\] and that ; \[\vec{OC}=4-2j\];i) find \[\vec{OB}\];(Note: Please enter your answer in the form of coordinates);ii) use a scalar product to find the acute angle between the diagonals of the parallelogram.Answers:

ID: 199703001016
Content:
The position vectors of the points A, B and C, relative to an origin O, are a, b and a + 2b respectively. AB and OC meet at D, where \[\frac{(AD)}{(AB)}=p\] and \[\frac{(OD)}{(OC)}=q\] ;Express \[\vec {OD}\]  in terms of;(a)	a, b and p,; (b)	a, b and q.;Hence evaluate p and q.Answers:

ID: 199801001014
Content:
(a)	The position vectors of points A and B relative to an origin O are \[\binom{2}{3}\] and \[\binom{5}{1}\] respectively. The point C is such that OACB is a parallelogram. Find the value of \[\vec{AB}\ast \vec{AC}\]and hence find the  \[\angle ACB\]; (b)	The position vectors of three points P, Q and R relative to an origin O are p, q and 4p respectively. The point M is the midpoint of PQ and the point S lies on OM produced such that \[OS = \frac{8}{5}OM\] ;Express \[\vec{OS}\]and \[\vec{QR}\]in terms of p and q. Deduce that S lies on QR and find the ratio QS : SR.Answers:

ID: 199803001015
Content:
The position vectors of the points L, M and N, relative to an origin O, are d, e and 2d + 2e respectively. The point P lies on LM and is such that \[\vec{LP}=\frac{2}{5}\vec{LM}\] The line OP is produced to meet the line LN and Q. Given that \[\vec{OQ}=\lambda\vec{OP}\] and that \[\vec{LQ}=\mu\vec{LN}\]express \[\vec{OQ}\] in terms of;(a) \[\lambda, d,e\];(b) \[\mu,d,e\]; Hence determine the value of \[\lambda\] and \[\mu\]Answers:

ID: 199901001016
Content:
img;The diagram shows the positions of the points A, B, C and D relative to the origin O.;The position vectors of A and B relative to O are 7i + 4j and I + 8j respectively. The point C is such that \[\vec{OC}=\frac{3}{2}\vec{AB}\] Find;i) \[\vec{OC}\];ii) \[\vec{CB}\];The point D is such that \[\vec{CD}=\lambda \vec{CB}\]  where \[\lambda >1\];iii) Obtain an expression, in terms of i, j and \[\lambda\] for \[\vec{OD}\];Given that \[\vec{AB}\]is perpendicular to \[\vec{OD}\];iv) use a scalar product to find the value of \[\lambda\];v) show that \[\vec{OD}\] bisects \[\vec{AB}\]Answers:

ID: 199903001003
Content:
Relative to an origin O the position vectors of the points P and Q are 3i + j and 7i-15j respectively. Given that R is the point such that \[3\vec{PR}=\vec{RQ}\] find a unit vector in the direction \[\vec{OR}\]Answers:

ID: 199903001016
Content:
The position vectors of points P and Q relative to an origin O are \[\binom{4}{2}\] and \[\binom{k}{4}\]respectively. Given that \[\vec{QP}\] is perpendicular to \[\vec{QP}\] use a scalar product to find ;i) the value of k,;ii) \[\angle QOP\];The position vector of the point R relative to the origin O is \[\binom{-8}{6}\] Given that X is a point on RP such that \[\vec{RX}=\lambda \vec{RP}\];iii) express \[\vec{OX}\] as a column vector in terms of \[\lambda\];Given also that \[\vec{OX}=\mu \vec{OQ}\];  iv) find the value of \[\lambda\] and of \[\mu\]Answers:

ID: 200001001006
Content:
The points M and N lie on a circle, centre O and radius 2. The position vectors of M and N, relative to O, are 1.92i + mj and ni + 1.6j respectively, where m and n are positive numbers.;(a) Find the value of m and of n.;(b)Use a scalar product to find the \[\angle MON\]Answers:

ID: 200001001011
Content:
a) The position vectors, relative to an origin O, of the three points P, Q and C are 3a, 2b and c respectively. Given that \[\vec{PC}=k\vec{PQ}\] express c in terms of a, b and k. Given also that a and b are the position vectors, relative to O, of the points A and B respectively, and that \[\vec{AC}=m\vec{AB}\] find the value of k and of m.;(b) ABCD is a rectangle whose diagonals intersect at O, and L is a point in the plane of ABCD. Show that \[\vec{AL}+\vec{BL}+\vec{CL}+\vec{DL}=4\vec{OL}\]Answers:

ID: 200003001010
Content:
(a) Find the positive value of p for which 0.6i + pj is a unit vector.;(b) The vector 3i + 4j is parallel to the vector a + 3b, where a = qi-j and b = i+ qj. Find the value of q.Answers:

ID: 200003001016
Content:
Solutions to this question by accurate drawing will not be accepted.;(a)	The four points O, A, B and D are such that \[\vec{OA}=\binom{8}{4}\];\[\vec{OB}=\binom{12}{6}\]; \[\vec{OC}=\binom{d}{3d}\];(i)	Show that O, A and B lie of the same straight line.;(ii)	Find the value of d given that \[|\vec{AD}|=|\vec{BD}|\];(b)	The position vectors of the points P and Q relative to an origin O are p and q respectively, The point R is such that \[\vec{OR}=\frac{3}{4}\vec{OP}\]  and the point S is the mid-point of PQ.;(i)	Express \[\vec{RQ}\] and \[\vec{OS}\] in terms of p and q.;RS is produced to meet OQ produced at the point T so that \[\vec{ST}=\lambda \vec{RS}\] and \[\vec{QT}=\mu \vec{OQ}\]  Express \[\vec{ST}\]in terms of ;(ii)	\[\lambda\], p and q,;(iii)	\[\mu\]  p and q.;Hence find the value of \[\lambda\] and of \[\mu\]Answers:

ID: 200101001010
Content:
img;In the triangle OAB shown above, \[\vec{OA}=\alpha,\vec{OB}=b\] and \[\vec{AP}=\frac{2}{5}\vec{AB}\]; (a) Find \[\vec{OP}\]in terms of a and b .;The point Q lies on OB such that QP is parallel to OA. Given that M is the mid-point of QP.;(b)	Find \[\vec{OM}\] in terms of a and b .Answers:

ID: 200101001014
Content:
img;(a)	The points A, B, C and D are the vertices of a parallelogram. The position vectors of A, B and D, relative to an origin O, are -2i + 3j, I + 5j, and 6i - j respectively.;(i)	Find the position vector of C.;(ii)	Evaluate \[\vec{AB}\ast \vec{AC}\] and hence find the \[\angle BAC\].;(b) The unit vector p is in the direction of 3i + 4j and the unit vector q is in the direction of 5i + 12j.;(i)	Find p and q.;(ii)	Use a scalar product to show that p - q is perpendicular to p + q.Answers:

ID: 200103001015
Content:
img;(a)	Points A, B and C have position vectors a, b and c respectively, relative to an origin O. ;The point P lies on BC such that BP : PC = 1 : 2. The point Q lies on AP produced such that AP : PQ = 1 : 2. Find, in terms of a, b and c,;(i) \[\vec{OP}\];(ii) \[\vec{OQ}\];Show that \[\vec{CQ}\] is parallel to \[\vec{AB}\];img;(b)	The points A, B, C and D, shown in the diagram, are vertices of a parallelogram ABCD. ;The position vectors of A, B and C relative to O, are 3i + 4j. 2i + j and 5i + 2j respectively. Find;(i) \[\vec{AC}\];(ii) \[\vec{BD}\];(iii) \[\vec{OD}\];The point A? is the reflection of A in the x-axis.;(iv)	Show that the points A?, C and D are collinear and find the ratio A?C : CD.Answers:

ID: 200201001004
Content:
The points P, Q and R are such that $$\vec{QR}=4\vec{PQ}$$. Given that the position vectors of P and Q relative to an origin O are $$\binom{6}{7}$$ and $$\binom{9}{20}$$ respectively, find the unit vector parallel to $$\vec{OR}$$Answers:

ID: 200202001005
Content:
A plane flies from an airport A to an airport B . The position vector of B relative to A is (1200i + 240j) km, where I is a unit vector due east and j is a unit vector due north. Because of the constant wind which is blowing, the flight takes 4 hours. The velocity in still air of the plane is $$(250i + 160j) kmh^{-1}$$. Find the speed of the wind and the bearing of the direction from which the wind is blowing.Answers:

ID: 200203001010
Content:
img;At 1200 hours, ship P is at the point with position vector 50j km and ship Q is a the point with position vector (80i + 20j) km, as shown in the diagram. Ship P is traveling with velocity (20i + 10j) km per h and ship Q is traveling with velocity (-10i + 30j) km per hour.;i.	Find an expression for the position vector of P and of Q at time t hours after 1200 s.;ii.	Use your answers to part (i) to determine the distance apart of P and Q at 1400 hours.;iii.	Determine, with full working, whether or not P and Q will meet.Answers:

ID: 200204001010
Content:
img;In the diagram, $$\vec{OA}=a$$, $$\vec{OB}=b$$, $$\vec{AM}=\vec{MB}$$ and $$\vec{OP}={\frac{1 }{3}}\vec{OB}$$;(a)	Express $$\vec{AP}$$ and $$\vec{OM}$$ in terms of a and b,;(b)	Given that $$\vec{OQ}=\lambda \vec{OM}$$, express $$\vec{OQ}$$ in terms of $$\lambda$$, a and b,;(c)	Given that $$\vec{AQ}=\mu \vec{AP}$$, express $$\vec{OQ}$$ in terms of $$\mu$$, a and b .;(d) 	Hence find the value of $$\lambda$$ and of $$\mu$$Answers:

ID: 200302001002
Content:
The position vectors of points A and B, relative to an origin O, are 6i- 3j and 15i + 9j respectively.;(a)	Find the unit vector parallel to $$\vec{AB}$$;The point C lies on AB such that $$\vec{AC}=2\vec{CB}$$;(b)	Find the position vector of C.Answers:

ID: 200303001006
Content:
In this question, i is a unit vector due east and j is a unit vector due north. A plane flies from P and Q. The velocity, in still air, of the plane is $$(280i -40j) kmh^{-1}$$ and there is a constant wind blowing with velocity $$(50i - 70j) kmh^{-1}$$. Find;(a)	the bearing of Q from P,;(b) the time of flight, to the nearest minute, given that the distance PQ is 273 km.Answers:

ID: 200403001001
Content:
The position vectors of points A,B and C, relative to an origin O, are i+9j, 5i-3j and k(i+3j) respectively, where k is a constant. Given that C lies on the AB, find the value of k.Answers:

ID: 200503001005
Content:
img;The diagram which is not drawn to scale, shows a horizontal rectangular surface. One corner of the surface is taken as the origin O and I and j are unit vectors along the edges of the surface.;A fly, F, starts at the point with position vector (i+12j) cm and crawls across the surface with a velocity of \[(3i+2j) cms^{-1}\]. At the instant that the fly starts crawling, a spider, S, at the surface with a velocity of \[(-5i+kj) cms^{-1}\] where k is a constant. Given that the spider catches the fly, calculate the value of k.Answers:

ID: 200504001007
Content:
img;In the diagram $$\vec{OP}=p, \vec{OQ}=q, \vec{PM}= \frac{1}{3}\vec{PQ}$$ and $$\vec{ON}= \frac{2}{5}\vec{OQ}$$;(i)	Given that $$\vec{OX}$$ and $$m \vec{OM}$$, express $$\vec{OX}$$ in terms of m, p and q.;(ii)	Given that $$\vec{PX}=n \vec{PN}$$,express $$\vec{OX}$$ in terms of n, p and q.;(iii)	Hence evaluate m and n.Answers:

ID: 200603001004
Content:
The vector $$\vec{OP}$$ has a magnitude of 10 units and is parallel to the vector 3i-4j. The vector $$\vec{OQ}$$ has a magnitude of 15 units and is parallel to the vector 4i+3j.;(i)	Express $$\vec{OP}$$ and $$\vec{OQ}$$ in terms of I and j.;(ii)	Given that the magnitude of $$\vec{PQ}$$ is $$\lambda \sqrt{13}$$,;Find the value of $$\lambda$$.Answers:

ID: 200703001011
Content:
img;In the diagram, $$\vec{OP}=p$$, $$\vec{OQ}=q$$, $$\vec{PR}=2\vec{OQ}$$ and $$\vec{OY}=\vec{YR}$$,;(i) Express $$\vec{OY}$$ and $$\vec{QR}$$ in terms of p and q.;(ii) Given that $$\vec{PX}=\lambda \vec{PY}$$, express $$\vec{PX}$$ in terms of $$\lambda$$, p and q.;(iii) Given that $$\vec{QX}=\mu \vec{QR}$$ , express $$\vec{PX}$$ in terms of $$\mu$$, p and q.;(iv) Hence find the value of $$\lambda$$ and of $$\mu$$.Answers:

ID: 200704001006
Content:
In this question i is a unit vector due east and j is a unit vector due north.;A plane flies from A to B, where B is 900 km due east of A. The velocity, in still air, of the plane is $$(270i-50j) kmh^{-1}$$ and there is a wind blowing with a constant velocity of $$(pi+qj) kmh^{-1}$$.;(i) Find the value of q.;(ii) Given that the journey takes 3 hours, show that p=30.;The plane returns from B to A with the same wind blowing and the velocity, in still air, of the plane is now $$(-270i-50j) kmh^{-1}$$.;(iii) Calculate the time taken for the return journey.Answers:

\end{document}
