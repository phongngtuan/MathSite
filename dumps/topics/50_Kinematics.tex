\documentclass{article}
\begin{document}
ID: 199501001013
Content:
A particle moves in a straight line so that t seconds after leaving a fixed point O, its velocity, \[v ms^{-1}\] is given by \[v=6+pt-t^{2}\] where p is a constant. The particle comes to an instantaneous rest at the point A, 6 seconds after leaving O.;
i) Evaluate p.;
ii) Calculate the distance OA.;
iii) Calculate the acceleration of the particle at A.
Answers:

ID: 199502001009
Content:
(a)	A particle P, moving in a straight line, passes a point A with a constant speed of \[8 ms^{-1}\] Three seconds later a second particle Q, traveling in the same direction as P, passes A with a constant speed of \[10 ms^{-1}\] The particle Q overtakes P at a point B, where AB = s metres, t seconds after P left A. Draw, on the same diagram, the distance-time graphs for the motions of P and Q from A to B, and evaluate s and t.;(b)	A car passes a service depot on a motorway with a speed of \[20 ms^{-1}\]  and a constant retardation of \[0.2 ms^{-2}\] At the same instant a motor-cyclist leaves the service depot, starting from rest, and with a constant acceleration of \[0.6 ms^{-2}\] The motor-cyclist passes the car at time T seconds after leaving the service depot at a point whose distance from the service depot is S metres. Find;(i) the value of T and of S,;(ii) the speed of each vehicle at time T seconds.;(iii) Sketch, on the same diagram, the velocity-time graphs of the motor-cyclist and the car over the time interval of T seconds.;(iv)	Find the time, after leaving the service depot, when the speeds of the vehicles are the same.Answers:

ID: 199502001010
Content:
A particle, P, is dropped from rest at A and strikes a horizontal plane at B, where AB = 20 m. At the same time that P is released, a second particle, Q is projected with a speed of \[15ms^{-1} \]  from a point D, level with A, directly down a smooth plane inclined at an angle of \[30^{\circ}\] to the horizontal. The point E on the plane is at the same level as B .;i) Show that P and Q arrive simultaneously at B and E respectively and find their speeds at these points.;The particle P rebounds from B at \[12ms^{-1} \] and rises to C, the highest point reached after leaving B .;ii) Calculate the distance BC and the time taken by P to go from B to C.;The particle Q continues to move down the plane without interruption and when P is at C, the particle Q is at F.  ;iii) Calculate the speed of Q at F and the distance EF.
Answers:

ID: 199502001011
Content:
An archer shoots an arrow with a speed of \[25ms^{-1} \] in a direction inclined at an angle\[\alpha\] to the horizontal, where \[\sin \alpha  = 0.28\] Find the vertical and horizontal components of the initial velocity.;The arrow strikes the horizontal plane through the point of projection. Find its time of flight and the horizontal distance it travels. ;The diagram shows the path of a second arrow, shot from A with the same speed and direction as the first arrow. This second arrow strikes a target at right angles at B . The target is inclined at an \[\angle \beta \] to the horizontal as shown, where \[\tan \beta =4\] ; Calculate ;i) the vertical component of the velocity of the arrow at B,;ii) the speed of the arrow when it hits the target,;iii) the time of flight from A to B,;iv) the horizontal and vertical distances of B from A. Answers:

ID: 199502001012
Content:
img;(a) The diagram shows a river flowing at \[1.5 ms^{-1}\] between parallel banks.; A rower, whose speed through still water is \[2 ms^{-1}\] rows from A to the point B immediately opposite A. The rower points the boat in a direction inclined at an \[\angle \alpha \] to the bank. Find the value of \[\alpha \] Given that AB = 120 m, find the time taken for the crossing, to the nearest second.;(b) At a particular instant particles P and Q are 36 cm apart and are moving in a horizontal plane with constant speeds and directions as shown in the diagram.;img; Given that they collide 3 seconds later, find;(i) the velocity, in magnitude and direction, of P relative to Q,;(ii) the value of v and of \[\theta\] (Note: Please enter the smaller value of \[\theta\] first in the answer space);(c) A car is traveling at \[80 kmh ^{-1}\] in a direction due north, and the wind is blowing at \[30 kmh ^{-1}\] from a direction \[240^{\circ}\].  Find the angle that a streamer, tied to the car roof, makes with the direction of motion of the car.Answers:

ID: 199502001013
Content:
a) Particles A and B, of mass 0.5 kg and 0.3 kg respectively, are connected by a light extensible string passing over a smooth fixed peg. The particles are held at the same height , 1.8 m, above a horizontal floor and then released. Calculate 
;i) the acceleration of the particles and the tension in the string during the motion,;ii) the time taken for A to reach the floor and its speed on impact.;After A strikes the floor, B rises freely under the action of gravity,;iii) Calculate the greatest height above the floor reached by B ;b) A particle of mass 3 kg slides down a line of greatest slope of a rough plane inclined at an angle of \[27^{\circ}\]  to the horizontal. Given that the coefficient of friction between the particle and the plane is 0.2, calculate the acceleration of the particle down the plane.
Answers:

ID: 199502001015
Content:
a) A bullet, of mass 120 grams, is fired horizontally into a stationary block of mass 2.88 kg. The block is free to move on a smooth horizontal plane. The bullet strikes the block with a speed of \[150ms^{-1} \] and is then embedded in the block. Find ;i) the final speed of the block and the bullet combined,;ii) the impulse exerted by the block on the bullet.;    This impulse occurs over a period of 0.02 seconds.;iii) Find the average resistance if the block to the bullet.;b) Particle A, of mass 0.21 kg, and b, of mass 0.7 kg, are suspended from a fixed point by two light inextensible strings of equal length, as shown in the diagram. Particle A is drawn aside, keeping the string taut, and released from rest at a vertical height of 1.25 m above B, which is at rest.;i) Show that A strikes B with a speed of \[5ms^{-1} \] Immediately after the collision the particles move in opposite directions, each with a speed of \[vms^{-1} \] Calculate  ;ii) the value of v,;iii) the height to which each particle rises after the collision.;The particle collide for a second time and A rebounds with a speed of \[3.6ms^{-1} \].;iv) Find the magnitude and direction of the velocity of B after this second collision.Answers:

ID: 199503001007
Content:
A particle move is a straight line so that , t seconds after leaving a fixed point O, its displacement , s m, is given by \[s=\frac{1}{3}t^3-2t^2+3t\];Given that the particle returns to O when t = T, find the value of T.;Using this value of T, find ;i) the maximum displacement from O of the particle during the interval \[0<=t<=T\] ;ii) the acceleration of the particle at time T seconds.Answers:

ID: 199504001007
Content:
A particle moves in a straight line so that, at time t seconds after leaving a fixed point O, its velocity v \[ms^{-1}\] is given by `\[v=\frac{27}{(2t+1)^2}-3\];Find ;i) the value of t for which the particle is at instantaneous rest,;ii) the initial acceleration of the particle,;iii) the displacement of the particle from O when \[t=\frac{1}{2}\]Answers:

ID: 199504001010
Content:
A car accelerates uniformly from rest, at a rate of \[1.6ms^{-2}\] to a speed of \[20ms^{-1}\] It then continues moving at \[20ms^{-1}\]  It then continues moving at \[20ms^{-1}\] for T seconds before decelerating uniformly to a speed of \[12ms^{-1}\] in a further 5 seconds.;(a)	Sketch the velocity-time diagram for the motion of the car.;Find;(b)	The time taken while accelerating,;(c)	The distance moved while accelerating,;(d)	The distance moved while decelerating.;Given that the total distance moved by the car is 1245 m, find the value of T.;On another occasion the car accelerates uniformly from rest at a point A until it reaches B with a speed of \[15ms^{-1}\]. At B the driver observes an obstruction ahead at a point C and immediately applies the brakes, causing the car to decelerate uniformly to rest 5 m before the obstruction. Given that the car is in motion for 12 seconds, sketch the velocity-time diagram for this motion and find the distance AC.;Given also that the magnitude of the deceleration is four times that of the acceleration, find the distance BC.Answers:

ID: 199504001011
Content:
(a)	    A particle is projected vertically upwards from the ground with an initial speed of \[54ms^{-1}\] Find;(i)	the speed of the particle when it is 117 m above the ground,;(ii)	the length of time for which the particle is above 117 m.;(b)	A stone, P is dropped from rest from the edge of a cliff 80 m high. Find ;(i)	the time P takes to reach the ground,;(ii)	the speed with which P strikes the ground.;One second after P is released, a second stone, Q is also dropped from rest from the edge of the cliff. Find ;(iii)	the height of Q above the ground at the moment when P strikes the ground. At the instant when P strikes the ground, a third stone, R, is projected vertically downwards from the edge of the cliff at a speed of \[u ms^{-1}\]. Given that Q and R strike the ground simultaneously,;(iv)	find the value of u,;(v)	show that the ratio of the final speed of R to the final speed of Q is 17 : 8.Answers:

ID: 199601001014
Content:
A particle travels in a straight line in such a way that, t seconds after passing through a fixed point O, its displacement from O is s meters. Given that \[s=4-8(t+2)^{-1}\];find;i)	expressions, in terms of t, for the velocity and acceleration of the particle,;ii)	the value of t when the velocity of the particle is \[0.125m s^{-1}\];iii)	the acceleration of the particle when it is 3.5 metres away from O.Answers:

ID: 199603001016
Content:
A girl runs in a straight line for 25 seconds. Her speed after t seconds, where \[0\leq t\leq 25\] is \[vms^{-1}\] where \[v=0.75t-0.03t^2\];Find;i) the time which the girl?s acceleration is zero,;ii) the distance the girl runs.Answers:

ID: 199701001007
Content:
A particle moves in a straight line so that ,t seconds after passing a fixed point O, its velocity, \[vcms^{-1}\] is given by \[v=t^2-5t+4\];Find,;i) the values of t when the particle is at instantaneous rest;(Note: In the case of multiple answers, please enter the answer in ascending order);ii) the distance between the positions at which the particle is at instantaneous rest.Answers:

ID: 199703001010
Content:
A particle moves in a straight line so that, t seconds after passing through a fixed point O, its velocity, \[vms^{-1}\] is given by \[v=5t^2+t(1-3p)+p\] where p is a constant.;i) Find an expression for the acceleration of the particle in terms of t and p; ii) Given that the acceleration of the particle is \[3ms^{-2}\]when t = 2, find the value of p; iii) Using your value of p, find the values of t when the particle is at instantaneous rest.(Note: Please enter the smaller value of t first in the answer space)Answers:

ID: 199801001006
Content:
A particle moves in a straight line so that, t seconds after passing through a fixed point O, its velocity, \[vms^{-1}\] is given by \[v=5t-3t^2+2\] The particle comes to instantaneous rest at the point Q. Find; i) the acceleration of the particle at Q,; ii) the distance OQ;iii) the total distance travelled in the time interval t = 0 to t = 3.Answers:

ID: 199803001007
Content:
img;The diagram shows two points A and B on a straight line, where AB = 4m. A particle P moves along the line so that the velocity, \[vms^{-1}\] is given by \[v=t^2-4t-5 \] \[t\geq 0\] where t is the time in seconds after leaving B . Initially particle P is at B, moving towards A.;Find an expression, in terms of t, for ;i) the acceleration of P;ii) the distance of P from A.;Find ;iii) the distance from A of the point where P comes to instantaneous to rest.;iv) the total distance travelled by P in the time interval t = 0 to t = 10.Answers:

ID: 199804001006
Content:
A particle moves in a straight line so that at time, t, seconds after leaving a fixed point O, its velocity \[vms^{-1}\] is given by \[v=20e^{\frac{-t}{4}}\]; i) Sketch the velocity - time curve.;ii) Find the value of t when v = 10.;iii) Find the acceleration of the particle when v = 10.;iv) Obtain an expression, in terms of t, for the displacement from O of the particle at time t seconds.Answers:

ID: 199901001008
Content:
The velocity \[vms^{-1}\] of a particle, travelling in a straight line, at time t s after leaving a fixed point O, is given by \[v=10+kt-3t^2\] where \[t\geq 0\] and k is a constant. When t = 0 the particle is at O and its acceleration is \[1 ms^{-2}\] Find;i) the value of k,;ii) the value of t when the particle is instantaneously at rest,;iii) the distance the particle has travelled when it is again at O.Answers:

ID: 199902001006
Content:
A particle moves in a straight line so that, at time t s after leaving a fixed point O, its displacement from O is s m and its velocity is \[v\frac{m}{s}\] Given that \[s=e^t-2e^{-t}+1\]  where \[t\geq 0\] find;(i)	the value of s when \[t = \ln 5\];(ii)	an expression for v in terms of t,;(iii)	the value of t for which v  = 4.5.Answers:

ID: 199903001013
Content:
A particle A moves in a straight line so that its displacement, s m, from a point O at time t s, where \[t\geq 0\] is given by \[s=t^3-4t^2-3t+5\];Find ;i) the value of t when the particle is instantaneously at rest and the distance the particle has then travelled,;ii) the value of t, to two decimal places, when the particle has returned to its initial position.;A particle B moves on a parallel straight line so that its acceleration \[a ms^{-2}\] at time t s is given by a= 2t + 1. Given that A and B have the same velocity when t = 5, obtain an expression, in terms of t, for the velocity of B .Answers:

ID: 200001001014
Content:
A particle P moves in a straight line so that, t seconds after passing through a fixed point O, its velocity \[vcms^{-1}\] is given by \[3t^2-15t+18\];Find;i) the value of t when the velocity of P is equal to its initial velocity,;ii) the values of t for which P is instantaneously at rest,;iii) an expression, in terms of t, for the distance of P from O at time t,;iv) the total distance travelled by P in the first 4 seconds after passing through O,;v) the distance P from O when the acceleration of P is zero.Answers:

ID: 200002001007
Content:
A particle moves in a straight line so that , at time t seconds after passing through a fixed point, its velocity \[v m s{-1}\] is given by \[v = 6 \cos 2t\] Find ;i) the two smallest positive values of t for which  the particle is at instantaneous rest,;ii) the distance between the positions of instantaneous rest corresponding to these two values of t,;iii) the greatest magnitude of the acceleration.Answers:

ID: 200003001007
Content:
A particle moves in a straight line so that, t seconds after leaving a fixed point O, its displacement, s metres from O, is given by \[s=9y^2+6t^3-2t^4\];Find;i) the positve value of t for which the particle is instantaneously at rest,;ii) the total distance travelled by the particle from t = 0 to t = 4.;iii) the acceleration of the particle when t = 1.Answers:

ID: 200101001007
Content:
A particle moves in a straight line so that, t seconds after leaving a fixed point O, its velocity,; \[v ms^{-1}\], is given by \[v=t^3-3t^2+2t\];Find;i) the acceleration of the particle when t = 1.5,;ii) the values of t when the particle is at instantaneous rest,(Note: Please enter your answers in ascending order);iii) the distance travelled in the interval \[0\leq t\leq 2\]Answers:

ID: 200103001009
Content:
The velocity \[vms{-1}\] of a particle, travelling in a straight line, at time t s after leaving a fixed point O, is given by \[v=3t^2-18t+32\] where \[t\geq 0\];Find ;i) the value of t which the acceleration is zero.;ii) the distance of the particle from O when its velocity is a minimum.Answers:

ID: 200104001010
Content:
A particle moves so that, t s after passing through a fixed point O, its velocity, $$vms^{-1}$$, is given by $$v=Ae^{-kt}$$, where A and k are constants. Given that when t = 0 the velocity is $$5 ms^{-1}$$ and that when t = 10 the velocity is $$3ms^{-1}$$,find ;i) the value of A and k,;ii) the acceleration of the particle when t = 10.Answers:

ID: 200202001013
Content:
A particle travels in a straight line, starting from rest at point A, passing through point B and coming to rest again at point C. The particle takes 5 s to travel from A to B with constant acceleration. The motion of the particle from B to C is such that its speed, \[v ms^{-1}\] t seconds after leaving A, is given by $$v=\frac{1}{225}(20-t)^3, 5\leq t\leq T$$;(a)	Find the speed of the particle at B and the value of T.;(b)	Find the acceleration of the particle when t = 14.;(c)	Sketch the velocity-time curve for $$0\leq t\leq T$$;(d)	Calculate the distance AC.Answers:

ID: 200203001003
Content:
The speed $$v=ms^{-1}$$ of a particle traveling from A to B, at time t s after leaving A, is given by $$v=10t-t^2$$. The particle starts from rest at A and comes to rest at B . Show that the particle has a speed of $$5ms^{-1}$$or greater for exactly $$4\sqrt5$$s.Answers:

ID: 200204001011
Content:
A car moves on a straight road. As the driver passes a point A on the road with speed of $$20ms^{-1}$$ he notices an accident ahead at a point B . He immediately applies the brakes and the car moves with an acceleration of $$a ms^{-2}$$ where $$a=\frac{3t }{2}-6$$ and t s is the time after passing A. When t = 4, the car passes the accident at B . The car then moves with a constant acceleration of $$2 ms^{-2}$$ until the original speed of $$20ms^{-1}$$ is regained at a point C. Find;(a)	the speed of the car at B ,;(b)	the distance AB .;(c)	The time taken for the car to travel from B to C .;Sketch the velocity-time graph for the journey from A to C .Answers:

ID: 200302001009
Content:
A motorcyclist travels on a straight road so that, t seconds after leaving a fixed point, his velocity, v $$ms^{-1}$$, is given by $$v=12t-t^2$$. On reaching his maximum speed at t = 6, the motorcyclist continues at this speed for another 6 seconds and then comes to rest with a constant deceleration of $$4ms^2$$;(a)	Find the total distance travelled.;(b)	Sketch the velocity-time graph for the whole of the motion.Answers:

ID: 200304001006
Content:
A particle travels in a straight line so that, t s after passing through a fixed point A, its speed $$vms^-1$$ is given by $$v=40(e^{-t}-0.1)$$ The particle comes to instantaneous rest at B . Calculate the distance AB .Answers:

ID: 200404001014
Content:
A particle, travelling in astraight line, passes a fixed point O on the line with a speed of $$0.5ms^{-1}$$.;The acceleration, $$ams^{-2}$$, of the particle, t s after passing O, is given by a=1.4-0.6t.;(i)	Show that the particle comes instantan to rest when t=5.;(ii)	Find the total distance travelled by the particle between t=0 and t=10.Answers:

ID: 200503001006
Content:
A particle starts from rest at a fixed point O and moves in a straight line towards a point A. The velocity, \[v m s^{-1}\] of the particle, t seconds after leaving O, is given by $$v=6-6e^{-3t}$$;Given that the particle A reaches A when t = ln 2, find;i) the acceleration of the particle at A,;ii) the distance OAAnswers:

ID: 200704001012
Content:
A particle starts from a fixed pointA and travels in a straight line. The velocity, $$vms^{-1}$$, of the particle, t s after leaving A, is given by $$v=1+t-\sqrt{4t+9}$$.;(i)	Find the acceleration of the particle when it is at instantaneous rest.;(ii) Obtain an expression, in terms of t, for the displacement, from A, of the particle t s after leaving A.Answers:

ID: 200904001008
Content:
A motorcycle is driven along a straight horizontal road. As it passes a point A the brakes are applied and the motorcycle slows down, coming to a rest at a point B. For the journey from A to B, the distance, s metres, of the motorcycle from A, t seconds after passing A, is given by $$s =400(1-e^{-\frac{t}{10}})-16t$$.;(i) Find an expression, in terms of t, for the velocity of the motorcycle during the journey from A to B. ;(ii) Find an expression, in terms of t, for the acceleration of the motorcycle during the journey from Ato B.;(iii) Find the velocity of the motorcycle at A.;(iv) Show that the time taken for the journey from A to B is approximately 9.163 seconds.;(v) Find the average speed of the motorcycle for the journey from A to BAnswers:

ID: 201004001008
Content:
Two particles, P and Q, leaves a point O at the same time and travel in the same direction along the same straight line. Particle P starts with a velocity of $$9 ms^{-1}$$ and moves with a constant acceleration of $$1.5 ms^{-2}$$. Particle Q starts from rest and moves with an acceleration of $$a ms^{-2}$$, where $$a=1+\frac{t}{2}$$ and t seconds is the time since leaving O. Find;(i) the velocity of each particle in terms of t,;(ii) the distance travelled by each particle in terms of t.Hence find;(iii) the distance from O at which Q collides with P,;(iv) the speed of each particle at the point of collisionAnswers:

\end{document}
