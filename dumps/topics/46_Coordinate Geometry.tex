\documentclass{article}
\begin{document}
ID: 199501001017
Content:
Solutions to this question by accurate drawing will not be accepted. ;img; The diagram shows a trapezium ABCE consisting of a parallelogram ABCD and a triangle ADE, where angle \[\angle DAE = 90^{\circ}\]; The equation of the line CDE is 5y = x + 13 and the equation of CB is y = 3x ? 3.  Calculate the coordinates of C.;Given also that the diagonals of the parallelogram intersect at M (5, 5), calculate the coordinates of A, B, D and E.Answers:

ID: 199502001003
Content:
(a) When the graph of \[y^{2}\]  against \[\sqrt{x}\]  is drawn, a straight line is obtained which has gradient 1.5 and which passes through the point (10, 16). Determine the relationship connecting x and y and evaluate x when y = 5;(b);img;The table shows experimental values of two variables x and y;It is known that x and y are related by the equation of the form form \[y = px^{2}+ q\sqrt{x}\] where p and q are constants. ;Draw the graph of \[\frac{y}{\sqrt{x}}\]  against \[x\sqrt{x}\]  for the given data and use it;(i) to evaluate p and q;(ii) to estimate the value of x when \[2y = 43\sqrt{x}\]Answers:

ID: 199502001008
Content:
A curve represented parametrically by  \[x = 1 + 3t, y = t^{2}+ 7t\] ;i) O is the origin and X is a point on the curve such that the gradient of OX is 1.5. Find the two possible values of the parameter t at X.(Note: Please enter the smaller value of t first in the answers space) ;ii) Obtain an expression for \[\frac{\mathrm{d}y }{\mathrm{d} x}\] ;iii) P is a point on the curve with parameter -2 and Q is the point on the curve with parameter -5.The tangents to the curve at P and Q meets at the point R. Find the coordinates of R and the area of the triangle PQR.  ;iv) Obtain the cartesian equation of the curve.Answers:

ID: 199503001001
Content:
Find the equation of the line which is parallel to 3y = 4x + 5 and which passes through the mid-point of the line joining (-2, 3) and (5, 9). Answers:

ID: 199503001015
Content:
img;The diagram shows a quadrilateral ABCD where A is (6, 1), B is on the x-axis and C is (1, 3). ;The diagonal BD bisects AC at right angles at M and BD = 3.5 BM. Find;(a)	the equation of BD,;(b)	the x-coordinate of B,;(c)	the coordinate of D,;(d)	the area of the quadrilateral ABCD.;By considering the area of triangle ABD, or otherwise, find the perpendicular distance from D to BA extended.Answers:

ID: 199504001004
Content:
img;a) The table shows experimental values of two variables x and y. ; It is known that x and y are related by the equation \[y=\frac{a}{\left ( x+b \right )}\]where a and b are constants.; i) Plot xy against y and obtain a straight line graph;ii) Use your graph to estimate the value of a and of b .;iii)Obtain the value of the gradient of the straight line obtained when \[\frac{1}{y}\]is plotted against x.;img;b) The variables x and y are related in such a way that when y-2x is plotted against \[x^{2}\], a straight line is obtained passing through (1, -2) and (4, 7).;Find ;i)	y in terms of x.;ii)	the values of x when y = 11. (Note: Please enter the smaller value of x first in the answer space)Answers:

ID: 199601001001
Content:
Three points are O(0, 0), A(4, 2) and B(1, 2). Find the equation of the line through B which is parallel to OA.;C is the point on this line whose x-coordinates is \[\frac{-11}{5}\]; Find the y-coordinate of C and show that OB=OC.Answers:

ID: 199601001006
Content:
A triangle has vertices A(0, 2), B(3, 7) and C(0, 6). Given that ABCD is a parallelogram, find;i)	the coordinates of D,;ii)	the area of the parallelogram ABCD.Answers:

ID: 199602001004
Content:
img;a) The table shows experimental values of two variables, x and y.;It is known that x and y are related by the equation \[y=Ae^{-kx}\] where A and k are constants. Using graph paper, plot in y against x for the above data and use your graph to estimate the value of A and of k.;img;b) The diagram shows part of the straight line graph drawn to represent the equation px + qy =xy. Given that the straight line passes through (2,0) and has a gradient of -1.5, find the value of p and of q.;c) Variables x and y are related in such a way that when x - y is plotted against xy a straight line is produced as as shown in the diagram;img; This line passes through the points (1,2) and (5,4). Find y in terms of x.Answers:

ID: 199603001001
Content:
The straight lines y=ax+9, where a is a constant, and \[y=\frac{1}{2}x-1\] are perpendicular. State the value of a and hence find the coordinates of the point of intersection of the lines.Answers:

ID: 199701001002
Content:
The points A, B and C have coordinates (-2, 1), (3, 11) and (-1, 8) respectively. The line from C which is perpendicular to AB meets AB at the point N.;(a)	Find the equation of AB and of CN.;(b)	Calculate the coordinates of N.Answers:

ID: 199701001004
Content:
Find the gradient of the curve \[y=x^2+\frac{24}{x}\] at the point P(2, 16). The tangent to the curve at P meets the x-axis at A and the y-axis at B . Calculate the area of the triangle AOB, where O is the origin.Answers:

ID: 199701001014
Content:
Solutions to this question by accurate drawing will not be accepted.;img;The diagram shows a triangle ABC, where A is (6, 9), B is (-2, 3) and C is (h, -5). Given that AB = BC, and that h is positive,;a)	find the value of h,;b)	show that \[ \angle ABC=90^{\circ}\];The midpoint of AB is M. The line through M, parallel to BC, meets AC at the point P and the x-axis at the point Q. Find;c)	the coordinates of M, P and Q,;d)	the ratio MP : PQ.Answers:

ID: 199702001004
Content:
img;(a)	The table shows experimental values of two variables x and y.;It is known that x and y are related by the equation \[y=Ab^x\] where A and b are constants.;(i)	Express the equation in a form suitable for drawing a straight line graph.;(ii)	Draw the graph and use it to estimate the value of A and of b .;(iii)	Find the value of y when x = 0.8.;(b)	Variables x and y are related in such a way that when \[\frac{y}{\sqrt{x}}\] is plotted against \[x^{2}\] a straight line is obtained. This line passes through the points (3, 5) and (5, 1) as shown on the diagram;img; Find;(i)	an expression for y in terms of x,;(ii)	the value of y when x = 4.Answers:

ID: 199703001002
Content:
A line drawn through the point A(4, 6), parallel to the line 2y = x -2, meets the y-axis at the point B .;(a)	Calculate the coordinates of B .;A line drawn through A, perpendicular to AB, meets the line 2y =x-2 at the point C .; (b)	Calculate the coordinates of C .Answers:

ID: 199703001003
Content:
Find the value of the constant c for which the line y = 2x + c is a tangent to the curve \[4x^2-6x+11\];This tangent meets the x-axis at A and the y-axis at B . Calculate the area of the triangle OAB, where O is the origin.Answers:

ID: 199703001015
Content:
Solutions to this question by accurate drawing will not be accepted.;The points A(-1, 4), B(2, 7), C and D(1, 0) are the four vertices of a parallelogram. The point E lies on BC such that \[BE =\frac{1}{3} BC\]  Lines are drawn, parallel to the y-axis, from A to meet the x-axis at N and from E to meet CD at F.; (a)	Calculate the coordinates of C and of E.;(b)	Find the equation of DC and calculate the coordinates of F. ;(c) Explain why AEFN is a parallelogram and calculate its area.Answers:

ID: 199704001003
Content:
img;a) The table shows experimental values of two variables x and y. ;Using the vertical axis for xy and the horizontal axis for \[x^3\] plot xy against \[x^3\]and obtain a straight line graph.;Make use of your graph to ;i) express x in terms of y ;ii) estimate the value of x when \[y=\frac{60}{x}\] ;b) The equation \[y=\frac{(x+c)}{(x+d)}\] where c and d are constant, can be represented by a straight line when xy-x is plotted against y as shown in the diagram.;img;Find the value of c and d.Answers:

ID: 199801001011
Content:
img;The figure shows a trapezium ABCD in which AD is parallel to BC and AB is parallel to the y-axis. The coordinates of A, D and C are (2, 12), (6, 4) and (13, 0) respectively. The point X lies on BC such that \[\angle BXA=90^{\circ}\];(a)	Find the equation of BC and of AX.;(b)	Deduce the coordinates of B and of X.;(c)	Determine the ratio BC : AD in the form n : 1.;(d)	Find the length of AX and deduce the area of the trapezium ABCD.Answers:

ID: 199802001003
Content:
img;(a) The table shows experimental values of two variables x and y.; It is known that x and y are related by an equation of the form \[y=ax+\frac{b}{\sqrt{x}}\] where a and b are constants. Using the vertical axis for \[y\sqrt{x}\] and the horizontal axis for \[x\sqrt{x}\] plot \[y\sqrt{x}\] against \[x\sqrt{x}\] and obtain a straight line graph. Make use of your graph to ;(i) evaluate a and b,;(ii) estimate the value of x when \[xy=4\sqrt{x}\];(b) The variables x and y are related in such a way that when \[\frac{y}{x}\] is plotted against x a straight line is obtained as shown in the diagram.;img; This line passes through the points (1, 1) and (5, -3). Express y in terms of x and sketch the graph of y against x.Answers:

ID: 199803001011
Content:
Solutions to this question by accurate drawing will not be accepted.;img;The diagram shows a parallelogram ABCD in which A is (8, 2) and B is (2, 6). The equation of BC is 2y = x + 10 and X is the point on BC such that AX is perpendicular to BC. Find;(a)	the equation of AX,;(b)	the coordinates of X.;Given also that BC = 5 BX, find;(c)	the coordinates of C and of D.;(d)	the area of the parallelogram ABCD.Answers:

ID: 199804001003
Content:
img;a) The table shows experimental values of two variables, x and y.  ;It is known that x and y are related by the equation \[y+10=Ak^x\] where A and k are constants. Using graph paper plot lg(y + 10) against x for the above data and use your graph to estimate ;i) the value of A and of k,;ii) the value of x when y = 0.;b) The variables x and y are related by the equation \[px^3+qy^2=1\];img;The diagram shows the straight-line graph of y2 against x3 which passes through the point \[(\frac{1}{2},\frac{1}{4})\] ;i) Given that the gradient of this line is \[\frac{3}{4}\] calculate the value of p and q,;ii) Given also that this line passes through (k, 4), find the value of k.Answers:

ID: 199901001002
Content:
A is the point (2, 5) and the line joining the points A and B has the gradient of \[\frac{1}{3}\] The perpendicular bisector of AB passes through the point (4, 9). Find;i) the equation of AB;ii) the coordinates of B .Answers:

ID: 199901001015
Content:
img;In the diagram the points A(11, 5), B(5,11), C and D are the vertices if a parallelogram. The points P(17, 8) and Q(21, 16) lie on AD and CD respectively.;i) Find the equation of AD and of CD;ii) Find the ratio DC:QC;iii) Show that triangle PDQ is isosceles and determine its area.Answers:

ID: 199902001002
Content:
img;(a)	The table shows experimental values of two variables x and y.;It is known that x and y are related by an equation of the form \[y=ax+\frac{b}{\sqrt{x}}\] where a and b are constants. Draw the straight line graph of \[y\sqrt{x}\] against \[x\sqrt{x}\] for the given data and use your graph to estimate;(i)	the value of a and of b,;(ii)	the value of x when \[y=\frac{2}{\sqrt{x}}\] `;(b)The variables x and y are related in such a way that when y + x is plotted against \[x^2\] a straight line is obtained which passes through (1, 1) and (13, 4), as shown in the diagram.;img; Express y in terms of x.;Sketch the graph of y against x, indicating on your graph the coordinates of the points where the curve cuts the coordinate axes.Answers:

ID: 199903001001
Content:
Find the equation of the perpendicular bisector of the line joining the point (-5, 4) to the point (9, -3).Answers:

ID: 199903001009
Content:
The point P(x, y) lies on the line 7y = x + 23 and is 5 units from the point (2, 0). Calculate the coordinates of the two possible positions of P.(Note: Please enter the answer with smaller value of x first)Answers:

ID: 199903001015
Content:
Solutions to this question by accurate drawing will not be accepted.;img;The diagram shows the trapezium ABCD in which A is the point (1, 2), B is (3, 8), D is (5, 4), \[\angle ABC=90^{\circ}\] and AB is parallel to DC.;(a)	Find the coordinate of C.;The point E lies on BD and is such that the area of triangle CDE is \[\frac{1}{4}\] of the area of triangle CDB;(b)	Find the coordinate of E.;The point F is such that CDFE is a parallelogram.;(c) Find the coordinates of F and the area of the parallelogram CDFE.Answers:

ID: 199904001004
Content:
(a)The table shows experimental values of two variables x and y. ;It is known that x and y are related by an equation of the form \[y=Ak^x\] where A and k are constants.;(i)	Express this equation in a form suitable for drawing a straight line graph.;(ii)	Draw this graph for the given data and use it to estimate A and k.;(iii) Estimate the value of x for which y = 600.;(b)	The variables x and y are related in such a way that, when \[y^2\] is plotted against \[x^2y\] a straight line is obtained which passes through the points (3, 5) and (5, 9). ;img;Find the two values of y for which \[x=\frac{1}{2}\sqrt5\] (Note: Please enter your answer in ascending order)Answers:

ID: 200001001002
Content:
Find the gradient of the curve \[y=4x^2-20x+27\] at the point P(2, 3). ;The tangent to the curve at P meets the x-axis at A. Calculate the area of the triangle AOP, where O is the origin.Answers:

ID: 200001001010
Content:
Solutions to this question by accurate drawing will not be accepted.;The points P(-3, 0), Q(-2, q), R(5, 6) and S are such that the perpendicular bisector of PR is QS, as shown in the diagram.;img;(a)	Find the value of q.;(b) Show that the area of triangle PQR is \[25units^2\];(c)	Find the coordinates of S, given that the area of the quadrilateral PQRS is \[75units^2\];d)	Find the length of the perimeter of PQRS.Answers:

ID: 200002001003
Content:
img;a) The table shows experimental values of the variables x and y;Using the vertical axis for \[\frac{1}{y}\] and the horizontal axis for \[\frac{1}{x}\]against \[\frac{1}{x}\] and draw a vertical straight line graph. Use this graph to express y in terms of x.;img;b) The diagram shows part of the straight line graph obtained by \[\lg y\] against \[\lg(x + 1)\] Find;i) y in terms of x,;ii) the value of y when x = 3.Answers:

ID: 200003001002
Content:
The points A and B have coordinates (2, 2) and (10, 8) respectively. Find the equation of the perpendicular bisector of AB .Answers:

ID: 200004001005
Content:
img;(a) The table shows experimental values of two variables x and y. ;Using the vertical axis for xy and the horizontal axis for \[x^2\] plot xy against \[x^2\] and obtain a straight line graph. Use your graph to ;(i)	express y in terms of x,;(ii)	estimate the value of x when \[y=\frac{30}{x}\];(b)	Variables x and y are related in such a way that, when y-x is plotted against \[x^2\] a straight line is produced which passes through the points A(4, 6), B(3, 4) and P(p, 4.48), as shown in the diagram.;img ;Find;(i)	y in terms of x,;(ii)	the value of p,;(iii)	the value of x and of y at the point P.Answers:

ID: 200101001002
Content:
The lines y - 3x = 1 and  y + 2x = 6 meet at the point A. Find;(a)	the equation of the line through A which passes through the point B(3, 8),;(b)	the equation of the line through A which is perpendicular to AB .Answers:

ID: 200101001013
Content:
Solutions to this question by accurate drawing will not be accepted.;img;The vertices of the triangle ABC have coordinates (8, 7), (5, -2) and (-2, 5), as shown in the diagram.;AD and CE are perpendicular to BC and AB respectively, and AD and CE meet at the point H. Find;(a) the coordinates of D and of H,;(b)  the ratio AD : HD,;(c) the area of triangle ABC and of triangle HBC.Answers:

ID: 200102001002
Content:
img;a) The table shows experimental values of two variables, x and y, which are known to be related by the equation \[y=\frac{a}{x}+\frac{b}{x^{2}}\];Using the vertical axis for xy, plot xy against \[\frac{1}{x}\] and draw a straight line graph. Use your graph to estimate ;i) the value of a and b,;ii) the value of x for which \[y=\frac{13}{x}\];b) Variables x and y are related by the equation \[y^2=\frac{a}{x^{n}}\]; When lg y is plotted against lg x, a straight line graph is obtained passing through \[(\frac{1}{2},\frac{1}{4})\] and (1, -1) as shown in the diagram.;img; Find the value of a and of n.Answers:

ID: 200103001003
Content:
img;The line joining A(5, 11) and B(0, 1) meets the x-axis at C. The point P lies on AC such that AP : PB = 3 : 2.;(a)	Find the coordinates of P.;A line through P meets the x-axis at Q and angle PCQ = angle PQC. Find;(b)	the equation of PQ.;(c)	The coordinates of Q.Answers:

ID: 200103001012
Content:
Solutions to this question by accurate drawing will not be accepted.;img;The diagram shows part of the curve \[y=x^2-3x-3\] passing through the points A(5, 7), P(2, -5) and C(0, -3). The line through A, parallel to the tangent at P, meets the curve again at B .;(a)	Find the equation of AB and the coordinates of B .;The line through P, parallel to the y-axis, meets AC at Q and AB at R.;(b)	Find the coordinates of R and of Q.;(c)	Evaluate the ratio CQ : QA.;(d)	Find the area of triangle ARQ.Answers:

ID: 200104001002
Content:
img;(a)	The table shows experimental values of two variables x and y. ;It is known that x and y are related by an equation of the form $$y=\frac{(ax)}{(x+b)}$$. Using the vertical axis for y and the horizontal axis for $$=\frac{x}{y}$$, draw a straight line graph of y against for the given data. Use the graph to estimate;(i)	the value of a and of b,;(ii)	the value of x for which y = 3x.;img;(b)	The diagram shows part of a straight line graph drawn to represent the equation cx + dy = xy. ;Find ;(i)	the value of c and of d,;(ii)	the value of y when x = 0.2.Answers:

ID: 200104001006
Content:
The parametric equations of a curve are $$x=t^2-4t+5, y=t^2+4$$;i) Express $$\frac{\mathrm{d} y}{\mathrm{d} x}$$ in terms of t.;A is the point on the curve where t = 1. The tangent to the curve at A meets the x- axis at B .;ii) Find the area of triangle AOB, where O is the origin.;C is the minimum point on the curve and D is the point on the curve at which the tangent is parallel to the y- axis.;iii) Find the coordinates of C and D and show that AC = AD.;iv) Find the value of the constant k for which the cartesian equation of the curve is $$(y-x+1)^2=16(y-k)$$Answers:

ID: 200201001012
Content:
The curve $$y=ax^n$$ where a and n are constants, passes through the points (2.25, 27), (4, 64) and (6.25, p). Calculate the value of a, of n and of p.Answers:

ID: 200201001014
Content:
Solutions to this question by accurate drawing will not be accepted;img;The diagram, which is not drawn to scale, shows a trapezium ABCD in which BC is parallel to AD. The side AD is perpendicular to DC. Point A is (1, 2), B is (4, 11) and D is (17, 10). Find;(a)	the coordinates of C.;The lines AB and DC are extended to meet at E. Find;(b)	the coordinates of E,;(c)	the ratio of the area of triangle EBC to the area of trapezium ABCD.Answers:

ID: 200203001009
Content:
The line $$2y = 3x- 6$$ intersects the curve $$xy = 12$$ at the points P and Q. Find the equation of the perpendicular bisector of PQ.Answers:

ID: 200203001013
Content:
A rectangle of area $$ym^2$$ has sides of length x m and (Ax + B)m, where A and B are constants and x and y are variables. Values of x and y are given in the table below.;img;i) Use the data above in order to draw, on graph paper, the straight line graph of $$\frac{y}{x}$$;ii) Use your graph to estimate the value of A and B;iii) On the same diagram, draw the straight line representing the equation $$y=x^2$$ and explain the significance of the value x given by the point of intersection of the two lines.;iv) State the value approached by the ratio of the two sides of the rectangle as x becomes increasingly large.Answers:

ID: 200301001010
Content:
img;The table above shows experimental values of the variables x and y. On graph paper draw the graph of xy against $$x^2$$;Hence ;i) express y in terms of x,;ii) find the value of x for which $$x=\frac{45}{y}$$Answers:

ID: 200302001011
Content:
Solutions to this solution_checking_question by accurate drawing will not be accepted.;img;The diagram shows a triangle ABC in which A is the point (3, 2), C is the point (7, 4) and angle ACB = $$90^{\circ}$$. The line BD is parallel to AC and D is the point (13.5, 11). The lines BA and DC are extended to meet at E. Find;(a)	the coordinates of B,;(b)	the ratio of the area of the quadrilateral ABCD to the area of the triangle EBD.Answers:

ID: 200303001011
Content:
Solutions to this question by accurate drawing will not be accepted.;img;The diagram, which is not drawn to scale, shows a parallelogram OABC where O is the origin and A is the point (2, 6). The equations of OA, OC and CB are y = 3x, y = 0.5x and y = 3x - 15 respectively. The perpendicular from A to OC meets OC at the point D. Find;(a)	the coordinates of C, B and D,;(b)	the perimeter of the parallelogram OABC, correct to 1 decimal place.Answers:

ID: 200404001009
Content:
In order that each of the equations;(i) $$y=ab^x$$;(ii) $$y=Ax^{(k)}$$;(iii) px + qy = xy;where a, b, A, k, p and q are unknown constants, may be represented by a straight line, they each need to be expressed in the form Y = mX + c, where X and Y are each functions of x and/or y, and m and c are constants. ;Copy the following table and insert in it an expression for Y, X, m and c for each case.;img;Answers:

ID: 200404001011
Content:
The diagram shows a trapezium OABC, where O is the origin. ;The equation of OA is y = 3x and the equation of OC is y + 2x = 0. The line through A perpendicular to OA meets the y-axis at B and BC is parallel to AO. Given that the length of OA is $$\sqrt{250}$$ units, calculate the coordinates of A, of B and of C.Answers:

ID: 200503001010
Content:
Solutions to this question by accurate drawing will not be accepted;;The diagram, which is not drawn to scale, shows a quadrilateral ABCD in which A is (0, 10), B is (2, 16) and C is (8, 14).;(a) Show that triangle ABC is isosceles.;The point D lies on the x-axis and is such that AD = CD. Find;(b) the coordinates of D,;(c) the ratio of the area of triangle ABC to the area of triangle ACD.Answers:

ID: 200503001012
Content:
Variables x ands y are related by the equation $$yx^n=a$$, where a and n are constants. The table below shows measured values of x and y.;img;i)	On graph paper plot lg y against lg x using a scale of 2 cm to represent 0.1 on the lg x axis and 1 cm to represent 0.1 on the lg y axis. Draw a straight line graph to represent the equation $$yx^n=a$$.;ii)	Use your graph to estimate the value of a and n.;iii)On the same diagram, draw the line representing the equation $$y = x^2$$;and hence find the value of x for which $$x^{(n+2)}=a$$.Answers:

ID: 200603001013
Content:
The variables x and y are related by the equation $$y=10^{-A}b^x$$, where A and b are constants. The table below shows values of x and y .;img;i) Draw a straight line graph of lg y against x, using a scale of 2 cm to represent 5 units on the x-axis and 2 cm to represent 0.5 units on the lg y-axis. ;ii) Use your graph to estimate the value of A and of b .;iii) Estimate the value of x when y = 10.;iv) On the same diagram, draw the line representing the equation $$y^5=10^{-x}$$and hence find the value of x for which $$10^{(A-( \frac{x}{5}))}=bx$$.Answers:

ID: 200703001013
Content:
The table shows experimental values of two variables, x and y.;It is known that x and y are related by the equation $$y=10+Ab^x$$, where A and b are constants.;(i) Using graph paper, draw the graph of lg (y-10) against x and use your graph to estimate the value of A and of b .;(ii) By drawing a suitable line on your graph, solve the equation.Answers:

ID: 200704001011
Content:
Solution to this question by accurate drawing will not be accepted.;img;The diagram which is not drawn to scale, shows a triangle ABC in which pointA is (9,9) and the pointB is (1,-3). The pointC lies on the perpendicular bisector of AB and equation of the line BC is y=8x-11. Find;(i) The equation of the perpendicular bisector of AB,;(ii) The coordinates of C.;(iii) Find the coordinates of D.;(iv) Show that AB=2CD.Answers:

ID: 200803001012
Content:
The variables x and y are connected by the equation $$y = kx^x$$, where k and b are constants. Experimental values of x and y were obtained. ;img;The diagram above shows the straight line graphs, passing through the points (0, 1.3) and (11, 0.8), obtained by plotting lg y against x. Estimate ;(i) the value, to 2 significant figures, of k and of b,;(ii) the value of y when x=8.Answers:

ID: 200804001011
Content:
img;The diagram shows two circles  $$C_1$$  and  $$C_2$$ . Circle  $$C_1$$  has its centre at the origin  O . Circle  $$C_2$$  passes through  O  and has its centre at  Q . The point P(8,-6) lies on both circles and  OP  is a diameter of  $$C_2$$ .;(i) Find the equation of  $$C_1$$ .;(ii) Find the equation of  $$C_2$$ .;The line through  Q  perpendicular to  OP  meets the circle  $$C_1$$  at the point  A  and  B .;(iii) Show that the x-coordinates of  A  and  B  are  $$a + b \sqrt3$$  and  $$a - b \sqrt3$$  respectively, where  a  and  b  are integers to be found.Answers:

ID: 200903001007
Content:
Find the coordinates of the mid-point of the straight line joining the points of intersection of the curve $$x^2 +2y^2 + 5x =68$$ and the line $$2y+3x=9$$.Answers:

ID: 200903001010
Content:
The mass,  m mg, of a radioactive substance decreases with time, t hours. Measured values of m and t are given in the following table.;img;It is known that m and t are related by the equation $$m = m_0 e^{-kt}$$, where $$m_0$$ and $$k$$ are constants.;(i)	Plot $$\ln m$$ against t for the given data and draw a straight line graph.;(ii)	Use your graph to estimate the value of k and of $$m_0$$.;(iii)	Estimate the number of hours for the mass of the substance to be halved.Answers:

ID: 200903001011
Content:
img;The diagram shows a trapezium ABCD in which AB is parallel to DC and angle $$BAD=90^{\circ}$$. The point A is (0,6) and the point D is (2,-2).;(i) Find the equation of AB.;Given that B lies on the line y=x, find;(ii) the coordinates of B.;Given that the length of DC is twice the length of AB, find ;(iii) the coordinates of C,;(iv) the area of the trapezium ABCD.Answers:

ID: 200904001009
Content:
img;The diagram shows a circle with centre C(2,-1) and radius 5.;(i) Given that the equation of the circle is $$x^2 +y^2+2gx+2fy+c=0$$, find the value of each of the constants g, f and c.;The points A and B lie on the circle such that the line AC is parallel to the x-axis and the line AB passes through the origin O;(ii) Write down the coordinates of A.;(iii) Find the equation of AB.;(iv) Find the coordinates of B.Answers:

ID: 200904001011
Content:
img;The diagram shows three fixed points O, A and D such that OA = 17cm, OD = 31cm and $$\angle AOD =90^{\circ}$$. The lines AB and DC are perpendicular to the line OC which makes an angle $$\theta$$ with the line OD. The angle $$\theta$$ can vary in such a way that the point B lies between the points O and C.;(i) Show that $$AB + BC + CD = (48\cos  \theta + 14 \sin  \theta)cm$$.;(ii) Find the values of $$\theta$$ for which AB+BC+CD = 49cm.;(iii) State the maximum value of AB+BC+CD and the corresponding value of $$\theta$$.Answers:

ID: 201003001007
Content:
img;The table shows experimental values of two variables, x and y, which are connected by an equation of the form $$yx^n=k$$, where n and k are constants.;(i) Using a scale of 1cm to 0.1unit on each axis, plot lg y against lg x and draw a straight line graph.;(ii) Use your graph to estimate the value of k and of n.Answers:

ID: 201003001012
Content:
(i)Write down the equation of the circle with centre A(-3,2) and radius 5.;This circle intersects the y-axis at points P and Q .;(ii) Find the length of PQ .;A second circle, centre B, also passes through P and Q .;(iii) State the y-coordinate of B .;Given that the x-coordinate of B is positive and that the radius of the second circle is $$\sqrt{80}$$, find ;(iv) the x-coordinate of B.;The equation of the circle, centre B, which passes through P and Q, may be written in the form $$x^2+y^2+2gx+2fy+c =0$$.;(v) State the value of g and of f, and find the value of c .Answers:

ID: 201004001009
Content:
img;The diagram shows a triangle ABC with vertices at A(0,5), B(8,14), C(k,15). Given that AB =BC,;(i) find the value of k.;A line is drawn from B to meet the x-axis at D such that AD=CD.;(ii) Find the equation of BD and the coordinates of D.;(iii) Show that the area of the triangle ABC is $$\frac{2}{7}$$ of the area of the quadrilateral ABC.Answers:

\end{document}
